% \documentclass{report}

% \usepackage{subcaption} % package for subfigures
% \usepackage{hyperref}  % package for linking figures etc
% \usepackage{enumitem}  % package for description with bullets
% \usepackage{graphicx}  % package for importing images
% \usepackage{mathtools} % package for math equation
% \usepackage{mathrsfs}  % package for math font
% \usepackage{indentfirst} % package for getting ident after section or paragraph
% \usepackage[export]{adjustbox}
% \usepackage{multirow}  % package for tables, multir
% \usepackage{amssymb}
% % \usepackage{tabu}   % for tables 
% \setlength{\parindent}{2em} % how much indent to use when we start a paragraph

% \graphicspath{ {./theory/figures/} }       % path for images

% \begin{document}

\chapter{Conclusion - Future work}

\section{Conclusion}
In this thesis we explore the problem of action recognition and localization. We design a network base on \cite{DBLP:journals/corr/HouCS17}
combined with some elements from \cite{DBLP:journals/corr/abs-1712-09184}, \cite{Ren:2015:FRT:2969239.2969250}, \cite{Girshick:2015:FR:2919332.2920125},
\cite{DBLP:journals/corr/abs-1903-00304} and \cite{hara3dcnns}. \par

We write a pytorch implementation expanding code only from \cite{jjfaster2rcnn}. Furthermore, we wrote our own code using some CUDA functions designed by us (like
calculating connection scores, modifying tubes etc). \par

We tried to design a design a Tube Proposal Network for proposing action tubes in given video segments, inspired by Faster R-CNN's RPN.
We designed it using general anchors and not dataset specific anchors in order to try to generalize our approach for several datasets, on the contrary with
the approach proposed by \cite{DBLP:journals/corr/abs-1712-09184}, in which it uses the most frequently appearing anchors as the general anchors. \par

On top of that, we designed a naive connection algorithm for connecting  our proposed action tubes based on the one proposed by \cite{DBLP:journals/corr/abs-1712-09184}.
In our approach, we use the same scoring policy, which is a combination between actioness and overlaping scores. The main difference is that we avoid to calculate
all the possible combinations using an updating threshold. We, also, tried another connection algorithm inspired by \cite{DBLP:journals/corr/abs-1903-00304}. However,
our implementation wasn't very good so, we didn't explore all of its potentials. \par

Finally, we explored several classifiers for the classification stage of our network, which are a RNN, a SVM and a MLP.  We used an implementation taken from Fast RCNN
for the SVM classifier, which included hard negatives mining training procedure. Furthermore, we explore some training techniques for best classification performance and
2 training approaches, the classic one and using pre-extracted features. 

\section{Future work}
There is a lot of room for improvement for our network, in order to achieve state-of-the-art results. The most important are described in next paragraphs.

\paragraph{Improving TPN proposals} We implemented 2 networks for proposing action tubes in a video segment. We managed to achieve about 63\% recall score for
sample duration = 16 and about 80\% recall for sample duration = 8. Theses scores show that there is plenty room for improvement especially for sample duration = 16.
Even though a lot of networks' architectures have been explored for regression, a good idea would be to try other networks, not necessarily inspired by object detection
networks like we did. On top of that, adding a $\lambda$ factor in training loss would be a good idea and exploring which is the best approach.
So training loss could be defined as:
\begin{equation} 
\begin{split}
 L  =  \sum_iL_{cls}(p_i, p_i^*) + \lambda_1 \sum_ip_i^*L_{reg}(t_i,t_i^*) + \lambda_2  \sum_iq_i^*L_{reg}(c_{i}, c_{i}^*) \\
\end{split}
\end{equation}

Furthermore, it would be a good idea to use SSD's (\cite{DBLP:journals/corr/LiuAESR15}) proposal network instead of RPN, in order to compare result. Finally,
we could experiment using Feature Pyramid Networks, which could be extracted in 3 dimensions as another feature extractor or some other type of 3D ResNet.

\paragraph{Changing Connection algorithm}
In this thesis, another challenge we came was connecting proposed ToIs for proposing action tubes. We implemented a very naive algorithm, which wasn't
able to give us very good proposals despite the changes we tried to do. We implemented another connection algorithm which was base in a estimation on temporal
progress of an action and their overlap. Although it also didn't give us very good proposals, we believe that we should explore this algorithm's ponytails. That's
because it takes advantage of the progress of the action, which the previous algorithm didn't.

\paragraph{Explore other  classification techniques}
For classification stage, we experiment mainly on a SVM classifier for JHMDB dataset and we didn't get involved a lot with UCF dataset. We found the best feature maps from
JHDMB and we used the same for UCF. We think that we should explore UCF's feature maps even though we believe that there will be the same. It is essential to confirm our
assumption. In addition, we could try other classification techniques like random forest or experiment more with RNN classifier for the UCF dataset.
Finally, another classification procedure would be a good idea, like extracting first all the possible action tubes and then using other network's features for classification
stage.

% \end{document}
