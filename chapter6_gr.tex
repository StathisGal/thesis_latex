% \documentclass{report}

% \usepackage[ english, greek]{babel}
% \usepackage[utf8]{inputenc}
% \usepackage[LGR, T1]{fontenc}

% % % 

% \newcommand{\tl}{\textlatin}
% \newcommand{\en}{\selectlanguage{english}}
% \newcommand{\gr}{\selectlanguage{greek}}

% \usepackage{hyperref}  % package for linking figures etc
% \usepackage{enumitem}  % package for description with bullets
% \usepackage{graphicx}  % package for importing images
% \usepackage{mathtools} % package for math equation
% \usepackage{mathrsfs}  % package for math font
% \usepackage{indentfirst} % package for getting ident after section or paragraph
% \usepackage{subcaption} % package for subfigures
% \usepackage[export]{adjustbox}
% \usepackage{longtable} % package for multi pages tables
% \usepackage{multirow}  % package for tables, multirow
% \usepackage{amssymb}
% \usepackage{esvect}
% \usepackage[
% backend=bibtex,
% citestyle=authoryear,
% % citestyle=authoryear-comp,
% % citestyle=authoryear-ibid,
% bibstyle=numeric,
% sorting=ynt,
% % style=numeric,
% % style=alphabetic ,
% ]{biblatex}
% \addbibresource{References}

% \graphicspath{ {./theory/figures/} }       % path for images

% \begin{document}
\gr 

\chapter{Επίλογος - Μελλοντικές επεκτάσεις}
\section{Επίλογος}
Σε αυτή τη διατριβή εξερευνήσαμε το πρόβλημα της αναγνώρισης και του εντοπισμού ανθρώπινης δράσης σε βίντεο.
Σχεδιάσαμε  ένα δίκτυο βασισμένο στην προσέγγιση των  \cite{DBLP:journals/corr/HouCS17} σε συνδυασμό με ορισμένα στοιχεία από τους \en\cite{DBLP:journals/corr/abs-1712-09184},
\cite{Ren:2015:FRT:2969239.2969250}, \cite{Girshick:2015:FR:2919332.2920125}, \cite{DBLP:journals/corr/abs-1903-00304} \gr
 και \en \cite{hara3dcnns}\gr. \par

 Γράψαμε μια \en pytorch \gr υλοποίηση παίρνοντας κώδικα  μόνο από το \en \cite{jjfaster2rcnn}\gr. Επιπλέον, γράψαμε τον δικό μας κώδικα χρησιμοποιώντας
 μερικές λειτουργίες της γλώσσας \en CUDA \gr που έχουν σχεδιαστεί από εμάς (όπως
υπολογισμός των βαθμολογιών σύνδεσης, τροποποίηση \tl{tubes} κλπ). \par

Προσπαθήσαμε να σχεδιάσουμε   το \en TPN\gr, ένα δίκτυο  που εξάγει \en ToIs, \gr ακολουθίες πλαισίων  δηλαδή, που πιθανώς να περιέχουν κάποια δράση  
σε δεδομένο τμήμα του βίντεο, εμπνευσμένο από το \en  RPN \gr  του \en Faster R-CNN\gr. To σχεδιάσαμε
χρησιμοποιώντας γενικευμένα \en anchors \gr  και όχι συγκεκριμένα για κάθε σύνολο δεδομένων. Προσπαθούμε δηλαδή
να γενικεύσουμε την προσέγγισή μας για διάφορα σύνολα δεδομένων, αντίθετα με την προσέγγιση
που προτείνεται από τους \en \cite{DBLP:journals/corr/abs-1712-09184}\gr, στην οποία χρησιμοποιoύνται τα πιο συχνά εμφανιζόμενα
\en anchors \gr για κάθε σύνολο δεδομένων.

Επιπροσθέτως, σχεδιάσαμε έναν αφελή αλγόριθμο σύνδεσης για τη σύνδεση
των προτεινόμενων \en ToIs \gr  με βάση αυτόν που προτάθηκε απ' τους \en \cite{DBLP:journals/corr/abs-1712-09184}\gr.
Στην προσέγγισή μας, χρησιμοποιούμε την ίδια πολιτική βαθμολόγησης, η οποία είναι ένας συνδυασμός των βαθμολογιών της πιθανότητας ύπαρξης δράσης και του σκορ επικάλυψης.
Η κύρια διαφορά είναι ότι αποφεύγουμε να υπολογίζουμε
πιθανούς συνδυασμούς, χρησιμοποιώντας ένα όριο ενημέρωσης, που μπορεί να ανανεώνεται. Επίσης, δοκιμάσαμε κι άλλον έναν
αλγόριθμο σύνδεσης εμπνευσμένος απ' τους \cite{DBLP:journals/corr/abs-1903-00304}.
Ωστόσο, η εφαρμογή μας δεν ήταν τόσο καλή όσο η προηγούμε, συνεπώς δεν εξερευνήσαμε όλες τις δυνατότητες του.

Τέλος, διερευνήσαμε αρκετούς ταξινομητές  για το στάδιο ταξινόμησης του
δικτύου. Αυτοί είναι: έναν \en RNN\gr, ένα Γραμμικό ταξινομητή, έναν \en SVM \gr  και έναν ταξινομητή \en MLP\gr.
Χρησιμοποιήσαμε μια εφαρμογή απ' το  Fast RCNN για τον ταξινομητή  \en SVM\gr, η οποία περιελάμβανε την διαδικασία
εκπαίδευσης μέσω σκληρών αρνητικών. Εξετάσαμε μερικές τεχνικές εκπαίδευσης για
βέλτιστη απόδοση ταξινόμησης και 2 εκπαιδευτικές προσεγγίσεις για τον ταξινομητή \en MLP\gr, την κλασσική και μία που
χρησιμοποιούμε προεξαγόμενα χαρακτηριστικά.

\section{Μελλοντικές επεκτάσεις}

Υπάρχουν πολλά περιθώρια βελτίωσης για το δίκτυό μας, προκειμένου να επιτευχθεί
τελευταίας τεχνολογίας αποτελέσματα. Οι σημαντικότερες περιγράφονται στις επόμενες παραγράφους.

\paragraph{Βελτίωση των προτάσεων του \en TPN\gr}

Υλοποιήσαμε 2 δίκτυα για την πρόταση ακολουθιών από πλαίσια σε ένα τμήμα βίντεο. Πετύχαμε περίπου 63\% βαθμολογία \en recall \gr
για τη διάρκεια του δείγματος ίση με 16 καρέ και περίπου 80\% \en recall \gr για τη διάρκεια του δείγματος ίση με 8. Αυτά τα σκορ 
δείχνουν ότι υπάρχει αρκετός χώρος για βελτίωση ειδικά για την περίπτωση με δείγμα 16 καρέ.
Παρόλο που έχουν διερευνηθεί πολλές αρχιτεκτονικές δικτύων για
παλινδρόμηση, μια καλή ιδέα θα ήταν να δοκιμάσουμε άλλα δίκτυα, τα οποία δεν είναι απαραίτητα εμπνευσμένη
από δίκτυα εντοπισμού αντικειμένων όπως κάναμε εμείς. 
Επιπλέον, προσθέτοντας έναν παράγοντα \textit{λ} στον τύπο του \en training loss \gr
 θα ήταν μια καλή ιδέα και θα διερευνούσε ποια είναι η καλύτερη προσέγγιση αυτού.
Έτσι, η απώλεια εκπαίδευσης θα μπορούσε να οριστεί ως:
\begin{equation} 
\begin{split}
 L  =  \sum_iL_{cls}(p_i, p_i^*) + \lambda_1 \sum_ip_i^*L_{reg}(t_i,t_i^*) + \lambda_2  \sum_iq_i^*L_{reg}(c_{i}, c_{i}^*) \\
\end{split}
\end{equation}
Επιπλέον, θα ήταν μια καλή ιδέα να χρησιμοποιήσουμε την μέθοδο του \en SSD  (\cite{DBLP:journals/corr/LiuAESR15}) \gr που προτείνει \en RoIs \gr 
αντί για το \en RPN\gr, για να συγκρίνουμε το αποτέλεσμα.  Τέλος, θα μπορούσαμε να πειραματιστούμε χρησιμοποιώντας τα δίκτυα \en Feature Pyramid (\cite{8099589}), \gr
τα οποία θα μπορούσαν να επεκταθούν σε 3 διαστάσεις ως ένα άλλο δίκτυο εξαγωγής χαρακτηριστικών ή κάποιο άλλο είδος \en 3D ResNet\gr.

\paragraph{Αλλαγή του αλγορίθμου σύνδεσης}
Σε αυτή τη διατριβή, μια άλλη πρόκληση που αντιμετωπίσαμε ήταν η σύνδεση των προτεινόμενων \en ToIs \gr για την πρόταση \en action tubes\gr. Υλοποιήσαμε έναν πολύ αφελή αλγόριθμο,
που δεν ήταν σε θέση να μπορεί να μας δώσει πολύ καλές προτάσεις παρά τις αλλαγές που προσπαθήσαμε να κάνουμε. Υλοποιήσαμε έναν άλλο αλγόριθμο σύνδεσης που ήταν βασισμένος στην εκτίμηση της χρονικής
πρόοδο ενός \en action tube \gr και την αλληλεπίδραση του με άλλα. Αν και δεν μας έδωσε και πολύ καλές προτάσεις, πιστεύουμε ότι πρέπει να εξερευνήσουμε τις δυνατότητες αυτού του αλγορίθμου. Κι αυτό 
επειδή είναι σε θέση να   εκμεταλλεύεται την πρόοδο της ενέργειας, την οποία δεν είχε ο προηγούμενος αλγόριθμος.

\paragraph{Εξερεύνηση άλλων τεχνικών ταξινόμησης}

Για το στάδιο ταξινόμησης, πειραματιστήκαμε κυρίως πάνω σε έναν ταξινομητή \en SVM  \gr για το σύνολο δεδομένων \en JHMDB \gr και δεν ασχοληθήκαμε καθόλου  με το σύνολο δεδομένων \en UCF-101\gr. Ο πρώτος μας
στόχος είναι να είμαστε σε θέση να εξάγουμε καλά αποτελέσματα ταξινόμησης για το σύνολο δεδομένων \en UCF-101\gr.  Πιστεύουμε ότι θα πρέπει να διερευνήσουμε τους χάρτες χαρακτηριστικών του \en UCF-101 \grκαι  τεχνικές  που εφαρμόζονται στους χάρτες χαρακτηριστικών πριν από την ταξινόμηση. Επιπλέον, θα μπορούσαμε να δοκιμάσουμε άλλες τεχνικές ταξινόμησης όπως \en Random Forests \gr ή να πειραματιστούμε περισσότερο με τον ταξινομητή \en RNN \gr για το σύνολο δεδομένων \en UCF-101\gr.
Τέλος, μια άλλη διαδικασία ταξινόμησης θα ήταν μια καλή ιδέα, όπως η εξαγωγή πρώτα όλων των πιθανών \en action tubes \gr και, στη συνέχεια, η χρήση άλλων δικτύων για εξαγωγή χαρακτηριστικών προκειμένου
να ταξινομήσουμε τα \en action tubes\gr.

\en
% \end{document}