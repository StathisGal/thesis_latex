\documentclass{report}

\usepackage[greek,english]{babel}

\newcommand{\tl}{\textlatin}
\newcommand{\en}{\selectlanguage{english}}
\newcommand{\gr}{\selectlanguage{greek}}

\usepackage{hyperref}  % package for linking figures etc
\usepackage{enumitem}  % package for description with bullets
\usepackage{graphicx}  % package for importing images
\usepackage{mathtools} % package for math equation
\usepackage{mathrsfs}  % package for math font
\usepackage{indentfirst} % package for getting ident after section or paragraph
\usepackage{subcaption} % package for subfigures
\usepackage[export]{adjustbox}
\usepackage{longtable} % package for multi pages tables
\usepackage{multirow}  % package for tables, multirow
\usepackage{amssymb}
\usepackage{esvect}
\usepackage[
backend=bibtex,
citestyle=authoryear,
% citestyle=authoryear-comp,
% citestyle=authoryear-ibid,
bibstyle=numeric,
sorting=ynt,
% style=numeric,
% style=alphabetic ,
]{biblatex}
\addbibresource{References}

\graphicspath{ {./theory/figures/} }       % path for images

\begin{document}
\gr
\section{Σχετική βιβλιογραφία}
Σε αυτή την ενότητα, παρουσιάζουμε ορισμένες από τις πιο συναφείς μεθόδους για την εργασία μας και άλλες που μελετήθηκαν για τον σχεδιασμό αυτής της προσέγγισης.
Οι μέθοδοι αυτές  χωρίζονται σε δύο ενότητες \textit{Aναγνώριση Δραστηριότητας} και \textit{Εντοπισμός Δραστηριότητας}. Το πρώτο μέρος αναφέρεται σε κλασικές μεθόδους
ταξινόμησης δράσης που εισήχθησαν μέχρι πρόσφατα και το δεύτερο μέρος, αντίστοιχα, σε πρόσφατες μεθόδους εντοπισμού της δράσης. 

\subsection{Αναγνώριση Δραστηριότητας}
Οι πρώτες προσεγγίσεις για την κατάταξη της δράσης αποτελούνταν από δύο βήματα α) αρχικά υπολογισμός σύνθετων ``χειροποίητων'' χαρακτηριστικών από ακατέργαστα καρέ βίντεο
και β) εκπαίδευση ενός ταξινομητή με βάση αυτά τα χαρακτηριστικά. Αυτά τα χαρακτηριστικά μπορούν να διαχωριστούν σε 3 κατηγορίες:
1) προσεγγίσεις χωροχρονικού όγκου \tl{(space-time volume)}, 2) τροχιές \tl{(trajectories)} και 3) χωροχρονικά  χαρακτηριστικά. Για τις μεθόδους χωροχρονικού όγκου,
η προσέγγιση είναι η εξής: Με βάση τα \tl{training} βίντεο, το σύστημα συνάπτει ένα μοντέλο τρισδιάστατου χωροχρόνου, συνενώνοντας δυσδιάστατες εικόνες
(διάσταση \tl{\textit{x-y}}) κατά τη διάρκεια του χρόνου (διάσταση \tl{\textit{t}} ή \tl{\textit{z}}), για την αναπαράσταση κάθε δράσης.
Όταν το σύστημα δέχεται ένα βίντεο που δεν έχει ετικέτα, κατασκευάζει μια τρισδιάστατη χωροχρονική αναπαράσταση που αντιστοιχεί σε αυτό το βίντεο.
Αυτό η νέα τρισδιάστατη αναπαράσταση, στη συνέχεια, συγκρίνεται με κάθε μοντέλο \tl{3D} χωροχρόνου, συγκρίνοντας την ομοιότητα στο σχήμα
και την εμφάνιση μεταξύ αυτών των δύο χωροχρονικών όγκων.
Το σύστημα εξάγει την κατηγορία του άγνωστου βίντεο, αντιστοιχώντας την με αυτήν της δράσης με την υψηλότερη ομοιότητα. Επιπλέον, υπάρχουν
διάφορες παραλλαγές των  χωροχρονικών αναπαραστάσεων. Αντί της αναπαράστασης \tl{space-time volume}, το σύστημα μπορεί να αναπαριστά τη κάθε δράση ως τροχιές
σε χωροχρονικές διαστάσεις ή ακόμη περισσότερο, η ενέργεια μπορεί να αναπαρασταθεί ως ένα σύνολο χαρακτηριστικών που εξάγονται από τον χωροχρονικό όγκο ή τις τροχιές.
Οι ``καθαρές'' χωροχρονικές αναπαραστάσεις περιλαμβάνουν μεθόδους σύγκρισης των περιοχών προσκηνίου ενός ατόμου (δηλ. σιλουέτες) όπως \en \cite{BobickAaron},\gr
συγκρίνοντας όγκους σε σχέση με επιφάνεια τους όπως οι \en \cite{1467296}\gr. Η μέθοδος \en \cite{4270510}\gr χρησιμοποιεί \tl{oversegmented} όγκους, αυτομάτως
υπολογίζοντας ένα σύνολο τμημάτων τρισδιάστατου όγκου \tl{ XYT} που αντιστοιχεί σε έναν κινούμενο άνθρωπο. \en\cite{4587727} \gr πρότειναν φίλτρα για
να αποτυπόνουν τα χαρακτηριστικά του χωροχρονικού όγκου, προκειμένου να τα ταιριάζουν πιο αξιόπιστα και αποδοτικά.
Από την άλλη πλευρά, οι προσεγγίσεις με βάση την τροχιά περιλαμβάνουν την αναπαράσταση μιας ενέργειας ως σύνολο 13 κοινών διαδρομών (\en\cite{1541250}\gr) ή
τη χρήση ενός συνόλου \tl{\textit{XYZT}}-διαστάσεων κοινών τροχιών που λαμβάνονται από κινούμενες κάμερες  (\en\cite{1541251}\gr).
Τέλος, διάφορες μέθοδοι χρησιμοποιούν τοπικά χαρακτηριστικά που εξάγονται από χωροχρονικούς όγκους τριών διαστάσεων,
όπως η εξαγωγή τοπικών χαρακτηριστικών σε κάθε καρέ του βίντεο και η ένωση του χρονικά (\en\cite{784616, 990935, 1544882}\gr,
η εξαγωγή αραιών χωροχρονικών τοπικών σημείων ενδιαφέροντος από τρισδιάστατους όγκους (\en\cite{1238378, 1570899,  Niebles, 1467373, Ryoo2006}\gr)
Οι προσεγγίσεις αυτές κατέστησαν την επιλογή των χαρακτηριστικών  σημαντικό παράγοντα για την απόδοση του δικτύου.
Αυτό συμβαίνει επειδή οι διαφορετικές κατηγορίες δράσεν μπορεί να διαφέρουν δραματικά από την άποψη της εμφάνισής τους και των μοτίβων κίνησης.
Ένα άλλο πρόβλημα ήταν ότι οι περισσότερες από αυτές τις προσεγγίσεις κάνουν υποθέσεις, υπό τις οποίες το βίντεο λήφθηκε λόγω προβλημάτων όπως το γεμάτo
φόντο,  γωνιές κάμερας κλπ. Μια ανασκόπηση των τεχνικών, που χρησιμοποιούνταν  μέχρι το 2011, παρουσιάζεται στο \en\cite{Aggarwal:2011:HAA:1922649.1922653}\gr.  \par

Τα πρόσφατα αποτελέσματα σε βαθιές αρχιτεκτονικές και ειδικά στον τομέα της ταξινόμηση εικόνας  έδωσε κίνητρο στους ερευνητές να εκπαιδεύσουν δίκτυα \tl{CNN} για
το πρόβλημα της αναγνώρισης δράσης. Η πρώτη σημαντική απόπειρα έγινε από τους \en\cite{6909619}\gr. Σχεδίασαν την αρχιτεκτονική τους με βάση το καλύτερο \tl{CNN}
στον διαγωνισμό  \tl{ImageNet}.
Εξερευνούν διάφορες μεθόδους για τη σύντηξη των χωροχρονικών λειτουργιών χρησιμοποιώντας δυσδιάσατες διαδικασίες κυρίως και τρισδιάστατη συνέλιξη μόνο σε αργή σύντηξη. Οι \en \cite{simonyan2014two} \gr
χρησιμοποιήσαν 2 CNNs, ένα για χωρικές πληροφορίες και ένα για οπτική ροή
και τα συνδύασαν  με τη χρήση της καθυστερημένης σύντηξης. Δείχνουν ότι η εξόρυξη χωρικού περιεχομένου από τα βίντεο και περιεχόμενο κίνησης από την οπτική ροή μπορεί να βελτιώσει
σημαντικά την ακρίβεια της αναγνώρισης της δράσης. Οι \en\cite{DBLP:journals/corr/FeichtenhoferPZ16} \gr επέκτειναν αυτή την προσέγγιση με τη χρήση πρώιμης σύντηξης στο τέλος των \en convolutional
layeers \gr αντί για  καθυστερημένης σύντηξης, η οποία λαμβάνει χώρα στο τελευταίο επίπεδο του δικτύου. Πάνω σ' αυτό, χρησιμοποίησαν ένα δεύτερο δίκτυο για το χρονικό περιεχόμενοοοο το οποίο
συνδέουν με το το άλλο δίκτυο με χρήση της καθυστερημένης σύντηξης. Επιπλέον, οι \en \cite{DBLP:journals/corr/WangXW0LTG16} \gr στήριξαν την μέθοδος τους
σε αυτήν που πρότειναν οι \en \cite{simonyan2014two}\gr. Ασχολούνται με το πρόβλημα του την εύρεσης χρονικού περιεχομένου και εκπαιδεύουν το δίκτυο τους, παρέχοντας του λίγα δείγματα.
Η προσέγγισή τους, την οποία ονόμασαν \en Temporal Segment Network (TSN), \gr διαχωρίζει το βίντεο εισόδου σε K τμήματα  και ένα σύντομο απόσπασμα από κάθε τμήμα επιλέγεται για ανάλυση.
Στη συνέχεια, συνδέουν  το εξαγόμενο χωροχρονικό περιεχόμενο, πραγματοποιώντας τελικά την πρόβλεψή τους. Πιο πρόσφατα, οι \en \cite{DBLP:journals/corr/ZhangWWQW16} \gr και οι \en \cite{DBLP:journals/corr/ZhuLNH17a} \gr χρησιμοποίησαν την \en two-stream \gr, επίσης. Οι \en \cite{DBLP:journals/corr/ZhangWWQW16} \gr αντικατέστησαν την οπτική ροή με ένα διάνυσμα κίνησης που μπορεί να ληφθεί απευθείας
από τα  συμπιεσμένα βίντεο χωρίς επιπλέον υπολογισμό και το τροφοδοτούν στο δίκτυο. Οι \en  \cite{DBLP:journals/corr/ZhuLNH17a} \gr εκπαίδευσαν ένα  \en CNN  \gr για τον υπολογισμό της οπτικής ροής,
καλώντας το \en MotionNet \gr και χρησιμοποίησαν ένα \en CNN \gr ως  χρονικό \en stream \gr για προβάλλουν τις πληροφορίες κίνησης έργου σε κατηγορίες δράσεωνο. Τέλος
χρησιμοποιούν Την καθυστερημένη σύντηξη μέσω της μέσης τιμής με βάρη των σκορ πρόβλεψης των χρονικών και χωρικών \en stream\gr. Από την άλλη πλευρά, μια νέα προσέγγιση εισήχθη από τους
\gr\cite{DBLP:journals/corr/abs-1711-01467}\gr ενσωματώνοντας χάρτες προσοχής με σκοπόν να βελτιώσουν σημαντικά  την απόδοση της αναγνώρισης δράσης. \par

Ορισμένες άλλες μέθοδοι περιλάμβαναν ένα δίκτυο \en RNN \gr  ή  \en LSTM  \gr για την ταξινόμηση κάνουν οι \en\cite{DBLP:journals/corr/DonahueHGRVSD14}\gr, οι \en \cite{DBLP:journals/corr/NgHVVMT15} \gr
και οι \en \cite{DBLP:journals/corr/MaCKA17}\gr. Οι  \en \cite{DBLP:journals/corr/DonahueHGRVSD14}\gr αντιμετωπίζουν τηv  πρόκλησης των μεταβλητών μήκη των ακολουθιών εισόδου και εξόδου,
εκμεταλλευόμενοι τα \en convolutional layers \gr  και μεγάλου εύρους χρονικές αναδρομές (\en recursions\gr). Προτείνουν ένα \en Long-term Recurrent
Convolutional Network (LRCN)\gr, το οποίο είναι ικανό να αντιμετωπίσει  τις εργασίες αναγνώρισης, λεζάντας εικόνας και περιγραφής βίντεο.
Για να ταξινομήσουν μια δεδομένη ακολουθία καρέ, το \en LRCN \gr λαμβάνει αρχικά ως είσοδο ένα καρέ, και πιο συγκεκριμένα τα κανάλια \en RGB \gr και την οπτική ροή του, και προβλέπει μια ετικέτα.
Μετά από αυτό, εξάγει την  κλάση  του βίντεο μέσω του  μέσου όρου των πιθανοτήτων των ετικετών, επιλέγοντας την πιο πιθανή κλάση.
Οι  \en\cite{DBLP:journals/corr/NgHVVMT15}  \gr πρώτα διερευνούν  διάφορες προσεγγίσεις για χρονική ομαδοποίησης (\en temporal pooling\gr) των χαρακτηριστικών.
Αυτές οι τεχνικές περιλαμβάνουν τον χειρισμό καρέ βίντεο ξεχωριστά από 2 αρχιτεκτονικές \en  CNN\gr: είτε απ' το \en AlexNet \gr είτε απ' το \en GoogleNet\gr, και αποτελούνται από 
πρώιμη σύντηξη, καθυστερημένη σύντηξη και  ενός συνδυασμού αυτών. Επιπλέον, προτείνουν ένα \en RNN \gr προκειμένου να εξετάσουν τα βίντεο κλιπ
ως ακολουθίες ενεργοποιήσεων CNN. Το προτεινόμενο \en LSTM \gr λαμβάνει ως είσοδο την έξοδο του τελικού \en CNN layer \gr για κάθε συνεχόμενο καρέ και μετά από 5 \en LSTM layers \gr και χρησιμοποιώντας
έναν \en softmax \gr ταξινομητή, προτείνει μία ετικέτα. Για την ταξινόμηση του βίντεο,  επιστρέφουν μια ετικέτα μετά το τελευταίο βήμα, εφαρμόζουν \en max-pooling \gr στις προβλέψεις στην διάσταση
του χρόνου, αθροίζουν τις προβλέψεις στην διάσταση του χρόνου και επιστρέφουν το μέγιστο ή έναν γραμμικό συνδυασμό με βάρη των προβλέψεων υπό έναν παράγοντα \tl{g}, τα αθροίζουν και επιστρέφουν το μέγιστο.
Έδειξαν ότι όλες οι προσεγγίσεις είναι 1\% διαφορετικές με προκατάληψη για τη χρήση των προβλέψεων με βάρη 
για την υποστήριξη της ιδέας ότι το \en LSTM \gr γίνεται προοδευτικά πιο ενημερωμένο. Τελευταίοι 
αλλά όχι λιγότερο σημαντικοι, οι \en \cite{DBLP:journals/corr/MaCKA17} \gr χρησιμοποίησαν ένα \en two-stream ConvNet \gr για  εξαγωγή χαρακτηριστικών και είτε ένα \en
LSTM \gr ή  \en convolutional \gr πάνω από το χρονικώς κατασκευασμένους πίνακες χαρακτηριστικών για τη σύντηξη χωρικών και χρονικών πληροφοριών. Χρησιμοποιούν ένα \en ResNet-101 \gr
για την εξόρυξη χαρτών ενεργοποίησης  τόσο για χωρικές όσο και για χρονικές ροές. Χωρίζουν το 
βίντεο σε διάφορα τμήματα, όπως έκαναν οι \en \cite{DBLP:journals/corr/WangXW0LTG16}\gr, και χρησιμοποίησαν ένα επίπεδο  \en temporal pooling \gr για την εξαγωγή διακεκριμένων χαρακτηριστικών.
Αφού λάβουν αυτά τα χαρακτηριστικά, το \en LSTM  \gr
εξάγει ενσωματωμένες δυνατότητες από όλα τα τμήματα. \par


Επιπλέον, οι \en \cite{Tran2014LearningSF} \gr  διευρένησαν τα \en  3D Convolutional \gr δίκτυα (\en \cite{pmid:22392705}\gr)
και εισήγαγαν το \en  C3D \gr δίκτυο που έχει \en  3D convolutional layers \gr  με πυρήνες $3 \times  3 \times 3$.
Αυτό το δίκτυο είναι σε θέση να μοντελοποιήσει την  εμφάνιση και την κίνηση ταυτόχρονα χρησιμοποιώντας τρισδιάσττες συνελίξεις και μπορεί να χρησιμοποιηθεί ως
εξαγωγέας χαρακτηριστικών. Συνδυάζοντας την αρχιτεκτονική δύο ροών και τις τρισδιάσττες συνελίξεις οι \cite{DBLP:journals/corr/CarreiraZ17} πρότειναν το  δίκτυο \en I3D\gr.
Πάνω σ' αυτό, oι δημιουργοί τονίζουν τα πλεονεκτήματα της μεταφοράς μάθησης για την εργασία της αναγνώρισης επαναλαμβάνοντας τα δυσδιάστατα προ-εκπαιδευμένα βάρη στην 3η διάσταση.
Οι \en \cite{DBLP:journals/corr/abs-1708-07632} \gr πρότειναν ένα δίκτυο \en 3D ResNet \gr για την αναγνώριση δράσης με βάση τα \en Residual \gr δίκτυα (\en ResNet)(\cite{DBLP:journals/corr/HeZRS15}\gr)
και διερευνούν την απόδοση  των δικτύων \en ResNet \gr με  \en 3D Convolutional \gr πυρήνες. Από την άλλη, οι \en \cite{DBLP:journals/corr/abs-1711-08200} \gr 
βάσισαν  την προσέγγισή τους στα \en DenseNets (\cite{DBLP:journals/corr/HuangLW16a}) \gr και επέκτειναν την αρχιτεκτονική του \en DenseNet \gr χρησιμοποιώντας τρισδιάστατα φίλτρα
και \en pooling \gr πυρήνες αντί για δισδιάστατους, ονομάζοντας αυτή την προσέγγιση ως \en DenseNet3D\gr. Επιπλέον, εισάγουν το \en Layer \gr χρονικής μετάβασης (TTL),
το οποίο συνενώνει χρονικά χάρτες χαρακτηριστικών που εξάγονται  σε διαφορετικά  χρονικά βάθη και αντικαθιστά το επίπεδο μετάβασης του \en  DenseNet\gr.
Παράλληλα, οι \en \cite{DBLP:DibaFSKAYG18} \gr εισήγαγαν  ένα νέο χρονικό \en layer \gr το οποίο μοντελοποιεί  μεταβλητούς χρονικούς  πυρήνες συνέλιξης. Τελευταίοι αλλά εξίσου σημαντικοί, 
oi \en \cite{DBLP:journals/corr/abs-1711-11248}\gr  πειραματίστηκαν  με διάφορες υπόλοιπες αρχιτεκτονικές \en Residual \gr δικτύου χρησιμοποιώντας
συνδυασμούς \en 2D  \gr και \gr 3D convolutional Layer\gr. Σκοπός τους είναι να δείξουν ότι η \en 2D \gr χωρική συνέλιξη ακολουθούμενη από \en 1D \gr χρονική  συνέλιξη επιτυγχάνει \en state-of-the-art \gr
αποτελέσματα, ονομάζοντας αυτού του τύπου το \en layer \gr ως \en R(2 + 1)D\gr. Πρόσφατα οι \en \cite{Guo_2018_ECCV} \gr  πρότειναν ένα \en framework \gr  που μπορεί να μάθει να
 αναγνωρίζει μια προηγουμένως αθέατη \en 3D \gr κλάση δράσης  με λίγα μόνο παραδείγματα εκμεταλλευόμενο την εγγενή δομή των \en 3D \gr δεδομένων μέσω μιας γραφικής αναπαράστασης.
Ακόμα πιο λεπτομερή  παρουσίαση των τεχνικών αναγνώρισης δράσης που χρησιμοποιήθηκαν μέχρι το 2018 πραγματοποιήθηκε από τους \en\cite{DBLP:journals/corr/abs-1806-11230}\gr.

\subsection{Εντοπισμός Δραστηριότητας}

Όπως προαναφέρθηκε, ο εντοπισμός δράσης μπορεί να θεωρηθεί ως προέκταση του
προβλήματος εντοπισμού αντικειμένων. Αντί να εξάγουμε δισδιάστατα πλαίσια οριοθέτησης  σε μία μόνο
εικόνας, ο στόχος των συστημάτων εντοπισμού δράσης είναι να εξάγουν \en action tubes, \gr τα οποία 
είναι ακολουθίες πλαισίων οριοθέτησης που περιέχουν μια ενέργεια που εκτελέστηκε. Έτσι, υπάρχουν
διάφορες προσεγγίσεις, συμπεριλαμβανομένου συνήθως ενός δικτύου ανιχνευτή αντικειμένων  και ενός ταξινομητή. \par

Οι πρώτες προσεγγίσεις ανίχνευσης αντικειμένων περιλάμβαναν την επέκταση ενός αλγορίθμου  πρότασης αντικειμένων
σε 3 διαστάσεις. Οι \en\cite{6619185} \gr επέκτείναν τα παραμορφώσιμα (\en deformable ) \gr μοντέλα (\en\cite{5255236}\gr) με το να αντιμετωπίζουν τις δράσεις  ως
xωροχρονικά μοτίβα και  δημιούργησαν ένα παραμορφώσιμου μοντέλο για κάθε δράση. Oi \en \cite{6909495} \gr εισήγαγαν την έννοια των \en tubelets\gr, γνωστά και ως ακολουθίες
πλαισίων οριοθέτησης και βάσισαν τη μέθοδό τους σε επιλεκτικό αλγόριθμο αναζήτησης (\en\cite{Uijlings13}\gr), επεκτείνοντας τα \en superpixels \gr σε \en supervoxels \gr
για την παραγωγή χωροχρονικών σχημάτων. Απ' την άλλη, οι \en \cite{Oneata} \gr επέκτειναν  μια τυχαιοποιημένη διαδικασία συγχώνευσης \en superpixels \gr που χρησιμοποιούταν για
που χρησιμοποιούνταν  για προτάσεις  αντικειμένων, όπως παρουσιάστηκαν απ' τους \en \cite{Manen:2013:POP:2586117.2587333}\gr.
Οι \en\cite{7298735} \gr  πρώτα προτείνουν πλαίσια οριοθέτησης για κάθε καρέ με χρήση ενός ανιχνευτή ανθρώπου  και κίνησης, ενώ, στη συνέχεια, με τη επιλογή των καλύτερων σε σκορ κουτιών,
πρότειναν έναν άπληστο συνδετικό αλγόριθμο με τη διατύπωση την εργασίας σύνδεσης ως  πρόβλημα  μέγιστης κάλυψης. Οι \en \cite{BMVC2015_177} \gr παράγουν  χωροχρονικές προτάσεις κατευθείαν
από τις πυκνές τροχειές, οι οποίες επίσης χρησιμοποιήθηκαν για ταξινόμηση.
Οι \en\cite{7410734} \gr δημιουργούν ένα γράφημα χωροχρονικής τροχιάς και επιλέγουν προτάσεις δράσεων που βασίζονται μόνο στην εσκεμμένη κίνηση που εξάγεται από το γράφημα.
Οι \en\cite{7410732} \gr διαχωρίζουν  τα τμήματα βίντεο σε \en supervoxels \gr και  χρησιμοποιούν το περιεχόμενο τους ως χωρική σχέση μεταξύ των \en supervoxels \gr σε σχέση με την δράση του προσκηνίου.
Δημιουργούν ένα γράφημα για κάθε βίντεο, όπου τα supervoxels σχηματίζουν τους κόμβους και οι  κατευθυνμένες άκρες απεικονίζουν  τις χωρικές σχέσεις μεταξύ τους. Κατά τη διάρκεια των δοκιμών,
κάνουν μια βόλτα στο περιβάλλον, όπου κάθε βήμα καθοδηγείται από τις σχέσεις περιβάλλοντος κατά τη διάρκεια της εκπαίδευσης, με αποτέλεσμα μια κατανομή  πιθανότητας μιας δράσης για όλα τα supervoxels.
Οι \en\cite{DBLP:journals/corr/MettesGS16} \gr αντί για τοποθέτηση  πλαισίων σε όλα τα καρέ των βίντεο, σχολίασαν σημεία σε ένα αραιό  υποσύνολο καρέ του  βίντεο και χρησιμοποίησαν
προτάσεις που λαμβάνονται μέσω ενός μέτρου επικάλυψης μεταξύ των προτάσεων δράσης και των σημείων.
Οι \en\cite{DBLP:journals/corr/BehlSSSCT17} \gr ασχολούνται με  την ανίχνευση και τον εντοπισμό  ενεργειών σε πραγματικό χρόνο μέσω της λήψης προτάσεων δράσης ανά καρέ και την πρόταση ενός αλγορίθμου
σύνδεσης που είναι σε θέση να κατασκευάσει και να ενημερώνει τα \en action tubes \gr ανά καρέ. Πιο πρόσφατα, οι  \en\cite{8237344} \gr προσπάθησαν  να ασχοληθούν  με το πρόβλημα της ανίχνευσης και
τον εντοπισμό δράσης χωρίς επίβλεψη. Η προσέγγισή τους περιελάμβανε αρχικά την εξόρυξη  κατακερματισμένων \en supervoxel \gr
και στη συνέχεια την ανάθεση ένα βάρους σε κάθε \en supervoxel\gr. Με την εξαγωγή \en supervoxels\gr, δημιουργούν ένα γράφημα και
στη συνέχεια χρησιμοποιούν μια διακριτική \en clustering \gr προσέγγιση εκπαιδεύεται ένας ταξινομητής.

Η εισαγωγή του \en R-CNN (\cite{DBLP:journals/corr/GirshickDDM13}) \gr κατάφερε σημαντικές βελτιώσεις
στην  απόδοση των δικτύων εντοπισμού αντικειμένων. Αυτή η αρχιτεκτονική,
πρώτον, προτείνει περιοχές στην εικόνα που είναι πιθανό να περιέχουν κάποιο αντικείμενο και
στη συνέχεια, τα ταξινομεί χρησιμοποιώντας ένα \en SVM\gr. Εμπνευσμένοι από αυτή την αρχιτεκτονική,
οι \en \cite{DBLP:journals/corr/GkioxariM14} \gr σχεδιάσαν ένα δίκτυο \en RCNN 2-stream \gr για να προτείνει 
προτάσεις δράσεων  για κάθε καρέ, ένα \en stream \gr για το επίπεδο καρέ και ένα για την 
οπτική ροή. Στη συνέχεια, τα συνδέουν χρησιμοποιώντας τον αλγόριθμο σύνδεσης \en Viterbi\gr.
Οι \en \cite{DBLP:journals/corr/WeinzaepfelHS15} \gr επεκτείνουν αυτή την προσέγγιση, εκτελώντας
προτάσεις στο επίπεδο καρέ  και χρησιμοποιώντας ένα tracker για τη σύνδεση των προτάσεων αυτών
χαρακτηριστικά της χωρικής και οπτικής ροής. Επίσης, η μέθοδός τους εκτελεί χρονικό εντοπισμό μέσω της χρήσης
ενός συρόμενου παράθυρου πάνω από τα εντοπισμένα \en tubes\gr.   \par

Η εισαγωγή του \en Faster RCNN (\cite{Ren:2015:FRT:2969239.2969250}) \gr συνήσφερε  πολύ
τη βελτίωση της απόδοσης των δικτύων εντοπισμού δράσης. Οι \en\cite{peng:hal-01349107} \gr και \en\cite{DBLP:journals/corr/SahaSSTC16} \gr 
χρησιμοποιούν το \en Faster R-CNN \gr αντί για το \en  RCNN \gr  για προτάσεις σε επίπεδο καρέ, χρησιμοποιώντας το \en RPN \gr για εικόνες \en RGB \gr και οπτικής ροής.
Αφού λάβουν χωρικές προτάσεις και προτάσεις κίνησης, οι \en \cite{peng:hal-01349107} \gr τις συγχωνεύουν και από κάθε προτεινόμενη \en ROI\gr, παράγουν 4 \en ROIs \gr για να επικεντρωθούν
σε συγκεκριμένο μέρος του σώματος του δρώντα. Μετά από αυτό, συνδέουν την πρόταση χρησιμοποιώντας τον αλγόριθμο \en Viterbi \gr για κάθε κλάση και εκτελούν χρονικό εντοπισμό
χρησιμοποιώντας ένα συρόμενο παράθυρο, με πολλαπλές χρονικές κλίμακες και διασκελισμό κάνοντας χρήση  μιας μεθόδου  μέγιστης υποσυστοιχίας.
Απ' την άλλη, οι \en \cite{DBLP:journals/corr/SahaSSTC16} \gr εκτελούν, επίσης, ταξινόμηση σε επίπεδο καρέ. Μετά απ' αυτό,  η μέθοδός τους εκτελεί σύντηξη με βάση έναν συνδυασμό
της εμφάνισης και της κίνησης με βάση τις προτάσεις και την βαθμολογία αλληλεπικάλυψης. Τέλος, η χρονική προσαρμογή λαμβάνει χώρα χρησιμοποιώντας δυναμικό προγραμματισμό.
Παράλληλα, οι \en \cite{DBLP:journals/corr/WeinzaepfelMS16} \gr χρησιμοποιούν το \en Faster RCNN \gr
για την εξαγωγή ανθρώπινων \en tubes \gr από βίντεο που εστιάζουν στο πρόβλημα του ασθενώς εποπτευόμενoυ εντοπισμού δράσης.
 Στη συνέχεια, χρησιμοποιώντας πυκνές τροχιές και μια \en multi-fold Multiple  Instance  Learning \gr προσέγγιση \en(\cite{7420739}\gr)  εκπαιδεύουν ένα
 ταξινομητή. Οι \en \cite{DBLP:journals/corr/MettesS17} \gr εισήγαγαν μια μέθοδο για \en zero-shot \gr Εντοπισμού δράσης. Η προσέγγισή τους περιλαμβάνει την βαθμολόγηση των προτεινόμενων
 \en action tubes \gr  σύμφωνα με τις αλληλεπιδράσεις μεταξύ των ατόμων που δρουν  και  αντικειμένων. Χρησιμοποίησαν το \en Faster-RCNN\gr,
στο πρώτο βήμα, για την ανίχνευση τόσο των ανθρώπων που δρουν όσο και των αντικειμένων και μετά, χρησιμοποιώντας χωρικές σχέσεις μεταξύ τους,  συνδέουν τα προτεινόμενα πλαίσια στον άξονα του χρόνου
βασιζόμενοι στην \en zero-shot \gr πιθανότητα της παρουσίας των ατόμων, συναφών αντικειμένων γύρω απ' αυτούς και τις αναμενόμενες χωρικές σχέσεις μεταξύ αντικειμένων και ανθρώπων που δρουν. Επιπλέον
οι \en\cite{DBLP:journals/corr/HeIDM17} \gr πρότειναν  το \en Tube Proposal Network (TPN) \gr για τη δημιουργία ανεξαρτήτου κλάσης προτάσεις \en tubelet\gr, οι οποίες χρησιμοποιούν το \en Faster R-CNN \gr
για να λάβουν δισδιάστατες προτάσεις περιοχών και έναν αλγόριθμο σύνδεση για τη σύνδεση των \en tubelets  \gr με τις  προτάσεις των περιοχών. Πιο πρόσφατα, oi \en \cite{DBLP:journals/corr/abs-1807-10066}
\gr πρότειναν μια μέθοδο για εντοπισμό δράσεων στο σύνολο δεδομένων \en AVA  (\cite{DBLP:journals/corr/GuSVPRTLRSSM17}\gr) συνδυάζοντας τις αρχιτεκτονικές
των \en I3D (\cite{DBLP:journals/corr/CarreiraZ17}\gr) και \en Faster RCNN\gr. Χρησιμοποιούν μπλοκ του \en I3D \gr για την λήψη αναπαράστασης βίντεο και το  \en RPN \gr του \en Faster-RCNN \gr
για να προτείνει προτάσεις «ανθρώπου» για το κεντρικό πλαίσιο. \par

Παράλληλα μ' αυτά, οι \en \cite{singh2016online} \gr και \en\cite{kalogeiton17iccv:hal-01519812} \gr  σχεδίασαν τα δίκτυα τους
με βάση το \en Single Shot Multibox Detector \cite{DBLP:journals/corr/LiuAESR15})\gr. Οι \en\cite{singh2016online} \gr δημιούργησαν
ένα  χωροχρονικό δίκτυο  πραγματικoύ χρόνου. Για να λειτουγεί το δίκτυο τους σε πραγματικού χρόνου, \en  \cite{singh2016online} \gr πρότειναν έναν νέο και αποδοτικό
αλγόριθμο με την προσθήκη πλαισίων σε \en tubes \gr σε κάθε καρέ, εάν επικαλύπτονται περισσότερο από ένα κατώφλιοή, ή  εναλλακτικά, τερματίζουν  το \en action tube \gr εάν για \tl{k} καρέ
δεν προσθέθηκε κανένα πλαίσιο. Οι \en\cite{kalogeiton17iccv:hal-01519812} \gr  σχεδίασαν ένα δίκτυο δύο ροών, το οποίο κάλεσαν
\en ACT-detector \gr, και εισήγαγαν τα κυβικά (\en cuboids) anchors\gr. Για K καρέ, και για τα δύο δίκτυα,
οι \en \cite{kalogeiton17iccv:hal-01519812} \gr εξάγουν  χωρικά χαρακτηριστικά σε επίπεδο καρέ, στη συνέχεια, τα στοιβάζουν.
Τέλος, με τη χρήση των κυβικών \en anchors\gr, το δίκτυο  εξάγει \en tubelets\gr,
με τις αντίστοιχες βαθμολογίες κατάταξης και στόχους παλινδρόμησης. Για τη σύνδεση των τουμπέλετς, oi \en \cite{kalogeiton17iccv:hal-01519812} \gr  ακολουθούν 
τα ίδια βήματα με τους \en \cite{singh2016online}\gr. Για χρονική εντοπισμό, χρησιμοποιούν μιά προσέγγιση χρονικής εξομάλυνσης. \par

Πιο πρόσφατα, το δίκτυο \en YOLO (\cite{DBLP:journals/corr/RedmonDGF15}) \gr έγινε η έμπνευση
για  τους \en \cite{DBLP:journals/corr/abs-1903-00304} \gr και τους \en \cite{DBLP:journals/corr/abs-1802-08362}\gr. Στην προσέγγιση που προτάθηκε από τους
\en  \cite{DBLP:journals/corr/abs-1903-00304}\gr, οι έννοιες της
εξέλιξης και τού ποσοστoύ προόδου εισήχθησαν. Εκτός από την πρόταση πλαισίων οριοθέτησης σε επίπεδο καρέ, χρησιμοποιούν το \en YOLO \gr μαζί με έναν ταξινομητή \en RNN \gr για
να εξάγουν χρονικές πληροφορίες για τις προτάσεις. Με βάση αυτές τις πληροφορίες, δημιουργούν \en action tubes, \gr χωρίζοντας τα  σε κλάσεις. Ορισμένες άλλες προσεγγίσεις περιλαμβάνουν
εκτίμηση πόζας αυτή των \en \cite{DBLP:journals/corr/abs-1802-09232}\gr. Πρότειναν μια μέθοδο
υπολογισμού των δισδιάστατων και τρισδιάστατων πόζων και στη συνέχεια εκτέλεσαν ταξινόμηση δράσεων.
Χρησιμοποιούν το διαφορίσιμο \en Soft-argamax \gr για   την εκτίμηση των \tl{2D} και \tl{3D}
αρθρώσεων, επειδή η συναρτηση \en argmax \gr δεν είναι διαφορίσιμη. Στη συνέχεια, για T παρακείμενες
δημιουργούν μια απεικόνιση εικόνας με το χρόνο και τις $N_j$  αρθρώσεις  ως $x-y$ άξονες, έχοντας  2 κανάλια για την \tl{2D} πόζα  και 3  για την \tl{ 3D} πόζα.
Χρησιμοποιούν \en Convolutional Layers \gr για να παράγουν  χάρτες θερμότητες δράσης   και στη συνέχεια χρησιμοποιώντας \en  max plus min pooling \gr και την συνάρτηση \en softmax \gr
 εκτελούν ταξινόμηση δράσης.
 Οι \en \cite{DBLP:journals/corr/ZolfaghariOSB17} \gr πρότειναν μια αρχιτεκτονική τριών ροών που περιλαμβάνει \en 2D \gr πόζα, οπτική ροή και πληροφορίες \en  RGB\gr. Αυτά τα \en streams \gr ενώνονται
 μέσω του μοντέλου της αλυσίδας Μάρκοφ. Επιπλέον, οι \en \cite{8237881} \gr πρότειναν μια αρχιτεκτονική με τη χρήση ενός χρονικού \en convolutional \gr δικτύου παλινδρόμησης,
 για να πιάνουν την  μακροπρόθεσμη εξάρτηση και πληροφίες  μεταξύ γειτονικών καρέ και ένα χωρικό δίκτυο παλινδρόμησης, για  προτάσεις ανά καρέ. Χρησιμοποιούν μεθόδους παρακολούθησης
και δυναμικoύ προγραμματισμού για τη δημιουργία προτάσεων δράσης. \par

Τα περισσότερα από τα προαναφερθέντα δίκτυα χρησιμοποιούν ανά καρέ χωρικές προτάσεις και εξάγουν 
τις χρονικές τους πληροφορίες υπολογίζοντας την οπτική ροή. Από την άλλη
oi \en \cite{DBLP:journals/corr/SahaSC17} \gr σχεδίασαν μια αρχιτεκτονική η οποία περιλαμβάνει προτάσεις σε επίπεδο τμήματος βίντεο, το οποίο σημαίνει  περισσότερα από ένα
καρέ ταυτόχρονα. Οι \en \cite{DBLP:journals/corr/SahaSC17} \gr  πρότειναν μια \en 3D-RPN \gr αρχιτεκτονική 
που είναι σε θέση να δημιουργήσει και να ταξινομήσει τρισδιάσττες προτάσεις   αποτελoύμενες από 2 συνεχόμενα  καρέ. Επίσης, πρότειναν έναν αλγόριθμο σύνδεσης,
τροποποιώντας αυτόν που πρότειναν oi \en \cite{DBLP:journals/corr/SahaSSTC16}\gr. Πάνω σ' αυτό, οι \en \cite{DBLP:journals/corr/HouCS17} \gr  σχεδιάσαν μια αρχιτεκτονική για τη δημιουργία
προτάσεων δράσης για περισσότερα από 2 καρέ, καλώντας το μοντέλο τους \en Tube CNN (T-CNN)\gr. Στην προσέγγισή τους, το επίπεδο του τμήματος βίντεο σημαίνει
ότι ολόκληρο το βίντεο χωρίζεται κλιπ βίντεο ίδιου αριθμού καρέ  και με τη χρήση
του \en C3D \gr για την εξόρυξη χαρακτηριστικών, επιστρέφουν  χωροχρονικές προτάσεις. Μετά την λήψη των 
προτάσων, oi \cite{DBLP:journals/corr/HouCS17} συνδέουν τις \en tube \gr προτάσεις τους  με έναν αλγόριθμο στηριζόμενος
στην πιθανότητα ύπαρξης δράσης  και την επικάλυψη μεταξύ των \en tubes\gr. Τέλος, η λειτουργία ταξινόμησης λαμβάνει χώρα για τα συνδεδεμένα \en action tubes\gr.


\printbibliography

\end{document}