\documentclass{report}

\usepackage{subcaption} % package for subfigures
\usepackage{hyperref}  % package for linking figures etc
\usepackage{enumitem}  % package for description with bullets
\usepackage{graphicx}  % package for importing images
\usepackage{mathtools} % package for math equation
\usepackage{mathrsfs}  % package for math font
\usepackage{indentfirst} % package for getting ident after section or paragraph
\usepackage[export]{adjustbox}
% \usepackage{amsmath}

\setlength{\parindent}{2em} % how much indent to use when we start a paragraph

\graphicspath{ {./theory/figures/} }       % path for images

\begin{document}

\chapter{Connecting Tubes}
As mentioned before, TPN gets as input a sequence of 16 frames and proposes TOIs. However, most actions in videos lasts more that 16 frames.
This means that, in overlaping video clips, there will be consequentive TOIs that represent the entire action. So, it is essential to create
an algorithm for finding and connecting these TOIs. Our algorithm is inspired by \cite{}, and uses a score in order to decide if a sequence of
TOIs is possible to contain an action. This score is a combination of 2 metrics:
\begin{description}
\item[ Actioness,  ] which is the TOI's possibility to contain an action. This score is produced by TPN's scoring layers.
\item [ TOIs' overlapping, ] which is the IoU of the last frames of the first TOI and the first frames of the second TOI.
\end{description}

The above scoring policy can be described by the following formula:
\[ S = \frac{1}{m} \sum_ {i=1}^{m} Actioness_i + \frac{1}{m-1} \sum_{j=1}^{m-1} Overlap_{j,j+1} \]


\end{document}
