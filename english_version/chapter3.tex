\documentclass{report}

\usepackage{hyperref}  % package for linking figures etc
\usepackage{enumitem}  % package for description with bullets
\usepackage{graphicx}  % package for importing images
\usepackage{mathtools} % package for math equation
\usepackage{mathrsfs}  % package for math font
\usepackage{indentfirst} % package for getting ident after section or paragraph
% \usepackage{amsmath}
\usepackage[
    backend=bibtex,
    citestyle=authoryear,
    % citestyle=authoryear-comp,
    % citestyle=authoryear-ibid,
    bibstyle=numeric,
    sorting=ynt,
    % style=numeric,
    % style=alphabetic ,
  ]{biblatex}
 
 \addbibresource{References}
\setlength{\parindent}{2em} % how much indent to use when we start a paragraph

\graphicspath{ {./theory/figures/} }       % path for images

\begin{document}

\chapter{Related work}


% The combination of CNN and Recurrent Neural Network (RNN) is a method widely been studied and achieved excellent results.
% The recently proposed methods can be seperated into 2 subcategories: (1) use a CNN detector in order to localize directly
% the action instance for each single frame and (2) modify the detector in order to get as input volute multiple frames and
% use 3D convolution.%  Both approaches produce the same type of output : i) an actioness score of a bounding box or a proposed
% tube ii) coordinate offsets for bounding box refinement and iii) clasification probabilities for all action classes.

% The task of action classification in approaches used features based on shape or motion sucha as HOG \cite{dalal2005histogramcvpr},
% SIFT \cite{Lowe2004}, MBH \cite{dalal2006human} and train a classifier as SVM.

% \paragraph{Frame based methods}
\section{Action Recognition}
First approaches for action classification consisted of two steps a) compute complex handcrafted features from raw video frames
such as SIFT, HOG, ORB features and b) train a classifier based on those features. These approaches made the choise of
features a signifact factor for network's performance. That's because different action classes may appear dramatically
different in terms of their appearance and motion patterns. Another problem was that most of those approaches make
assumptions about the circumstances under which the video was taken due to problems such as cluttered
background, camera viewpoint variations etc. A review of the techniques, used until 2011, is presented in \cite{Aggarwal:2011:HAA:1922649.1922653}. \par

Recent results in deep architectures and especially in image classification motivated researchers to train CNN networks for
the task of action recognition. The first significant attempt was made by \cite{6909619}. They design their architecture based on the best-scoring CNN
in the ImageNet competition. they explore several methods for fusion of spatio-temporal features using 2D operations mostly and 3D convolution only in slow fusion.
\cite{simonyan2014two}  used a 2 CNNs, one for spatial information and one for optical flow and combined them using late fusion.
They show that extacting spatial context from videos and motion context from optical flow can improve significantly action recognition accuracy.
\cite{DBLP:journals/corr/FeichtenhoferPZ16} extend this approach by using early fusion at the end of convolutional layers,  instead of late fusion which
takes places at the last layer of the network. On top that, they used a second network for temporal context which they fuse with the other network using late
fusion. Futhermore, \cite{DBLP:journals/corr/WangXW0LTG16} based their method on \cite{simonyan2014two}, too. They deal with the problem of capturing long-range
temporal context and training their network given limited training samples. Their approach, which they named Temporal Segment Network (TSN), seperates the input
video in K segments and a short snippet from each segment is chosen for analysis. Then they fuse  the extracted spatio-temporal context, making, eventually, their
prediction. \par
Some other methods included a RNN or LSTM network for classification like \cite{DBLP:journals/corr/DonahueHGRVSD14}, \cite{DBLP:journals/corr/NgHVVMT15} and
\cite{DBLP:journals/corr/MaCKA17}. \textbf{Pending... description} \par
% A comparison between RNN networks and dilated convolutions is presented in \cite{DBLP:journals/corr/abs-1803-01271}, even though they don't refer at.\textbf{Pending... description}
Additionally, \cite{Tran2014LearningSF} explored 3D Convolutional Networks (\cite{pmid:22392705}) and introduced C3D network which  has
3D convolutional layers with kernels $ 3 \times 3 \times 3$.
This network is able to  model appearence and motion context simultaneously using 3D convolutions and it can be used as a feature extractor, too.
Combining Two-stream architecture and 3D Convolutions, \cite{DBLP:journals/corr/CarreiraZ17} proposed
I3D network. On top of that, the authors emphasize in the advantages of transfer learning for the task of action recognition by repeating 2D pre-trained weights
in the 3rd dimension. \cite{DBLP:journals/corr/abs-1708-07632} proposed a 3D ResNet Network for action recognition based on Residual Networks (ResNet)
(\cite{DBLP:journals/corr/HeZRS15}) and explore the effectiveness of ResNet with 3D Convolutional kernels.
\cite{DBLP:journals/corr/abs-1711-08200} \textbf{Pending ...} \cite{DBLP:journals/corr/abs-1711-11248}
experiment with several residual network architectures using combinations of 2D and 3D convolutional layer. Their purpose is
to show that a 2D spatial convolution followed by a 1D temporal convolution achieves state of the art classification performance, naming
this type of convolution layer as R(2+1)D. A more detailed presentation for Action Recognition techniques used until 2018 is included in
\cite{DBLP:journals/corr/abs-1806-11230}.

\section{Action Localization}

As mentioned before, Action Localization can be seen as an extention of the object detection problem. Instead of outputing 2D bounding
boxes in a single image, the goal of action localization systems is to output action tubes which are sequences of bounding boxes that
contain an performed action. So, there are several approaches including an object-detector network for single frame
action proposal and a classifier. \par
The introduction of R-CNN (\cite{DBLP:journals/corr/GirshickDDM13}) achieve significant improvement
in the performance of Object Detection Networks. This architecture, firstly, proposes regions in the image which are likely to
contain an object and then it classifies them using a SVM classifier. Inspired by this architecture, \cite{DBLP:journals/corr/GkioxariM14}
design a 2-stream RCNN network in order to generate action proposals for each frame, one stream for frame level and one for optical flow.
Then they  connect them using the viterbi connection algorithm. \cite{DBLP:journals/corr/WeinzaepfelHS15} extend this approach, by performing
frame-level proposals and using a tracker for connecting those proposals using both spatial and optical flow features. Also their method performs
temporal localization using a sliding window over the tracked tubes. \par
\cite{peng:hal-01349107} and \cite{DBLP:journals/corr/SahaSSTC16} use Faster R-CNN (\cite{Ren:2015:FRT:2969239.2969250}) instead of RCNN
for frame-level proposals, using RPN for both RGB and optical flow images.
After getting spatial and motion proposals,\cite{peng:hal-01349107} fuse them exploring and from each proposed ROI, generate 4 ROIs in order to focus in specific
body parts of the actor. After that, they connect the proposal using Viterbi algorithm for each class and perform temporal localization by using a sliding window, with multiple
temporal scales and stride using a maximum subarray method. From the other hand, \cite{DBLP:journals/corr/SahaSSTC16} perform, too, frame-level classification. After that,
their method performs fusion based on a combination between the actioness scores of the appearence and motion based proposals and their overlap score. Finally, temporal localization
takes place using dynamic programming. \par
On top of that, \cite{singh2016online} and \cite{kalogeiton17iccv:hal-01519812} design their networks based on the Single Shot Multibox Detector \cite{DBLP:journals/corr/LiuAESR15}).
\cite{singh2016online} created an online real-time spatio-temporal network. In order their network to execute real-time,  \cite{singh2016online} propose a novel and efficient algorithm
by adding boxes in tubes in every frame if they overlap more than a threshold, or alternatively, terminate the action tube if for k-frames no box was added.  \cite{kalogeiton17iccv:hal-01519812}
designed a two-stream network, which they called ACT-detector, and introduced anchor cuboids. For K frames, for both networks, \cite{kalogeiton17iccv:hal-01519812} extract spatial
features in frame-level, then they stack these features. Finally, using cuboid anchors, the network extracts tubelets, that is a sequence of boxes, with their corresponding classification
scores and regression targets. For linking the tubelets, \cite{kalogeiton17iccv:hal-01519812} follow about the same steps as \cite{singh2016online} did. For temporal localization, they use
a temporal smoothing approach. \par

Most recently, YOLO Network (\cite{DBLP:journals/corr/RedmonDGF15}) became the inspiration for \cite{DBLP:journals/corr/abs-1903-00304} and
\cite{DBLP:journals/corr/abs-1802-08362}. In \cite{DBLP:journals/corr/abs-1903-00304}, concepts of progression and progress
rate were introduced. Except from proposing bounding boxes in frame level, they use YOLO together with a RNN classifier for extracting temporal information for the proposals.
Based on this information, they create action tubes, seperated into classes. Some other approaches include pose estimation like \cite{DBLP:journals/corr/abs-1802-09232}, \cite{} and
\cite{}. In \cite{DBLP:journals/corr/abs-1802-09232} uses \textbf{pending description...}. \par
Most of aforementioned networks use per-frame spatial proposals and extract their temporal infomation by calculating optical flow. On the other hand, \cite{DBLP:journals/corr/HouCS17} design
an architecture which icludes proposal in video segment level, which they called Tube CNN (T-CNN). Video segment level means that the whole video is seperated into equal length video clips, and
using a C3D for extracting features, it returns spatio-temporal proposals. After getting proposals, \cite{DBLP:journals/corr/HouCS17} link the tube proposals by an algorithm based on tubes'
actioness score and overlap. Finally, classification operation is performed for the linked video proposals.
% Edw na diavasw ...\cite{Li_2018_ECCV}

\section{Our implementation}
We propose a network similar to \cite{DBLP:journals/corr/HouCS17}. Our architecture is consisted by the following basic elements:
\begin{itemize}
\item One 3D Convolutional Network, which is used for feature extraction. In our implementaion we use a 3D Resnet network which is taken from
  \cite{hara3dcnns} and it is based on ResNet CNNs for Image Classification \cite{DBLP:journals/corr/HeZRS15}.
\item Tube Proposal Network for proposing action tubes (based on the idea presented in \cite{DBLP:journals/corr/HouCS17}).
\item A classifier for classifying video tubes.
\end{itemize}

\textbf{Pending ... more commentary and a figure}

\printbibliography

\end{document}