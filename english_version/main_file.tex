\documentclass[diploma]{softlab-thesis}


%%%
%%%  The document
%%%

\usepackage{hyperref}  % package for linking figures etc
\usepackage{enumitem}  % package for description with bullets
\usepackage{graphicx}  % package for importing images
\usepackage{mathtools} % package for math equation
\usepackage{mathrsfs}  % package for math font
\usepackage{indentfirst} % package for getting ident after section or paragraph
\usepackage{subcaption} % package for subfigures
\usepackage[export]{adjustbox}
\usepackage{longtable} % package for multi pages tables
\usepackage{multirow}  % package for tables, multirow
\usepackage{amssymb}

% \usepackage{amsmath}
\usepackage[
    backend=bibtex,
    citestyle=authoryear,
    % citestyle=authoryear-comp,
    % citestyle=authoryear-ibid,
    bibstyle=numeric,
    sorting=ynt,
    % style=numeric,
    % style=alphabetic ,
  ]{biblatex}
 \addbibresource{References}
\graphicspath{ {./theory/figures/} }       % path for images
\begin{document}

%%%  Title page

\frontmatter

\title{Action Localization and Recognition in Videos}
\author{Efstatios Galanakis}
\date{July 2019}
\datedefense{-1}{12}{2019}

\supervisor{}
\supervisorpos{}

\committeeone{}
\committeeonepos{}
\committeetwo{}
\committeetwopos{}
\committeethree{}
\committeethreepos{}

\TRnumber{CSD-SW-TR-42-14}  % number-year, ask nickie for the number
\department{}

\maketitle

% \begin{abstractgr}%
%   Σκοπός της παρούσας εργασίας είναι αφενός η σχεδίαση μίας απλής
%   γλώσσας υψηλού επιπέδου με υποστήριξη για προγραμματισμό με
%   αποδείξεις, αφετέρου η υλοποίηση ενός μεταγλωττιστή για τη γλώσσα
%   αυτή που θα παράγει κώδικα για μία γλώσσα ενδιάμεσου επιπέδου
%   κατάλληλη για δημιουργία πιστοποιημένων εκτελέσιμων.

%   Στη σημερινή εποχή, η ανάγκη για αξιόπιστο και πιστοποιημένα ασφαλή
%   κώδικα γίνεται διαρκώς ευρύτερα αντιληπτή. Τόσο κατά το παρελθόν όσο
%   και πρόσφατα έχουν γίνει γνωστά προβλήματα ασφάλειας και
%   συμβατότητας προγραμμάτων που είχαν ως αποτέλεσμα προβλήματα στην
%   λειτουργία μεγάλων συστημάτων και συνεπώς οικονομικές επιπτώσεις
%   στους οργανισμούς που τα χρησιμοποιούσαν. Τα προβλήματα αυτά
%   οφείλονται σε μεγάλο βαθμό στην έλλειψη δυνατότητας προδιαγραφής και
%   απόδειξης της ορθότητας των προγραμμάτων που χαρακτηρίζει τις
%   σύγχρονες γλώσσες προγραμματισμού. Για το σκοπό αυτό, έχουν προταθεί
%   συστήματα πιστοποιημένων εκτελέσιμων, στα οποία έχουμε τη δυνατότητα
%   να προδιαγράφουμε την ορθότητα των προγραμμάτων, και να παρέχουμε
%   μία τυπική απόδειξη αυτής, η οποία μπορεί να ελεγχθεί μηχανιστικά
%   πριν το χρόνο εκτέλεσης.

%   Τα συστήματα που έχουν προταθεί είναι ενδιάμεσου επιπέδου οπότε η
%   διαδικασία προγραμματισμού σε αυτά είναι ιδιαίτερα πολύπλοκη. Οι
%   γλώσσες υψηλού επιπέδου που συνοδεύουν αυτά τα συστήματα, ενώ είναι
%   ιδιαίτερα εκφραστικές, παραμένουν δύσκολες στον προγραμματισμό.  Μία
%   απλούστερη γλώσσα υψηλού επιπέδου, όπως αυτή που προτείνουμε σε αυτή
%   την εργασία, θα επέτρεπε ευρύτερη εξάπλωση του συγκεκριμένου
%   ιδιώματος προγραμματισμού.

%   Στη γλώσσα που προτείνουμε, ο προγραμματιστής προδιαγράφει τη μερική
%   ορθότητα του προγράμματος, δίνοντας προσυνθήκες και μετασυνθήκες για
%   τις παραμέτρους και τα αποτελέσματα των συναρτήσεων που ορίζει.
%   Επίσης δίνει ένα σύνολο θεωρημάτων βάσει του οποίου κατασκευάζονται
%   αποδείξεις της ορθής υλοποίησης και χρήσης των συναρτήσεων αυτών. Ως
%   μέρος της εργασίας, έχουμε υλοποιήσει σε γλώσσα OCaml ένα
%   μεταφραστή αυτής της γλώσσας στο σύστημα πιστοποιημένων
%   εκτελέσιμων NFLINT.

%   Επιτύχαμε να διατηρήσουμε τη γλώσσα κοντά στο ύφος των ευρέως
%   διαδεδομένων συναρτησιακών γλωσσών, καθώς και να διαχωρίσουμε τη
%   φάση προγραμματισμού από τη φάση απόδειξης της ορθότητας των
%   προγραμμάτων. Έτσι ένας μέσος προγραμματιστής μπορεί εύκολα να
%   προγραμματίζει στη γλώσσα που προτείνουμε με τον τρόπο που ήδη
%   γνωρίζει, και ένας γνώστης μαθηματικής λογικής να αποδεικνύει σε
%   επόμενη φάση την μερική ορθότητα των προγραμμάτων. Ως απόδειξη της
%   πρακτικότητας της προσέγγισης αυτής, παραθέτουμε ένα σύνολο
%   παραδειγμάτων στη γλώσσα με απόδειξη μερικής ορθότητας.
% \begin{keywordsgr}
% Γλώσσες προγραμματισμού, Προγραμματισμός με αποδείξεις, Ασφαλείς γλώσσες
% προγραμματισμού, Πιστοποιημένος κώδικας.
% \end{keywordsgr}
% \end{abstractgr}


%%%  Abstract, in English

\begin{abstracten}%

  The purpose of this diploma thesis is the design of a network for recognizing and localising human actions in videos.
  Our network aims to spatio-temporally localize a recognized action within a video
  producing a sequence of 2D boxes, one per frame, which includes the actor
  performing the recognized action.\par

  Detecting and Recognizing actions in videos is one of the biggest
  challenges in the field of Computer Vision. Most recent approaches
  includes an object detection network which proposes bounding boxes
  per frame, a linking method for creating candidate action tubes and
  a classifier for classifying these. On top of that, most of these
  approaches extract temporal information from a network which
  estimates optical flow in frame level. The introduction of 3D
  Convolutional Networks has helped us estimating spatio-temporal
  information from videos and simultaneously extract spatio-temporal
  features. Our approach tries to combine the benefits from using
  object detection networks and 3D Convolution.\par

  % \textbf{TODO change that}
  We design a network whose structure is based on standard action localization networks. Its first
  element is a 3D ResNet34 which is used for spatio-temporal feature extraction. Also,
  we design a network for proposing action tubes based on spatio-temporal features, called Tube Proposal Network.
  This network is an  expansion of Region Proposal Network and it gets as input the extracted features and
  outputs k-proposed action tubes.
  % introduced by Faster RCNN(\cite{Ren:2015:FRT:2969239.2969250})
  We explore 2 approaches for
  defining 3D anchors, which TPN uses. On top of that, we design a linking algorithm for
  connecting proposed action tubes. Finally, we explore several classification techniques
  including a SVM classifier and a MLP.

\begin{keywordsen}
Action Localization, Action Recognition, Action Tubes
\end{keywordsen}
\end{abstracten}


%%%  Acknowledgements

% \begin{acknowledgementsgr}
% \end{acknowledgementsgr}
%%%  Acknowledgements

\begin{acknowledgementsen}
Pending...
\end{acknowledgementsen}


%%%  Various tables


%%%  Various tables

\tableofcontents
\listoftables
\listoffigures


%%%  Main part of the book

\mainmatter

% \documentclass{report}

% \usepackage{subcaption} % package for subfigures
% \usepackage{hyperref}  % package for linking figures etc
% \usepackage{enumitem}  % package for description with bullets
% \usepackage{graphicx}  % package for importing images
% \usepackage{mathtools} % package for math equation
% \usepackage{mathrsfs}  % package for math font
% \usepackage{indentfirst} % package for getting ident after section or paragraph
% \usepackage[export]{adjustbox}
% % \usepackage{amsmath}

% \setlength{\parindent}{2em} % how much indent to use when we start a paragraph

% \graphicspath{ {./theory/figures/} }       % path for images

% \begin{document}

\chapter{Introduction}
Nowadays, the enormous increase of computing power help us deal with a lot of difficult situations appeared in our daily life.
A lot of areas of science have managed to tackle with problems, which were consided non trivial 20 years ago. One of
these area is Computer Vision and an important problem is human action recognition and localization.
\section{Problem statement}
The area of human action recognition and locatization has 2 main goals:
\begin{enumerate}
\item Automatically detect and classify any human activity, which appears in a video.
\item Automatically locate in the video, where the previous action is performed.
\end{enumerate}

\subsection{Human Action Recognition}
Considering human action recognition, a video may be consisted of only by 1 person doing something. However, this is a ideal
situation. In most cases, videos contain multiple people, who perform multiple actions or may not act at all in some segments.
So, our goal is not only to classify an action, but to dertemine the temporal boundaries of each action.
\subsection{Human Action Localization}
Alongside with Human Action Recognition, another problem is to present spatial boundaries of each action. Usually, this means
presenting a 2D bounding box for each video frame, which contains the actor. Of course, this bounding box moves alongside with
the actor.

\section{Applications}
The field of Human Action Recognition and Localization has a lot of applications which include 
 content based video analysis,automated video segmentation, security and surveillance systems,
human-computer interaction.

The huge availability of data (especially of videos) create the  necessity to find ways to take advantage of them.
About 2.5 billion images are uploaded at Facebook database every month, more than 34K hours of video in YouTube and
about 5K images every minute. On top of that, there are about 30 million surveillance cameras in US, which means
about 700K video hours per day. All those data need to be seperated in categories according to their content in
order to search them more easily. This process takes place by hand, by a user who attaches
keywords or tags to each video. However, most users avoid doing that, so many videos end up without any tagging information.
This situation creates the need to create algorithms for automated indexing based on the content of the video.

Another application is video summury. This area take place usually in movies or sports events. In movies,
video analysis algorithms can create a small video containing all the important moments of the movie. This
can be achieved by choosing video segments which an important action takes place such as killing the villain
of the movie. In sports events, video summury applications include creating highlight videos automatically, like
a video containing all achieved goals in football match.

On top of that, human action recognition can replace human operators in surveillance systems. Until now,
security systems include a system of multiple cameras handled by a human operator, who judges if a person
is acting normally or not. Automatic action classification systems can act like human, and immediately
judge if there is any human behavioral anomaly.

Last but not least, another field of application is related with human-computer interaction. Robotic applications
help elderly people deal with their daily needs. Also, gaming applications using Kinect create new kinds of
gaming experience without the need of a physical game controller.

\section{Challenges and Datasets}
There are various types of human activities. Depending on their complexity, weconceptually categorize human activities into four different
levels: gestures, actions, interactions, and group activities. Gestures are elementary movements of a person’s body part, and are the atomic
components describing the meaningful motion of a person. ``Stretching an arm'' and ``raising a leg'' are good examples of gestures.
Actions are single person activities that may be composed of multiple gestures organized temporally, such as ``walking'', ``waving'', and
``punching''. Interactions are human activities that involve two or more persons and/or objects. For example, ``two persons fighting'' is
an interaction between two humans and ``a person stealing a suitcase from another'' is a human-object interaction involving two humans and one
object. Finally, group activities are the activities performed by conceptual groups composed of multiple persons and/or objects. ``A group of persons marching'', ``a group having a meeting'', and ``two groups fighting'' are typical examples of them.
The wide variety of human activities and applications creates a lot of challenges which involve action recognition systems.
The most important challenges include large variations in appearence of the actors, camera view-point changes, occlusions,
non-rigid camera motions etc. On top of that, a big problem is that there are too many action classes which means
that manual collection of training sample is prohibitive. Also, some times, action vocabulary is not well defined.
As figure \ref{fig:open_example} shows, ``Open'' action can include a lot of kinds of actions, so we must carefully
choose which granularity of the action we will consider.

\begin{figure}[h]
  \centering
  \includegraphics[scale=0.3]{open_example}
  \caption{Examples of ``Open'' action}
  \label{fig:open_example}

\end{figure}

In order to deal with those challenges, several standard action datasets have been created in order to delevop
robust human action recognition systems and detection algorithms.
The first datasets included 1 actor performing using a static camera over homogeneous backgrounds.
Even though, those datasets helped us design the first action recognition algorithms, they were not able to deal with the above
challenges.
This lead us to design datasets containing more ambigious videos, such as Joint-annotated Human Motion Database(JHMDB) (\cite{Kuehne11})
and UCF-101 (\cite{soomro2012ucf101}). These datasets contain only human actions, the second category presented above.

\subsection{JHMDB Dataset}
The JHMDB dataset (\cite{Jhuang:ICCV:2013}) is a fully annotated dataset for human actions and human poses. It is consisted of 21 action categories and 928
clips extracted from Human Motion Database (HMDB51) \cite{Kuehne11}. This dataset contains trimmed videos with duration between
15 to 40 frames. Each clip is annotated for each frame using a 2D pose and contains only 1 action.
In order to train our model for action localization, we modify 2D poses into 2D boxes containing the whole pose in each frame.
There are available 3 different splits for training data, proposed by the authors. We chose the first split which contains 660
videos for training set and 268 for validation . 

\subsection{UCF-101 Dataset}
The UCF-101 dataset (\cite{soomro2012ucf101}) contains 13320 videos from 101 action categories.
From those, for 24 classes and 3194 video spatio-temporal annotations are included. This means that there is a 2D bounding box surrounding the actor for each frame in which an action is taking place.
We seperate them in 2284 videos for training set and 910 for validation test according to the
first proposed training split. For training data, there are videos up to 641 frames, while in validation data max number of frames is 900.
Each video, both training and validation, is untrimmed, including sometimes more than 1 actions taking place simultaneously.
We took annotations from
\cite{singh2016online} because the by the authors proposed annotations contain some mistakes.

\section{Motivation ans Contibutions}
The current achievements in Object Recognition Networks and in 3D Convolution Networks for Action Recognition have triggered us to try
to combine them in order to achieve state-of-the-art results for action localization. We introduce a new network structure inspired by
\cite{DBLP:journals/corr/HouCS17}, \cite{DBLP:journals/corr/abs-1712-09184},\cite{Ren:2015:FRT:2969239.2969250} and for implementation
by \cite{jjfaster2rcnn}.

Our contributions are the following:
\begin{enumerate}
\item We create a new framework for action localization extending the code taken from faster RCNN implementation. Based on the structure
  proposed by \cite{DBLP:journals/corr/HouCS17}, we modified it, using a 3D Resnet34 instead of C3D, which previous approach used.

\item Furthermore, we proposed our own TPN Network, a Network for proposing candidate action tubes give a small video segment.
  Following the approach \cite{DBLP:journals/corr/HouCS17} proposed, we firstly implement an architecture which uses
  cubboids as anchors, which then using a regressor it becomes a sequence of bounding boxes, likely to contain an action.
  We experiment with two candidate regressor's architecture and proposed and implement a 3D RoiAlign which uses trilinear
  interpolation for extracting each proposed action tube's activation maps. 
  Inspired by \cite{DBLP:journals/corr/abs-1712-09184}, we proposed and implement a TPN which uses predefined sequences of bounding
  boxes as 3D anchors. We proposed anchors that last equal with and less than video segment's duration in order our architecture to be able to
  perform temporal localization.  we explore two different regressors' architectures for better spatial precision using activation
  maps extracted from 2D RoiAlign, treating each frame seperatly.
\item Inspired by liniking algorithm proposed by \cite{DBLP:journals/corr/HouCS17}, we introduce our own linking algorithm, which
  uses a combination of actioness and overlap scores in order to decide if 2 proposed action tubes would connect or not and some updatable lists.
  Our approach includes gathering all candidate action tubes whose score is bigger than a threshold, and use them as active action tubes for
  new possible connections. When the number of gathered active action tube is bigger than a threshold, we keep the k-best scoring action tubes
  and remove the rest.  We implement this algorithm using, also, CUDA code in order to calculate connection score faster. We proposed 3 versions of this algorith:
  \begin{enumerate}
  \item An approach which uses an updatable scoring threshold, in order not to calculate unnecessary connection scores
  \item An approach which doesn't use an updatable scoring threshold, but it just updateds ``active'' action tube more frequently.
  \item An approach which, also, uses NMS or softmax-NMS algorithms for getting wider action tube proposals.
  \end{enumerate}
  Also, we implement, from scratch, another connection algorithm proposed by \cite{DBLP:journals/corr/abs-1903-00304} and extending it in order to work for ToIs instead of frames, which they proposed.
  We modified our TPN structure in order to calculate progression and progress rate scores in order to calculate connection scores and generate candidate action tubes.
\item We experiment using several classifier in order to find the most suitable. We considered 2 feature maps extracted using 3D RoiAlign and proposed action tubes, without any other
  modification. Also, we explore the different ratios and number of groundtruth foreground tubes that should be used during training
  stage. Finally, we tried to perform only temporal localization using temporal information generated from proposed action tubes.
\end{enumerate}

\section{Thesis structure}
The rest of Thesis is organized as follows. Chapter 2 provides an general introduction to Machine Learning techniques currently used.
After that, we present the basic elements of object recognition systems and alongside with loss functions and evaluation metrics that
we used. Also, Chapter 2 presents an brief overview of literature on human action recognition and localization. Chapter 3 introduces the first basic element of our network, Tube Proposal Network (TPN), a network which proposes Tubes of Interest (ToIs), which are sequences of bounding boxes, with are likely to contain a performed action. Furthermore, it contains all the proposed architectures for achieving this.
Chapter 4 proposes algorithms for linking the proposed TOIs from every video segment and proposal performance is presented.
In Chapter 5, we present all the classification approaches we used for designing our architecture and some classification results.
Chapter 6 is used for conclusions, summary of our contribution alongside with possible future work.

% \end{document}

\chapter{Εισαγωγή}

Μπλα μπλα μπλα, μπλα μπλα, μπλα μπλα μπλα, μπλα μπλα μπλα μπλα,
μπλα μπλα μπλα, μπλα μπλα μπλα, μπλα, μπλα μπλα μπλα, μπλα μπλα,
μπλα μπλα μπλα, μπλα, μπλα μπλα μπλα, μπλα μπλα, μπλα μπλα μπλα,
μπλα, μπλα μπλα μπλα, μπλα μπλα, μπλα μπλα μπλα, μπλα, μπλα μπλα
μπλα, μπλα μπλα, μπλα μπλα μπλα, μπλα, μπλα μπλα μπλα, μπλα μπλα,
μπλα μπλα μπλα, μπλα, μπλα μπλα μπλα, μπλα μπλα, μπλα μπλα μπλα,
μπλα, μπλα μπλα μπλα, μπλα μπλα, μπλα μπλα μπλα, μπλα, μπλα μπλα
μπλα, μπλα μπλα, μπλα μπλα μπλα, μπλα, μπλα μπλα μπλα, μπλα μπλα,
μπλα μπλα μπλα, μπλα, μπλα μπλα μπλα, μπλα μπλα, μπλα μπλα μπλα,
μπλα, μπλα μπλα μπλα, μπλα μπλα, μπλα μπλα μπλα, μπλα, μπλα μπλα
μπλα, μπλα μπλα, μπλα μπλα μπλα, μπλα, μπλα μπλα μπλα.


\section{Η γλώσσα προγραμματισμού C}

Μπλα μπλα μπλα, μπλα μπλα, μπλα μπλα μπλα, μπλα μπλα μπλα μπλα,
μπλα μπλα μπλα, μπλα μπλα μπλα, μπλα, μπλα μπλα μπλα, μπλα μπλα,
μπλα μπλα μπλα, μπλα, μπλα μπλα μπλα, μπλα μπλα, μπλα μπλα μπλα,
μπλα, μπλα μπλα μπλα, μπλα μπλα, μπλα μπλα μπλα, μπλα, μπλα μπλα
μπλα, μπλα μπλα, μπλα μπλα μπλα, μπλα, μπλα μπλα μπλα, μπλα μπλα,
μπλα μπλα μπλα, μπλα, μπλα μπλα μπλα, μπλα μπλα, μπλα μπλα μπλα,
μπλα, μπλα μπλα μπλα, μπλα μπλα, μπλα μπλα μπλα, μπλα, μπλα μπλα
μπλα, μπλα μπλα μπλα, μπλα μπλα, μπλα μπλα μπλα, μπλα, μπλα μπλα
μπλα, μπλα μπλα, μπλα μπλα μπλα, μπλα, μπλα μπλα μπλα, μπλα μπλα,
μπλα μπλα μπλα, μπλα, μπλα μπλα μπλα, μπλα μπλα, μπλα μπλα μπλα,
μπλα, μπλα μπλα μπλα, μπλα μπλα, μπλα μπλα μπλα, μπλα, μπλα μπλα
μπλα, μπλα μπλα, μπλα μπλα μπλα, μπλα, μπλα μπλα μπλα, μπλα μπλα,
μπλα μπλα μπλα, μπλα, μπλα μπλα μπλα, μπλα μπλα, μπλα μπλα μπλα,
μπλα, μπλα μπλα μπλα, μπλα μπλα, μπλα μπλα μπλα, μπλα, μπλα μπλα
μπλα, μπλα μπλα, μπλα μπλα μπλα, μπλα, μπλα μπλα μπλα.

Μπλα μπλα μπλα, μπλα μπλα, μπλα μπλα μπλα, μπλα μπλα μπλα μπλα,
μπλα μπλα μπλα, μπλα μπλα μπλα, μπλα, μπλα μπλα μπλα, μπλα μπλα,
μπλα μπλα μπλα, μπλα, μπλα μπλα μπλα, μπλα μπλα, μπλα μπλα μπλα,
μπλα, μπλα μπλα μπλα, μπλα μπλα, μπλα μπλα μπλα, μπλα, μπλα μπλα
μπλα μπλα μπλα, μπλα, μπλα μπλα μπλα, μπλα μπλα, μπλα μπλα μπλα,
μπλα, μπλα μπλα μπλα, μπλα μπλα, μπλα μπλα μπλα, μπλα, μπλα μπλα
μπλα, μπλα μπλα, μπλα μπλα μπλα, μπλα, μπλα μπλα μπλα, μπλα μπλα,
μπλα μπλα μπλα, μπλα, μπλα μπλα μπλα, μπλα μπλα, μπλα μπλα μπλα,
μπλα, μπλα μπλα μπλα, μπλα μπλα, μπλα μπλα μπλα, μπλα, μπλα μπλα
μπλα, μπλα μπλα, μπλα μπλα μπλα, μπλα, μπλα μπλα μπλα, μπλα μπλα,
μπλα μπλα μπλα, μπλα, μπλα μπλα μπλα, μπλα μπλα, μπλα μπλα μπλα,
μπλα, μπλα μπλα, μπλα μπλα μπλα, μπλα, μπλα μπλα μπλα, μπλα μπλα,
μπλα μπλα μπλα, μπλα, μπλα μπλα μπλα, μπλα μπλα, μπλα μπλα μπλα,
μπλα, μπλα μπλα, μπλα μπλα μπλα, μπλα, μπλα μπλα μπλα, μπλα μπλα,
μπλα μπλα μπλα, μπλα, μπλα μπλα μπλα, μπλα μπλα, μπλα μπλα μπλα,
μπλα, μπλα μπλα μπλα, μπλα μπλα, μπλα μπλα μπλα, μπλα, μπλα μπλα
μπλα, μπλα μπλα, μπλα μπλα μπλα, μπλα, μπλα μπλα μπλα.

Μπλα μπλα μπλα, μπλα μπλα, μπλα μπλα μπλα, μπλα μπλα μπλα μπλα,
μπλα μπλα μπλα, μπλα μπλα μπλα, μπλα, μπλα μπλα μπλα, μπλα μπλα,
μπλα μπλα μπλα, μπλα, μπλα μπλα μπλα, μπλα μπλα, μπλα μπλα μπλα,
μπλα, μπλα μπλα μπλα, μπλα μπλα, μπλα μπλα μπλα, μπλα, μπλα μπλα
μπλα μπλα μπλα, μπλα, μπλα μπλα μπλα, μπλα μπλα, μπλα μπλα μπλα,
μπλα, μπλα μπλα μπλα, μπλα μπλα, μπλα μπλα μπλα, μπλα, μπλα μπλα
μπλα, μπλα μπλα, μπλα μπλα μπλα, μπλα, μπλα μπλα μπλα, μπλα μπλα,
μπλα μπλα μπλα, μπλα, μπλα μπλα μπλα, μπλα μπλα, μπλα μπλα μπλα,
μπλα, μπλα μπλα μπλα, μπλα μπλα, μπλα μπλα μπλα, μπλα, μπλα μπλα
μπλα, μπλα μπλα, μπλα μπλα μπλα, μπλα, μπλα μπλα μπλα, μπλα μπλα,
μπλα μπλα μπλα, μπλα, μπλα μπλα μπλα, μπλα μπλα, μπλα μπλα μπλα,
μπλα, μπλα μπλα μπλα, μπλα μπλα, μπλα μπλα μπλα, μπλα, μπλα μπλα
μπλα, μπλα μπλα, μπλα μπλα μπλα, μπλα, μπλα μπλα μπλα.


\section{Σημασιολογία γλωσσών προγραμματισμού}

Μπλα μπλα μπλα, μπλα μπλα, μπλα μπλα μπλα, μπλα μπλα μπλα μπλα,
μπλα μπλα μπλα, μπλα μπλα μπλα, μπλα, μπλα μπλα μπλα, μπλα μπλα,
μπλα μπλα μπλα, μπλα, μπλα μπλα μπλα, μπλα μπλα, μπλα μπλα μπλα,
μπλα, μπλα μπλα μπλα, μπλα μπλα, μπλα μπλα μπλα, μπλα, μπλα μπλα
μπλα μπλα μπλα, μπλα, μπλα μπλα μπλα, μπλα μπλα, μπλα μπλα μπλα,
μπλα, μπλα μπλα μπλα, μπλα μπλα, μπλα μπλα μπλα, μπλα, μπλα μπλα
μπλα, μπλα μπλα, μπλα μπλα μπλα, μπλα, μπλα μπλα μπλα, μπλα μπλα,
μπλα \nocite{*} μπλα μπλα, μπλα, μπλα μπλα μπλα, μπλα μπλα, μπλα
μπλα μπλα, μπλα, μπλα μπλα μπλα, μπλα μπλα, μπλα μπλα μπλα, μπλα,
μπλα μπλα μπλα, μπλα μπλα, μπλα μπλα μπλα, μπλα, μπλα μπλα μπλα,
μπλα μπλα, μπλα μπλα μπλα, μπλα, μπλα μπλα μπλα, μπλα μπλα, μπλα
μπλα μπλα, μπλα, μπλα μπλα μπλα, μπλα μπλα, μπλα μπλα μπλα, μπλα,
μπλα μπλα μπλα, μπλα μπλα, μπλα μπλα μπλα, μπλα, μπλα μπλα μπλα.

Μπλα μπλα μπλα, μπλα μπλα, μπλα μπλα μπλα, μπλα μπλα μπλα μπλα,
μπλα μπλα μπλα, μπλα μπλα μπλα, μπλα, μπλα μπλα μπλα, μπλα μπλα,
μπλα μπλα μπλα, μπλα, μπλα μπλα μπλα, μπλα μπλα, μπλα μπλα μπλα,
μπλα, μπλα μπλα μπλα, μπλα μπλα, μπλα μπλα μπλα, μπλα, μπλα μπλα
μπλα, μπλα μπλα, μπλα μπλα μπλα, μπλα, μπλα μπλα μπλα, μπλα μπλα,
μπλα μπλα μπλα, μπλα, μπλα μπλα μπλα, μπλα μπλα, μπλα μπλα μπλα,
μπλα, μπλα μπλα μπλα, μπλα μπλα, μπλα μπλα μπλα, μπλα, μπλα μπλα
μπλα, μπλα μπλα, μπλα μπλα μπλα, μπλα, μπλα μπλα μπλα, μπλα μπλα,
μπλα μπλα μπλα, μπλα, μπλα μπλα μπλα, μπλα μπλα, μπλα μπλα μπλα,
μπλα, μπλα μπλα μπλα, μπλα μπλα, μπλα μπλα μπλα, μπλα, μπλα μπλα
μπλα, μπλα μπλα, μπλα μπλα μπλα, μπλα, μπλα μπλα μπλα.

Μπλα μπλα μπλα, μπλα μπλα, μπλα μπλα μπλα, μπλα μπλα μπλα μπλα,
μπλα μπλα μπλα, μπλα μπλα μπλα, μπλα, μπλα μπλα μπλα, μπλα μπλα,
μπλα μπλα μπλα, μπλα, μπλα μπλα μπλα, μπλα μπλα, μπλα μπλα μπλα,
μπλα, μπλα μπλα μπλα, μπλα μπλα, μπλα μπλα μπλα, μπλα, μπλα μπλα
μπλα μπλα μπλα, μπλα, μπλα μπλα μπλα, μπλα μπλα, μπλα μπλα μπλα,
μπλα, μπλα μπλα μπλα, μπλα μπλα, μπλα μπλα μπλα, μπλα, μπλα μπλα
μπλα, μπλα μπλα μπλα, μπλα μπλα, μπλα μπλα μπλα, μπλα, μπλα μπλα
μπλα μπλα μπλα, μπλα, μπλα μπλα μπλα, μπλα μπλα, μπλα μπλα μπλα,
μπλα, μπλα μπλα μπλα, μπλα μπλα, μπλα μπλα μπλα, μπλα, μπλα μπλα
μπλα μπλα μπλα, μπλα, μπλα μπλα μπλα, μπλα μπλα, μπλα μπλα μπλα,
μπλα, μπλα μπλα, μπλα μπλα μπλα, μπλα, μπλα μπλα μπλα, μπλα μπλα,
μπλα μπλα μπλα, μπλα, μπλα μπλα μπλα, μπλα μπλα, μπλα μπλα μπλα,
μπλα, μπλα μπλα μπλα, μπλα μπλα, μπλα μπλα μπλα, μπλα, μπλα μπλα
μπλα, μπλα μπλα, μπλα μπλα μπλα, μπλα, μπλα μπλα μπλα, μπλα μπλα,
μπλα μπλα μπλα, μπλα, μπλα μπλα μπλα, μπλα μπλα, μπλα μπλα μπλα,
μπλα, μπλα μπλα μπλα, μπλα μπλα, μπλα μπλα μπλα, μπλα, μπλα μπλα
μπλα, μπλα μπλα, μπλα μπλα μπλα, μπλα, μπλα μπλα μπλα.


\section{Θεωρία πεδίων}

Μπλα μπλα μπλα, μπλα μπλα, μπλα μπλα μπλα, μπλα μπλα μπλα μπλα,
μπλα μπλα μπλα, μπλα μπλα μπλα, μπλα, μπλα μπλα μπλα, μπλα μπλα,
μπλα μπλα μπλα, μπλα, μπλα μπλα μπλα, μπλα μπλα, μπλα μπλα μπλα,
μπλα, μπλα μπλα μπλα, μπλα μπλα, μπλα μπλα μπλα, μπλα, μπλα μπλα
μπλα μπλα μπλα, μπλα, μπλα μπλα μπλα, μπλα μπλα, μπλα μπλα μπλα,
μπλα, μπλα μπλα μπλα, μπλα μπλα, μπλα μπλα μπλα, μπλα, μπλα μπλα
μπλα, μπλα μπλα, μπλα μπλα μπλα, μπλα, μπλα μπλα μπλα, μπλα μπλα,
μπλα μπλα μπλα, μπλα, μπλα μπλα μπλα, μπλα μπλα, μπλα μπλα μπλα,
μπλα, μπλα μπλα μπλα, μπλα μπλα, μπλα μπλα μπλα, μπλα, μπλα μπλα
μπλα, μπλα μπλα, μπλα μπλα μπλα, μπλα, μπλα μπλα μπλα, μπλα μπλα,
μπλα μπλα μπλα, μπλα, μπλα μπλα μπλα, μπλα μπλα, μπλα μπλα μπλα,
μπλα, μπλα μπλα μπλα, μπλα μπλα, μπλα μπλα μπλα, μπλα, μπλα μπλα
μπλα, μπλα μπλα, μπλα μπλα μπλα, μπλα, μπλα μπλα μπλα.

Μπλα μπλα μπλα, μπλα μπλα, μπλα μπλα μπλα, μπλα μπλα μπλα μπλα,
μπλα μπλα μπλα, μπλα μπλα μπλα, μπλα, μπλα μπλα μπλα, μπλα μπλα,
μπλα μπλα μπλα, μπλα, μπλα μπλα μπλα, μπλα μπλα, μπλα μπλα μπλα,
μπλα, μπλα μπλα μπλα, μπλα μπλα, μπλα μπλα μπλα, μπλα, μπλα μπλα
μπλα, μπλα μπλα, μπλα μπλα μπλα, μπλα, μπλα μπλα μπλα, μπλα μπλα,
μπλα μπλα μπλα, μπλα, μπλα μπλα μπλα, μπλα μπλα, μπλα μπλα μπλα,
μπλα, μπλα μπλα μπλα, μπλα μπλα, μπλα μπλα μπλα, μπλα, μπλα μπλα
μπλα, μπλα μπλα, μπλα μπλα μπλα, μπλα, μπλα μπλα μπλα, μπλα μπλα,
μπλα μπλα μπλα, μπλα, μπλα μπλα μπλα, μπλα μπλα, μπλα μπλα μπλα,
μπλα, μπλα μπλα μπλα, μπλα μπλα, μπλα μπλα μπλα, μπλα, μπλα μπλα
μπλα, μπλα μπλα, μπλα μπλα μπλα, μπλα, μπλα μπλα μπλα.

Μπλα μπλα μπλα, μπλα μπλα, μπλα μπλα μπλα, μπλα μπλα μπλα μπλα,
μπλα μπλα μπλα, μπλα μπλα μπλα, μπλα, μπλα μπλα μπλα, μπλα μπλα,
μπλα μπλα μπλα, μπλα, μπλα μπλα μπλα, μπλα μπλα, μπλα μπλα μπλα,
μπλα, μπλα μπλα μπλα, μπλα μπλα, μπλα μπλα μπλα, μπλα, μπλα μπλα
μπλα μπλα μπλα, μπλα, μπλα μπλα μπλα, μπλα μπλα, μπλα μπλα μπλα,
μπλα, μπλα μπλα μπλα, μπλα μπλα, μπλα μπλα μπλα, μπλα, μπλα μπλα
μπλα, μπλα μπλα μπλα, μπλα μπλα, μπλα μπλα μπλα, μπλα, μπλα μπλα
μπλα μπλα μπλα, μπλα, μπλα μπλα μπλα, μπλα μπλα, μπλα μπλα μπλα,
μπλα, μπλα μπλα μπλα, μπλα μπλα, μπλα μπλα μπλα, μπλα, μπλα μπλα
μπλα μπλα μπλα, μπλα, μπλα μπλα μπλα, μπλα μπλα, μπλα μπλα μπλα,
μπλα, μπλα μπλα, μπλα μπλα μπλα, μπλα, μπλα μπλα μπλα, μπλα μπλα,
μπλα μπλα μπλα, μπλα, μπλα μπλα μπλα, μπλα μπλα, μπλα μπλα μπλα,
μπλα, μπλα μπλα μπλα, μπλα μπλα, μπλα μπλα μπλα, μπλα, μπλα μπλα
μπλα, μπλα μπλα, μπλα μπλα μπλα, μπλα, μπλα μπλα μπλα, μπλα μπλα,
μπλα μπλα μπλα, μπλα, μπλα μπλα μπλα, μπλα μπλα, μπλα μπλα μπλα,
μπλα, μπλα μπλα μπλα, μπλα μπλα, μπλα μπλα μπλα, μπλα, μπλα μπλα
μπλα, μπλα μπλα, μπλα μπλα μπλα, μπλα, μπλα μπλα μπλα.


% \documentclass{report}

% \usepackage{hyperref}  % package for linking figures etc
% \usepackage{enumitem}  % package for description with bullets
% \usepackage{graphicx}  % package for importing images
% \usepackage{mathtools} % package for math equation
% \usepackage{mathrsfs}  % package for math font
% \usepackage{indentfirst} % package for getting ident after section or paragraph
% % \usepackage{amsmath}
% \usepackage[
%     backend=bibtex,
%     style=authoryear,
%     sorting=ynt,
%     % style=numeric,
%     % style=alphabetic ,
%   ]{biblatex}
 
%  \addbibresource{References}
% \setlength{\parindent}{2em} % how much indent to use when we start a paragraph

% \graphicspath{ {./theory/figures/} }       % path for images

% \begin{document}

\chapter{Related work}


% The combination of CNN and Recurrent Neural Network (RNN) is a method widely been studied and achieved excellent results.
% The recently proposed methods can be seperated into 2 subcategories: (1) use a CNN detector in order to localize directly
% the action instance for each single frame and (2) modify the detector in order to get as input volute multiple frames and
% use 3D convolution.%  Both approaches produce the same type of output : i) an actioness score of a bounding box or a proposed
% tube ii) coordinate offsets for bounding box refinement and iii) clasification probabilities for all action classes.

% The task of action classification in approaches used features based on shape or motion sucha as HOG \cite{dalal2005histogramcvpr},
% SIFT \cite{Lowe2004}, MBH \cite{dalal2006human} and train a classifier as SVM.

% \paragraph{Frame based methods}
\section{Action Recognition}
First approaches for action classification consisted of 2 steps a) compute complex handcrafted features from raw video frames
such as SIFT, HOG, optical flow and b) train a classifier based on those features. These approaches made the choise of
features a signifact factor for network's performance. That's because different action classes may appear dramatically
different in terms of their appearences and motion patterns. Another problem was that most of those approaches take
assumptions about the circumstances under which the video was taken because there was problems such as cluttered
background, camera viewpoit variations etc. A review techniques used until 2011 made by \cite{Aggarwal:2011:HAA:1922649.1922653}.

Recent results in deep architectures and especially in image classification made us attempt to train CNN networks for
the task of action classification. First significant attempt made by \cite{6909619}.  % \cite{DBLP:journals/corr/WangXW0LTG16}.
 \cite{simonyan2014two} and \cite{DBLP:journals/corr/FeichtenhoferPZ16} both added optical flow in order to achieve better results.
On top of that, the increase  in computing performance contributed to the design more complicated architectures including
3D Convolutions as presented in \cite{6165309} as done by \cite{DBLP:journals/corr/TranBFTP14}.

R(2+1) \cite{DBLP:journals/corr/abs-1711-11248}
\textbf(Pending...More before 3D ResNet)
Recent day 3D ResNet has been introduced by \cite{Hara_2018_CVPR} 
\section{Action Localization}
As mentioned before, Action Localization can be seen as an extention of object detection problem, where the outputs are action tubes
that consist of a sequence of bounding boxes. So, there are several approaches including an object-detector network for single frame
action proposal and a classifier. The introduction of R-CNN (\cite{DBLP:journals/corr/GirshickDDM13}) achieve significant improvemet
in the performance of Object Detection Networks. This architectures, firstly, proposes regions in the image which are likely to
contain an object and then it classify it using a SVM classifier. Inspired by this architecture, \cite{DBLP:journals/corr/GkioxariM14}
desing a 2-stream RCNN network in order to generate action proposals for each frame, one stream for frame level and one for optical flow.
Then they  connect them using viterbi connection algorithm. \cite{DBLP:journals/corr/WeinzaepfelHS15} extend this approach, performing
frame-level proposals and using a tracker for connecting those proposals using both spatial and optical flow features. Also it performs
temporal localization using a sliding window over the tracked tubes. \cite{peng:hal-01349107} used Faster R-CNN (\cite{Ren:2015:FRT:2969239.2969250})
instead of RCNN for frame-level proposals, and they use Viterbi algorithm for linking proposals, too. For temporal localization, they
use a maximum subarray method.







\cite{6909495} introduces the tubelets.

% \\
% \cite{peng:hal-01349107} NA DW TI KANEI.

\cite{singh2016online} uses SSD 

Some approaches include tracking \cite{DBLP:journals/corr/WeinzaepfelHS15}.
Other approaches treat a video as a sequence of frames such as in \cite{DBLP:journals/corr/KalogeitonWFS17} and in \cite{DBLP:journals/corr/HouCS17}.

\paragraph{3d-2d pose}



% % \paragraph{Tube-level proposals}
% The previous approaches do not consider the motion information enconded in multiple contiguous frames because they treat video
% frames as still images. In \cite{pmid:22392705} was introuduce 3D Convolution which captures spatial and temporal information.

% On top of that, and based also in \cite{simonyan2014two},  \cite{DBLP:journals/corr/KalogeitonWFS17} presented action tubes

% % We can devide
% % the previous methods in 3 subcategories (1) temporal modeling for action representation (2) object detection and (3) spatial-temporal
% % localization.


\section{Our implementation}
We propose a network similar to \cite{DBLP:journals/corr/HouCS17}. Our architecture is consisted by the following basic elements:
\begin{itemize}
\item One 3D Convolutional Network, which is used for feature extraction. In our implementaion we use a 3D Resnet network which is taken from
  \cite{hara3dcnns} and it is based on ResNet CNNs for Image Classification \cite{DBLP:journals/corr/HeZRS15}.
\item Tube Proposal Network for proposing action tubes (based on the idea presented in \cite{DBLP:journals/corr/HouCS17}).
\item A classifier for classifying video tubes.
\end{itemize}
  


% \printbibliography

% \end{document}
\documentclass{report}

\usepackage{subcaption} % package for subfigures
\usepackage{hyperref}  % package for linking figures etc
\usepackage{enumitem}  % package for description with bullets
\usepackage{graphicx}  % package for importing images
\usepackage{mathtools} % package for math equation
\usepackage{mathrsfs}  % package for math font
\usepackage{indentfirst} % package for getting ident after section or paragraph
\usepackage[export]{adjustbox}
\usepackage{multirow}  % package for tables, multir
\usepackage{amssymb}
% \usepackage{tabu}   % for tables 
\setlength{\parindent}{2em} % how much indent to use when we start a paragraph

\graphicspath{ {./theory/figures/} }       % path for images

\begin{document}

\chapter{Tube Proposal Network}

One of the basic elements of ActionNet is \textbf{Tube Proposal Network}(TPN). The main purpose of this network is to propose
\textbf{Tube of Interest}(TOIs). These tubes are likely to contain an known action and are consisted of some 2D boxes
(1 for each frame). TPN is inspired from RPN introduced by FasterRCNN (\cite{Ren:2015:FRT:2969239.2969250}), but instead of images, TPN
is used in videos as show in \cite{DBLP:journals/corr/HouCS17}. In full correspondence with RPN, the structure
of TPN is similar to RPN. The only difference, is that TPN uses 3D Convolutional Layers and 3D anchors instead of 2D. \par
We designed 2 main structures for TPN. Each approach has a different definition of the used 3D anchors.
The rest structure of the TPN is mainly the same with some little differences in the regression layer. \par
Before describing TPN, we present the preprocess procedure which is the same for both approaches.

\section{Preparation for TPN}

\subsection{Preparing data}

Our architecture gets as input a sequnece of frames which has a fixed size in widht, height and duration. However, each video has different resolution. That's creates the
need to resize each frame before. As mentioned in previous chapter, the first element of our network is a 3D RenNet taken from \cite{hara3dcnns}. This network is designed to
get images with dimensions (112,112). As a result, we resize each frame from datasets' videos into (112,112) frames. In order to keep aspect ratio, we pad each frame either
left and right, either above and bellow depending which dimension is bigger. In figure  \ref{fig:Preprocess_example} we can see the original frame and the resize and padded one.
In full correspondance, we resize the groundtruth bounding boxes for each frame (figure \ref{fig:original_image_rois} and \ref{fig:trans_image_rois} show that).

\begin{figure}[h]
  \centering
  \begin{subfigure}{0.35\textwidth}
    \includegraphics[width=\textwidth]{./figures/original_image.jpg}
    \caption{}
    \label{fig:original_image}
  \end{subfigure}
  \hfill
  \begin{subfigure}{0.35\textwidth}
    \includegraphics[width=\textwidth]{./figures/original_image_rois.jpg}
    \caption{}
    \label{fig:original_image_rois}
  \end{subfigure}
  \hfill
  \begin{subfigure}{0.35\textwidth}
    \includegraphics[width=\textwidth]{./figures/transformed_image.jpg}
    \caption{}
    \label{fig:trans_image}
  \end{subfigure}
  \hfill
  \begin{subfigure}{0.35\textwidth}
    \includegraphics[width=\textwidth]{./figures/transformed_image_rois.jpg}
    \caption{}
    \label{fig:trans_image_rois}
  \end{subfigure}

  \caption{ At (a), (b) frame is its original size and at (c), (d) same frame after preprocessing part}
  \label{fig:Preprocess_example}
\end{figure}


\subsection{3D ResNet}
Before using Tube Proposal Network, we spatio-temporal features from the video. In order to do so, we extract the 3 first Layers of a
pretrained 3D ResNet. It is pretrained in Kinetics dataset \cite{DBLP:journals/corr/KayCSZHVVGBNSZ17} for sample duration = 16 and
sample size = (112,122). \par
This network normally is used for classifying the whole video, so some of its layers use temporal stride = 2.
We set their temporal stride equal to 1 because we don't want to miss any temporal information during the process.
So, the output of the third layer is a feature maps with dimesions (256,16,7,7). We feed this feature map to TPN, which is described
in following sections.

\section{ 3D anchors as 6-dim vector}
\subsection{First Description}
We started desinging our TPN inspired by \cite{DBLP:journals/corr/HouCS17}. We consider each anchor as a 3D bounding box written as
$(x_1, y_1, t_1, x_2, y_2, t_2)$ where $x_1, y_1, t_1$
are the upper front left coordinates of the 3D and $x_2, y_2, t_2$ are the lower back left as shown in figure \ref{fig:anchor_6d}.
\begin{figure}[h]
  \centering
  \includegraphics[scale=0.5]{anchor_6d}
  \caption{An example of the anchor $(x_1,y_1,t_1,x_2,y_2,t_2)$}
  \label{fig:anchor_6d}
\end{figure}

The main advantage of this approach is that except from x-y dims, dimension of time is mutable. As a result, the proposed TOIs have
no fixed time duration. This will help us deal with untrimmed videos, because proposed TOIs would exclude background frames.
For this approach, we use \textbf{n=4k=60} anchors for each pixel in the feature map of TPN. We have k anchors for each sample
duration( 5 scales of 1, 2, 4, 8, 16, 3 aspect ratios of 1:1, 1:2, 2:1 and 4 durations of 16,12,8,4 frames).
In \cite{DBLP:journals/corr/HouCS17},  network's anchors are defined according to the dataset most common anchors. This, however,
creates the need to redesign the network for each dataset. In our approach, we use the same anchors for both datasets, because we want our network not
to be dataset-specific but to be able to generalize for several datasets. As sample duration, we chose 16 frames per video segment because
our pre-trained ResNet is trained for video clips with that duration.
So the structure of TPN is:
\begin{itemize}
\item 1 3D Convolutional Layer with kernel size = 3, stride = 3 and padding = 1
\item 1 classification layer outputs \textit{2n scores} whether there is an action or not for \textit{n tubes}.
\item 1 regression layer outputs \textit{6n coordinates} ($x_1,y_1,t_1,x_2,y_2,t_2$) for \textit{n tubes}.
\end{itemize}

which is shown in figure \ref{fig:tpn_1_1}
\begin{figure}[h]

  \includegraphics[width=1.\textwidth]{tpn_1_1}
  \caption{Structure of TPN}
  \label{fig:tpn_1_1}
\end{figure}

The output of TPN is the k-best scoring cuboid, in which it is likely to contain an action.
% \subsection{Preparing Data for training}


\subsection{Training}
As mentioned before, TPN extracts TOIs as 6-dim vectors. For that reason, we modify out groundtruth ROIs to groundtruth Tubes.
We take for granted that the actor cannot move a lot during 16 frames, so that's why we use this kind of tubes. As shown 
in figure \ref{fig:gt_tubes_and_rois}, these tubes are 3D boxes which include all the groundtruth rois, which are different
for each frame.

\begin{figure}[h]
  \centering
  \begin{subfigure}{0.15\textwidth}
    \includegraphics[width=\textwidth]{output/img_0.jpg}
  \end{subfigure}
  \begin{subfigure}{0.15\textwidth}
    \includegraphics[width=\textwidth]{output/img_3.jpg}
  \end{subfigure}
  \begin{subfigure}{0.15\textwidth}
    \includegraphics[width=\textwidth]{output/img_5.jpg}
  \end{subfigure}
  \begin{subfigure}{0.15\textwidth}
    \includegraphics[width=\textwidth]{output/img_7.jpg}
  \end{subfigure}
  \begin{subfigure}{0.15\textwidth}
    \includegraphics[width=\textwidth]{output/img_11.jpg}
  \end{subfigure}
  \begin{subfigure}{0.15\textwidth}
    \includegraphics[width=\textwidth]{output/img_15.jpg}
  \end{subfigure}
  \caption{Groundtruth tube is coloured with blue and groundtruth rois with colour green}
  \label{fig:gt_tubes_and_rois}
\end{figure}

For training procedure, for each video, we randomly select a part of it which has duration 16 frames. For each video, we train TPN in
order to score all the anchors using IoU criterion (as explained in chapter 2 when 3D boxes are cubboids) and we use Cross Entropy Loss as a loss function.
For regression, we use smooth-L1 loss. For regression targets, we use pytorch FasterRCNN implementation (\cite{jjfaster2rcnn}) for bounding box regression.
We modified the code in order to extend it for 3 dimensions. \textbf{TODO more details}
So we have:
\[ \begin{matrix}
    t_x = (x-x_a)/w_a, & t_y = (y-y_a)/h_a, & t_z= (z-z_a)/d_a, \\
    t_w= log(w/w_a), & t_h= log(h/h_a), & t_d = log(d/d_a), \\
    t^*_x = (x^* - x_a)/w_a, & t^*_y = (y^* - y_a)/h_a, & t^*_z = (z^* - z_a)/d_a, \\
    t^*_w = log(w^* /w_a), & t^*_h = log(h^*/h_a), & t^*_d = log(d^*/d_a),
    % t∗x= (x∗−xa)/wa,  t∗y= (y∗−ya)/ha,t∗w= log(w∗/wa),  t∗h= log(h∗/ha)
  \end{matrix}
\]
where \textit{x, y, z, w, h, d} denote the 3D box's center coordinates and its widht, height and duration. Variables $x, x_a, \text{ and } x^*$
are for the predicted box, anchor box, and groundthruth box respectively (likewise for \textit{y, z, w, h, d}). \par

\textbf{TODO Training Loss formula}

\subsection{Validation}

Validation procedure is a bit similar to training procedure.
We randomly select 16 frames from a validation video and we examine if there is at least 1 proposed TOI
which overlaps $\ge$ 0.5 with each groundtruth action tube and we get recall score. 
In order to get good proposals, after TPN we use Non-Maximum Suppresion (NMS) algorith. We set NMS threshold equal with 0.7, so we reject
thems with overlapping score > 0.7 .

\subsection{Modified Intersection over Union(mIoU) TODO check again} 
During training, we get numerous anchors. We have to classify them as foreground anchors or
background anchors. Foreground anchors are those which contain some action, and, respectively, background
don't. As presented before, IoU for cuboids calculates the ratio between volume of overlap and volume of
union.
Intuitively, this criterion is good for evaluating 2 tubes if they overlap but it has one big drawback:
it considers x-y dimesions to have same importance with time dimension, which we do not desire. That's becase
firstly we care to be accurate in time dimension, and then we can fix x-y domain.
As a result, we change the way we calculate the Intesection Over Union. We calculate seperately
the IoU in x-y domain (IoU-xy) and in t-domain (IoU-t). Finally, we multiply them in order to get the final IoU.
So the formula for 2 tubes ($x_1, y_1, t_1, x_2, y_2, t_2$) and ($x'_1, y'_1, t'_1, x'_2, y'_2, t'_2$) is:
\[ IoU_{xy} = \frac{ \text{Area of Overlap in x-y}} { \text{Area of Union in x-y}}  \]
\[ IoU_t = \frac { max(t_1, t'_1) - min(t_2, t'_2)} {min(t_1,t'_1) - max(t_2,t'_2)} \]
\[ IoU = IoU_{xy} \cdot  IoU_t \]
The above criterion help us balance the impact of time domain in IoU. For example, let us consider 2 anchors:
a = (22, 41, 1, 34, 70, 5) and b = (20, 45, 2, 32, 72, 5). These 2 anchors in x-y domain have IoU score equal to 0.61.
But they are not exactly overlaped in time dim. Using the first approach we get 0.5057 IoU score and using the
second approach we get 0.4889. So, the second criterion would reject this anchor, because there is a difference in time
duration.  \par

In order to verify our idea, we train TPN using both IoU and mIoU criterion for tube-overlapping. At Table \ref{table:iou_miou}
we can see the performance in each case for both datasets, JHMDB and UCF. Recall threshold for this case is 0.5 and during validation,
we use regular IoU for defining if 2 tubes overlap.
\begin{table}[h]
\centering
  \begin{tabular}{|| c | c | c ||}
    \hline
    \textbf{Dataset} & \textbf{Criterion} & \textbf{Recall(0.5)} \\
    \hline  \hline
    \multirow{2}{4em}{JHMDB} & IoU & TODO \\
    \cline{2-3}
    {} & mIoU & TODO \\
    \hline
    \multirow{2}{4em}{UCF} & IoU & TODO \\
    \cline{2-3}
    {} & mIoU & TODO \\
    \hline      
  \end{tabular}
  \caption{Recall results for both datasets using IoU and mIoU metrics}
  \label{table:iou_miou}
\end{table}

Table \ref{table:iou_miou} shows that modified-IoU give us slightly better recall performance, so that's the overlapping scoring policy
for now on, during Training mode.

\subsection{Improving TPN score}
After first test, we came with the idea that in a video lasting 16 frames, in time domain, all kind of actions can be seperated in the following categories:
\begin{enumerate}
\item Action starts in the n-th  frame and finishes after the 16th frame of the sampled video.
\item Action has already begun before the 1st frame of the video and ends in the n-th frame.
\item Action has already begun before the 1st frame of the video and finishes after the 16th video frame.
\item Action starts and ends in that 16 frames of the video.
\end{enumerate}

On top of that, we noticed that most of actions, in our datasets, last more that 16 frame. So, we added 1 scoring layer and 1 reggression layer as shown in 
figure \ref{fig:tpn_1_2}. These two layers have anchors with fixed time duration. Their purpose is to be trained only in x-y domain, keeping time duration
steady. 
\begin{figure}[h]
  \centering
  \includegraphics[scale=0.5]{tpn_1_2}
  \caption{TPN structure after adding 2 new layers, where k = 5n.}
  \label{fig:tpn_1_2}
\end{figure}

Training and Validation procedures remain the same. The only big difference is that now we have from 2 difference system proposed TOIs. So, we first concate
them and, then, we follow the same procedure. For training loss, we have 2 different cross-entropy losses and 2 different smooth-L1 losses, each for every
layer correspondly. \\

\textbf{TODO Training Loss formula}

\paragraph {2 approaches}

\textbf{Kernel vs mean}

\begin{table}[h]
  \centering
  \begin{tabular}{||c | c | c | c ||}
    \hline
    \textbf{Dataset} & \textbf{Fix-time anchors} & \textbf{Type} & \textbf{Recall(0.5)} \\
    \hline  \hline
    \multirow{3}{4em}{JHMDB} & No &  - & TODO \\
    \cline{2-4}
    {} & \multirow{2}{*}{Yes} & Kernel & TODO \\
    \cline{3-4}
    {} & {} & Mean & TODO \\
    \hline
    \multirow{3}{4em}{UCF} & No & - & TODO \\
    \cline{2-4}
    {} & \multirow{2}{*}{Yes} & Kernel & TODO \\
    \cline{3-4}
    {} & {} & Mean & TODO \\
    \hline      
  \end{tabular}
  \caption{Recall results after adding fixed time duration anchors}
  \label{table:add_16}
\end{table}

AS we can see from the previous results, the new layers increased recall performance significantly.

\subsection{Adding regressor}
The output of TPN is $\alpha$-highest scoring anchors moved according to their regression prediction. After that, we have to translate the anchor into tubes.
In order to do so, we add a regressor system which gets as input TOIs' feature maps and returns a sequence of 2D boxes, each for every frame.
The only problem is that the regressor needs a fixed input size of featuremaps. This problem is already solven by R-CNNs which use roi pooling and roi align
in order to get fixed size feature maps from ROIs with changing sizes. In our situation, we extend roi align operation, presented by Mask R-CNN, and we
call it \textbf{3D Roi Align}.

\paragraph{3D Roi Align}
3D Roi align is a modification of roi align presented by Mask R-CNN (\cite{DBLP:journals/corr/HeGDG17}). The main difference between those two is that Mask R-CNN's roi align uses
bilinear interpolation for extracting ROI's features and ours 3D roi align uses trilinear interpolation for the same reason. Again, the 3rd dimension is
time.
So, we have as input a feature map extracted from ResNet34 with dimensions (64,16,28,28) and a tensor containing the proposed TOIs.
For each TOI whose actiovation map whose size is (64,16,7,7), we get as output a feature map with size (64, 16, 7, 7). \par
\subsubsection{Regression procedure}
At first, for each proposed ToI, we get its corresponding activation maps using 3D Roi Align. These features are given as input to a regressor. This regressor returns 16 $\cdot$ 4 predicted
transforms $(\delta_x,\delta_y, \delta_w,\delta_h)$, 4 for each frame, where $ \delta_x, \delta_y$ specify the coordinates of proposal's center and $\delta_w, \delta_h$ its width and height, as specified
in \cite{DBLP:journals/corr/GirshickDDM13}.  We keep only the predicted translations, for the frames that are $\ge t_1$ and $< t_2$ and for the other frames, we set a zero-ed 2D box. 
After that, we modify each anchor from a cuboid written like $(x_1,y_1,t_1, x_2, y_2, t_2)$ to a sequence of 2D boxes, like: \\
$(0,0,0,0, ..., x_{T_1},y_{T_1},x'_{T_1},y'_{T_1}, ... ,x_{i},y_{i},x'_{i}, ..., x_{T_2},y_{T_2},x'_{T_2},y'_{T_2}, 0,0,0,0, ....)$, where
\begin{itemize}
\item $ T_1 \le i \le T_2$, for $T_1 < t_1 + 1,  T_2 < t_2 \text{ and }T_1,T_2 \in \mathbb{Z} $
\item $ x_i = x_1, y_i= y_1, x'_i = x_2, y'_i = y_2 $.
\end{itemize}

\paragraph{ Training}
mIou, 16mean
After getting proposed TOIs from TPN, we pick, randomly, 16 tubes which will be input in the regressor.
Finally, we find the traslation for each rois and, again, we use smooth-L1 loss for loss function.\par

\textbf{TODO Training Loss formula}

\subsubsection{First regression Network} \textbf{Na pw gia to normalize kai to pooling kernel}\par
The architecture of reggression network is show in Figure \ref{fig:regressor_3d}, and it is described below:
\begin{figure}[h]
  \centering
  \includegraphics[scale=0.48]{regressor_1_1}
  \caption{Structure of Regressor}
  \label{fig:regressor_3d}
\end{figure}

\begin{enumerate}
\item Regressor is consisted, at first, with a 3D convolutional layer with kernel = 1, stride = 1 and no padding. This layer gets as input ToI's activation map extracted by 3D Roi Align.
\item After that, we calculate the average value in time domain, so from a feature map with dimensions (64,16,7,7), we get as output a feature map (64,7,7).
\item These feature maps are given as input to a Linear layer followerd by a Relu Layer, a Dropout Layer, another Linear Layer and Relu Layer and a final Linear.
\end{enumerate}

\textbf{NA DW TA FEATURE MAPS POIA XRISIMOPOIISA}
\begin{table}[h]
  \centering
  \begin{tabular} {||c | c | c ||}
    \hline
    \textbf{Dataset} & \textbf{Pooling} & Recall \\
    \hline                
    \multirow{2}{*}{JHMDB} & avg & TODO \\
    \cline{2-3}
    {} & mean & TODO \\
    \cline{1-3}
    \multirow{2}{*}{UCF} & avg & TODO \\
    \cline{2-3}
    {} & mean & TODO \\
    \cline{1-3}
                                   
  \end{tabular}
  \caption{}
  \label{table:reg_1_1}
\end{table}

\textbf{TODO sxoliasmos apotelesmatwn}
As the above results show, when we translate a TOI into a sequence of ROIs, recall reduces about 20-30\%, which is a big problem. 
 % As a result, we should find a solution, in order to
\subsection{Changing Regressor - from 3D to 2d}
After getting first recall results, we experiment using another architecture for the regressor network. Instead of having a 3D Convolutional
Layer, we will use a 2D Convolutional Layer in order to treat the whole time dimension as one during convolution operation. So, as shown in Figure \ref{fig:reg_1_2},
the $2^{nd}$ Regression Network is about the same with first one, with 2 big differences:
\begin{enumerate}
\item We performing a pooling operation at the feature maps extracted by 3D Roi Align operation
\item Instead of a 3D Convolutional Layer, we have a 2D Convolutional Layer with kernel size = 1, stride = 1 and no padding.
\end{enumerate}

\begin{figure}[h]

  \centering
  \includegraphics[scale=0.48]{regressor_1_2}
  \caption{Structure of Regressor}
  \label{fig:regressor_2d}
\end{figure}

On top of that, we tried to determine which feature map is the most suitable  for getting best-scorig recall performance. This feature map will be given as
input to Roi Algin operation.  At Table \ref{table:reg_1_2}, we can see the recall performance for different feature maps and different pooling methods.

\begin{table}[h]
  \centering
  \begin{tabular}{||c | c | c || c||}
    \textbf{Dataset} & \textbf{Pooling} & \textbf{Feature Map} & \textbf{Recall (0.5)} \\
    \hline
    \multirow{6}{*}{JHMDB} & \multirow{3}{*}{mean} & 64 & TODO \\
    \cline{3-4}
    {} & {} & 128 & TODO \\
    \cline{3-4}
    {} & {} & 256 & TODO \\
    \cline{2-4}
    {} & \multirow{3}{*}{max} & 64 & TODO \\
    \cline{3-4}
    {} & {} & 128 & TODO \\
    \cline{3-4}
    {} & {} & 256 & TODO \\
    \hline
    \multirow{6}{*}{UCF} & \multirow{3}{*}{mean} & 64 & TODO \\
    \cline{3-4}
    {} & {} & 128 & TODO \\
    \cline{3-4}
    {} & {} & 256 & TODO \\
    \cline{2-4}
    {} & \multirow{3}{*}{max} & 64 & TODO \\
    \cline{3-4}
    {} & {} & 128 & TODO \\
    \cline{3-4}
    {} & {} & 256 & TODO \\
    \hline


  \end{tabular}
  \caption{}
  \label{table:reg_1_2}
\end{table}

\textbf{TODO add more comment}
As we noticed from the above results, our system has difficulty in translating 3D boxes into 2D sequence of ROIs. So, that makes us rethink the way we designed
our TPN.

\section{ 3D anchors as 4k-dim vector}
In this approach, we set 3D anchors as 4k coordinates (k = 16 frames = sample duration). So a typical anchor is written as ($x_1, y_1, x'_1, y'_1, x_2, y_2, ...$)
where $x_1, y_1, x'_1, y'_1 $ are the coordinates for the 1st frame, $x_2, y_2, x'_2, y'_2$ are the coordinates for the 2nd frame etc as presented in \cite{DBLP:journals/corr/abs-1712-09184}.
In figure \ref{fig:anchor_4k} we can an example of this type of anchor.

\begin{figure}[h]
  \centering
  \includegraphics[scale=0.5]{anchor_4k}
  \caption{An example of the anchor $(x_1,y_1,x'_1,y'_1,x_2,y_2, ...)$}
  \label{fig:anchor_4k}
\end{figure}

The main advantage of this approach is that we don't need to translate the 3D anchors into 2D boxes. However, it has a big drawback, which is the fact that this type of anchors
has fixed time duration. In order to deal with this problem, we set anchors with different time durations, which are 16, 12, 8 and 4. Anchors with duration $ < $
sample duration (16 frames) can be written as 4k vector with zeroed coordinateds in the frames bigger that the time duration. For example, an anchor with
2 frames duration, starting from the 2nd frame and ending at the 3rd can be written as (0, 0, 0, 0, $x_1, y_1, x'_1, y'_1, x_2, y_2, x'_2, y'_2$, 0, 0, 0, 0) if sample
duration is 4 frames. 

\begin{figure}[h]
  \centering
  \includegraphics[width=1.\textwidth]{tpn_2}
  % \includegraphics[scale=0.35]{tpn_2}
  \caption{The structure of TPN according to new approach}
  \label{fig:New_structure}
\end{figure}

This new approach led us to change the structure of TPN. The new one can is presented in figure \ref{fig:New_structure}. As we can see, we \
added scoring and regression layers for each duration.

\subsection{Training}
In following figures we can see recall performance for sample duration = 16 when using max or avg pooling. 
% \begin{center}
%   % \begin{tabu} to 0.8\textwidth { | X[l] | X[c] | X[r] | }
%   \begin{tabular} { | c | c | c | c |}
%     \hline
%     \textbf{Duration} & \textbf{Type of Pooling} & \textbf{Dataset}  &  \textbf{Threshold } & \textbf{recall}\\
%     \hline
%     16  & Avg  & jHMDB  &  0.5 & \\
%     \hline
%     % \end{tabu}
%   \end{tabular}
% \end{center}
From the above results, we can see that using max pooling achieves better results.

\begin{table}[h]
  \centering
  \begin{tabular}{||c | c | c | c ||}
    \hline
    \textbf{Dataset} & \textbf{Pooling} &  \textbf{Recall(0.5)} \\
    \hline  \hline
    \multirow{2}{4em}{JHMDB} & mean &  TODO \\
    \cline{2-3}
    {} & max & TODO \\
    \hline
    \multirow{2}{4em}{UCF} & mean &  TODO \\
    \cline{2-3}
    {} & max & TODO \\
    \hline
  \end{tabular}
  \caption{Recall results after adding fixed time duration anchors}
  \label{table:tpn_2_1}
\end{table}

\subsection{Adding regressor}
In full correspondance with the previous approach, we added an regressor for trying to find better results. We 

\begin{table}[h]
  \centering
  \begin{tabular}{||c | c | c || c||}
    \textbf{Dataset} & \textbf{Pooling} & \textbf{Feature Map} & \textbf{Recall (0.5)} \\
    \hline
    \multirow{6}{*}{JHMDB} & \multirow{3}{*}{mean} & 64 & TODO \\
    \cline{3-4}
    {} & {} & 128 & TODO \\
    \cline{3-4}
    {} & {} & 256 & TODO \\
    \cline{2-4}
    {} & \multirow{3}{*}{max} & 64 & TODO \\
    \cline{3-4}
    {} & {} & 128 & TODO \\
    \cline{3-4}
    {} & {} & 256 & TODO \\
    \hline
    \multirow{6}{*}{UCF} & \multirow{3}{*}{mean} & 64 & TODO \\
    \cline{3-4}
    {} & {} & 128 & TODO \\
    \cline{3-4}
    {} & {} & 256 & TODO \\
    \cline{2-4}
    {} & \multirow{3}{*}{max} & 64 & TODO \\
    \cline{3-4}
    {} & {} & 128 & TODO \\
    \cline{3-4}
    {} & {} & 256 & TODO \\
    \hline


  \end{tabular}
  \caption{}
  \label{table:reg_2_1}
\end{table}

% \subsection{Changing Regressor - from 3D to 2d}
% After getting first recall results, we experiment using another architecture for the regressor network. Our idea was that we can treat feature maps like not having
% temporal dependencies between their frames. So, at first, we get from Roi Align operation activation maps with dimensions \textit{(k,64,16,7,7)} where k is the number
% of ToIs proposed for this video segment. We reshape this feature map to \textit{ (k*16,64,7,7) }, and in same time, we reshape their proposed action tubes from \textit{(k,16,4)}
% to \textit{(k*16,4)}. So, the new regression Network is consisted with:
% % \begin{enumerate}
% % \item a 2D Convolutional Network
% \paragraph{From 3D to 2d}
% The first idea we thought, was to change the first Convolutional layer from 3D to 2D. This means that we consider  features  not to have temporal dependencies for
% each frame. As we can see in the figure \ref{}, we got worse results, so, we rejected this idea.

% \subsection{Trying to  improve performance}
% TODO
% \subsection{Changing training procedure}
% TODO
\subsection{Changing sample duration}
After trying all the previous version, we noticed that we get about the same recall performances. So, we thought that we could try
to reduce the sample duration. On top of that, we trained our network for sample duration = 8 and 4 frames.

\begin{table}
  \centering
  \begin{tabular}{|c | c| c|}
    \hline
    \textbf{Dataset} & \textbf{Sample dur} & \textbf{Recall} \\
    \hline
    \multirow{2}{*}{JHMDB} & 8 & TODO \\
    \cline{2-3}
    {} & 4 & TODO \\
    \hline
    \multirow{2}{*}{UCF} & 8 & TODO \\
    \cline{2-3}
    {} & 4 & TODO \\
    \hline
    
  \end{tabular}
  \caption{}
  \label{table:reg_2_sample}
\end{table}
                                             
                                             
    

    


\end{document}


% \documentclass{report}

% \usepackage{subcaption} % package for subfigures
% \usepackage{hyperref}  % package for linking figures etc
% \usepackage{enumitem}  % package for description with bullets
% \usepackage{graphicx}  % package for importing images
% \usepackage{mathtools} % package for math equation
% \usepackage{mathrsfs}  % package for math font
% \usepackage{indentfirst} % package for getting ident after section or paragraph
% \usepackage{multirow}  % package for tables, multirow
% \usepackage{longtable} % package for multi pages tables


% \usepackage[export]{adjustbox}
% % \usepackage{amsmath}

% \setlength{\parindent}{2em} % how much indent to use when we start a paragraph

% \graphicspath{ {./theory/figures/} }       % path for images

% \begin{document}

\chapter{Connecting Tubes}
\section{Introduction}
In the previous chapter, we described methods for generating candidate action tubes given a small video segment lasting 8 or 16 frames. However, actual videos
and actual human actions, in the  wild, last more than 16 frames most of the times. Current networks are unable to process a whole video at once, in order to generate action tubes
due to memory and computing power issues.  As mentioned in chapter 2, a lot action localization approaches deal with this situation given a video, either by 
proposing candidate areas in the frame-level and then  connecting them in order to generate action tube proposal either, separating the video into video segments,
proposing  sequences of bounding boxes for each video segment and then linking them in order to generate action proposals. Both aforementioned techniques make
the suitable choice of linking method an important factor for the performance of the network. That's because, even though frame-level or video segment-level proposals
might be very good, if the proposed connection algorithm doesn't work well, final action tube proposals won't be efficient, so the final model will never
be able to achieve high classification performance. In other words, if connecting algorithm doesn't generate action proposals with great recall and MABO performance,
the model's classifier won't be able to perform suitable classification, because probably it would be given action tubes without any context.
In this chapter, we present 3 different approaches used for linking proposed ToIs generated from TPN in the previous chapter.

\section{First approach: combine overlap and actioness}
Our algorithm is inspired by \cite{DBLP:journals/corr/HouCS17}, which calculates all possible sequences of ToIs. In order to find the best candidate action tubes,
it uses a score, which tells us how likely a sequence of ToIs is  to contain an action. This score is a combination of 2 metrics:
\begin{description}
\item[ Actioness,  ] which is the ToI's possibility to contain an action. This score is produced by TPN's scoring layers.
\item [ ToIs' overlapping, ] which is the IoU of the last bounding boxes of the first ToI and the first frames of the second ToI.
\end{description}

The above scoring policy can be described by the following formula:
\[ S = \frac{1}{m} \sum_ {i=1}^{m} Actioness_i + \frac{1}{m-1} \sum_{j=1}^{m-1} Overlap_{j,j+1} \]

For every possible combination of ToIs we calculate their score as shown in figure \ref{fig:connection_algo}.

\begin{figure}[h]
  \centering
  \includegraphics[scale=0.225]{connection_algo}
  \caption{An example of calculating connection score for 3 random ToIs taken from \cite{DBLP:journals/corr/HouCS17}}
  \label{fig:connection_algo}
\end{figure}

The above approach, however, needs too much memory for all needed calculations, so a memory usage  problem is
appeared. The reason is, for every new video segment, we propose \textit{k ToIs} (16 during training and 150 during validation).
As a result, for a small video separated into  \textbf{10 segments}, we need to calculate 
\textbf{  150\textsuperscript{10} scores} during validation stage. This causes our system to overload and it takes too much time to process
just one video. \par

In order to deal with this problem, we create a greedy algorithm in order to find the candidates tubes. Intuitively, this algorithm after
getting  a new video segment, keeps tubes with a score higher than a threshold, and deletes the rest. So, we don't need to calculate combinations with a
very low score. We wrote code for calculating tubes' scores in CUDA language, which has the ability to
parallel process the same code using different data. Our algorithm is described below:

\begin{enumerate}
\item Firstly,  initialize empty lists for the final tubes, final tubes' duration, their scores, active tubes, their corresponding duration,
  active tubes' overlapping sum and actioness sum where:
  \begin{itemize}
  \item Final tubes list contains all tubes which are the most likely to contain an action, and their score list contains their
    corresponding scores. We refer to each tube by its index, which is related a tensor, in which we saved all the ToIs proposed
    from TPN for each video segment.
  \item Active tubes list contains all tubes that will be matched with the new ToIs. Their overlapping sum list and actioness sum list
    contain their sums in order to avoid calculating then for each loop. 
  \end{itemize}
Also, we initialize  connection threshold equal to 0.5 .
\item For the first video segment, we add all the ToIs to both active tubes and final tubes. Their scores are only their actioness because
  there are no tubes for calculating their overlapping score. So, we set their overlaping sum equal to 0.
\item For each next video, after getting the proposed ToIs, firstly we calculate their overlapping score with each active tube. Then, we
  empty active tubes, active tubes' duration, overlapping sum and actioness score lists.  For each new tube that has a score higher than the
  connection threshold,  we add both to final tubes and to active tubes and their corresponding lists, and we increase their duration.
\item If the number of active tubes is higher than a threshold, we set connection threshold equal to the score of
  the 100th higher score. On top of that, we update the final tubes list, removing all tubes that have score lower than the threshold.
\item After that, we add in active tubes, the current video segment's proposed ToIs, alongside with their actioness scores in the  actioness sum list and
  zero values in corresponding positions in the overlaps sum list (such as in the 1st step).
\item We repeat the previous 3 steps until there is no video segment left.
\item Finally, as we mentioned before, we have a list which contains the indexes of the saved tubes. So, we modify them in order to have
  the corresponding bounding boxes. However, 2 succeeding ToIs do not, always, have exactly the same bounding boxes in the frames that overlap. For example,
  ToIs from the $1^{st}$ video segment start from frame 1 to frame 16. If we have video step equal to 8, these ToIs overlap temporally
  with the ToIs from the  succeeding video segment in frames 8-16. In those frames, in final tube, we choose the area that contains both bounding boxes which is
  denoted as $(min(x_1,x'_1), min(y_1,y'_1), max(x_2,x'_2), max(y_2,y'_2))$ for bounding boxes $(x_1,y_1,x_2,y_2)$ and $(x'_1,y'_1,x'_2,y'_2)$.
\end{enumerate}
% We implement this algorithm using CUDA language for counting the scores. In \cite{DBLP:journals/corr/HouCS17}, they use temporal \textit{stride = sample duration} during testing. We want 
\subsection{JHMDB Dataset}

In order to validate our algorithm, we firstly experiment in JHMDB dataset's videos in order to define the best overlapping policy and
the video overlapping step. Again, we use recall as evaluation metric. A groundtruth action tube is considered to be found, as well as positive,
if there is at least 1 video tube which overlaps with it over a predefined threshold, otherwise it is considered not found.  These thresholds are again 0.5, 0.4 and 0.3.
We set TPN to return 30 ToIs per video segment.
We chose to update threshold when active tubes are more than 500 and to keep the first 100 tubes as active. We did so, because, a big part of the
code is performing in the CPU. That's because, we use lists, which are very easy to handle for adding and removing elements. So, if we use bigger update
limits, it takes much more time to process them.

\paragraph{Sample duration equal to 16 frames} At first, we use as sample duration = 16 frames and video step = 8. As overlapping frames we count frames
\textit{(8...15)} so we have overlapping scores from \#8 frames, which we calculate their average value. Also, we use only \#4 frames with combinations \textit{(8...11), (10...13) and (12...15)} and 
\#2 frames with combinations \textit{(8,9), (10,11), (12,13), and (14,15)}. The results are shown in Table \ref{table:step8_16} (in bold are
the frames with which we calculate the overlap score).

\newpage
\begin{center}
\begin{longtable}{||c||c c c||}

  \hline
  \multirow{2}{5em}{combination} & {} &overlap thresh & {} \\
                                    &  0.5  &  0.4 &  0.3 \\         
  \hline  \hline
  0,1,...,\textbf{\{8,...,15\}}               & {} & {} & {} \\
  \textbf{\{8,9,...,15\}},16,...,23           & 0.3172 & 0.4142 & 0.6418 \\
  \hline     \hline                          


  0,1,...,\textbf{\{8,...,11,\}}...,14,15     & {} & {} & {} \\
  \textbf{\{8,...,11\}},12,...,22,23          & 0.3172 & 0.4142& 0.6381 \\
  \hline
  0,1,...,\textbf{\{10,...,13,\}}14,15,       & {} & {} & {} \\
  8,9,\textbf{\{10,...,13\}},14,...,22,23     & 0.3209 &0.4179   & 0.6418 \\
  \hline
  0,1,...,\textbf{\{12,...,15,\}}             & {} & {} & {} \\
  8,9,...,\textbf{\{12,...,15\}},16,...,23,   & 0.3284 & 0.4216 & 0.6381 \\
  \hline     \hline                          

  0,1,...,\textbf{\{8,...,11,\}},...,14,15,   & {} & {} & {} \\
  \textbf{\{8,9,...,11,\}}12,...,22,23        & 0.3172   & 0.4142& 0.6381 \\
  \hline
  0,1,...,\textbf{\{10,...,13,\}}14,15,       & {} & {} & {} \\
  \textbf{\{10,...,13\}},14,...,22,23         & 0.3209 &0.4179   & 0.6418 \\
  \hline
  0,1,...,\textbf{\{12,...,15\}}              & {} & {} & {} \\
  8,9,...,\textbf{\{12,...,15\}},16,...       & 0.3284 & 0.4216 & 0.6381 \\
  \hline \hline
  
  0,1,...,\textbf{\{8,9,\}},10,...,14,15,     & {} & {} & {} \\
  \textbf{\{8,9,\}}10,11,...,22,23            & 0.3134   & 0.4104 & 0.6381 \\
  \hline
  0,1,...,\textbf{\{10,11,\}},12,...,14,15,   & {} & {} & {} \\
  8,9,\textbf{\{10,11,\}}12,...,22,23         & 0.3209   & 0.4216 & 0.6418 \\
  \hline
  0,1,...,\textbf{\{12,13,\}},14,15,          & {} & {} & {} \\
  8,9,...,\textbf{\{12,13,\}}14,...,22,23     & 0.3246   & 0.4179 & 0.6418 \\
  \hline
  0,1,...,13,\textbf{\{14,15,\}}              & {} & {} & {} \\
  8,9,...,\textbf{\{14,15,\}}16,...,22,23     & 0.3321   & 0.4216 & 0.6306 \\
  \hline \hline
  \caption{Recall results for step = 8}
  \label{table:step8_16}
\end{longtable} 
\end{center}

As we can from the above table, generally we get very bad performance and we got the best performance when we calculate the overlap between only 2 frames (either \textit{14,15} or \textit{12,13}).
So, we thought that we should increase the video step because, probably, the connection algorithm is too strict into big movement variations during  the video. As a result, we set video step = 12 which
means that we have only 4 frames overlap. In this case,  for \#4 frames, we only have the combination \textit{(12...15)}, for \#2 frames we have \textit{(12,13), (13,14) and (14,15)} as shown in
Table \ref{table:step12_16}.

\begin{center}
\begin{longtable}{||c||c c c||}

  \hline
  \multirow{2}{5em}{combination} & {} &overlap thresh & {} \\
                                    &  0.5  &  0.4 &  0.3 \\         
  \hline  \hline
  0,1,...,11,\textbf{\{12,...,15\}}           & {} & {} & {} \\
  \textbf{\{12,13,...,15\}},16,...,26,27         & 0.3769 & 0.4627 & 0.6828 \\
  \hline     \hline                          

  0,1,...,\textbf{\{12,13,\}},14,15,          & {} & {} & {} \\
  \textbf{\{12,13,\}}14,15,...,26,27          & 0.3694   & 0.4627 & 0.6903 \\
  \hline                          
  0,1,...,12\textbf{\{13,14,\}},15,           & {} & {} & {} \\
  12,\textbf{\{13,14,\}}15,...,26,27          & 0.3843   & 0.4627 & 0.6828 \\
  \hline                          
  0,1,...,12,13\textbf{\{14,15,\}}            & {} & {} & {} \\
  12,13,\textbf{\{14,15,\}}16,...,26,27       & 0.3694   & 0.459 & 0.6828 \\
  \hline     \hline                          

  \caption{Recall results for step = 12}
  \label{table:step12_16}
\end{longtable} 
\end{center}

As we can see, recall performance is increased so that means that our assumption was correct. So again, we increase video step into 14, 15 and 16 frames
and recall score is shown in Table \ref{table:step14_16}
\begin{center}
\begin{longtable}{||c||c c c||}

  \hline
  \multirow{2}{5em}{combination} & {} &overlap thresh & {} \\
                                    &  0.5  &  0.4 &  0.3 \\         
  \hline  \hline
  0,1,...,13\textbf{\{14,15\}}                & {} & {} & {} \\
  \textbf{\{14,15\}},16,...,28,29                & 0.3731 & 0.5336 & 0.6493 \\
  \hline     \hline                          

  0,1,...,13,\textbf{\{14,\}}15,              & {} & {} & {} \\
  \textbf{\{14,\}}15,...,28,29                & 0.3694   & 0.5299 & 0.6455 \\
  \hline                          
  0,1,...,14,\textbf{\{15\}}                  & {} & {} & {} \\
  14,\textbf{\{15,\}}16,...,28,29             & 0.3731   & 0.5187 & 0.6381 \\
  \hline  \hline

  0,1,...,14,\textbf{\{15\}}                & {} & {} & {} \\
  \textbf{\{15\}},16,...,30                 & 0.3918 & 0.5187 & 0.6381 \\
  \hline     \hline                          
  0,1,...,14,\textbf{\{15\}}                & {} & {} & {} \\
  \textbf{\{16\}},17,...,31                 & 0.4067 & 0.7313 & 0.8731 \\
  \hline                          
  \caption{Recall results for steps = 14, 15 and 16}
  \label{table:step14_16}
\end{longtable} 
\end{center}

The results show that we get the best recall performance when we have no overlapping steps and video step = 16 = sample duration. We try to improve
more our results, using smaller duration because, as we saw from TPN recall performance, we get better results when we have sample duration = 8 or 4.

\paragraph{Sample duration equal to  8}
We wanted to confirm that we get the best results, when we have no overlapping frames and step = sample duration. So Table \ref{table:step4_8}
shows recall performance for sample duration = 8 and video step = 4 and Table \ref{table:step8_678 } for video steps = 6, 7 and 8.


\begin{center}
\begin{longtable}{||c||c c c||}

  \hline
  \multirow{2}{5em}{combination} & {} &overlap thresh & {} \\
                                    &  0.5  &  0.4 &  0.3 \\         
  \hline  \hline
  0,1,2,3,13\textbf{\{4,5,6,7\}}                & {} & {} & {} \\
  \textbf{\{4,5,6,7\}},8,9,10,11                & 0.2015   & 0.3582 & 0.5858 \\
  \hline     \hline                          

  0,1,2,3,\textbf{\{4,5,\}}6,7                  & {} & {} & {} \\
  \textbf{\{4,5,\}}6,7,8,9,10,11                & 0.1978   & 0.3582 & 0.5933 \\
  \hline                          
  0,1,2,3,4\textbf{\{5,6,\}}7                   & {} & {} & {} \\
  4,\textbf{\{5,6,\}}7,8,9,10,11                & 0.1978   & 0.3507 & 0.5821 \\
  \hline                          
  0,1,2,3,4,5\textbf{\{6,7\}}                   & {} & {} & {} \\
  4,5,\textbf{\{6,7,\}}8,9,10,11                & 0.194   & 0.3433 & 0.585 \\
  \hline                           
  \caption{Recall results for step = 4}
  \label{table:step4_8}
\end{longtable} 
\end{center}

\begin{center}
\begin{longtable}{||c||c c c||}

  \hline
  \multirow{2}{5em}{combination} & {} &overlap thresh & {} \\
                                    &   0.5  &  0.4 &  0.3 \\         
  \hline  \hline
  0,1,2,3,4,5\textbf{\{6,7\}}           & {} & {} & {} \\
  \textbf{\{6,7\}},8,9,10,11,12,13      & 0.3134  & 0.7015 & 0.8619 \\
  \hline     \hline                          

  0,1,2,3,4,5,\textbf{\{6,\}}7          & {} & {} & {} \\
  \textbf{\{6,\}}7,8,9,10,11,12,13      & 0.3209  & 0.6679 & 0.847 \\
  \hline                          
  0,1,2,3,4,5,6,\textbf{\{7\}}          & {} & {} & {} \\
  6,\textbf{\{7\}}8,9,10,11,12,13       & 0.3172  & 0.6567 & 0.8507 \\
  \hline                          
  0,1,2,3,4,5,6\textbf{\{7\}}           & {} & {} & {} \\
  \textbf{\{7,\}}8,9,10,11,12,13,14     & 0.5597  & 0.7687 & 0.903 \\
  \hline                           
  0,1,2,3,4,5,6\textbf{\{7\}}           & {} & {} & {} \\
  \textbf{\{8\}}9,10,11,12,13,14,15     & 0.653	  & 0.8396 &0.9179  \\
  \hline                           
  \caption{Recall results for steps = 6, 7 and 8}
  \label{table:step8_678 }
\end{longtable} 
\end{center}

According to Tables \ref{table:step4_8} and \ref{table:step8_678 }, it is clearly shown that, we achieve the  best results, for $step = sample duration$ and overlapping scores is calculated between the last box of the current tubes and the first box of next tubes.

\subsection{UCF-101 dataset}
In previous steps, we tried to find the best overlap policy for our algorithm in JHMDB dataset. After that, it's time to apply our algorithm in UCF-101 dataset using the best scoring
overlap policy. We did some modifications in the code, in order to use less memory and we moved most parts of the code to the GPU. This happened by using tensors instead of lists for scores while
most operations are, from now on, matrix operations. On top that, the last step of the algorithm, which is the modification from indices to actual action tubes is written, now, in CUDA code so
it takes place in the GPU, too. So, we are now able to increase the number of ToIs returned by TPN, the max number of active tubes before updating threshold and the max number of final
tubes. \par
The first experiments we performed were related to the number of the final tubes, our network proposes alongside with TPN's proposed
tubes' number. We experiment for cases, in which TPN proposes 30, 100 and 150 ToIs, our final network proposes 500, 2000 and 4000
candidate action tubes for sample durations equal to 8 and 16 frames.
For the sample duration equal to 8 we return 100 ToIs because, when we tried to return 150 proposed ToIs, we got OutOfMemory error.
Table \ref{table:ucf_recall} show the spatiotemporal recall and MABO performance of those approaches. Furthermore, Table \ref{table:ucf_temp_recall } show their temporal recall and MABO performance. We are interested in temporal performance, because UCF-101 consists of
untrimmed videos, unlike JHMDB which has only trimmed videos. So, we want to know how well our network is able to propose action tubes that
overlap temporally with the groundtruth action tubes over a ``big'' threshold. For temporal localization, we don't use 0.5, 0.4 and 0.3
overlapping threshold, but instead, we use 0.9, 0.8 and 0.7, because it is very important our network to be able to propose tubes that
 contain an action, at least from the temporal perspective. In order to calculate the temporal overlap, we use IoU for 1 dimension as described before.

\begin{center}
\begin{longtable}{||c | c | c ||c c c | c|}

  \hline
  \multirow{2}{*}{combination} & \multirow{2}{2.5em}{TPN tubes} & \multirow{2}{2.5em}{Final tubes} &  {} & overlap thresh & {} & \multirow{2}{*}{MABO} \\
  {} & {} & {} &  0.5 &  0.4 & 0.3 & {}\\         
  \hline
  
  \multirow{6}{7em}{0,1,...,6,\textbf{\{7,\}}
  \textbf{\{8,\}}9,...,14,15 }  & \multirow{3}{*}{30} & 500   & 0.2829  & 0.4395 & 0.5817  & 0.3501 \\
  \cline{3-7}
  {} &  {}   & 2000   & 0.3567  & 0.4996 & 0.6289 & 0.3815\\
  \cline{3-7}
  {} &  {}   & 4000   & 0.3749  & 0.5316 & 0.6487 & 0.3934 \\
  \cline{2-7}
  {} &  \multirow{3}{*}{100}   & 500   & 0.2966  & 0.451 & 0.5947 & 0.356 \\
  \cline{3-7}
  {} &  {}   & 2000   & 0.3757  & 0.5163 & 0.6471 & 0.3902 \\
  \cline{3-7}
  {} &  {}   & 4000  & 0.3977  & 0.5506 & 0.6624 & 0.4029 \\
  \hline                                    
  \multirow{6}{7em}{0,1,...,14,\textbf{\{15,\}}
  \textbf{\{16,\}}17,18,...,23 }  & \multirow{3}{*}{30} & 500   & 0.362  & 0.5042 & 0.6243 & 0.3866 \\
  \cline{3-7}
  {} &  {}   & 2000   & 0.416  & 0.5468 & 0.6631 & 0.4108  \\
  \cline{3-7}
  {} &  {}   & 4000   & 0.4281  & 0.5589  & 0.6779 & 0.4182 \\
  \cline{2-7}
  {} &  \multirow{3}{*}{150}   & 500 & 0.3589  & 0.4981 & 0.6198 & 0.3845 \\
  \cline{3-7}
  {} &  {}   & 2000   & 0.4129  & 0.5392  & 0.6563 & 0.4085 \\
  \cline{3-7}
  {} &  {}   & 4000   & 0.4266  & 0.5521 & 0.6722 & 0.4162\\
  \hline                                    

  \caption{Recall results for UCF-101 dataset}
  \label{table:ucf_recall}
\end{longtable} 
\end{center}

\begin{center}
\begin{longtable}{||c | c | c ||c c c| c|}

% \begin{table}
%   \centering
%   \setlength{\tabcolsep}{4pt}
%   \begin{tabular}{||c | c | c ||c c c| c|}

  \hline
  \multirow{2}{*}{combination} & \multirow{2}{2.5em}{TPN tubes} & \multirow{2}{2.5em}{Final tubes} &  {} &overlap thresh & {} & \multirow{2}{*}{MABO} \\
  {} & {} & {} &  0.9 &  0.8 & 0.7 & {}\\         
  \hline
  
  
  \multirow{6}{7em}{0,1,...,6,\textbf{\{7,\}}
  \textbf{\{8,\}}9,...,15 }  & \multirow{3}{*}{30} & 500   & 0.4464  & 0.581 & 0.6844  & 0.7787 \\
  \cline{3-7}
  {} &  {}   & 2000   & 0.635  & 0.7665 & 0.8403 & 0.8693 \\
  \cline{3-7}
  {} &  {}   & 4000   & 0.7034  & 0.8228 & 0.8875 & 0.8973 \\
  \cline{2-7}
  {} &  \multirow{3}{*}{100}   & 500   & 0.454 & 0.5924 & 0.692 & 0.783 \\
  \cline{3-7}
  {} &  {}   & 2000   & 0.651 & 0.7696 & 0.8441 &0.8734 \\
  \cline{3-7}
  {} &  {}   & 4000   & 0.7209 &0.8312 & 0.8913 & 0.9026 \\

  \hline                                    
  \multirow{6}{7em}{0,1,...,14,\textbf{\{15,\}}
  \textbf{\{16,\}}17,18,...,23 }  & \multirow{3}{*}{30} & 500   & 0.6844 &0.8327 & 0.9027 & 0.8992 \\
  \cline{3-7}
                                    {} &  {}   & 2000   & 0.7475 & 0.8684 & 0.9217 & 0.9175 \\
  \cline{3-7}
                                    {} &  {}   & 4000   & 0.7567  & 0.8745  & 0.9255 & 0.9211 \\
  \cline{2-7}
                                    {} &  \multirow{3}{*}{150}   & 500   & 0.7498 &0.8707 &0.9171 & 0.9125 \\
  \cline{3-7}
                                    {} &  {}   & 2000   & 0.8243 & 0.911 & 0.9392 & 0.9342\\
  \cline{3-7}
                                    {} &  {}   & 4000   &  0.8403  & 0.9179 & 0.9437 & 0.9389\\
  \hline                                    
  % \end{tabular}
  \caption{Temporal Recall results for UCF-101 dataset}
  \label{table:ucf_temp_recall }
% \end{table}
\end{longtable} 
\end{center}

According to Table \ref{table:ucf_recall}, we achieve better recall and MABO performance when we set sample duration equal to 16.
In all cases,  recall performance of simulations with sample duration equal to  16 outweight the corresponding with 8, with the difference
varying from 2\% to 8\%. In addition, we get best recall and MABO performance when our system proposes 4000 tubes. As we can see,
the ratio of good proposals increases about 5\%-7\% when we change number of proposed tubes from 500 to 2000. This ratio increases more
when we double returned action tubes, from 2000 to 4000. However, this increase is only about 1\%-2\%, which make us rethink if this increase
is worth to be performed. That's because, this modification increases memory usage, because of 4000 proposed action tubes, instead of
2000. Finally, Table \ref{table:ucf_recall} shows that, for sample duration = 8, changing the number of ToIs produced by TPN, slightly
helps our network to achieve better results. This contribution is measured about 1\%-2\%.
On the contrary, when we set sample duration equal to 16, it slightly reduces network's performance. Taking all the aforementioned
results into account, we think that the most suitable choices for connection approaches are, for the sample duration equal to 8, the one in
which TPN returns 100 ToIs and our network proposes 4000 action tubes, and for the sample duration equal to 16, the one in which,
TPN returns 30 ToIs and the network 4000 action tubes. \par
Additionally, Table \ref{table:ucf_temp_recall } shows some interesting facts, too. At first, it confirms that increasing the number of proposed
action tubes, from 500 to 4000, increases recall and MABO performance. Also, we get better results when network has 16 frames as sample
duration, too. However, unlike Table \ref{table:ucf_recall}, Table \ref{table:ucf_temp_recall } shows that when we increase TPN's number
of proposed ToIs, it increases performances for both sample durations. For sample duration equal to 8, this increase results in improving
recall performances by 2\% and MABO performance by 1\% like spatiotemporal recall and MABO. For the sample duration equal to 16, recall
performance is increasing by about 8\% and MABO by 1\%-2\%.  \par
Taking both tables into consideration, we think that the best approach is TPN returning 30 proposed ToIs, network returning 4000 proposed
action tubes and sample duration equals with 16. We didn't choose TPN returning 150 proposed ToIs because, based on MABO performances,
they different only by 1\%, difference which is insignificant.


\subsubsection{Adding NMS algorithm}

Previous section describes the performances of network's proposals for variations in the number of  TPN's  returned ToIs, number of returned
proposed action tubes and sample duration. For each situation, we choose the k-best scoring action tubes, without taking into account
any relation between these action tubes, like their spatiotemporal overlap. So, like TPN's approach, we thought that we should apply
nms algorithm before choosing k-best scoring tubes, in order to further improve  spatiotemporal and temporal, recall and MABO  performance.
We experiment using again two sample durations, 16 and 8 frames per video segment, number or TPN's returning tubes equal to 30 and the
number of final picked action tubes equal to 4000. NMS algorithm uses a threshold in order to choose if 2 action tubes overlap enough. We
experiment setting this threshold equal to 0.7 and 0.8 and  the results are shown in Table \ref{table:ucf_nms_recall} for Spatiotemporal
performance and at Table \ref{table:ucf_nms_temp_recall} for temporal performance.

\begin{center}
  \setlength{\tabcolsep}{2pt}
\begin{longtable}{||c | c | c | c c c| c|}

  \hline
  \multirow{2}{*}{combination} & \multirow{2}{2.5em}{NMS thresh} & \multirow{2}{3.5em}{PreNMS tubes} &  {} &overlap thresh & {} & \multirow{2}{*}{MABO} \\
  {} & {} & {} &  0.5 &  0.4 & 0.3 & {}\\         
  \hline
  \multirow{3}{7em}{0,1,...,6,\textbf{\{7,\}}
  \textbf{\{8,\}}9,...,15 }  & 0.7 &\multirow{3}{*}{20000}  & 0.346 & 0.5202 & 0.657 & 0.3824685269 \\
  \cline{2-2} \cline{4-7} 
  {} &  0.8   & {}   & 0.3643 & 0.5392 & 0.6578 & 0.3904727407 \\
  \cline{2-2} \cline{4-7} 
  {} &  0.9   & {}   & 0.397  & 0.5574 & 0.6677 & 0.4031543642 \\
  \hline                                    
  \multirow{3}{7em}{0,1,...,14,\textbf{\{15,\}}
  \textbf{\{16,\}}17,...,23 }  & 0.7 & \multirow{3}{*}{20000}   & 0.3939 & 0.5559  & 0.6882 & 0.404689056 \\
  \cline{2-2} \cline{4-7} 
                                    {} &  0.8   & {}   & 0.4259 & 0.5764 & 0.6981 & 0.419487652 \\
  \cline{2-2} \cline{4-7} 
                                    {} &  0.9   & {}   & 0.4494 & 0.5856 & 0.7019 & 0.4302611039 \\

  \hline                                    

  \caption{Spatiotemporal Recall results for UCF-101 dataset}
  \label{table:ucf_nms_recall}
\end{longtable} 
\end{center}

\newpage
\begin{center}
  \setlength{\tabcolsep}{2.2pt}
\begin{longtable}{||c | c | c | c c c| c|}

  \hline
  \multirow{2}{*}{combination} & \multirow{2}{2.5em}{NMS thresh} & \multirow{2}{3.5em}{PreNMS tubes} &  {} &overlap thresh & {} & \multirow{2}{*}{MABO} \\
  {} & {} & {} &  0.9 &  0.8 & 0.7 & {}\\         
  \hline
  \multirow{3}{7em}{0,1,...,6,\textbf{\{7,\}}
  \textbf{\{8,\}}9,...,15 }  & 0.7 &\multirow{3}{*}{20000}  & 0.6281 & 0.8251 & 0.9027 & 0.8885141223  \\
  \cline{2-2} \cline{4-7} 
  {} &  0.8   & {}   & 0.7369 & 0.8616 & 0.9148 & 0.9106069806 \\
  \cline{2-2} \cline{4-7} 
  {} &  0.9   & {}   &  0.7787 & 0.8753 & 0.9209 & 0.9212593589 \\
  \hline                                    
  \multirow{3}{7em}{0,1,...,14,\textbf{\{15,\}}
  \textbf{\{16,\}}17,...,23 }  & 0.7 & \multirow{3}{*}{20000}   & 0.7452 & 0.8920 & 0.9361 & 0.920331595 \\
  \cline{2-2} \cline{4-7} 
  {} &  0.8   & {}   & 0.8160 & 0.9278 & 0.9506 & 0.93612757 \\
  \cline{2-2} \cline{4-7} 
  {} &  0.9   & {}   & 0.854 & 0.9346 & 0.9529 & 0.9434986107 \\
  \hline                                    

  \caption{Temporal Recall results for UCF-101 dataset}
  \label{table:ucf_nms_temp_recall}
\end{longtable} 
\end{center}

Comparing Table \ref{table:ucf_nms_recall} with Table \ref{table:ucf_recall},  we notice that NMS algorithm improves recall and MABO
performance when NMS threshold is equal to 0.9. When we set it equal to 0.7 or 0.8, we get worse results. This happens, probably, because
nms algorithm removes some good proposals. Comparing these results with results presented at Tables \ref{table:ucf_recall} and \ref{table:ucf_temp_recall } it becomes clear that using NMS algorithm is very useful. That's because, even though we get the same number of proposed action tubes,
these tubes are not very close spatiotemporally, so this makes proposed action tubes more likely to contain actual foreground action tubes.

\subsubsection{Stop updating threshold}

In previous approaches, scoring threshold was updated each time our algorithm gathered a significant number of ``active'' tubes in order not
to add action tubes with score below this score. However, after serious consideration, we came to the conclusion that sometimes, the updated
threshold leads to not detecting action tubes that start after some frames. That's because, until then, linking threshold may be too big that
won't let new action tubes to be created. So, we came with the modification of not updating linking threshold, but just filtering proposed
tubes, by keeping k-best scoring each time their number is bigger that the a specific number. The rest algorithm remains the same. Tables
\ref{table:ucf_nms_noup_recall} and \ref{table:ucf_nms_noup_temp_recall} show spatiotemporal and temporal recall and MABO performance respectively.
We experiment for cases in which either we don't use the  NMS algorithm at all, either we set overlap threshold equal to 0.7 and 0.9 as shown
below.

\begin{center}
  \setlength{\tabcolsep}{2pt}
\begin{longtable}{||c | c | c | c c c| c|}

  \hline
  \multirow{2}{*}{combination} & \multirow{2}{2.5em}{NMS thresh} & \multirow{2}{3.5em}{PreNMS tubes} &  {} &overlap thresh & {} & \multirow{2}{*}{MABO} \\
  {} & {} & {} &  0.5 &  0.4 & 0.3 & {}\\         
  \hline
  \multirow{3}{7em}{0,1,...,6,\textbf{\{7,\}}
  \textbf{\{8,\}}9,...,15 }   &   \multicolumn{2}{|c|}{-}     &  0.3779 & 0.5316 & 0.6471 & 0.393082961 \\
  \cline{2-7}
  {} & 0.7 &\multirow{2}{*}{20000}  & 0.3483  & 0.5194 & 0.6471 & 0.3783524086 \\
  \cline{2-2} \cline{4-7} 
  {} &  0.9   & {}   & 0.416 & 0.5605 & 0.6722 & 0.4074053106 \\
  \hline                                    
  \multirow{3}{7em}{0,1,...,14,\textbf{\{15,\}}
  \textbf{\{16,\}}17,...,23 }  &   \multicolumn{2}{|c|}{-} & 0.438 & 0.5635 & 0.6829 & 0.4231788 \\
  \cline{2-7}
  {} & 0.7 & \multirow{2}{*}{20000}   & 0.4525 & 0.5848 & 0.7034 & 0.429747438 \\
  \cline{2-2} \cline{4-7} 
  {} &  0.9   & {}   & 0.3802 & 0.5133 & 0.6068 & 0.3862278851848662 \\

  \hline                                    

  \caption{Spatiotemporal Recall results for UCF-101 dataset}
  \label{table:ucf_nms_noup_recall}
\end{longtable} 
\end{center}

\begin{center}
  \setlength{\tabcolsep}{2.2pt}
\begin{longtable}{||c | c | c | c c c| c|}

  \hline
  \multirow{2}{*}{combination} & \multirow{2}{2.5em}{NMS thresh} & \multirow{2}{3.5em}{PreNMS tubes} &  {} &overlap thresh & {} & \multirow{2}{*}{MABO} \\
  {} & {} & {} &  0.9 &  0.8 & 0.7 & {}\\         
  \hline

  \multirow{3}{7em}{0,1,...,6,\textbf{\{7,\}}
    \textbf{\{8,\}}9,...,15 }  &   -   & -    & 0.7087 & 0.8281 & 0.8913 & 0.899210587 \\
  \cline{2-7} 
  {} & 0.7 &\multirow{2}{*}{20000}  & 0.6586 & 0.854 & 0.9278 & 0.903373468 \\
  \cline{2-2} \cline{4-7} 
  {} &  0.9   & {}   &  0.8137 & 0.8973 & 0.9361 & 0.9333068498 \\
  \hline                                    
  \multirow{3}{7em}{0,1,...,14,\textbf{\{15,\}}
  \textbf{\{16,\}}17,...,23 }  &   \multicolumn{2}{|c|}{-} & 0.8327 & 0.9156 &0.9399 & 0.940143272 \\
  \cline{2-7}
  {} & 0.7 & \multirow{2}{*}{20000}& 0.8646 & 0.9369 & 0.9567 & 0.946701832 \\
  \cline{2-2} \cline{4-7} 
  {} &  0.9   & {}   & 0.6183 & 0.7696 & 0.8388 & 0.8628507037919737 \\
  \hline                                    

  \caption{Temporal Recall results for UCF-101 dataset}
  \label{table:ucf_nms_noup_temp_recall}
\end{longtable} 
\end{center}

Comparing recall and MABO performances shown at Table \ref{table:ucf_nms_noup_recall} with those included in Tables \ref{table:ucf_nms_recall}
and \ref{table:ucf_recall}, we deduce that for the sample duration equal to 8, stopping updating linking threshold results in worse performance
when we set nms threshold equal to 0.7, but it achieves the best performances  when setting NMS threshold equal to 0.9 . Furthermore, for sample duration
equal to 16, we get, now,  best performance when we set nms threshold equal to 0.7 and worse performance for nms threshold equal to 0.9 .

\subsubsection{Soft-nms instead of nms}

After widely experiment using NMS algorithm, we thought that we should try to use Soft-NMS algorithm, introduced by \cite{DBLP:journals/corr/BodlaSCD17} and described in chapter 2. We implement our own soft-nms algorithm modifying it in order to calculate spatiotemporal overlapping
scores, and not just spatial, like the one implemented by \cite{DBLP:journals/corr/BodlaSCD17}. As mentioned before, instead of removing action tubes, Soft-NMS algorithm just reduces their score for those which overlap over a predefined threshol. We experiment for the sample duration
equal to 8 and thresholds equal to 0.7 and 0.9, because, our implementation ran out of memory for sample duration equal to 16.
Recall and MABO performance are presented in Tables \ref{table:ucf_softnms_recall} and \ref{table:ucf_softnms_temp_recall}

\begin{center}
  \setlength{\tabcolsep}{2pt}
\begin{longtable}{||c | c | c | c c c| c|}

  \hline
  \multirow{2}{*}{combination} & \multirow{2}{2.5em}{NMS thresh} & \multirow{2}{3.5em}{PreNMS tubes} &  {} &overlap thresh & {} & \multirow{2}{*}{MABO} \\
  {} & {} & {} &  0.9 &  0.8 & 0.7 & {}\\         
  \hline
  \multirow{2}{7em}{0,1,...,6,\textbf{\{7,\}}
  \textbf{\{8,\}}9,...,15 }  & 0.7 &\multirow{2}{*}{20000}  & 0.3916 & 0.5384 & 0.6464 & 0.3964639 \\
  \cline{2-2} \cline{4-7} 
  {} &  0.9   & {}   & 0.4023 & 0.5430 & 0.6502 & 0.398845313 \\
  \hline                                    

  \caption{Spatiotemporal Recall results for UCF-101 dataset using Soft-NMS}
  \label{table:ucf_softnms_recall}
\end{longtable} 
\end{center}

\begin{center}
  \setlength{\tabcolsep}{2.2pt}
\begin{longtable}{||c | c | c | c c c| c|}

  \hline
  \multirow{2}{*}{combination} & \multirow{2}{2.5em}{NMS thresh} & \multirow{2}{3.5em}{PreNMS tubes} &  {} &overlap thresh & {} & \multirow{2}{*}{MABO} \\
  {} & {} & {} &  0.9 &  0.8 & 0.7 & {}\\         
  \hline

  \multirow{2}{7em}{0,1,...,6,\textbf{\{7,\}}
    \textbf{\{8,\}}9,...,15 }  & 0.7 &\multirow{2}{*}{20000}  & 0.7521 & 0.8586 & 0.9110 & 0.915746097  \\
  \cline{2-2} \cline{4-7} 
  {} &  0.9   & {}   & 0.7741 & 0.8768 & 0.9255 & 0.922677864 \\
  \hline                                    

  \caption{Temporal Recall results for UCF-101 dataset using SoftNMS}
  \label{table:ucf_softnms_temp_recall}
\end{longtable} 
\end{center}

Taking results at Tables \ref{table:ucf_softnms_recall} and \ref{table:ucf_softnms_temp_recall} into consideration, alongside with those
at Tables \ref{table:ucf_nms_noup_recall} and \ref{table:ucf_nms_noup_temp_recall} for sample duration equal to 8, we notice that
using softNMS results in slightly better results. This happens when we set overlapping threshold equal to 0.9, otherwise, for
overlapping threshold 0.7, we get worst performance. Despite the fact that softnms results in better recall and MABO performance,
our implementation is very slow, which means that for a 201-frame video, softNMS part lasts about 32 seconds on the contrary with
standard NMS algorithm without updating linking threshold, in which this part last only 2 seconds. So, we choose to use the standard NMS
algorithm without updating liking threshold as an algorithm for  removing overlapping action tubes.

\section{Second approach: use progression and progress rate}
As we saw before, our first connecting algorithm doesn't have very good recall results. So, we created another algorithm which is based on
the one proposed by \cite{DBLP:journals/corr/abs-1903-00304}. This
algorithm introduces two new metrics according to \cite{DBLP:journals/corr/abs-1903-00304}:

% TODO add more description
\begin{description}
\item[ Progression,  ] which describes the probability of a specific action being performed in the ToI. 
  We add this factor because we have noticed that actioness is tolerant to false positives. Progression is
  mainly a rescoring mechanism for each class (as mentioned in \cite{DBLP:journals/corr/abs-1903-00304})

\item [ Progress rate, ] which is defined as the progress proportion that each action class has been performed.
  
\end{description}

So, each action tube is described as a set of ToIs
\[  T = {\{ {\bf t}_i^{(k)} | {\bf t}_i^{(k)} = ( t_i^{(k)}, s_i^{(k)}, r_i^{(k)} ) \}}_{i=1:n^{(k)},k=1:K} \]
where $ t_i^{(k)} $ contains ToI's spatiotemporal information, $ s_i^{(k)} $ its confidence score and $ r_i^{(k)} $ its progress rate.

In this approach, each class is handled separately, so for the rest section, we discuss action tube generation for one class only. In order to link 2 ToIs, for
a video with N video segments, the following steps are applied:
\begin{enumerate}
\item For the first video segment (k = 1), initialize an array with the M best scoring ToIs, which will be considered as active action tubes ( AT ).
  Correspondingly, initialize an array with M progress rates  and M confidence scores.
\item For k = 2:N, execute (a) to (c) steps:
  \begin{enumerate}
  \item Calculate overlaps between $ AT^{(k)} $ and $ ToIs^{(k)}. $
  \item Connect all tubes which satisfy the following criteria:
    \begin{enumerate}
    \item $ overlap score(at_i^{(k)},t_j^{(k)})   > \theta, 
      at  \in AT^{(k)}, t \in ToIs^{(k)}  $
    \item $r(at_i^{(k)}) < r(t_j^{(k)}) $ or 
      $r(t_i^{(k)}) - r(at_i{(k)}) < \lambda $
    \end{enumerate}
    
  \item For all new tubes update confidence score and progress rate as follows:
    \begin{description}
    \item The new confidence score is the average score of all connected ToIs:
      \[  s_z^{(k+1)} = \frac {1} {n} \sum_{n=0}^{k} s_i^{(n)}\]
    \item New progress rate is the highest progress rate:
      \[r(at_z^{(k+1)} = max(r(at_i^{(k)}), r(t_j^{(k)})) \]
    \end{description}
    % \item If $ progressrate(at_i^{(k)}) < progressrate(t_i^{(k)}) $ then $ progressrate(at_i^{(k+1)}) =
  \item Keep M best scoring action tubes as active tubes and keep K best scoring action tubes for classification.
  \end{enumerate}
  
\end{enumerate}

This approach has the advantage that we don't need to perform classification again because we already know the class of
each final tube. In order to validate our results, now, we calculate the recall only from the tubes which have the same
class as the groundtruth tube. Again, we considered a groundtruth  tube as positive if there is at least one proposed  tube
that overlaps with it over the predefined threshold

\begin{center}
\begin{longtable}{||c c||c c c||}
  \hline
  \multicolumn{2}{||c||}{\textbf{combination}} &\multicolumn{3}{|c||}{\textbf{overlap thresh}}\\

  \hline
  sample dur & step &   0.5  &  0.4 &  0.3 \\
  \hline   \hline
  8 & 6 & 0.3284 & 0.5 & 0.6082  \\
  \hline
  8 & 7 & 0.209	& 0.459 & 0.6119 \\
  \hline
  8 & 8 & 0.3060 & 0.5672 & 0.6866 \\
  \hline
  16 & 8  & 0.194 & 0.4366 & 0.7164 \\
  \hline
  16 & 12 & 0.3358 & 0.5336 & 0.7537 \\
  \hline
  16 & 16 & 0.2649 & 0.4664 & 0.709 \\
  
  \hline 

  \caption{Recall results for second approach with step = 8, 16 and their corresponding steps }
  \label{table:conn_app2}
\end{longtable} 
\end{center}

According to  Table \ref{table:conn_app2}, we get the best performance when we set sample duration equal to  16 and overlap step equal to 12.
Comparing this performance with the first approach, for both sample durations equal to 8 and 16, we notice that second approach falls short
comparing to the first one. 

\section{Third approach : use naive algorithm - only for JHMDB}

As mention in the first approach, \cite{DBLP:journals/corr/HouCS17} calculate all possible sequences of ToIs in order to the find the best
candidates. We rethought about this approach and we concluded that it could be implemented for JHMDB dataset if we reduce the number of proposed
ToIs, produced by TPN,  to 30 for each video clip. We exploited the fact that JHMDB dataset's videos are trimmed, so we do not need to look
for action tubes starting in the second video clip which saves us a lot of memory. On top of that, we modified our code
in order to be more memory efficient  writing some parts using CUDA programming language, saving a lot of processing power, too. \par
So, after computing all possible combinations starting of the first video clip and ending in the last video clip, we keep only the
\textbf{k-best scoring tubes (k = 500) }. We run experiments which have sample duration equal to 8 and 16 frames and we modify the video step each time.
For sample duration = 8, we return only 15 ToIs and for sample duration = 16, we return 30 because, if we return more, we get ``out of memory error''.
In the following table, we can see the recall results. \par  

\begin{center}
\begin{longtable}{||c c||c c c||}

  \hline
  \multicolumn{2}{||c||}{\textbf{combination}} &\multicolumn{3}{|c||}{\textbf{overlap thresh}}\\
  \hline
  sample dur & step &  0.5  &  0.4 &  0.3 \\
  \hline   \hline

  8 & 6 & 0.7873 & 0.8657 & 0.9366  \\
  \hline
  8 & 7 & 0.7836 & 0.8731 & 0.9366  \\
  \hline
  8 &  8 & 0.7910 & 0.8806 & 0.9515 \\
  \hline 

  16 & 8  & 0.7873 & 0.8843 & 0.9291 \\
  \hline
  16 & 12 & 0.7948 & 0.8881 & 0.9403 \\
  \hline
  16 & 16 & 0.7985 & 0.8918 & 0.9515 \\
  \hline 
  \caption{Recall results for third approach with  step = 8, 16 and their
corresponding steps}
  \label{table:conn_app3}
\end{longtable} 
\end{center}

From the above table, firstly, we confirm that when video step is equal to the sample duration gives us the best recall results.
Also, we notice that when sample duration is equal to 16  frames recall gets slightly better that when sample duration is
equal to 8. However, using  16 frames per video segment sample increases the memory usage even though it reduces the number of video segments, because of the
need to process bigger videos, bigger feature maps etc. So for the classification stage we will experiment using mostly sample duration equal
with 8 frames.

\section {General comments}

\begin{figure}[h]
  % \includegraphics[scale=0.7]{convolutional_neural_network_structure} \]
  \centering
  % \includegraphics[scale=0.2]{tube_ex2_half}
  \includegraphics[width= 0.8\textwidth, height=0.45\textheight]{tube_ex2_half}
  \caption{Example of connected tubes}
  \label{fig:tube_ex}
\end{figure}

Figure \ref{fig:tube_ex} shows the example presented in chapter 3, after linking first video segment's first ToI with a ToI proposed for the second video segment. For this case, we used the third proposed method, including calculating all possible combinations. As shown in the Figure,
our algorithm manages to track the actor performing the action efficiently enough. This means that even though the actor moves during the
video, our approach manages not to loose contact with him. On top of that, it clear that the silhouette of the actor changes, and so does the area of proposed action tubes. The only problem which appears is that proposed action tubes sometimes exceeds the area of the actual video.
In order to deal with this problem, we set bounding boxes not to exceed these areas by keeping the limits of the original picture. So,
from now on, no bounding box will overlap with padding area.

% \end{document}

\documentclass{report}

\usepackage{subcaption} % package for subfigures
\usepackage{hyperref}  % package for linking figures etc
\usepackage{enumitem}  % package for description with bullets
\usepackage{graphicx}  % package for importing images
\usepackage{mathtools} % package for math equation
\usepackage{mathrsfs}  % package for math font
\usepackage{indentfirst} % package for getting ident after section or paragraph
\usepackage[export]{adjustbox}
\usepackage{longtable} % package for multi pages tables
\usepackage{multirow}  % package for tables, multirow
% \usepackage{amsmath}

\setlength{\parindent}{2em} % how much indent to use when we start a paragraph

\graphicspath{ {./theory/figures/} }       % path for images

\begin{document}

\chapter{Classification stage}
\section{Description}
After getting all proposed tubes, it's time to do classification. As classifiers we use several approaches includingn
a Recursive Neural Network (RNN) Classifier, a Support Vector Machine (SVM) Classifier and a Multilayer perceptron (MLP).

\begin{figure}[h]
  % \includegraphics[scale=0.7]{convolutional_neural_network_structure} \]
  \centering
  \includegraphics[scale=0.42]{model_prenms}
  \caption{Structure of the whole network}
  \label{fig:whole_network}
\end{figure}

The whole procedure of classification is consisted from the following steps:
\begin{enumerate}
\item Seperate video into small video clips. Feed TPN network those video clips and get as output
  k-proposed ToIs and their corresponding features for each video clip.
\item Connect the proposed ToIs in order to get video tubes which may contain an action.
\item For each candidate video tube, which is a sequence of ToIs, feed it into the classifier
  for verification.
\end{enumerate}

The general structure of the whole network is depicted in figure \ref{fig:whole_network}, in which we can see the previous steps if we
follow the arrows.  \par
In first steps of classification stage we refer only to JHMDB dataset because it has smaller number of video than UCF dataset which
helped us save a lot of time and resources. That's because  we performed most experiments only JHMDB and after we found the optimal
situation, we implemented to UCF-dataset, too. 

\section{Preparing data for classification -  RNN Classifier}

% \textbf{(Pending... Introduction about Linear and RNN classifiers)}
\textbf{(Pending.. also an image of RNN classifier)}

In order to train our classifier, we have to execute the previous steps for each video. However, each video
has different number of frames and reserves too much memory in the GPU. In order to deal with this situation,
we give as input one video per GPU. So we can handle 4 videos simultaneously. This means that a regular
training session takes too much time for just 1 epoch. \par
The solution we came with, is to precompute the features for both positive video tubes and negative video tubes.
Then we feed those features to our classifier and we train it in order to disciminate their classes.
At first, we extract only groundtruth video tubes' features and the double number of background video tubes. We chose this
ratio between positive and negative tubes inspired by \cite{jjfaster2rcnn}, in which it has 0.25 ratio between foreground
and background rois and chooses 128 roi in total. Respectively, we chose a little bigger ratio because we have only 1 groundtruth
video tube in each video. So, for each video we got 3 video tubes in total, 1 for positive and 2 for background. We considered
background tubes those whose overlap scores with groundtruth tubes are $ \ge 0.1 $ and $ \le 0.3 $. \par

Then, after extracting those features, we trained both linear and RNN classifiers. The Linear classifier needs a fixed input size,
so we used a pooling function in the dimension of the videos. So, at first we had a feature map of 3,512,16 dimensions and then we
get as output a feature maps of 512,16 dimensions. We used both max and mean pooling as show in the results below. For the RNN
classifier, we do not use any pooling function before feeding it. For both classifiers, at first, we didn't considered a fixed
threshold for confidence score.

\textbf{(Pending results in Linear... Table)}

The results are disappointing.
As we can see in the table, RNN classifier cannot classify very well because, probably, the duration of the videos are so small
so we stopped using it in jHMDB dataset. In the Linear classifier, we noticed that every tube is considered as background tube.
That means that Linear classifier gets overfitted with trained data and cannot handle unknown data. So, we thought that we
need a classifier which can \"learn\" very easily, with little data. So we chose to try a support vector machine classifier.

\section{Support Vector Machine (SVM)}
\subsection{First steps}
SVMs are classifiers defined by a separating hyperplane between trained data in a N-dimensional space. The main advantage of using a SVM
is that can get very good classification results when we have few data available. 
\textbf{write more introduction and a pic, Pending...} \par
The use of SVM is inspired from \cite{Girshick:2015:FR:2919332.2920125} and it is trained using hard negative mining. 
This means that we have 1 classifier per class which has only 2 labels, positive and negative. We mark as positive the feature maps of the
groundtruth action, and as negative groundtruth actions from other classes, and feature maps from background classes.
As we know, SVM is driven by small number of examples near decision boundary. Our goal is to find a set of negatives that are the closest to
the seperating hyperplane. So in each iteration, we update this set of negatives adding those which our SVM didn't perform very well. Each
SVM is trained independently. \par
SVM code is take from Microsoft's Azure \href{https://github.com/Azure/ObjectDetectionUsingCntk} {github page} in which there is an implementation
of Fast RCNN using a SVM classifier. We didn't modify its parameters which means that it has a linear kernelr, uses  L2-norm as penalty and L1-norm
as loss during training. Also, we consider as hard-negatives the tubes that got score $ >  -1.0 $ during classification.\par
This whole process makes the choise of the negatives a crutial factor. In order to find the best policy,  we came with 5 different cases to consider
as negatives:
\begin{enumerate}
\item Negatives are other classes's positives and all the background tubes
\item Negatives are only all the background videos
\item Negatives are only other classes's positives
\item Negatives are other classes's positives and background tubes taken only from videos that contain a positive tube
\item Negatives are only background tubes taken from videos that contain a positive tube
\end{enumerate}

On top of that, we use 2 pooling functions in order to have a fixed input size. \par
In the next tables, we show our architecture's  mAP performance when we follow each one of the above policies. Also,
we experimented for 2 feature maps, \textit{(64,8,7,7)} and \textit{(256,8,7,7)} where 8 equals with the sample duration.
Both feature maps were extracted by using 3D RoiAlign procedure from feature maps with dimensions \textit{(64,8,28,28)} and
\textit{(256,8,7,7)} respectively (in the second case, we just add zeros in the feature map outsize from the bounding boxes for
each frame). Table \ref{table:svm_first_results} contains the first classification results. At first column we have the dimensions
of feature maps before pooling fuction, where k = 1,2,..5 . At second column we have feature maps' dimensions after pooling, and at
the next 2 column, the type of pooling function and the policy we followed. Finally in the last 3 collumns we have the mAP performance
when we have threshold equal with 0.3, 0.4 and 0.5 respectively. During validation, we keep only the best scoring tube because we know that
we have only 1 action per video.

\begin{center}
\begin{longtable}{||c | c | c| c||c c c||}

  \hline
  \multicolumn{2}{||c|}{\textbf{Dimensions}} & \multirow{2}{*}{ \textbf{Pooling}} &\multirow{2}{*}{\textbf{Type}} & \multicolumn{3}{|c||}{\textbf{mAP precision}}\\

   before & after &  {} & {} &  0.3 &  0.4 & 0.5 \\
 \hline   \hline
 \multirow{5}{*}{(k,64,8,7,7)} & \multirow{5}{*}{(1,64,8,7,7)} & \multirow{5}{*}{mean}  & 1 &  3.16 & 4.2 & 4.4    \\
  \cline{4-7}
  {} & {} & {} & 2 & 2.29 & 2.68 & 2.86    \\
    \cline{4-7}
  {} & {} & {} & 3 & 1.04 & 1.04 & 1.04    \\
    \cline{4-7}
  {} & {} & {} & 4 & TODO & TODO & TODO    \\
    \cline{4-7}
  {} & {} & {} & 5 & TODO & TODO & TODO    \\
  \hline
 \multirow{5}{*}{(k,64,8,7,7)} & \multirow{5}{*}{(1,64,8,7,7)} & \multirow{5}{*}{max}  & 1 & TODO & TODO & TODO \\
    \cline{4-7}
  {} & {} & {} & 2 & TODO & TODO & TODO    \\
    \cline{4-7}
  {} & {} & {} & 3 & TODO & TODO & TODO    \\
    \cline{4-7}
  {} & {} & {} & 4 & TODO & TODO & TODO    \\
    \cline{4-7}
  {} & {} & {} & 5 & TODO & TODO & TODO     \\

  \hline   \hline

 \multirow{5}{*}{(k,256,8,7,7)} & \multirow{5}{*}{(1,256,8,7,7)} & \multirow{5}{*}{mean}  & 1 &  11.41 & 11.73 & 11.73 \\

    \cline{4-7}
  {} & {} & {} & 2 & 10.35 & 10.92 &11.89 \\
    \cline{4-7}
  {} & {} & {} & 3 & 8.93 9.64 9.94 \\
    \cline{4-7}
  {} & {} & {} & 4 & TODO & TODO & TODO \\
    \cline{4-7}
  {} & {} & {} & 5 & 5.92 & 6.92 & 7.79 \\
    \hline
 \multirow{5}{*}{(k,256,8,7,7)} & \multirow{5}{*}{(1,256,8,7,7)} & \multirow{5}{*}{max}  & 1  & 22.07 & 24.4 & 25.77  \\
    \cline{4-7}
  {} & {} & {} & 2  & 14.07 & 16.56 & 17.74 \\
    \cline{4-7}
  {} & {} & {} & 3  & 14.22 & 18.94 &21.6 \\
    \cline{4-7}
  {} & {} & {} & 4  & 21.05 & 24.63 & 25.93 \\
    \cline{4-7}
  {} & {} & {} & 5  & 11.6 & 13.92 & 15.81 \\
  \hline   
  
  \caption{Our architecture's performance using 5 different policies and 2 different feature maps while pooling in
  tubes' dimension. With bold is the best scoring case}
  \label{table:svm_first_results}

\end{longtable} 
\end{center}

\subsection{Modifying 3D Roi Align}
As we mentioned before, we extract from each tube its activation maps using 3D Roi Align procedure and we set equal to zero the pixels outside
of bounding boxes for each frame. We came with the idea that the enviroment surrounding the actor sometimes help us determine the class
of the action which is performed. This is base in the idea that 3D Convolutional Networks use the whole scene in order to classify the action
that is performed. We thought to extend a little each bounding box both in width and height. So, during Roi Align procedure, after resizing
the bounding box into the desired spatial scale  ( in our case 1/16 because original sample size = 112 and resized sample size = 7 )
we increase by 1 both width and height. According to that if we have a resized bounding box $( x_1,y_1,x_2,y_2) $ our new bounding box becomes
$ (max(0,x_1-0.5),max(0,y_1-0.5),min(7,x_2+0.5),min(7,y_2+0.5)) $ ( we use \textit{ min} and \textit{max} functions in order to avoid exceeding feature maps' limits).
We just experiment in policies 1 and 4 for both (256,8,7,7) and (64,8,7,7) feature maps as show in  Table \ref{table:svm_mod_roialign}


\begin{center}
\begin{longtable}{||c | c | c| c||c c c||}

  \hline
  \multicolumn{2}{||c|}{\textbf{Dimensions}} & \multirow{2}{*}{ \textbf{Pooling}} &\multirow{2}{*}{\textbf{Type}} & \multicolumn{3}{|c||}{\textbf{mAP precision}}\\

   before & after &  {} & {} &  0.3 &  0.4 & 0.5 \\
 \hline   \hline
 \multirow{5}{*}{(k,64,8,7,7)} & \multirow{5}{*}{(1,64,8,7,7)} & \multirow{5}{*}{mean}  & 1 & TODO & TODO & TODO  \\
  \cline{4-7}
  {} & {} & {} & 2 & TODO & TODO & TODO \\
    \cline{4-7}
  {} & {} & {} & 3 & 0.96 & 2.11 & 2.9   \\
    \cline{4-7}
  {} & {} & {} & 4 &  2.4  & 4.53 & 5.16    \\
    \cline{4-7}
  {} & {} & {} & 5 &  0.6  & 0.76 & 0.93    \\
  \hline
 \multirow{5}{*}{(k,64,8,7,7)} & \multirow{5}{*}{(1,64,8,7,7)} & \multirow{5}{*}{max}  & 1 &  1.07 & 1.85 & 2.26    \\
    \cline{4-7}
  {} & {} & {} & 2 &  3.2 & 3.39 & 3.66    \\
    \cline{4-7}
  {} & {} & {} & 3 &  0.86 & 1.98 & 1.98    \\
    \cline{4-7}
  {} & {} & {} & 4 &  1.24 & 2.68 & 3.09    \\
    \cline{4-7}
  {} & {} & {} & 5 &  0.21 & 0.27 & 0.32    \\

  \hline   \hline

 \multirow{5}{*}{(k,256,8,7,7)} & \multirow{5}{*}{(1,256,8,7,7)} & \multirow{5}{*}{mean}  & 1 &  10.62 & 10.94 & 10.94    \\

    \cline{4-7}
  {} & {} & {} & 2 &  9.09  & 10.02 & 10.83   \\
    \cline{4-7}
  {} & {} & {} & 3 &  9.05  & 9.65  & 9.69    \\
    \cline{4-7}
  {} & {} & {} & 4 &  11.19 & 11.51 & 11.51   \\
    \cline{4-7}
  {} & {} & {} & 5 &  4.84  & 6.13  & 6.13   \\
    \hline
 \multirow{5}{*}{(k,256,8,7,7)} & \multirow{5}{*}{(1,256,8,7,7)} & \multirow{5}{*}{max}  & 1  & \bf 20.94 & \bf 24.96 & \bf 26.46   \\
    \cline{4-7}
  {} & {} & {} & 2  & 14.94 & 17.78 & 19.38   \\
    \cline{4-7}
  {} & {} & {} & 3  & 14.9 & 17.39 & 19.88   \\
    \cline{4-7}
  {} & {} & {} & 4  & 19.43 & 23.91 & 25.31   \\
    \cline{4-7}
  {} & {} & {} & 5  & 10.41 & 10.46 & 11.29   \\
  \hline   
  
  \caption{Our architecture's performance using 5 different policies and 2 different feature maps extracted using modified Roi Align.}
  \label{table:svm_mod_roialign}

\end{longtable} 
\end{center}

From the above results we notice that features map with dimension (256,8,7,7) outperform in all cases, both for mean and max pooling and
for all the policies. Also, we can see that max pooling outperforms mean pooling in all cases, too. Last but not least, we notice that policies
2, 3 and 5 give us the worst results which means that svm needs both data from other classes positives and from background tubes. 

\subsection{Temporal pooling}
After getting first results, we implement a temporal pooling function inspired from \cite{DBLP:journals/corr/HouCS17}. We need a
fixed input size for the SVM. However, our tubes' temporal stride varies from 2 to 5. So we use as fixed temporal pooling equal
with 2. As pooling function we use 3D max pooling, one for each filter of the feature map.  So for example, for an action tube
with 4 consecutive ToIs, we  have 4,256,8,7,7 as feature size. We seperate the feature map into 2 groups using \textit{linspace}
function and we reshape the feature map into 256,k,8,7,7 where k is the size of each group, After using 3D max pooling, we get
a feature map 256,8,7,7 so finally we concat them and get 2,256,8,7,7. In this case we didn't experiment with (64,8,7,7) feature
maps because it wouldn't performed better that (256,8,7,7) ferature maps as noticed from the previous section.

\begin{center}
\begin{longtable}{||c | c| c| c||c c c||}

  \hline
 \multicolumn{2}{||c|}{\textbf{Dimensions}} & \multirow{2}{*}{\textbf{Pooling}} &\multirow{2}{*}{ \textbf{Type}} &\multicolumn{3}{|c||}{\textbf{mAP precision}}\\

  before & after & {} & {} & 0.3 &  0.4 & 0.5 \\
  \hline   \hline

  \multirow{5}{*}{k,256,8,7,7} & \multirow{5}{*}{2,256,8,7,7} & \multirow{5}{*}{mean}  & 1 & TODO & TODO & TODO \\
  \cline{4-7}

  {} & {} & {} & 2 & TODO & TODO & TODO \\
  \cline{4-7}
  {} & {} & {} & 3 & TODO & TODO & TODO \\
  \cline{4-7}
  {} & {} & {} & 4 & TODO & TODO & TODO \\
  \cline{4-7}
  {} & {} & {} & 5 & TODO & TODO & TODO \\
  \hline

  \multirow{5}{*}{k,256,8,7,7} & \multirow{5}{*}{2,256,8,7,7} & \multirow{5}{*}{max}  & 1 & 25.07 & 26.91 & 29.11 \\
  \cline{4-7}

  {} & {} & {} & 2 & TODO & TODO & TODO \\
  \cline{4-7}
  {} & {} & {} & 3 & TODO & TODO & TODO \\
  \cline{4-7}
  {} & {} & {} & 4 & TODO & TODO & TODO \\
  \cline{4-7}
  {} & {} & {} & 5 & TODO & TODO & TODO \\
  \hline

  \caption{mAP results for second approach using temporal pooling }
  \label{table:svm_temp_pooling}
\end{longtable} 
\end{center}

\subsection{Increasing sample duration to 16 frames}

\begin{center}
\begin{longtable}{||c | c| c| c||c c c||}

  \hline
 \multicolumn{2}{||c|}{\textbf{Dimensions}} & \multirow{2}{4.5em}{\textbf{Temporal Pooling}} &\multirow{2}{*}{ \textbf{Type}} &\multicolumn{3}{|c||}{\textbf{mAP precision}}\\

  before & after & {} & {} & 0.3 &  0.4 & 0.5 \\
  \hline   \hline

  \multirow{5}{*}{k,256,16,7,7} & \multirow{5}{*}{1,256,16,7,7} & \multirow{5}{*}{No}  & 1 & 23.4 & 27.57 &28.65  \\
  \cline{4-7}

  {} & {} & {} & 2 & TODO & TODO & TODO \\
  \cline{4-7}
  {} & {} & {} & 3 & TODO & TODO & TODO \\
  \cline{4-7}
  {} & {} & {} & 4 & 22.7 & 26.95 & 28.05 \\
  \cline{4-7}
  {} & {} & {} & 5 & TODO & TODO & TODO \\
  \hline

  \multirow{5}{*}{k,256,16,7,7} & \multirow{5}{*}{2,256,16,7,7} & \multirow{5}{*}{Yes}  & 1 & 21.12 & 24.07 & 24.36  \\
  \cline{4-7}

  {} & {} & {} & 2 & TODO & TODO & TODO \\
  \cline{4-7}
  {} & {} & {} & 3 & TODO & TODO & TODO \\
  \cline{4-7}
  {} & {} & {} & 4 & 18.36 & 23.09 & 23.75 \\
  \cline{4-7}
  
  {} & {} & {} & 5 & TODO & TODO & TODO \\
  \hline
  \caption{mAP results for   }
  \label{table:svm_temp_pooling_16}
\end{longtable} 
\end{center}

\subsection{Adding more groundtruth tubes}
From above results, we notice that SVM improve a lot the perfomance of our model. In order to futher improve our results, we will
add more groundtruth action tubes. We consider as groundtruth action tubes all the tubes whose overlap score  with a groundtruth tube is
greater that 0.7 . Also, we increase the total number of tube to 8. Table \ref{table:svm_increased}

\begin{center}
\begin{longtable}{||c | c| c| c||c c c||}

  \hline
 \multicolumn{2}{||c|}{\textbf{Dimensions}} & \multirow{2}{*}{\textbf{Pooling}} &\multirow{2}{*}{ \textbf{Type}} &\multicolumn{3}{|c||}{\textbf{mAP precision}}\\

  before & after & {} & {} & 0.3 &  0.4 & 0.5 \\
  \hline   \hline

  \multirow{2}{*}{k,64,8,7,7} & \multirow{2}{*}{2,64,7,7} & \multirow{2}{*}{max}  & 1 & TODO & TODO & TODO \\
  \cline{4-7}
  {} & {} & {} & 4 & TODO & TODO & TODO   \\
  \hline   
  \multirow{2}{*}{k,256,8,7,7} & \multirow{2}{*}{2,256,7,7} & \multirow{2}{*}{max}  & 1 & TODO & TODO & TODO \\
  \cline{4-7}
  {} & {} & {} & 4 & TODO & TODO & TODO   \\
  \hline   
  \caption{Results after increasing the number of total tubes }
  \label{table:svm_increased}

\end{longtable} 
\end{center}

\section{MultiLayer Perceptron (MLP)}
Last but not least approach is a Multilayer perceptorn (MLP). More specifically, we extract the 3 last residulal layers of 3D ResNet34
and we add a classification layer.  

\subsection{Extract features}
We

\subsection{Regular training}

\section{Classifying dataset UCF}
As mentioned before, all the above results are from JHMDB dataset. 
\section{Final Improvements}
After classification, we relize that a lot of classified tubes overlap and represent the same action. So, we use again NMS algorithm in order
to remove unnecessary tubes. The new model can be seen in figure \ref{fig:network_nms}.

\begin{figure}[h]
  % \includegraphics[scale=0.7]{convolutional_neural_network_structure} \]
  \centering
  \includegraphics[scale=0.7]{model_nms}
  \caption{Structure of the network with NMS}
  \label{fig:network_nms}
\end{figure}

\end{document}

% \documentclass{report}

% \usepackage{subcaption} % package for subfigures
% \usepackage{hyperref}  % package for linking figures etc
% \usepackage{enumitem}  % package for description with bullets
% \usepackage{graphicx}  % package for importing images
% \usepackage{mathtools} % package for math equation
% \usepackage{mathrsfs}  % package for math font
% \usepackage{indentfirst} % package for getting ident after section or paragraph
% \usepackage[export]{adjustbox}
% \usepackage{multirow}  % package for tables, multir
% \usepackage{amssymb}
% % \usepackage{tabu}   % for tables 
% \setlength{\parindent}{2em} % how much indent to use when we start a paragraph

% \graphicspath{ {./theory/figures/} }       % path for images

% \begin{document}

\chapter{Conclusion - Future work}

\section{Conclusion}
In this thesis, we explored the problem of action recognition and localization. We design a network based on the approach proposed by \cite{DBLP:journals/corr/HouCS17}
combined with some elements presented by \cite{DBLP:journals/corr/abs-1712-09184}, \cite{Ren:2015:FRT:2969239.2969250}, \cite{Girshick:2015:FR:2919332.2920125},
\cite{DBLP:journals/corr/abs-1903-00304} and \cite{hara3dcnns}. \par

We wrote a pytorch implementation, expanding code only from \cite{jjfaster2rcnn}. Furthermore, we wrote our own code using some CUDA functions designed by us (like
calculating connection scores, modifying tubes etc). \par

We tried to  a design the  Tube Proposal Network for proposing ToIs in given video segments, inspired by Faster R-CNN's RPN.
We designed it using general anchors and not dataset specific anchors. This means that we try to generalize our approach for several datasets, on the contrary with
the approach proposed by \cite{DBLP:journals/corr/abs-1712-09184}, in which it uses the most frequently appearing anchors for each dataset.

On top of that, we designed a naive connection algorithm for connecting  proposed ToIs based on the algorithm proposed by \cite{DBLP:journals/corr/abs-1712-09184}.
In our approach, we use the same scoring policy, which is a combination between actioness and overlapping scores. The main difference is that we avoid to calculate
all the possible combinations using an updatable threshold. We, also, tried another connection algorithm inspired by \cite{DBLP:journals/corr/abs-1903-00304}. However,
our implementation wasn't very good so, we didn't explore all of its potentials. \par

Finally, we explored several classifiers for the classification stage of our network, which are a RNN, an SVM and an MLP.  We used an implementation taken from Fast RCNN
for the SVM classifier, which included hard negatives mining training procedure. We explored some training techniques for best classification performance and
2 training approaches for MLP classifier, the classic one and using pre-extracted features. 

\section{Future work}
There is a lot of room for improvement for our network, in order to achieve state-of-the-art results. The most important are described in the next paragraphs.

\paragraph{Improving TPN proposals} We implemented 2 networks for proposing action tubes in a video segment. We managed to achieve about 63\% recall score for
sample duration = 16 and about 80\% recall for sample duration = 8. Theses scores show that there is plenty room for improvement, especially for sample duration = 16.
Even though a lot of networks' architectures have been explored for regression, a good idea would be to try other networks, not necessarily inspired by object detection
networks like we did. On top of that, adding a $\lambda$ factor in training loss would be a good idea and exploring which is the best approach.
So training loss could be defined as:
\begin{equation} 
\begin{split}
 L  =  \sum_iL_{cls}(p_i, p_i^*) + \lambda_1 \sum_ip_i^*L_{reg}(t_i,t_i^*) + \lambda_2  \sum_iq_i^*L_{reg}(c_{i}, c_{i}^*) \\
\end{split}
\end{equation}

Furthermore, it would be a good idea to use SSD's (\cite{DBLP:journals/corr/LiuAESR15}) proposal network instead of RPN, in order to compare result. Finally,
we could experiment using Feature Pyramid Networks(\cite{8099589}), which could be exteded in 3 dimensions as another feature extractor or some other type of 3D ResNet.

\paragraph{Changing Connection algorithm}
In this thesis, another challenge we came was connecting proposed ToIs for proposing action tubes. We implemented a very naive algorithm, which wasn't
able to give us very good proposals despite the changes we tried to do. We implemented another connection algorithm which was based on estimationing temporal
progress of an action tube  and its overlap with others. Although it also didn't give us very good proposals, we believe that we should explore this algorithm's potential. That's
because it takes advantage of the progress of the action, which the previous algorithm didn't.

\paragraph{Explore other  classification techniques}
For classification stage, we experimented mainly on an SVM classifier for JHMDB dataset and we didn't get involved with UCF dataset. Our first
goal is to be able to get good classification results for UCF dataset, too.  We think that we should explore UCF's feature maps and techniques applied at feature maps before  classification. In addition, we could try other classification techniques like random forest or experiment more with RNN classifier for the UCF dataset.
Finally, another classification procedure would be a good idea, like extracting first all the possible action tubes and then using other network's features for classification
stage.

% \end{document}

%%%  Bibliography

% \bibliographystyle{softlab-thesis}
% \bibliography{References}
%%% mine
\printbibliography

% %%%  Appendices

% \backmatter

% \appendix

% \chapter{}

% $A \rightarrow B$ : συνάρτηση από το πεδίο $A$ στο πεδίο $B$.

% \chapter{}

% \textbf{Haskell} : η γλώσσα της ζωής μου.

% \chapter{}

% 42 : life, the universe and everything.


%%%  End of document

\end{document}
