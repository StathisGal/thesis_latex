\documentclass{report}

\usepackage[greek,english]{babel}

\newcommand{\tl}{\textlatin}
\newcommand{\en}{\selectlanguage{english}}
\newcommand{\gr}{\selectlanguage{greek}}

\usepackage{hyperref}  % package for linking figures etc
\usepackage{enumitem}  % package for description with bullets
\usepackage{graphicx}  % package for importing images
\usepackage{mathtools} % package for math equation
\usepackage{mathrsfs}  % package for math font
\usepackage{indentfirst} % package for getting ident after section or paragraph
\usepackage{subcaption} % package for subfigures
\usepackage[export]{adjustbox}
\usepackage{longtable} % package for multi pages tables
\usepackage{multirow}  % package for tables, multirow
\usepackage{amssymb}
\usepackage{esvect}
\usepackage[
backend=bibtex,
citestyle=authoryear,
% citestyle=authoryear-comp,
% citestyle=authoryear-ibid,
bibstyle=numeric,
sorting=ynt,
% style=numeric,
% style=alphabetic ,
]{biblatex}
\addbibresource{References}

\graphicspath{ {./theory/figures/} }       % path for images

\begin{document}
\gr
\section{Σχετική βιβλιογραφία}
Σε αυτή την ενότητα, παρουσιάζουμε ορισμένες από τις πιο συναφείς μεθόδους για την εργασία μας και άλλες που μελετήθηκαν για τον σχεδιασμό αυτής της προσέγγισης.
Οι μέθοδοι αυτές  χωρίζονται σε δύο ενότητες \textit{Aναγνώριση Δραστηριότητας} και \textit{Εντοπισμός Δραστηριότητας}. Το πρώτο μέρος αναφέρεται σε κλασικές μεθόδους
ταξινόμησης δράσης που εισήχθησαν μέχρι πρόσφατα και το δεύτερο μέρος, αντίστοιχα, σε πρόσφατες μεθόδους εντοπισμού της δράσης. 

\subsection{Αναγνώριση Δραστηριότητας}
Οι πρώτες προσεγγίσεις για την κατάταξη της δράσης αποτελούνταν από δύο βήματα α) αρχικά υπολογισμός σύνθετων ``χειροποίητων'' χαρακτηριστικών από ακατέργαστα καρέ βίντεο
και β) εκπαίδευση ενός ταξινομητή με βάση αυτά τα χαρακτηριστικά. Αυτά τα χαρακτηριστικά μπορούν να διαχωριστούν σε 3 κατηγορίες:
1) προσεγγίσεις χωροχρονικού όγκου \tl{(space-time volume)}, 2) τροχιές \tl{(trajectories)} και 3) χωροχρονικά  χαρακτηριστικά. Για τις μεθόδους χωροχρονικού όγκου,
η προσέγγιση είναι η εξής: Με βάση τα \tl{training} βίντεο, το σύστημα συνάπτει ένα μοντέλο τρισδιάστατου χωροχρόνου, συνενώνοντας δυσδιάστατες εικόνες
(διάσταση \tl{\textit{x-y}}) κατά τη διάρκεια του χρόνου (διάσταση \tl{\textit{t}} ή \tl{\textit{z}}), για την αναπαράσταση κάθε δράσης.
Όταν το σύστημα δέχεται ένα βίντεο που δεν έχει ετικέτα, κατασκευάζει μια τρισδιάστατη χωροχρονική αναπαράσταση που αντιστοιχεί σε αυτό το βίντεο.
Αυτό η νέα τρισδιάστατη αναπαράσταση, στη συνέχεια, συγκρίνεται με κάθε μοντέλο \tl{3D} χωροχρόνου, συγκρίνοντας την ομοιότητα στο σχήμα
και την εμφάνιση μεταξύ αυτών των δύο χωροχρονικών όγκων.
Το σύστημα εξάγει την κατηγορία του άγνωστου βίντεο, αντιστοιχώντας την με αυτήν της δράσης με την υψηλότερη ομοιότητα. Επιπλέον, υπάρχουν
διάφορες παραλλαγές των  χωροχρονικών αναπαραστάσεων. Αντί της αναπαράστασης \tl{space-time volume}, το σύστημα μπορεί να αναπαριστά τη κάθε δράση ως τροχιές
σε χωροχρονικές διαστάσεις ή ακόμη περισσότερο, η ενέργεια μπορεί να αναπαρασταθεί ως ένα σύνολο χαρακτηριστικών που εξάγονται από τον χωροχρονικό όγκο ή τις τροχιές.
Οι ``καθαρές'' χωροχρονικές αναπαραστάσεις περιλαμβάνουν μεθόδους σύγκρισης των περιοχών προσκηνίου ενός ατόμου (δηλ. σιλουέτες) όπως \en \cite{BobickAaron},\gr
συγκρίνοντας όγκους σε σχέση με επιφάνεια τους όπως οι \en \cite{1467296}\gr. Η μέθοδος \en \cite{4270510}\gr χρησιμοποιεί \tl{oversegmented} όγκους, αυτομάτως
υπολογίζοντας ένα σύνολο τμημάτων τρισδιάστατου όγκου \tl{ XYT} που αντιστοιχεί σε έναν κινούμενο άνθρωπο. \en\cite{4587727} \gr πρότειναν φίλτρα για
να αποτυπόνουν τα χαρακτηριστικά του χωροχρονικού όγκου, προκειμένου να τα ταιριάζουν πιο αξιόπιστα και αποδοτικά.
Από την άλλη πλευρά, οι προσεγγίσεις με βάση την τροχιά περιλαμβάνουν την αναπαράσταση μιας ενέργειας ως σύνολο 13 κοινών διαδρομών (\en\cite{1541250}\gr) ή
τη χρήση ενός συνόλου \tl{\textit{XYZT}}-διαστάσεων κοινών τροχιών που λαμβάνονται από κινούμενες κάμερες  (\en\cite{1541251}\gr).
Τέλος, διάφορες μέθοδοι χρησιμοποιούν τοπικά χαρακτηριστικά που εξάγονται από χωροχρονικούς όγκους τριών διαστάσεων,
όπως η εξαγωγή τοπικών χαρακτηριστικών σε κάθε καρέ του βίντεο και η ένωση του χρονικά (\en\cite{784616, 990935, 1544882}\gr,
η εξαγωγή αραιών χωροχρονικών τοπικών σημείων ενδιαφέροντος από τρισδιάστατους όγκους (\en\cite{1238378, 1570899,  Niebles, 1467373, Ryoo2006}\gr)
Οι προσεγγίσεις αυτές κατέστησαν την επιλογή των χαρακτηριστικών  σημαντικό παράγοντα για την απόδοση του δικτύου.
Αυτό συμβαίνει επειδή οι διαφορετικές κατηγορίες δράσεν μπορεί να διαφέρουν δραματικά από την άποψη της εμφάνισής τους και των μοτίβων κίνησης.
Ένα άλλο πρόβλημα ήταν ότι οι περισσότερες από αυτές τις προσεγγίσεις κάνουν υποθέσεις, υπό τις οποίες το βίντεο λήφθηκε λόγω προβλημάτων όπως το γεμάτo
φόντο,  γωνιές κάμερας κλπ. Μια ανασκόπηση των τεχνικών, που χρησιμοποιούνταν  μέχρι το 2011, παρουσιάζεται στο \en\cite{Aggarwal:2011:HAA:1922649.1922653}\gr.  \par

Τα πρόσφατα αποτελέσματα σε βαθιές αρχιτεκτονικές και ειδικά στον τομέα της ταξινόμηση εικόνας  έδωσε κίνητρο στους ερευνητές να εκπαιδεύσουν δίκτυα \tl{CNN} για
το πρόβλημα της αναγνώρισης δράσης. Η πρώτη σημαντική απόπειρα έγινε από τους \en\cite{6909619}\gr. Σχεδίασαν την αρχιτεκτονική τους με βάση το καλύτερο \tl{CNN}
στον ανταγωνισμό της \tl{ImageNet}. % έχουν διερευνήσει διάφορες μεθόδους για τη σύντηξη των χωρικών χαρακτηριστικών χρησιμοποιώντας 2D λειτουργίες κυρίως και 3D συνέλιξη μόνο σε αργή σύντηξη.
% cite{ssinιy2014 δύο χρησιμοποίησαν ένα 2 CNNs, ένα για χωρική πληροφορίες και ένα για την οπτική ροή και συνδύασε τα με τη χρήση της καθυστερημένης σύντηξης.
% Δείχνουν ότι η εξόρυξη χωρικών περιβάλλοντος από βίντεο και το περιβάλλον κίνησης από την οπτική ροή μπορεί να βελτιώσει σημαντικά την ακρίβεια της αναγνώρισης δράσης.
% cite{DBLP: εγγραφές/Corr/FeichtenhoferPZ16} επεκτείνουν αυτή την προσέγγιση χρησιμοποιώντας την πρώιμη σύντηξη στο τέλος των convolutional στρωμάτων, αντί της καθυστερημένης σύντηξης που
% λαμβάνει θέσεις στο τελευταίο επίπεδο του δικτύου. Από πάνω, χρησιμοποίησαν ένα δεύτερο δίκτυο για το χρονικό πλαίσιο, το οποίο ασφαλούσε με το άλλο δίκτυο χρησιμοποιώντας
% Σύντηξης. Επιπλέον, ccite{dblp: εγγραφές/Corr/WangXW0LTG16} με βάση τη μέθοδο τους στο ccite{ssinymrincmn δύο}. Ασχολούνται με το πρόβλημα της σύλληψης μεγάλης εμβέλειας
% πλαίσιο και την κατάρτιση του δικτύου τους, τους δόθηκαν περιορισμένα δείγματα εκπαίδευσης. Η προσέγγισή τους, την οποία ονόμασαν δίκτυο χρονικών τμημάτων (TSN), διαχωρίζει
% βίντεο σε τμήματα K και ένα σύντομο απόσπασμα από κάθε τμήμα επιλέγεται για ανάλυση. Στη συνέχεια, ασφαλίζονται το εξαγόμενο χωροχρονικό πλαίσιο, καθιστώντας, τελικά, την
% Πρόβλεψη.
% Πιο πρόσφατα, cite{DBLP: εγγραφές/Corr/ZhangWWQW16} και cite{DBLP: εγγραφές/Corr/ZhuLNH17a} χρησιμοποιήθηκαν και για την προσέγγιση δύο ροών. cite{DBLP: εγγραφές/Corr/ZhangWWQW16} αντικαταστήστε την οπτική ροή με διάνυσμα κίνησης που μπορεί να ληφθεί απευθείας από συμπιεσμένα βίντεο χωρίς επιπλέον υπολογισμό και τροφοδοτούν το. cite{DBLP: εγγραφές/Corr/ZhuLNH17a} εκπαίδευσε ένα CNN για τον υπολογισμό της οπτικής ροής, καλώντας το
% MotionNet και χρησιμοποιήστε ένα κροταφικό ρεύμα CNN για πληροφορίες κίνησης έργου σε ετικέτες δράσης. Τέλος, χρησιμοποιούν την καθυστερημένη σύντηξη μέσω του σταθμισμένου μέσου όρου των βαθμολογιών πρόβλεψης των χρονικών και χωρικών ρευμάτων. Από την άλλη πλευρά, εισήχθη μια νέα προσέγγιση από το cite{DBLP: εγγραφές/Corr/ABS-1711-01467} που ενσωματώνουν χάρτες προσοχής για να δώσουν σημαντική βελτίωση στην απόδοση της αναγνώρισης δράσης  par 

% Ορισμένες άλλες μέθοδοι περιλάμβαναν ένα δίκτυο RNN ή LSTM για ταξινόμηση όπως cite{DBLP: εγγραφές/Corr/DonahueHGRVSD14}, cite{DBLP: εγγραφές/Corr/NgHVVMT15} και cite{DBLP: εγγραφές/Corr/MaCKA17}.  cite{DBLP: εγγραφές/Corr/DonahueHGRVSD14} αντιμετωπίζουν την πρόκληση του μεταβλητού μήκους του
% ακολουθίες εισόδου και εξόδου, εκμετάλλευση convolutional στρώσεων και χρονικές ανατομές μεγάλης εμβέλειας. Προτείνουν μια μακροπρόθεσμη επαναλαμβανόμενη
% Convolutional Network (ΣΑΑ), το οποίο είναι ικανό να ασχοληθεί με τα καθήκοντα της αναγνώρισης ενέργειας, της λεζάντας εικόνας και της περιγραφής βίντεο. Για να ταξινομήσετε μια δεδομένη ακολουθία πλαισίων, η ΣΑΙΡΣΟ αρχικά λαμβάνει ως εισαγωγή ένα πλαίσιο, και ιδίως τα κανάλια RGB και την οπτική ροή και προβλέπει μια ετικέτα κλάσης. Μετά από αυτό, εξάγει κλάση βίντεο με μέσο όρο πιθανότητες ετικέτας, επιλέγοντας την πιο πιθανή κλάση.
% cite{DBLP: εγγραφές/Corr/NgHVVMT15} πρώτα Εξερευνήστε διάφορες προσεγγίσεις για τη συγκέντρωση χρονικών δυνατοτήτων. Αυτές οι τεχνικές περιλαμβάνουν τη διαχείριση βίντεο
% πλαίσια ξεχωριστά από 2 αρχιτεκτονικές CNN: είτε η Αλεξέ ή το GoogleNet, και αποτελούνταν από πρώιμη σύντηξη, η καθυστερημένη σύντηξη ενός συνδυασμού
% Τους. Επιπλέον, προτείνουν μια υποτροπιάζουσα νευρωνική αρχιτεκτονική δικτύου, προκειμένου να εξετάσουν τα βίντεο κλιπ ως ακολουθίες ενεργοποιήσεων CNN.
% Η προτεινόμενη LSTM λαμβάνει μια εισροή της εξόδου του τελικού στρώματος CNN σε κάθε συνεχόμενο πλαίσιο βίντεο και μετά από πέντε στοιβαγμένα επίπεδα LSTM με τη χρήση ενός
% Η ταξινόμηση με μαλακό Max, προτείνει μια ετικέτα κλάσης. Για την ταξινόμηση βίντεο, επιστρέφουν μια ετικέτα μετά το τελευταίο βήμα, Max-pool τις προβλέψεις
% με την πάροδο του χρόνου, αθροίστε τις προβλέψεις με την πάροδο του χρόνου και επιστρέψτε το μέγιστο ή γραμμικά βάρος τις προβλέψεις με την πάροδο του χρόνου από έναν παράγοντα g, αθροίστε τους και επιστρέψτε το μέγ.
% Έδειξαν ότι όλες οι προσεγγίσεις είναι 1\% διαφορετικές με μια προκατάληψη για τη χρήση προβλέψεων στάθμισης για την υποστήριξη της ιδέας ότι το LSTM γίνεται προοδευτικά πιο ενημερωμένο. Τελευταίο αλλά όχι λιγότερο σημαντικό, cite{DBLP: εγγραφές/Corr/MaCKA17} Χρησιμοποιήστε ένα διέδι για εκχύλιση με δυνατότητες και είτε ένα LSTM ή convolutional στρώματα πάνω από χρονικά κατασκευασμένες μήτρες, για τη σύντηξη χωρικών και χρονικών πληροφοριών. Χρησιμοποιούν ένα ResNet-101 για
% την εξόρυξη χαρτών δυνατοτήτων τόσο για χωρικές όσο και για χρονικές ροές. Διαιρούν τα καρέ βίντεο σε πολλά τμήματα όπως cite{DBLP: εγγραφές/Corr/WangXW0LTG16} και χρησιμοποιούν ένα επίπεδο χρονικής ομαδοποίησης για την εξαγωγή διακεκριμένων δυνατοτήτων. Λήψη αυτών των δυνατοτήτων, LSTM εξάγει ενσωματωμένα χαρακτηριστικά από όλα τα τμήματα.  par

% Επιπλέον, Ccite{tran2014 εκμάθηση 3D Convolutional δίκτυα (cite{PMID: 22392705}) και εισήγαγε C3D δίκτυο το οποίο έχει
% 3D convolutional στρώματα με πυρήνες $3  φορές 3  φορές $3.
% Αυτό το δίκτυο είναι σε θέση να μοντέλο εμφάνιση και το περιβάλλον κίνησης ταυτόχρονα χρησιμοποιώντας 3D περιελιγμοί και μπορεί να χρησιμοποιηθεί ως ένας απορροφητήρας δυνατότητα, πάρα πολύ.
% Συνδυασμός αρχιτεκτονικής δύο ροών και 3D περιελιγμοί, cite{DBLP: εγγραφές/Corr/CarreiraZ17} που προτείνονται
% Δίκτυο I3D. Επιπλέον, οι συγγραφείς τονίζουν τα πλεονεκτήματα της μεταφοράς μάθησης για το έργο της αναγνώρισης δράσης με την επανάληψη 2D προ-εκπαιδευμένο βάρη
% στην 3ο διάσταση. cite{DBLP: εγγραφές/Corr/ABS-1708-07632} πρότεινε ένα δίκτυο 3D ResNet για την αναγνώριση ενέργειας που βασίζεται σε εναπομείναντα δίκτυα (ResNet)
% (cite{DBLP: εγγραφές/Corr/HeZRS15}) και Εξερευνήστε την αποτελεσματικότητα του ResNet με 3D Convolutional πυρήνες.
% Από την άλλη πλευρά, cite{DBLP: εγγραφές/Corr/ABS-1711-08200} με βάση την προσέγγισή τους στο DenseNets (cite{DBLP: εγγραφές/Corr/HuangLW16a}) και επεκτείνετε
% DenseNet αρχιτεκτονική χρησιμοποιώντας 3D φίλτρα και ομαδοποίηση πυρήνες αντί 2D, ονοματοδοσίας αυτής της προσέγγισης ως DenseNet3D. Τα περισσότερα, εισάγουν
% Επίπεδο χρονικής μετάβασης (TTL), το οποίο συνενώνει χρονικά χαρακτηριστικά-χάρτες που εξάγονται σε διαφορετικές χρονικές περιοχές βάθους και αντικαθιστά την
% επίπεδο μετάβασης. Στην κορυφή αυτού του αρχείου cite{DBLP: DibaFSKAYG18} εισήχθη μια νέα χρονική στρώση που μοντέλα μεταβλητού χρονικού βάθους πυρήνα.
% Τελευταίο αλλά όχι λιγότερο σημαντικό, cite{DBLP: εγγραφές/Corr/ABS-1711-11248} πειραματιστείτε με αρκετές αρχιτεκτονικές δικτύου υπολειπόμενες χρησιμοποιώντας συνδυασμούς 2D και 3D convolutional Layer. Ο σκοπός τους είναι
% για να δείξει ότι ένα 2D χωρική συνέλιξη ακολουθούμενη από μια 1D κροταφική συνέλιξη επιτυγχάνει την κατάσταση της απόδοσης της ταξινόμησης της τέχνης, ονοματοδοσίας
% Αυτός ο τύπος επιπέδου συγέλιξης ως R (2 + 1) D. 
% Πρόσφατα Cite{guo_2018 _ecccv} πρότεινε ένα πλαίσιο το οποίο μπορεί να μάθει να αναγνωρίζει μια προγενέστερη κλάση δράσης 3D με λίγα μόνο παραδείγματα
% αξιοποιώντας την εγγενή δομή των δεδομένων 3D μέσω γραφικής αναπαράστασης. Μια πιο λεπτομερής παρουσίαση για τις τεχνικές αναγνώρισης δράσης που χρησιμοποιούνται μέχρι το 2018
% cite{DBLP: εγγραφές/Corr/ABS-1806-11230}.

\subsection{Εντοπισμός Δραστηριότητας}

% Όπως προαναφέρθηκε, η μετάφραση ενέργειας μπορεί να θεωρηθεί ως προέκταση του προβλήματος εντοπισμού αντικειμένου. Αντί να επισκιάζω 2D οριοθέτησης
% πλαισίων σε μία μόνο εικόνα, ο στόχος των συστημάτων τοπικοποίησης δράσης είναι οι σωλήνες δράσης εξόδου που είναι ακολουθίες πλαισίων που
% περιέχουν μια ενέργεια που εκτελέστηκε. Έτσι, υπάρχουν διάφορες προσεγγίσεις που περιλαμβάνουν ένα δίκτυο ανιχνευτή αντικειμένων για ένα πλαίσιο
% πρόταση δράσης και μια τάξη.  par
% Οι πρώτες προσεγγίσεις ανίχνευσης αντικειμένων περιλάμβαναν την επέκταση ενός αλγορίθμου πρότασης αντικειμένου σε 3 διαστάσεις. cite{6619185} Extended μοντέλα παραμορφώσιμου Part (cite{6619185}) με τη θεραπεία ενεργειών ως χωροχρονικών μοτίβων και τη δημιουργία ενός παραμορφώσιμου τμήματος για κάθε ενέργεια. cite{6909495} εισήγαγε την έννοια των tubelets, γνωστή και ως ακολουθίες πλαισίων οριοθέτησης και βασίστηκε στη μέθοδό τους στον επιλεκτικό αλγόριθμο αναζήτησης
% (Ccite{uijlings13}), επεκτείνοντας τα υπερ-εικονοστοιχεία σε υπερ-voxels για την παραγωγή χωροχρονικών σχημάτων. Από την άλλη πλευρά, cite{Oneata}
% επεκτείνετε μια τυχαιοποιημένη διαδικασία συγχώνευσης υπερpixel που χρησιμοποιήθηκε για προτάσεις αντικειμένου όπως παρουσιάζονται από cite{Manen: 2013: POP: 2586117.2587333}.
% cite{7298735} προτείνουν πρώτα πλαίσια οριοθέτησης για κάθε πλαίσιο με τη χρήση ανιχνευτή ανθρώπινου και κίνησης και, στη συνέχεια, επιλέγοντας τα πλαίσια οριοθέτησης με την καλύτερη βαθμολογία,
% πρότειναν έναν άπληστο αλγόριθμο σύνδεσης, διατυπώνοντας την εργασία που του αρέσει ως μέγιστο πρόβλημα κάλυψης. Cite{BMW vc2015 _177} παράγουν χωροχρονικές προτάσεις απευθείας από πυκνές τροχιές, οι οποίες επίσης χρησιμοποιούνται για ταξινόμηση. cite{7410734} Δημιουργήστε μια χωροχρονική τροχιά
% και να επιλέξουν προτάσεις δράσης με βάση μόνο την εσκεμμένη κίνηση που εξάγεται από το γράφημα. cite{7410732} Διαχωρίστε τα τμήματα βίντεο
% σε supervoxels και να χρησιμοποιούν το πλαίσιό τους ως χωρική σχέση μεταξύ supervoxels σε σχέση με την δράση του πρώτου πλάνου. Δημιουργούν ένα γράφημα για κάθε
% βίντεο, όπου τα supervoxels σχηματίζουν τες και οι κατευθυνμένες άκρες αιχμαλωτίζει τις χωρικές σχέσεις μεταξύ τους. Κατά τη διάρκεια των δοκιμών, εκτελούν ένα περιβάλλον
% με τα πόδια όπου κάθε βήμα καθοδηγείται από τις σχέσεις περιβάλλοντος που διδαχθήκαμε κατά τη διάρκεια της εκπαίδευσης, με αποτέλεσμα μια κατανομή πιθανότητας μιας ενέργειας σε όλα τα supervoxels. cite{DBLP: εγγραφές/Corr/MettesGS16}, αντί για σχολιασμό σε όλα τα πλαίσια, σχολιάστε σημεία σε ένα κατακερματισμένο υποσύνολο του βίντεο
% και να χρησιμοποιούν προτάσεις που λαμβάνονται με ένα μέτρο επικάλυψης μεταξύ των προτάσεων δράσης και των σημείων. cite{DBLP: εγγραφές/Corr/BehlSSSCT17} ασχολούνται με
% ηλεκτρονική ανίχνευση και εντοπισμό ενεργειών, με τη λήψη πρότασης δράσης ανά καρέ και την πρόταση ενός αλγορίθμου σύνδεσης που είναι σε θέση να κατασκευάσει και να ενημερώσει τους σωλήνες ενεργειών σε κάθε πλαίσιο.\par

% Η εισαγωγή του R-CNN (cite{DBLP: εγγραφές/Corr/GirshickDDM13}) επιτυγχάνει σημαντική βελτίωση
% στις επιδόσεις των δικτύων εντοπισμού αντικειμένων. Η αρχιτεκτονική αυτή, πρώτον, προτείνει περιφέρειες της εικόνας που είναι πιθανόν να
% περιέχουν ένα αντικείμενο και στη συνέχεια τα ταξινομεί χρησιμοποιώντας μια ταξινόμηση SVM. Εμπνευσμένη από αυτήν την αρχιτεκτονική, cite{DBLP: εγγραφές/Corr/GkioxariM14}
% σχεδιάσετε ένα δίκτυο RCNN 2 ροών για να δημιουργήσετε προτάσεις δράσης για κάθε πλαίσιο, μία ροή για επίπεδο καρέ και μία για οπτική ροή.
% Στη συνέχεια, τα συνδέουν χρησιμοποιώντας τον αλγόριθμο σύνδεσης Viterbi. cite{DBLP: εγγραφές/Corr/WeinzaepfelHS15} επεκτείνουν αυτή την προσέγγιση, εκτελώντας
% προτάσεις σε επίπεδο καρέ και χρησιμοποιώντας ένα tracker για τη σύνδεση αυτών των προτάσεων χρησιμοποιώντας τόσο χωρική όσο και οπτική ροή. Επίσης, η μέθοδός τους εκτελεί
% χρησιμοποιώντας ένα συρόμενο παράθυρο πάνω από τους εντοπισμένες σωλήνες. Πιο πρόσφατα, το cite{8237344} προσπάθησε να ασχοληθεί με το πρόβλημα της μη εποπτευόμενων
% εντοπισμού και εντοπισμού ενεργειών. Η προσέγγισή τους περιελάμβανε την εξαγωγή της κατάτμησης supervoτρέξελ και στη συνέχεια την εκχώρηση βάρους σε κάθε supervoτρέξελ.
% Χρησιμοποιώντας supervoxels που έχουν εξαχθεί, δημιουργούν ένα γράφημα και, στη συνέχεια, χρησιμοποιώντας μια προσέγγιση συμπλέγματος διακρίσεις μια τάξη είναι εκπαιδευμένη.  \par

% Η εισαγωγή της ταχύτερης RCNN (cite{Ren: 2015: FRT: 2969239.2969250}) συνεισφέρει πολλά στη βελτίωση της απόδοσης των δικτύων μετάφρασης Actiol
% ccite{PΙ: HAL-01349107}, cite{DBLP: εγγραφές/Corr/SahaSSTC16} και χρήση ταχύτερης R-CNN αντί για RCNN
% για προτάσεις σε επίπεδο καρέ, χρησιμοποιώντας RPN για εικόνες RGB και οπτικής ροής.
% Μετά τις προτάσεις για χωρικές και κινήσεις κίνησης, cite{tόγκ: HAL-01349107} την ασφάλεια της εξερεύνησης και από κάθε προτεινόμενη ROI, παράγουν 4 ROIs για να επικεντρωθεί σε συγκεκριμένες
% μέρη του σώματος του ηθοποιού. Μετά από αυτό, συνδέουν την πρόταση χρησιμοποιώντας τον αλγόριθμο Viterbi για κάθε κλάση και να εκτελέσει χρονική μετάφραση, χρησιμοποιώντας ένα συρόμενο παράθυρο, με πολλαπλές
% χρονικές κλίμακες και διασκελισμό με τη μέγιστη μέθοδο υποσυστοιχίας. Από την άλλη πλευρά, cite{DBLP: εγγραφές/Corr/SahaSSTC16} εκτελούν, επίσης, ταξινόμηση σε επίπεδο καρέ. Μετά από αυτό,
% η μέθοδός τους εκτελεί σύντηξη με βάση ένα συνδυασμό μεταξύ των βαθμολογιών της εμφάνισης και της πρότασης με βάση τις προτάσεις και τη βαθμολογία αλληλεπικάλυψης τους. Τέλος, η χρονική εντοπισμός
% λαμβάνει χώρα χρησιμοποιώντας δυναμικό προγραμματισμό. Από την άλλη πλευρά, η χρήση cite{DBLP: εγγραφές/Corr/WeinzaepfelMS16} χρησιμοποιεί
% Ταχύτερη RCNN για την εξόρυξη ανθρώπινων σωλήνων από βίντεο που επικεντρώνονται σε ασθενώς εποπτεύεται πρόβλημα εντοπισμού δράσης.
% Στη συνέχεια, η χρήση πυκνών διαδρομών και μια πολλαπλή μαθησιακή προσέγγιση πολλαπλών παρουσιών (cite{7420739}) εκπαιδεύουν μια τάξη.
% cite{DBLP: εγγραφές/Corr/MettesS17} εισήγαγε μια μέθοδο για την εντοπισμό ενέργειας με μηδενική δόση. Η προσέγγισή τους περιλαμβάνει τη βαθμολόγηση των προτεινόμενων σωλήνων δράσης σύμφωνα με τις αλληλεπιδράσεις
% φορείς και τοπικά αντικείμενα. Χρησιμοποίησαν το γρηγορότερο-RCNN, στο πρώτο βήμα, για την ανίχνευση τόσο των ηθοποιών όσο και των αντικειμένων και στη συνέχεια τη χρήση χωρικών σχέσεων μεταξύ τους συνδέουν τα προτεινόμενα κουτιά πάνω
% χρόνο με βάση την πιθανότητα μηδενικού κινδύνου από την παρουσία παραγόντων, συναφών αντικειμένων γύρω από τους παράγοντες και τις αναμενόμενες χωρικές σχέσεις μεταξύ αντικειμένων και παραγόντων.
% Επιπλέον, Ccite{dblp: εγγραφές/Corr/HeIDM17} πρότεινε το δίκτυο πρόταση Tube (TPN) για τη δημιουργία γενικών προτάσεις tubelet ανεξάρτητη κλάση, η οποία χρησιμοποιεί ταχύτερη-RCNN για να πάρει
% 2D προτάσεις περιφέρειας και έναν αλγόριθμο σύνδεσης για τη σύνδεση tubelets με αυτές τις προτάσεις περιοχή. Πιο πρόσφατα, cite{DBLP: εγγραφές/Corr/ABS-1807-10066} πρότεινε μια μέθοδο μετάφρασης ενεργειών
% στο σύνολο δεδομένων AVA (cite{DBLP: εγγραφές/Corr/GuSVPRTLRSSM17}) συνδυάζοντας I3D (cite{DBLP: εγγραφές/Corr/CarreiraZ17}) και αρχιτεκτονικές ταχύτερα RCNN. Χρησιμοποιούν μπλοκ I3D για να πάρει αναπαράσταση βίντεο
% και RCNN του Fast-RCNN για την παραγωγή "πρόσωπο" προτάσεις για το κεντρικό πλαίσιο.\par

% Πάνω από αυτό, cite{singh2016 στη διεύθυνση και ccite{kictatesdinfon17: HAL-01519812} Σχεδιάστε τα δίκτυά τους με βάση τον ανιχνευτή πολλαπλών πλαισίων μονού πυροβολισμό cite{DBLP: εγγραφές/Corr/LiuAESR15}).
% η διεύθυνση cite{singh2016 στο διαδίκτυο δημιούργησε ένα διαδικτυακό χωροκροταφικό δίκτυο σε πραγματικό χρόνο. Προκειμένου το δίκτυό τους να εκτελέσει σε πραγματικό χρόνο, cite{singh2016 στο διαδίκτυο προτείνει ένα νέο και αποτελεσματικό αλγόριθμο
% προσθέτοντας κουτιά σε σωλήνες σε κάθε πλαίσιο, εάν επικαλύπτονται περισσότερο από ένα κατώφλι, ή εναλλακτικά, τερματίζουν το σωλήνα δράσης, εάν για τα καρέ k δεν προστέθηκε κουτί.  ccite{καλογειτονιά 17iccv: HAL-01519812}
% σχεδίασε ένα δίκτυο δύο ροών, το οποίο αποκαλούσαν ανιχνευτή ΔΡΆΣΗς, και εισήγαγε κουμποειδή αγκύρωσης. Για καρέ K, και για τα δύο δίκτυα, ccite{kicyyesdinfon17: HAL-01519812} απόσπασμα χωρική
% χαρακτηριστικά σε επίπεδο καρέ και στη συνέχεια στοιβάζονται αυτές τις δυνατότητες. Τέλος, με τη χρήση αγκυρών άγκυρες, το δίκτυο εξάγει tubelets, δηλαδή μια ακολουθία κουτιών, με την αντίστοιχη ταξινόμησή τους
% βαθμολογίες και στόχους παλινδρόμησης. Για τη σύνδεση των tubelets, cite{καλογειτονιά 17iccv: HAL-01519812} ακολουθήστε τα ίδια βήματα με το cite{singh2016 στο διαδίκτυο. Για χρονική εντοπισμό, χρησιμοποιούν
% μια διαχρονική προσέγγιση εξομάλυνσης.  \par

% Πιο πρόσφατα, το δίκτυο YOLO (cite{DBLP: εγγραφές/Corr/RedmonDGF15}) αποτέλεσε την έμπνευση για το cite{DBLP: εγγραφές/Corr/ABS-1903-00304} και
% cite{DBLP: εγγραφές/Corr/ABS-1802-08362}. In cite{DBLP: εγγραφές/Corr/ABS-1903-00304}, έννοιες εξέλιξης και προόδου
% ποσοστό που θεσπίστηκε. Εκτός από την πρόταση πλαισίων οριοθέτησης σε επίπεδο καρέ, χρησιμοποιούν το YOLO μαζί με μια ταξινόμηση RNN για την εξαγωγή χρονικών πληροφοριών για τις προτάσεις.
% Με βάση αυτές τις πληροφορίες, δημιουργούν σωλήνες δράσης, χωρίζονται σε τάξεις. Ορισμένες άλλες προσεγγίσεις περιλαμβάνουν την εκτίμηση όπως cite{DBLP: εγγραφές/Corr/ABS-1802-09232}. 
% Πρότειναν μια μέθοδο για calcualatin 2D και 3D ποζάρει και στη συνέχεια εκτέλεσε την ταξινόμηση της δράσης. Χρησιμοποιούν τη διάχυση Soft-argamax λειτουργία για την εκτίμηση 2D και 3D αρθρώσεις, επειδή
% η λειτουργία αργμέγ δεν είναι διαφορή. Στη συνέχεια, για το T παρακείμενες στάσεις, δημιουργούν μια αναπαράσταση εικόνας με το χρόνο και $N _j $ ενώνει ως $x-y $ διαστάσεις και έχοντας 2 channes για 2D ποζάρει και 3
% καναλιών για 3D ποζάρει. Χρησιμοποιούν Convolutional στρώματα για να παράγουν δράση θερμαίνει και στη συνέχεια χρησιμοποιώντας μέγιστη συν-ελάχιστη συγκέντρωση και μια ενεργοποίηση μαλακή μέγ εκτελούν ταξινόμηση δράσης.
% cite{DBLP: εγγραφές/Corr/ZolfaghariOSB17} πρότεινε μια αρχιτεκτονική τριών ροών που περιλαμβάνει 2D πόζα, οπτική ροή και πληροφορίες RGB. Αυτά τα ρεύματα ενσωματώνονται διαδοχικά μέσω ενός
% μοντέλο αλυσίδας. Επιπλέον, το cite{8237881} πρότεινε μια αρχιτεκτονική με τη χρήση ενός χρονικού δικτύου παλινδρόμησης convolutional, για τη μακροπρόθεσμη εξάρτηση και περιβάλλοντα μεταξύ γειτονικών
% πλαισίων και ενός δικτύου χωρικών παλινδρόμησης,
% λήψη προτάσεων ανά καρέ. Χρησιμοποιούν μεθόδους παρακολούθησης και δυναμικό προγραμματισμό για τη δημιουργία προτάσεων δράσης.  par

% Τα περισσότερα από τα προαναφερθέντα δίκτυα χρησιμοποιούν ανά πλαίσιο χωρικές προτάσεις και εξάγουν τις χρονικές τους πληροφορίες υπολογίζοντας την οπτική ροή. Από την άλλη πλευρά, cite{DBLP: εγγραφές/Corr/SahaSC17} και cite{DBLP: εγγραφές/Corr/HouCS17} Σχεδιάστε μια αρχιτεκτονική η οποία έχει πρόταση icludes σε επίπεδο τμήματος βίντεο, χειρισμός περισσότερο του 1 καρέ ταυτόχρονα. cite{DBLP: εγγραφές/Corr/SahaSC17} πρότεινε ένα 3D-RPN το οποίο
% είναι σε θέση να δημιουργήσει και να ταξινομήσει προτάσεις της περιφέρειας 3D αποτελούνταν από δύο διαδοχικά πλαίσια. Επίσης, πρότειναν έναν αλγόριθμο σύνδεσης, τροποποιώντας αυτόν που προτάθηκε από το cite{DBLP: εγγραφές/Corr/SahaSSTC16}.
% Επιπλέον, cite{DBLP: εγγραφές/Corr/HouCS17} Σχεδιάστε μια αρχιτεκτονική για τη δημιουργία προτάσεων δράσης για περισσότερα από 2 καρέ, τα οποία αποκαλούσαν κανάλι CNN (T-CNN). Στην προσέγγισή τους, το επίπεδο τμήματος βίντεο σημαίνει ότι ολόκληρο το βίντεο χωρίζεται σε ίδια μήκη κλιπ βίντεο και
% χρησιμοποιώντας ένα C3D για την εξόρυξη χαρακτηριστικών, επιστρέφει χωροχρονικές προτάσεις. Μετά τη λήψη προτάσεων, cite{DBLP: εγγραφές/Corr/HouCS17} Συνδέστε τις προτάσεις του σωλήνα με έναν αλγόριθμο που βασίζεται σε σωλήνες '
% Βαθμολογία και επικάλυψη. Τέλος, η λειτουργία ταξινόμησης πραγματοποιείται για τις συνδεδεμένες προτάσεις βίντεο.



\printbibliography

\end{document}