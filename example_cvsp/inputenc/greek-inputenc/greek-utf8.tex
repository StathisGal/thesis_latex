\documentclass[a4paper]{scrartcl}
\usepackage{cmap} % fix search and cut-and-paste in Acrobat Reader
\usepackage[LGR,T1]{fontenc}
\usepackage[utf8]{inputenc}
\usepackage{textalpha}
\usepackage{alphabeta}
\usepackage{textcomp}
\usepackage[colorlinks]{hyperref}
\usepackage{bookmark}
% \usepackage{parskip}
\usepackage{booktabs}

\usepackage{lmodern}
% \usepackage{kerkis}
% \usepackage{gfsdidot}

% Greek utf8 definitions work with and without "Babel", 
% with monotonic, polytonic, and ancient Greek variants.
% However, this document uses babel language switches, so it requires Babel:
\usepackage[greek,english]{babel}
% \languageattribute{greek}{polutoniko}
\languageattribute{greek}{ancient}

\begin{document}

\title{Greek Unicode with 8-bit TeX and \emph{inputenc}}
\author{Günter Milde}
\maketitle

\begin{abstract}
The definitions in \texttt{lgrenc.dfu} provide UTF-8 support for
the Greek script based on \emph{inputenc} and the \emph{LaTeX internal
character representation} macros (LICRs) defined in the \emph{greek-fontenc}
package.
\end{abstract}

\section{Requirements}

The \href{http://www.ctan.org/pkg/inputenc}{\emph{inputenc}} standard
package enables the use of non-ASCII characters with 8-bit TeX. However, it
misses definitions for Greek characters. The \emph{greek-inputenc} package
extends \emph{inputenc} to allow the use of Greek literals in the document
source.

As with all \emph{inputenc} definitions, this only works if the active font
encoding supports the characters. For the Greek script, this is usually the
non-standard \emph{LGR} font encoding set up by
\href{http://www.ctan.org/pkg/greek-fontenc}{\emph{greek-fontenc}}.
% e.g. Π produces:
% ! LaTeX Error: Command \textPi unavailable in encoding T1.
% just like Ж produces:
% ! LaTeX Error: Command \CYRZH unavailable in encoding T1.

\section{Usage}

There are several alternatives to activate Greek Unicode input for 8-bit TeX%
\footnote{
  The XeTeX and LuaTeX engines use utf8 as native input encoding. They do
  not require (and, except in 8-bit compatibility mode, do not work with)
  the \emph{inputenc} and \emph{greek-inputenc} packages.}
%
(see also the source document \url{greek-utf8.tex}):


\begin{itemize}

\item
Define the LGR font encoding and the UTF8 input encoding (the order does not
matter), e.g.,
%
\begin{verbatim}
  \usepackage[T1,LGR]{fontenc}
  \usepackage[utf8]{inputenc}
\end{verbatim}
%
Ensure that LGR is the active font encoding whenever a Greek character is
used in the text (see below).

\item
For text in the Greek language, it is recommended to use the
\href{http://www.ctan.org/pkg/babel}{\emph{Babel}} package with the Greek
language definitions in
\href{http://www.ctan.org/pkg/babel-greek}{\emph{babel-greek}}. Babel sets
the font encoding automatically to LGR and Greek Unicode characters work as
expected. Write in the preamble, e.g.,
%
\begin{verbatim}
  \usepackage[utf8]{inputenc}
  \usepackage[LGR,T1]{fontenc}
  \usepackage[english,greek,german]{babel}
\end{verbatim}
%
and use \verb+\foreignlanguage+ or \verb+\selectlanguage+ to set the text
language to Greek (see the
\href{http://www.ctan.org/pkg/babel-greek}{\emph{babel-greek}} documentation
for detailled examples).

\begin{quote}
  \greekscript
  Τί φήις; Ἱδὼν ἐνθέδε παῖδ’ ἐλευθέραν
  τὰς πλησίον Νύμφας στεφανοῦσαν, Σώστρατε,
  ἐρῶν άπῆλθες εὐθύς;
\end{quote}

\item
In combination with the \emph{textalpha} package from
\href{http://www.ctan.org/pkg/greek-fontenc}{\emph{greek-fontenc}}, Greek Unicode
characters can be used in text with any font encoding -- just like the
symbols provided by the ``textcomp'' package (i.e. with some limitations
described in
\href{http://mirrors.ctan.org/language/greek/greek-fontenc/textalpha-doc.pdf}%
{textalpha-doc}). With the preamble lines
%
\begin{verbatim}
  \usepackage[utf8]{inputenc}
  \usepackage{textalpha}
\end{verbatim}
%
it is straightforward to write about π-mesons, γ-radiation, or a 50\,kΩ
resistor.

\item
In combination with the \emph{alphabeta} package (also from
\href{http://www.ctan.org/pkg/greek-fontenc}{\emph{greek-fontenc}}),
Greek Unicode literals can also be used in math mode:
%
\begin{verbatim}
  \usepackage[utf8]{inputenc}
  \usepackage{alphabeta}
\end{verbatim}
\[
   \tan β = \frac{\sin β}{\cos β}.
\]

\end{itemize}

\section{Warning: unsafe ASCII input}

LGR is no ``standard font encoding''. Latin characters and some other ASCII
symbols are mapped to Greek equivalents if LGR is the active font encoding.
(See
\href{http://mirrors.ctan.org/language/babel/contrib/greek/usage.pdf}{usage.pdf}
for a description of this Latin-Greek transliteration.)

This means you need an explicit language and/or font-encoding switch for
Latin words and abbreviations in Greek text, e.g., not
\foreignlanguage{greek}{((ηία αντίσταση 750-kΩ))} but
\foreignlanguage{greek}{((ηία αντίσταση 750-\textlatin{k}Ω))}

Special care is also required with the question mark characters:
\begin{itemize}
  \item The Unicode standard says character 003B SEMICOLON and not 037E GREEK
  	QUESTION MARK, is the preferred character for a `Greek question
	mark' (erotimatiko),
  \item The LGR font encoding maps a SEMICOLON to a middle dot (ano teleia),
	while the Latin question mark ``?'' is mapped to the erotimatiko.
\end{itemize}
As a result, only the deprecated character 037E GREEK QUESTION MARK works
with both, Xe/LuaTeX and 8-bit TeX. Compare the source \url{greek-utf8.tex}
and the PDF output:

\begin{center}
\begin{tabular}{lll}
  Latin (T1) & Greek (LGR) & question mark character \\
  \midrule
  Τί φήις; & \foreignlanguage{greek}{Τί φήις;} & 037E GREEK QUESTION MARK \\
  Τί φήις; & \foreignlanguage{greek}{Τί φήις;} & 003B SEMICOLON \\
  Τί φήις? & \foreignlanguage{greek}{Τί φήις?} & 003F QUESTION MARK \\
\end{tabular}
\end{center}


\section{Supported Characters}

Unicode definitions exist for all non-ASCII characters that can be rendered
with an LGR-encoded font.


\subsection{Greek and Coptic}

\begin{tabular}{rrrrrrrrrrrrrrrrr}
\toprule
    & 0 & 1 & 2 & 3 & 4 & 5 & 6 & 7 & 8 & 9 & A & B & \textlatin C & D & E & F\\
\midrule
\textlatin{370} & * & * & * & * & ʹ & ͵ & * & * &   &   & ͺ & * & * & * & ; &  \\
\textlatin{380} &   &   &   &   & ΄ & ΅ & Ά & · & Έ & Ή & Ί &   & Ό &   & Ύ & Ώ\\
\textlatin{390} & ΐ & Α & Β & Γ & Δ & Ε & Ζ & Η & Θ & Ι & Κ & Λ & Μ & Ν & Ξ & Ο\\
\textlatin{3A0} & Π & Ρ &   & Σ & Τ & Υ & Φ & Χ & Ψ & Ω & Ϊ & Ϋ & ά & έ & ή & ί\\
\textlatin{3B0} & ΰ & α & β & γ & δ & ε & ζ & η & θ & ι & κ & λ & μ & ν & ξ & ο\\
\textlatin{3C0} & π & ρ & ς & σ & τ & υ & φ & χ & ψ & ω & ϊ & ϋ & ό & ύ & ώ &  \\
\textlatin{3D0} & * & * & * & * & * & * & * & * & Ϙ & ϙ & Ϛ & ϛ & Ϝ & ϝ & * & ϟ\\
\textlatin{3E0} & Ϡ & ϡ & * & * & * & * & * & * & * & * & * & * & * & * & * & *\\
\textlatin{3F0} & * & * & * & * & * & * & * & * & * & * & * & * & * & * & * & *\\
\bottomrule
\end{tabular}

 \noindent
legend: * glyph missing in LGR, [space] Unicode point not defined


\subsection{Greek Extended}

\begin{tabular}{rrrrrrrrrrrrrrrrr}
\toprule
      & 0 & 1 & 2 & 3 & 4 & 5 & 6 & 7 & 8 & 9 & A & B & \textlatin C & D & E & F\\
\midrule
\textlatin{1F00} & ἀ & ἁ & ἂ & ἃ & ἄ & ἅ & ἆ & ἇ & Ἀ & Ἁ & Ἂ & Ἃ & Ἄ & Ἅ & Ἆ & Ἇ\\
\textlatin{1F10} & ἐ & ἑ & ἒ & ἓ & ἔ & ἕ &   &   & Ἐ & Ἑ & Ἒ & Ἓ & Ἔ & Ἕ &   &  \\
\textlatin{1F20} & ἠ & ἡ & ἢ & ἣ & ἤ & ἥ & ἦ & ἧ & Ἠ & Ἡ & Ἢ & Ἣ & Ἤ & Ἥ & Ἦ & Ἧ\\
\textlatin{1F30} & ἰ & ἱ & ἲ & ἳ & ἴ & ἵ & ἶ & ἷ & Ἰ & Ἱ & Ἲ & Ἳ & Ἴ & Ἵ & Ἶ & Ἷ\\
\textlatin{1F40} & ὀ & ὁ & ὂ & ὃ & ὄ & ὅ &   &   & Ὀ & Ὁ & Ὂ & Ὃ & Ὄ & Ὅ &   &  \\
\textlatin{1F50} & ὐ & ὑ & ὒ & ὓ & ὔ & ὕ & ὖ & ὗ &   & Ὑ &   & Ὓ &   & Ὕ &   & Ὗ\\
\textlatin{1F60} & ὠ & ὡ & ὢ & ὣ & ὤ & ὥ & ὦ & ὧ & Ὠ & Ὡ & Ὢ & Ὣ & Ὤ & Ὥ & Ὦ & Ὧ\\
\textlatin{1F70} & ὰ & ά & ὲ & έ & ὴ & ή & ὶ & ί & ὸ & ό & ὺ & ύ & ὼ & ώ &   &  \\
\textlatin{1F80} & ᾀ & ᾁ & ᾂ & ᾃ & ᾄ & ᾅ & ᾆ & ᾇ & ᾈ & ᾉ & ᾊ & ᾋ & ᾌ & ᾍ & ᾎ & ᾏ\\
\textlatin{1F90} & ᾐ & ᾑ & ᾒ & ᾓ & ᾔ & ᾕ & ᾖ & ᾗ & ᾘ & ᾙ & ᾚ & ᾛ & ᾜ & ᾝ & ᾞ & ᾟ\\
\textlatin{1FA0} & ᾠ & ᾡ & ᾢ & ᾣ & ᾤ & ᾥ & ᾦ & ᾧ & ᾨ & ᾩ & ᾪ & ᾫ & ᾬ & ᾭ & ᾮ & ᾯ\\
\textlatin{1FB0} & ᾰ & ᾱ & ᾲ & ᾳ & ᾴ &   & ᾶ & ᾷ & Ᾰ & Ᾱ & Ὰ & Ά & ᾼ & ᾽ & ι & ᾿\\
\textlatin{1FC0} & ῀ & ῁ & ῂ & ῃ & ῄ &   & ῆ & ῇ & Ὲ & Έ & Ὴ & Ή & ῌ & ῍ & ῎ & ῏\\
\textlatin{1FD0} & ῐ & ῑ & ῒ & ΐ &   &   & ῖ & ῗ & Ῐ & Ῑ & Ὶ & Ί &   & ῝ & ῞ & ῟\\
\textlatin{1FE0} & ῠ & ῡ & ῢ & ΰ & ῤ & ῥ & ῦ & ῧ & Ῠ & Ῡ & Ὺ & Ύ & Ῥ & ῭ & ΅ & `\\
\textlatin{1FF0} &   &   & ῲ & ῳ & ῴ &   & ῶ & ῷ & Ὸ & Ό & Ὼ & Ώ & ῼ & ´ & ῾ &  \\
\bottomrule
\end{tabular}


\subsection{Other Unicode Blocks}

\begin{description}

\item [Latin-1 Supplement]: \ensuregreek{¨ « ¯ ´ · »}
\item [IPA Extensions]: \ensuregreek{ə} LATIN SMALL LETTER SCHWA
\item [Spacing Modifier Letters]:
      \ensuregreek{˘a} (here followed by letter alpha)
\item [General Punctuation]:
      \ensuregreek{– — ‘ ’ ‰} ZWNJ (zero width no joiner, prevents kerning
      and ligatures, e.g. \ensuregreek{A‌‌U} vs. \ensuregreek{AU} and
      \ensuregreek{'‌a} vs. \ensuregreek{'a})
\item [Currency Symbols]: \ensuregreek{€}
\item [Letter-like Symbols]: Ω  % OHM SIGN, preferred representation is 03A9
\item [Ancient Greek Numbers]: \ensuregreek{
      𐅄 % \textPiDelta{} % GREEK ACROPHONIC ATTIC FIFTY
      𐅅 % \textPiEta{}   % GREEK ACROPHONIC ATTIC FIVE HUNDRED
      𐅆 % \textPiChi{}   % GREEK ACROPHONIC ATTIC FIVE THOUSAND
      𐅇 % \textPiMu{}    % GREEK ACROPHONIC ATTIC FIFTY THOUSAND
      }
\end{description}

\newpage

\section{Test up/downcasing}

Capital Greek letters have diacritics (except the dialytika) to the left
(instead of above) and drop them in uppercase, e.g.
\foreignlanguage{greek}{μαΐστρος $\mapsto$ \MakeUppercase{μαΐστρος}}.

Tonos and dasia on the first vowel of a diphthong (\ensuregreek{άι, άυ, έι})
imply a \emph{hiatus}. A dialytika must be placed on the second
vowel if they are dropped (\ensuregreek{\MakeUppercase{\'ai, \'au, \'ei}}).

The auto-hiatus feature in lgrxenc.def works with the Latin
transcription and with character-macros (%
\ensuregreek{\MakeUppercase{%
  \'ai,
  \'\textalpha \textupsilon,
  \'\textepsilon \textiota
}})
and also if the first character is wrapped in \verb+\ensuregreek+ (as done by
the lgrenc.dfu definition for accented characters) or a literal Unicode
character
(\ensuregreek{\MakeUppercase{%
  \ensuregreek{\'\textalpha}\textiota,
  ά\textupsilon,
  ά\textiota
}})
but not if the second character of the diphthong is a Unicode literal
(\ensuregreek{\MakeUppercase{%
  \'\textalpha ι,
  άυ,
  \'\textepsilon ι
}}).

Therefore, the diaresis is missing in the following examples:
\ensuregreek{άυλος  $\mapsto$ \MakeUppercase{άυλος},
           ἄυλος  $\mapsto$ \MakeUppercase{ἄυλος},
           μάινα  $\mapsto$ \MakeUppercase{μάινα},
           κέικ,  $\mapsto$ \MakeUppercase{κέικ},
           ἀυπνία $\mapsto$ \MakeUppercase{ἀυπνία}}.

Fixing this shortcoming requires knowledge of what
\verb+\LGR@ifnextchar+ ``sees'' when the next character is an upcased
Unicode literal.

As an ugly workaround, use \verb+\textiota+ resp. \verb+\textupsilon+
for the character that should get the diaresis:
\ensuregreek{ἀ\textupsilon{}πνία $\mapsto$ \MakeUppercase{ἀ\textupsilon{}πνία}}.



The following subsections test MakeUppercase and MakeLowercase with all
characters defined in lgrenc.dfu:

\subsection{Greek and Coptic}

\newcommand{\GreekAndCoptic}{ʹ͵ͺ; ΄ ΅Ά·ΈΉΊΌΎΏΐΑΒΓΔΕΖΗΘΙΚΛΜΝΞΟΠΡΣΤΥΦΧΨΩΪΫϘϚϜϠ}
\newcommand{\greekandcoptic}{άέήίΰαβγδεζηθικλμνξοπρςστυφχψωϊϋόύώϙϛϝϟϡ}

Characters of the Greek and Coptic Unicode Block:
\begin{quote}
  \selectlanguage{greek}
  \GreekAndCoptic\\
  \greekandcoptic
\end{quote}
MakeUppercase:
\begin{quote}
  \selectlanguage{greek}
  \MakeUppercase{\GreekAndCoptic}\\
  \MakeUppercase{\greekandcoptic}
\end{quote}
Letters and ypogegrammeni upcased, tonos dropped, dialytika kept.

There is no capital Koppa in LGR, therefore \ensuregreek{ϟ} is left unchanged
with MakeUppercase.


MakeLowercase:

\begin{quote}
  \selectlanguage{greek}
  \MakeLowercase{\GreekAndCoptic}\\
  \MakeLowercase{\greekandcoptic}
\end{quote}

The lowercase of \ensuregreek{Σ} is the «auto-sigma» (\verb+\textautosigma+):
\ensuregreek{ΣΣ $\mapsto$ \MakeLowercase{ΣΣ}}. Add a ZWNJ or use the
\verb+\noboundary+ macro to prevent conversion to final sigma:
\ensuregreek{\MakeLowercase{ΣΣ‌}}. The lowercase of GREEK LETTER STIGMA
\ensuregreek{Ϛ} is \ensuregreek{\MakeLowercase{Ϛ}}.


\subsection{Greek extended}

MakeUppercase:

% \selectlanguage{greek}

\MakeUppercase{ ἀ ἁ ἂ ἃ ἄ ἅ ἆ ἇ Ἀ Ἁ Ἂ Ἃ Ἄ Ἅ Ἆ Ἇ }\\
\MakeUppercase{ ἐ ἑ ἒ ἓ ἔ ἕ     Ἐ Ἑ Ἒ Ἓ Ἔ Ἕ     }\\
\MakeUppercase{ ἠ ἡ ἢ ἣ ἤ ἥ ἦ ἧ Ἠ Ἡ Ἢ Ἣ Ἤ Ἥ Ἦ Ἧ }\\
\MakeUppercase{ ἰ ἱ ἲ ἳ ἴ ἵ ἶ ἷ Ἰ Ἱ Ἲ Ἳ Ἴ Ἵ Ἶ Ἷ }\\
\MakeUppercase{ ὀ ὁ ὂ ὃ ὄ ὅ     Ὀ Ὁ Ὂ Ὃ Ὄ Ὅ     }\\
\MakeUppercase{ ὐ ὑ ὒ ὓ ὔ ὕ ὖ ὗ   Ὑ   Ὓ   Ὕ   Ὗ }\\
\MakeUppercase{ ὠ ὡ ὢ ὣ ὤ ὥ ὦ ὧ Ὠ Ὡ Ὢ Ὣ Ὤ Ὥ Ὦ Ὧ }\\
\MakeUppercase{ ὰ ά ὲ έ ὴ ή ὶ ί ὸ ό ὺ ύ ὼ ώ     }\\
\MakeUppercase{ ᾀ ᾁ ᾂ ᾃ ᾄ ᾅ ᾆ ᾇ ᾈ ᾉ ᾊ ᾋ ᾌ ᾍ ᾎ ᾏ }\\
\MakeUppercase{ ᾐ ᾑ ᾒ ᾓ ᾔ ᾕ ᾖ ᾗ ᾘ ᾙ ᾚ ᾛ ᾜ ᾝ ᾞ ᾟ }\\
\MakeUppercase{ ᾠ ᾡ ᾢ ᾣ ᾤ ᾥ ᾦ ᾧ ᾨ ᾩ ᾪ ᾫ ᾬ ᾭ ᾮ ᾯ }\\
\MakeUppercase{ ᾰ ᾱ ᾲ ᾳ ᾴ   ᾶ ᾷ Ᾰ Ᾱ Ὰ Ά ᾼ ᾽ ι ᾿ }\\
\MakeUppercase{ ῀ ῁ ῂ ῃ ῄ   ῆ ῇ Ὲ Έ Ὴ Ή ῌ ῍ ῎ ῏ }\\
\MakeUppercase{ ῐ ῑ ῒ ΐ     ῖ ῗ Ῐ Ῑ Ὶ Ί   ῝ ῞ ῟ }\\
\MakeUppercase{ ῠ ῡ ῢ ΰ ῤ ῥ ῦ ῧ Ῠ Ῡ Ὺ Ύ Ῥ ῭ ΅ ` }\\
\MakeUppercase{     ῲ ῳ ῴ   ῶ ῷ Ὸ Ό Ὼ Ώ ῼ ´ ῾   }
\selectlanguage{english}

MakeLowercase:

% \selectlanguage{greek}
\MakeLowercase{ ἀ ἁ ἂ ἃ ἄ ἅ ἆ ἇ Ἀ Ἁ Ἂ Ἃ Ἄ Ἅ Ἆ Ἇ }\\
\MakeLowercase{ ἐ ἑ ἒ ἓ ἔ ἕ     Ἐ Ἑ Ἒ Ἓ Ἔ Ἕ     }\\
\MakeLowercase{ ἠ ἡ ἢ ἣ ἤ ἥ ἦ ἧ Ἠ Ἡ Ἢ Ἣ Ἤ Ἥ Ἦ Ἧ }\\
\MakeLowercase{ ἰ ἱ ἲ ἳ ἴ ἵ ἶ ἷ Ἰ Ἱ Ἲ Ἳ Ἴ Ἵ Ἶ Ἷ }\\
\MakeLowercase{ ὀ ὁ ὂ ὃ ὄ ὅ     Ὀ Ὁ Ὂ Ὃ Ὄ Ὅ     }\\
\MakeLowercase{ ὐ ὑ ὒ ὓ ὔ ὕ ὖ ὗ   Ὑ   Ὓ   Ὕ   Ὗ }\\
\MakeLowercase{ ὠ ὡ ὢ ὣ ὤ ὥ ὦ ὧ Ὠ Ὡ Ὢ Ὣ Ὤ Ὥ Ὦ Ὧ }\\
\MakeLowercase{ ὰ ά ὲ έ ὴ ή ὶ ί ὸ ό ὺ ύ ὼ ώ     }\\
\MakeLowercase{ ᾀ ᾁ ᾂ ᾃ ᾄ ᾅ ᾆ ᾇ ᾈ ᾉ ᾊ ᾋ ᾌ ᾍ ᾎ ᾏ }\\
\MakeLowercase{ ᾐ ᾑ ᾒ ᾓ ᾔ ᾕ ᾖ ᾗ ᾘ ᾙ ᾚ ᾛ ᾜ ᾝ ᾞ ᾟ }\\
\MakeLowercase{ ᾠ ᾡ ᾢ ᾣ ᾤ ᾥ ᾦ ᾧ ᾨ ᾩ ᾪ ᾫ ᾬ ᾭ ᾮ ᾯ }\\
\MakeLowercase{ ᾰ ᾱ ᾲ ᾳ ᾴ   ᾶ ᾷ Ᾰ Ᾱ Ὰ Ά ᾼ ᾽ ι ᾿ }\\
\MakeLowercase{ ῀ ῁ ῂ ῃ ῄ   ῆ ῇ Ὲ Έ Ὴ Ή ῌ ῍ ῎ ῏ }\\
\MakeLowercase{ ῐ ῑ ῒ ΐ     ῖ ῗ Ῐ Ῑ Ὶ Ί   ῝ ῞ ῟ }\\
\MakeLowercase{ ῠ ῡ ῢ ΰ ῤ ῥ ῦ ῧ Ῠ Ῡ Ὺ Ύ Ῥ ῭ ΅ ` }\\
\MakeLowercase{     ῲ ῳ ῴ   ῶ ῷ Ὸ Ό Ὼ Ώ ῼ ´ ῾   }
\selectlanguage{english}

\subsection{Other Unicode Blocks}

MakeUppercase does not change non-letter symbols and the letter shwa:
\begin{quote}
  \greekscript
  \MakeUppercase{¨ « ¯ ´ · »}
  \MakeUppercase{ə}
  \MakeUppercase{˘a}
  \MakeUppercase{– — ‘ ’ ‰ a‌u}
  \MakeUppercase{€}
  % \MakeUppercase{Ω}
  \MakeUppercase{
  𐅄 % GREEK ACROPHONIC ATTIC FIFTY
  𐅅 % GREEK ACROPHONIC ATTIC FIVE HUNDRED
  𐅆 % GREEK ACROPHONIC ATTIC FIVE THOUSAND
  𐅇 % GREEK ACROPHONIC ATTIC FIFTY THOUSAND
  }
\end{quote}
MakeLowercase does not change non-letter symbols, either:
\begin{quote}
  \greekscript
  \MakeLowercase{¨ « ¯ ´ · »}
  \MakeLowercase{ə}
  \MakeLowercase{˘A}
  \MakeLowercase{– — ‘ ’ ‰ A‌‌U}
  \MakeLowercase{€}
  % \MakeLowercase{Ω}
  \MakeLowercase{
  𐅄 % GREEK ACROPHONIC ATTIC FIFTY
  𐅅 % GREEK ACROPHONIC ATTIC FIVE HUNDRED
  𐅆 % GREEK ACROPHONIC ATTIC FIVE THOUSAND
  𐅇 % GREEK ACROPHONIC ATTIC FIFTY THOUSAND
  }
\end{quote}

\section{Test kerning/ligatures}


check for kerning and unwanted ligatures:

\begin{quote}
  \greekscript

Αἀα Αἁα Αἂα Αἃα Αἄα Αἅα Αἆα Αἇα ΑἈα ΑἉα ΑἊα ΑἋα ΑἌα ΑἍα ΑἎα ΑἏα

Αἐα Αἑα Αἒα Αἓα Αἔα Αἕα ΑἘα ΑἙα ΑἚα ΑἛα ΑἜα ΑἝα

Αἠα Αἡα Αἢα Αἣα Αἤα Αἥα Αἦα Αἧα ΑἨα ΑἩα ΑἪα ΑἫα ΑἬα ΑἭα ΑἮα ΑἯα

Αἰα Αἱα Αἲα Αἳα Αἴα Αἵα Αἶα Αἷα ΑἸα ΑἹα ΑἺα ΑἻα ΑἼα ΑἽα ΑἾα ΑἿα

Αὀα Αὁα Αὂα Αὃα Αὄα Αὅα ΑὈα ΑὉα ΑὊα ΑὋα ΑὌα ΑὍα

Αὐα Αὑα Αὒα Αὓα Αὔα Αὕα Αὖα Αὗα ΑὙα ΑὛα ΑὝα ΑὟα

Αὠα Αὡα Αὢα Αὣα Αὤα Αὥα Αὦα Αὧα ΑὨα ΑὩα ΑὪα ΑὫα ΑὬα ΑὭα ΑὮα ΑὯα

Αὰα Αάα Αὲα Αέα Αὴα Αήα Αὶα Αία Αὸα Αόα Αὺα Αύα Αὼα Αώα

Αᾀα Αᾁα Αᾂα Αᾃα Αᾄα Αᾅα Αᾆα Αᾇα Αᾈα Αᾉα Αᾊα Αᾋα Αᾌα Αᾍα Αᾎα Αᾏα

Αᾐα Αᾑα Αᾒα Αᾓα Αᾔα Αᾕα Αᾖα Αᾗα Αᾘα Αᾙα Αᾚα Αᾛα Αᾜα Αᾝα Αᾞα Αᾟα

Αᾠα Αᾡα Αᾢα Αᾣα Αᾤα Αᾥα Αᾦα Αᾧα Αᾨα Αᾩα Αᾪα Αᾫα Αᾬα Αᾭα Αᾮα Αᾯα

Αᾰα Αᾱα Αᾲα Αᾳα Αᾴα Αᾶα Αᾷα ΑᾸα ΑᾹα ΑᾺα ΑΆα Αᾼα Α᾽α Αια Α᾿α

Α῀α Α῁α Αῂα Αῃα Αῄα Αῆα Αῇα ΑῈα ΑΈα ΑῊα ΑΉα Αῌα Α῍α Α῎α Α῏α

Αῐα Αῑα Αῒα Αΐα Αῖα Αῗα ΑῘα ΑῙα ΑῚα ΑΊα Α῝α Α῞α Α῟α

Αῠα Αῡα Αῢα Αΰα Αῤα Αῥα Αῦα Αῧα ΑῨα ΑῩα ΑῪα ΑΎα ΑῬα Α῭α Α΅α Α`α

Αῲα Αῳα Αῴα Αῶα Αῷα ΑῸα ΑΌα ΑῺα ΑΏα Αῼα Α´α Α῾α

\end{quote}

\end{document}


Problems with text-extraction from PDF with Kerkis:

     0  1 2 3 4  5   6 7  8  9  A  B  C D E    F
370  *  * * *  ΄  ͵  * *         ι *  * *   ;
380            ΄ ΅  ΄Α   ΄Ε ΄Η  ΄Ι   ΄Ο   ΄Υ  ΄Ω
390   ΐ Α Β Γ ∆  Ε   Ζ Η Θ    Ι Κ  Λ Μ  Ν Ξ    Ο
3Α0  Π  Ρ   Σ Τ  Υ  Φ  Χ Ψ   Ω   Ϊ Ϋ  ά έ  ή    ί
3Β0  ΰ  α ϐ γ δ  ε   Ϲ η  ϑ   ι κ  λ  µ ν  ξ   ο
3῝0  π  ϱ ς σ τ  υ   ϕ χ  ψ  ω   ϊ ϋ  ό ύ  ώ
3∆0  *  * * * *  *   * *     Ϟ  Ϝ  ϝ  Ϝ ϝ  *   ϟ
3Ε0     ϡ * * *  *   * *  *  *  *  *  * *  *   *
3Φ0  *  * * * *  *   * *  *  *  *  *  * *  *   *


0387    GREEK ANO TELEIA    missing
03B6    zeta                      replaced by 03F9      GREEK CAPITAL LUNATE SIGMA SYMBOL
03B8    GREEK SMALL LETTER THETA  replaced by 03D1      GREEK THETA SYMBOL
03C1    GREEK SMALL LETTER RHO    replaced by 03F1      GREEK RHO SYMBOL
03C6    GREEK SMALL LETTER PHI    replaced by 03D5      GREEK PHI SYMBOL


and GFS Didot:

       0   1   2  3  4  5   6 7  8  9  A  B  C D  E   F
370 *      *   *  *   ´  ͵ *  *         ι *  * *   ;
380                   ´  ῆ Α
                         ´ ´    ´Ε ´Η  ´Ι   ´Ο   ´Υ  ´Ω
390 ῆ  ´ι Α   Β   Γ  ∆  Ε Ζ   Η  Θ  Ι  Κ  Λ Μ  Ν  Ξ   Ο
3Α0 Π      Ρ       Σ  Τ  Υ Φ   Χ Ψ   Ω   ῆ
                                        Ι ῆ
                                          Υ  ά έ  ή    ί
3Β0 ῆ ´υ  α    β  γ  δ  ε ζ   η  ϑ   ι κ  λ  μ ν  ξ   ο
 3῝0 π    ρ    ς  σ  τ  υ φ   χ  ψ  ω   ι
                                        ῆ υ
                                          ῆ  ό ύ ώ
3∆0 *      *   *  *  *  * *   *           Ϛ  Ϝ Ϝ  *   Ϟ
3Ε0            *  *  *  * *   *  *  *   * *  * *  *   *
3Φ0 *      *   *  *  *  * *   *  *  *   * *  * *  *   *
