\documentclass[10pt, twoside, a4paper]{cvsp-thesis}



\usepackage{graphicx}
\usepackage{fancyhdr}
\usepackage{geometry}
\usepackage{amsmath, amsfonts}
\usepackage{epstopdf}
\usepackage{algorithm}
\floatname{algorithm}{\textgreek{Αλγόριθμος}}
\usepackage{algorithmic}

\newcommand{\tl}{\textlatin}
\newcommand{\en}{\selectlanguage{english}}
\newcommand{\gr}{\selectlanguage{greek}}



\hyphenpenalty=424242
\sloppy
\usepackage{fancyhdr}
\fancyhf{}
\fancyhead[LE,RO]{\thepage}
\fancyhead[RE]{\leftmark}
\fancyhead[LO]{\rightmark}
\pagestyle{fancy}

% dika moy packages
\usepackage{hyperref}  % package for linking figures etc
\usepackage{enumitem}  % package for description with bullets
\usepackage{graphicx}  % package for importing images
\usepackage{mathtools} % package for math equation
\usepackage{mathrsfs}  % package for math font
\usepackage{indentfirst} % package for getting ident after section or paragraph
\usepackage{subcaption} % package for subfigures
\usepackage[export]{adjustbox}
\usepackage{longtable} % package for multi pages tables
\usepackage{multirow}  % package for tables, multirow
\usepackage{amssymb}
\usepackage{esvect}
\usepackage[
    backend=bibtex,
    citestyle=authoryear,
    % citestyle=authoryear-comp,
    % citestyle=authoryear-ibid,
    bibstyle=numeric,
    sorting=ynt,
    % style=numeric,
    % style=alphabetic ,
  ]{biblatex}
 \addbibresource{References}

% \includeonly{chapter1,chapter2_3,chapter4,chapter5,chapter6,chapter7,References}


\title{Αναγνώριση και εντοπισμός ανθρώπινης δραστηριότητας σε βίντεο}
\author{Ευστάθιος Ε. Γαλανάκης}
\supervisor{Πέτρος Μαραγκός}
\epitropiF{-}
\epitropiS{-}

\graphicspath{ {./theory/figures/} }       % path for images

\begin{document}
\gr
\maketitle

\begin{acknowledgements}
-
\end{acknowledgements}

%%%  Abstract, in Greek
\begin{abstract}
  
Σκοπός αυτής της διπλωματικής εργασίας είναι ο σχεδιασμός ενός δικτύου αναγνώρισης
και εντοπισμού των ανθρώπινων ενεργειών σε βίντεο. Το δίκτυό μας στοχεύει να προσδιορίσει χωροχρονικά
μια αναγνωρισμένη ενέργεια μέσα σε ένα βίντεο παράγoντας ακολουθίες δισδιάστατων πλαισίων, ένα για κάθε καρέ βίντεο, που περικλείει το άτομο που εκτελεί την αναγνωρισμένη ενέργεια. \par

Η ανίχνευση και η αναγνώριση των ενεργειών σε βίντεο είναι μια από τις μεγαλύτερες προκλήσεις
στο πεδίο της Όρασης Υπολογιστών. Οι πιο πρόσφατες προσεγγίσεις περιλαμβάνουν ένα δίκτυο ανίχνευσης αντικειμένων
τo οποίο προτείνει δισδίαστα κουτάκια ανά καρέ, έναν αλγόριθμο σύνδεσης για τη δημιουργία
υποψήφιων  \en action tubes  \gr και έναν ταξινομητή για την ταξινόμησή τους. Πάνω σ' αυτό, οι  περισσότερες
από αυτές τις προσεγγίσεις εξαγάγουν τις χρονικές πληροφορίες από ένα δίκτυο το οποίο εκτιμά
οπτική ροή σε επίπεδο πλαισίου. Η εισαγωγή των τρισδιάστατων συνελικτικών δικτύων 
μας έχει βοηθήσει να μπορούμε να υπολογίσουμε τις χωροχρονικές πληροφορίες
από τα βίντεο και ταυτόχρονα να εξάγουμε χωροχρονικά χαρακτηριστικά. Η προσέγγισή μας προσπαθεί να συνδυάσει τα οφέλη
του να χρησιμοποιείς δίκτυα ανίχνευσης αντικειμένων και τρισδιάστατες συνελίξεις.
Σχεδιάζουμε ένα δίκτυο του οποίου η δομή βασίζεται στα κλασσικά δίκτυα  εντοπισμού  δράσης
 και το ονομάζουμε \en ActionNet\gr. Το πρώτο στοιχείο είναι ένα τρισδιάστατο \en ResNet34 \gr  το οποίο
χρησιμοποιείται για τη χωροχρονική εξαγωγή χαρακτηριστικών. Επίσης, σχεδιάζουμε ένα δίκτυο για να 
προτείνει υποψήφιες ακολουθίες από δισδιάστατα πλαίσια με βάση τα χωροχρονικά χαρακτηριστικά, που το ονομάζουμε
\en Tube Proposal Network\gr. Αυτό το δίκτυο είναι μια επέκταση του \en Region Proposal Network \gr παίρνοντας ως είσοδο
τα εξαγόμενα χαρακτηριστικά και εξάγει \tl{k} προτεινόμενες ακολουθίες από δισδιάστατα κουτιά. Εξετάζουμε
2 προσεγγίσεις για τον καθορισμό των τρισδιάστατων προκαθορισμένων κουτιών του, τα  οποία χρησιμοποιεί το \en TPN\gr. Επιπλέον, σχεδιάζουμε
έναν αλγόριθμο σύνδεσης για τη σύνδεση και δημιουργία των προτεινόμενων \en action tubes\gr. Τέλος, διερευνούμε 
αρκετές τεχνικές ταξινόμησης, συμπεριλαμβανομένου ενός ταξινομητή \en SVM\gr, ενός \en Linear\gr, ενός \en RNN \gr και ενός \en MLP \gr
για τα σύνολα δεδομένων \en JHMDB \gr και \en UCF101\gr.
\begin{keywords}
Αναγνώριση δράσης, Εντοπισμός δράσης, \en Action Tubes\gr, \en TPN\gr, \en ActionNet\gr
\end{keywords}
\end{abstract}
\en
\begin{abstracteng}

  The purpose of this diploma thesis is the design of a network for recognizing and localising human actions in videos.
  Our network aims to spatiotemporally localize a recognized action within a video
  producing sequences of 2D boxes, one per frame, which includes the actor
  performing the recognized action. \par

  Detecting and Recognizing actions in videos is one of the biggest
  challenges in the field of Computer Vision. Most recent approaches
  includes an object detection network which proposes bounding boxes
  per frame, a linking method for creating candidate action tubes and
  a classifier for classifying these. On top of that, most of these
  approaches extract temporal information from a network which
  estimates optical flow in frame level. The introduction of 3D
  Convolutional Networks has helped us estimat spatiotemporal
  information from videos and simultaneously extract their spatiotemporal
  feature maps. Our approach tries to combine the benefits from using
  object detection networks and 3D Convolution.\par

  % \textbf{TODO change that}
  We designed a network whose structure is based on standard action localization networks and we name it ActionNet. Its first
  element is a 3D ResNet34 which is used for spatiotemporal feature extraction. Also,
  we introduced a network for proposing action tubes based on spatiotemporal features, called Tube Proposal Network.
  This network is an  expansion of Region Proposal Network and it gets as input the extracted features and
  outputs k-proposed action tubes.
  We explore 2 approaches for  defining 3D anchors, which TPN uses. On top of that,
  we design a linking algorithm for
  connecting proposed action tubes. Finally, we explore several classification techniques
  including a SVM classifier and a MLP for datasets JHMDB and UCF101.

\begin{keywordseng}
Action Localization, Action Recognition, Action Tubes, TPN, ActionNet,
\end{keywordseng}

\end{abstracteng}

\mainmatter
\en
\addcontentsline{toc}{chapter}{\contentsname}
\en
\tableofcontents
\en
\addcontentsline{toc}{chapter}{\listtablename}
\en
\listoftables
\en
\addcontentsline{toc}{chapter}{\listfigurename}
\en
\listoffigures
\gr
%%%  Main part of the book

% \documentclass{report}

% \usepackage{subcaption} % package for subfigures
% \usepackage{hyperref}  % package for linking figures etc
% \usepackage{enumitem}  % package for description with bullets
% \usepackage{graphicx}  % package for importing images
% \usepackage{mathtools} % package for math equation
% \usepackage{mathrsfs}  % package for math font
% \usepackage{indentfirst} % package for getting ident after section or paragraph
% \usepackage[export]{adjustbox}
% % \usepackage{amsmath}

% \setlength{\parindent}{2em} % how much indent to use when we start a paragraph

% \graphicspath{ {./theory/figures/} }       % path for images

% \begin{document}

\chapter{Εισαγωγή}
Στις μέρες μας, η τεράστια αύξηση της υπολογιστικής ισχύος των Η/Υ μας βοηθά να αντιμετωπίσουμε πολλές δύσκολες καταστάσεις που εμφανίζονται στην καθημερινότητά μας.
Πολλοί τομείς της επιστήμης κατάφεραν να αντιμετωπίσουν σημαντικά προβλήματα πριν από 20 χρόνια και σήμερα θεωρούνται ασήμαντα. Ένας επιστομικός που επιρρεάστηκε
αρκέτα είναι ο τομέας της Όρασης των Υπολογιστών (\tl{Computer Vision}) και πιο συγκεκριμένα, το πρόβλημα της αναγνώρισης και  εντοπισμού ανθρώπινης δράσης σε βίντεο.
\section{Περιγραφή Προβλήματος}
H πρόκληση της αναγνώρισης και εντοπισμού ανθρώπινης δράσης έχει δύο κύριους στόχους:
\begin{enumerate}
\item Την αυτόματη αναγνώριση και ταξινόμησή οποιασδήποτε ανθρώπινης δραστηριότητας στο βίντεο.
\item Τον αυτόματο εντοπισμό αυτής της δράσης στο βίντεο
\end{enumerate}


\subsection{Αναγνώριση ανθρώπινης δραστηριότητας}
Λαμβάνοντας υπόψη την αναγνώριση της ανθρώπινης δράσης, ένα βίντεο μπορεί να αποτελείται μόνο από ένα άτομο που κάνει κάτι. Ωστόσο, αυτό είναι ένα ιδανικό
σενάριο. Στις περισσότερες περιπτώσεις, τα βίντεο περιέχουν πολλά άτομα, που εκτελούν πολλαπλές ενέργειες ή ενδέχεται να μην δρουν καθόλου σε ορισμένα τμήματα
του βίντεο.
Έτσι, ο στόχος μας δεν είναι μόνο να ταξινομήσουμε μια δράση, αλλά να αποκομίσουμε τα χρονικά όρια κάθε δράσης

\subsection{Εντοπισμός ανθρώπινης δραστηριότητας}
Παράλληλα με την αναγνώριση της ανθρώπινης δράσης, ένα άλλο πρόβλημα είναι να προσδιορίσουμε χωρικά όρια κάθε δράσης. Συνήθως, αυτό σημαίνει
να καθορίσουμε  ένα δισδιάστατο πλαίσιο οριοθέτησης για κάθε εικόνα βίντεο, το οποίο περιέχει τον δρόντα. Φυσικά, αυτό το κουτί οριοθέτησης κινείται μαζί με
τον ηθοποιό.

\section{Εφαρμογές}
Το πεδίο της Ανθρώπινης Δράσης Αναγνώρισης και Εντοπισμού έχει πολλές εφαρμογές που περιλαμβάνουν
  ανάλυση περιεχομένου με βάση το περιεχόμενο, αυτοματοποιημένη κατάτμηση βίντεο, συστήματα ασφάλειας και επιτήρησης,
αλληλεπίδρασης ανθρώπου υπολογιστή.
Η τεράστια διαθεσιμότητα δεδομένων (ειδικά των βίντεο) δημιουργεί την ανάγκη να βρεθούν τρόποι για να επωφεληθούμε απ' αυτά.
Περίπου 2,5 δισεκατομμύρια εικόνες μεταφορτώνονται στη βάση δεδομένων του \tl{Facebook} κάθε μήνα, περισσότερες από 34K ώρες βίντεο στο \tl{YouTube} και 
περίπου 5K εικόνες κάθε λεπτό. Επιπλέον, υπάρχουν περίπου 30 εκατομμύρια κάμερες παρακολούθησης στις ΗΠΑ, πράγμα που σημαίνει
περίπου 700 ώρες βίντεο ανά ημέρα. Όλα αυτά τα δεδομένα πρέπει να χωριστούν σε κατηγορίες ανάλογα με το περιεχόμενό τους
προκειμένου να γίνουν πιο εύκολα προς αναζήτηση. Η διαδικασία αυτή γίνεται, συνήθως, χειρωνακτικά από έναν χρήστη που συνδέει το κάθε βίντεο με 
λέξεις-κλειδιά ή ετικέτες. Ωστόσο, οι περισσότεροι χρήστες αποφεύγουν να το κάνουν αυτό, τόσα πολλά βίντεο καταλήγουν χωρίς πληροφορίες σχετικά με τις ετικέτες.
Αυτή η κατάσταση δημιουργεί την ανάγκη δημιουργίας αλγορίθμων για αυτοματοποιημένη εύρεση του κατάλληλου βίντεο με βάση το περιεχόμενο του.

Ένα άλλο πεδίο εφαρμογών είναι η περίληψη βίντεο. Αυτές οι εφαρμογές χρησιμοποιούνται συνήθως σε ταινίες ή αθλητικές εκδηλώσεις. Στις ταινίες,
οι αλγόριθμοι ανάλυσης βίντεο μπορούν να δημιουργήσουν ένα μικρό βίντεο που περιέχει όλες τις σημαντικές στιγμές της ταινίας. Αυτό
μπορεί να επιτευχθεί επιλέγοντας τμήματα βίντεο στα οποία λαμβάνει χώρα μια σημαντική ενέργεια, όπως η δολοφονία του κακοποιού
της ταινίας. Στις αθλητικές εκδηλώσεις, οι εφαρμογές περίληψης βίντεο  περιλαμβάνουν τη δημιουργία αυτόματων βίντεο προβολής, όπως π.χ.
ένα βίντεο που περιέχει όλα τα επιτευχθέντα γκολ σ' έναν ποδοσφαιρικό αγώνα.

Επιπλέον, η αναγνώριση της ανθρώπινης δράσης μπορεί να αντικαταστήσει τους ανθρώπινους χειριστές στα συστήματα επιτήρησης. Μέχρι τώρα,
τα συστήματα ασφαλείας περιλαμβάνουν ένα σύστημα πολλαπλών καμερών που τα χειρίζεται ένας άνθρωπος χειριστής, ο οποίος κρίνει εάν ένα άτομο
ενεργεί κανονικά ή όχι. Τα συστήματα αυτόματης ταξινόμησης ενέργειας μπορούν να ενεργούν όπως o άνθρωπος, και αμέσως να κρίνουν
εάν υπάρχει κάποιους είδους περίεργη συμπεριφορά
κρίνετε εάν υπάρχει ανωμαλία στον άνθρωπο.

Τελευταίο αλλά όχι ασήμαντο, ένα άλλο πεδίο εφαρμογής σχετίζεται με την αλληλεπίδραση ανθρώπου-υπολογιστή. Ρομποτικές εφαρμογές
βοηθούν τους ηλικιωμένους να αντιμετωπίζουν τις καθημερινές τους ανάγκες. Επίσης, οι εφαρμογές παιχνιδιών που χρησιμοποιούν το \tl{Kinect} δημιουργούν νέα επίπεδα
εμπειρίας παιχνιδιού χωρίς την ανάγκη ενός ελεγκτή φυσικού παιχνιδιού.


% \section{Challenges and Datasets}
\section{Προκλήσεις και \tl{Datasets}}

Υπάρχουν διάφοροι τύποι ανθρώπινων δραστηριοτήτων. Ανάλογα με την πολυπλοκότητά τους, θεωρούμε ότι οι ανθρώπινες δραστηριότητες ταξινομούνται σε τέσσερις διαφορετικές κατηγορίες
επίπεδα: χειρονομίες, ενέργειες, αλληλεπιδράσεις και δραστηριότητες ομάδας. Οι χειρονομίες είναι στοιχειώδεις κινήσεις του σώματος ενός ατόμου και είναι το ατομικό
στοιχεία που περιγράφουν την ουσιαστική κίνηση ενός ατόμου. «Το στρίψιμο του βραχίονα» και «της ανύψωσης του ποδιού» είναι καλά παραδείγματα χειρονομιών.
Οι ενέργειες είναι δραστηριότητες ενός ατόμου που μπορούν να αποτελούνται από πολλαπλές χειρονομίες που οργανώνονται προσωρινά, όπως «περπάτημα», «χαιρετισμός» και
``μπουνία''. Οι αλληλεπιδράσεις είναι ανθρώπινες δραστηριότητες που περιλαμβάνουν δύο ή περισσότερα άτομα και / ή αντικείμενα. Για παράδειγμα, «δύο άτομα που αγωνίζονται» είναι 
μια αλληλεπίδραση μεταξύ δύο ανθρώπων και  «ενός ατόμου που κλέβει μια βαλίτσα από κάποιον άλλο» είναι μια αλληλεπίδραση ανθρώπου-αντικειμένου που περιλαμβάνει δύο ανθρώπους και ένα
αντικείμενο. Τέλος, οι δραστηριότητες ομάδας είναι οι δραστηριότητες που εκτελούνται από εννοιολογικές ομάδες που αποτελούνται από πολλαπλά πρόσωπα και / ή αντικείμενα. «Μια ομάδα ανθρώπων
που κάνουν πορεία», «μια ομάδα που έχει συνάντηση» και «δύο ομάδες που παίζουν ξύλο» είναι τυπικά παραδείγματα αυτών.

Η μεγάλη ποικιλία ανθρώπινων δραστηριοτήτων και εφαρμογών δημιουργεί πολλές προκλήσεις που περιλαμβάνουν συστήματα αναγνώρισης της δράσης.
Οι σημαντικότερες προκλήσεις περιλαμβάνουν μεγάλες διακυμάνσεις της εμφάνισης των ανθρώπων που δρουν, αλλαγές στην οπτική γωνία της κάμερας, αποκλείσεις,
μη-άκαμπτες κινήσεις κάμερας κλπ. Επιπλέον, ένα μεγάλο πρόβλημα είναι ότι υπάρχουν πάρα πολλές κατηγορίες δράσης που σημαίνει
ότι η χειροκίνητη συλλογή του δείγματος εκπαίδευσης είναι απαγορευτική. Επίσης, ορισμένες φορές, το λεξιλόγιο περιγραφής των δράσεων δεν είναι καλά καθορισμένο.
Όπως δείχνει το σχήμα \ref{fig:open_example}, η ενέργεια «Ανοίγω» μπορεί να περιλαμβάνει πολλά είδη ενεργειών, γι 'αυτό πρέπει προσεκτικά
να αποφασίσουμε ποια έννοια αυτής της πράξης θα λάβουμε υπόψη.

\begin{figure}[h]
  \centering
  \includegraphics[scale=0.3]{open_example}
  \caption{Παραδείγματα της δράσης «Ανοίγω»}
  \label{fig:open_example}

\end{figure}

Προκειμένου να αντιμετωπιστούν αυτές οι προκλήσεις, έχουν δημιουργηθεί διάφορα σύνολα δεδομένων για ανθρώπινες δράσεις, προκειμένου να αναπτυχθούν
ισχυρά συστήματα αναγνώρισης της ανθρώπινης δράσης και αλγόριθμοι ανίχνευσης.
Τα πρώτα σύνολα δεδομένων περιέλαμβαν έναν δρων χρησιμοποιώντας  μιας στατικής κάμερα πάνω σε ομοιογενή υπόβαθρα.
Παρόλο που αυτά τα σύνολα δεδομένων συνείσφεραν στο να σχεδιάσουμε τους πρώτους αλγορίθμους αναγνώρισης δράσης, δεν ήταν σε θέση να αντιμετωπίσουν αποτελεσματικά τις παραπάνω
προκλήσεις.
Έτσι λοιπόν οδηγηθήκαν στον να δημιουργήσουμε σύνολα δεδομένων που περιέχουν πιο αμφιλεγόμενα βίντεο, όπως το \tl{ Joint-annotated Human Motion Database(JHMDB) (\cite{Kuehne11})}
και \tl{UCF-101 (\cite{soomro2012ucf101})}. Αυτά τα \tl{dataset} περιλαμβάνουν μόνο ανθρώπινες ενέργειες, η δεύτερη κατηγορία που αναφέρθηκε πιο πριν.

\subsection{\tl{JHMDB Dataset}}
Το σύνολο δεδομένων \tl{JHMDB (\cite{Jhuang: ICCV: 2013})} είναι ένα πλήρες σχολιασμένο σύνολο δεδομένων για ανθρώπινες ενέργειες και ανθρώπινες πόζες. Αποτελείται από 21 κατηγορίες δράσεων
και 928 κλιπ που εξάγονται από την βάση δεδομένων κίνησης του ανθρώπου \tl{(HMDB51) \cite{Kuehne11}}. Αυτό το σύνολο δεδομένων περιέχει κομμένα βίντεο με διάρκεια μεταξύ
15 έως 40 καρέ. Κάθε κλιπ σχολιάζεται για κάθε καρέ χρησιμοποιώντας μια δισδιάστατη στάση και περιέχει μόνο 1 ενέργεια.
Προκειμένου να εκπαιδεύσουμε το μοντέλο μας για τον εντοπισμό των ενεργειών, τροποποιούμε της δισδιάστατες πόζες σε δισδιάσταστα πλαίσια που περιέχουν ολόκληρη τη στάση του δρώντα σε κάθε καρέ.
Υπάρχουν διαθέσιμα 3 διαφορετικά χωρίσματα για να εκπαιδευτεί ένα μοντέλο, τα οποία προτείνουν οι συγγραφείς. Επιλέξαμε το πρώτο  που περιέχει 660
βίντεο στο εκπαιδευτικό σετ και 268 για επικύρωση.

\textbf{Meqri edw eqw elegxei}
\subsection{\tl{UCF-101 Dataset}}
Το σύνολο δεδομένων \tl{UCF-101 (\cite{soomro2012ucf101})} περιέχει 13320 βίντεο από 101 κατηγορίες δράσεων.
Από αυτά, για 24 τάξεις και 3194 βίντεο χωροχρονικές σχολιασμοί περιλαμβάνονται. Αυτό σημαίνει ότι υπάρχει ένα 2D οριοθετημένο πλαίσιο που περιβάλλει τον ηθοποιό για κάθε πλαίσιο στο οποίο λαμβάνει χώρα μια δράση.
Τους διαχωρίζουμε σε 2284 βίντεο για εκπαιδευτικό σετ και 910 για δοκιμή επικύρωσης σύμφωνα με το
αρχικά προτεινόμενο σχίσιμο. Για τα δεδομένα εκπαίδευσης, υπάρχουν βίντεο μέχρι 641 καρέ, ενώ σε στοιχεία επικύρωσης ο μέγιστος αριθμός πλαισίων είναι 900.
Κάθε βίντεο, τόσο για την εκπαίδευση όσο και για την επικύρωση, δεν είναι έγκυρη, συμπεριλαμβανομένων μερικές φορές περισσότερες από 1 ενέργειες που πραγματοποιούνται ταυτόχρονα.
Λάβαμε σχολιασμούς από \tl{\cite{singh2016online}} επειδή οι συγγραφείς που πρότειναν σχολιασμούς περιέχουν κάποια λάθη.

\section{Motivation ans Contibutions}
Τα τρέχοντα επιτεύγματα στα δίκτυα αναγνώρισης αντικειμένων και στα δίκτυα 3D συνάθροισης για την αναγνώριση ενεργειών μας προκάλεσαν να δοκιμάσουμε
για να τα συνδυάσουμε, προκειμένου να επιτύχουμε τα καλύτερα αποτελέσματα για τον εντοπισμό της δράσης. Εισάγουμε μια νέα δομή δικτύου εμπνευσμένη από το
\ cite {DBLP: journals / corr / HouCS17}, \ cite {DBLP: journals / corr / abs-1712-09184}, \ cite {Ren: 2015: FRT: 2969239.2969250}
από \ cite {jjfaster2rcnn}.

Οι συνεισφορές μας είναι οι εξής: 1) Δημιουργούμε ένα νέο πλαίσιο για τον εντοπισμό των ενεργειών που επεκτείνει τον κώδικα που λαμβάνεται από τα γρηγορότερα RCNN,
2) Προσπαθούμε να δημιουργήσουμε ένα δίκτυο για την πρόταση ακολουθιών πλαισίων οριοθέτησης σε βίντεο κλιπ τα οποία μπορεί να περιέχουν μια ανάρτηση που εκμεταλλεύεται
των χωροχρονικών χαρακτηριστικών που μας παρέχουν οι 3D Convolutions, 3) δημιουργούμε έναν αλγόριθμο σύνδεσης για τη σύνδεση προτεινόμενων ακολουθιών
οριοθετώντας κουτιά για να εξάγουμε σωλήνες υποψήφιας ενέργειας και 4) προσπαθούμε να βρούμε τους καταλληλότερους χάρτες χαρακτηριστικών για την ταξινόμησή τους.
\section{Thesis structure}
Η υπόλοιπη διατριβή οργανώνεται ως εξής. Το κεφάλαιο 2 παρέχει μια γενική εισαγωγή στις τεχνικές μάθησης μηχανής που χρησιμοποιούνται σήμερα.
Στη συνέχεια, παρουσιάζουμε τα βασικά στοιχεία των συστημάτων αναγνώρισης αντικειμένων και παράλληλα με τις λειτουργίες απώλειας και τις μετρήσεις αξιολόγησης
συνηθίζαμε. Επίσης, το Κεφάλαιο 2 παρουσιάζει μια σύντομη επισκόπηση της βιβλιογραφίας σχετικά με την αναγνώριση και τον εντοπισμό της ανθρώπινης δράσης. Το Κεφάλαιο 3 εισάγει το πρώτο βασικό στοιχείο του δικτύου μας, Tube Proposal Network (TPN), ένα δίκτυο που προτείνει Tubes of Interest (TIs), οι οποίες είναι ακολουθίες των πλαισίων οριοθέτησης, με πιθανό να περιέχουν μια εκτελεσθείσα ενέργεια. Επιπλέον, περιέχει όλες τις προτεινόμενες αρχιτεκτονικές για την επίτευξη αυτού του στόχου.
Το κεφάλαιο 4 προτείνει αλγόριθμους για τη σύνδεση των προτεινόμενων TOI από κάθε τμήμα βίντεο και παρουσιάζονται οι επιδόσεις των προτάσεων.
Στο Κεφάλαιο 5 παρουσιάζουμε όλες τις προσεγγίσεις ταξινόμησης που χρησιμοποιήσαμε για τον σχεδιασμό της αρχιτεκτονικής μας και ορισμένα αποτελέσματα ταξινόμησης.
Το Κεφάλαιο 6 χρησιμοποιείται για συμπεράσματα, περίληψη της συμβολής μας μαζί με πιθανές μελλοντικές εργασίες.
\end{document}

\documentclass{report}

\usepackage[greek,english]{babel}

\newcommand{\tl}{\textlatin}
\newcommand{\en}{\selectlanguage{english}}
\newcommand{\gr}{\selectlanguage{greek}}

\usepackage{hyperref}  % package for linking figures etc
\usepackage{enumitem}  % package for description with bullets
\usepackage{graphicx}  % package for importing images
\usepackage{mathtools} % package for math equation
\usepackage{mathrsfs}  % package for math font
\usepackage{indentfirst} % package for getting ident after section or paragraph
\usepackage{subcaption} % package for subfigures
\usepackage[export]{adjustbox}
\usepackage{longtable} % package for multi pages tables
\usepackage{multirow}  % package for tables, multirow
\usepackage{amssymb}
\usepackage{esvect}
\usepackage[
backend=bibtex,
citestyle=authoryear,
% citestyle=authoryear-comp,
% citestyle=authoryear-ibid,
bibstyle=numeric,
sorting=ynt,
% style=numeric,
% style=alphabetic ,
]{biblatex}
\addbibresource{References}

\graphicspath{ {./theory/figures/} }       % path for images

\begin{document}
\gr
\section{Σχετική βιβλιογραφία}
Σε αυτή την ενότητα, παρουσιάζουμε ορισμένες από τις πιο συναφείς μεθόδους για την εργασία μας και άλλες που μελετήθηκαν για τον σχεδιασμό αυτής της προσέγγισης.
Οι μέθοδοι αυτές  χωρίζονται σε δύο ενότητες \textit{Aναγνώριση Δραστηριότητας} και \textit{Εντοπισμός Δραστηριότητας}. Το πρώτο μέρος αναφέρεται σε κλασικές μεθόδους
ταξινόμησης δράσης που εισήχθησαν μέχρι πρόσφατα και το δεύτερο μέρος, αντίστοιχα, σε πρόσφατες μεθόδους εντοπισμού της δράσης. 

\subsection{Αναγνώριση Δραστηριότητας}
Οι πρώτες προσεγγίσεις για την κατάταξη της δράσης αποτελούνταν από δύο βήματα α) αρχικά υπολογισμός σύνθετων ``χειροποίητων'' χαρακτηριστικών από ακατέργαστα καρέ βίντεο
και β) εκπαίδευση ενός ταξινομητή με βάση αυτά τα χαρακτηριστικά. Αυτά τα χαρακτηριστικά μπορούν να διαχωριστούν σε 3 κατηγορίες:
1) προσεγγίσεις χωροχρονικού όγκου \tl{(space-time volume)}, 2) τροχιές \tl{(trajectories)} και 3) χωροχρονικά  χαρακτηριστικά. Για τις μεθόδους χωροχρονικού όγκου,
η προσέγγιση είναι η εξής: Με βάση τα \tl{training} βίντεο, το σύστημα συνάπτει ένα μοντέλο τρισδιάστατου χωροχρόνου, συνενώνοντας δυσδιάστατες εικόνες
(διάσταση \tl{\textit{x-y}}) κατά τη διάρκεια του χρόνου (διάσταση \tl{\textit{t}} ή \tl{\textit{z}}), για την αναπαράσταση κάθε δράσης.
Όταν το σύστημα δέχεται ένα βίντεο που δεν έχει ετικέτα, κατασκευάζει μια τρισδιάστατη χωροχρονική αναπαράσταση που αντιστοιχεί σε αυτό το βίντεο.
Αυτό η νέα τρισδιάστατη αναπαράσταση, στη συνέχεια, συγκρίνεται με κάθε μοντέλο \tl{3D} χωροχρόνου, συγκρίνοντας την ομοιότητα στο σχήμα
και την εμφάνιση μεταξύ αυτών των δύο χωροχρονικών όγκων.
Το σύστημα εξάγει την κατηγορία του άγνωστου βίντεο, αντιστοιχώντας την με αυτήν της δράσης με την υψηλότερη ομοιότητα. Επιπλέον, υπάρχουν
διάφορες παραλλαγές των  χωροχρονικών αναπαραστάσεων. Αντί της αναπαράστασης \tl{space-time volume}, το σύστημα μπορεί να αναπαριστά τη κάθε δράση ως τροχιές
σε χωροχρονικές διαστάσεις ή ακόμη περισσότερο, η ενέργεια μπορεί να αναπαρασταθεί ως ένα σύνολο χαρακτηριστικών που εξάγονται από τον χωροχρονικό όγκο ή τις τροχιές.
Οι ``καθαρές'' χωροχρονικές αναπαραστάσεις περιλαμβάνουν μεθόδους σύγκρισης των περιοχών προσκηνίου ενός ατόμου (δηλ. σιλουέτες) όπως \en \cite{BobickAaron},\gr
συγκρίνοντας όγκους σε σχέση με επιφάνεια τους όπως οι \en \cite{1467296}\gr. Η μέθοδος \en \cite{4270510}\gr χρησιμοποιεί \tl{oversegmented} όγκους, αυτομάτως
υπολογίζοντας ένα σύνολο τμημάτων τρισδιάστατου όγκου \tl{ XYT} που αντιστοιχεί σε έναν κινούμενο άνθρωπο. \en\cite{4587727} \gr πρότειναν φίλτρα για
να αποτυπόνουν τα χαρακτηριστικά του χωροχρονικού όγκου, προκειμένου να τα ταιριάζουν πιο αξιόπιστα και αποδοτικά.
Από την άλλη πλευρά, οι προσεγγίσεις με βάση την τροχιά περιλαμβάνουν την αναπαράσταση μιας ενέργειας ως σύνολο 13 κοινών διαδρομών (\en\cite{1541250}\gr) ή
τη χρήση ενός συνόλου \tl{\textit{XYZT}}-διαστάσεων κοινών τροχιών που λαμβάνονται από κινούμενες κάμερες  (\en\cite{1541251}\gr).
Τέλος, διάφορες μέθοδοι χρησιμοποιούν τοπικά χαρακτηριστικά που εξάγονται από χωροχρονικούς όγκους τριών διαστάσεων,
όπως η εξαγωγή τοπικών χαρακτηριστικών σε κάθε καρέ του βίντεο και η ένωση του χρονικά (\en\cite{784616, 990935, 1544882}\gr,
η εξαγωγή αραιών χωροχρονικών τοπικών σημείων ενδιαφέροντος από τρισδιάστατους όγκους (\en\cite{1238378, 1570899,  Niebles, 1467373, Ryoo2006}\gr)
Οι προσεγγίσεις αυτές κατέστησαν την επιλογή των χαρακτηριστικών  σημαντικό παράγοντα για την απόδοση του δικτύου.
Αυτό συμβαίνει επειδή οι διαφορετικές κατηγορίες δράσεν μπορεί να διαφέρουν δραματικά από την άποψη της εμφάνισής τους και των μοτίβων κίνησης.
Ένα άλλο πρόβλημα ήταν ότι οι περισσότερες από αυτές τις προσεγγίσεις κάνουν υποθέσεις, υπό τις οποίες το βίντεο λήφθηκε λόγω προβλημάτων όπως το γεμάτo
φόντο,  γωνιές κάμερας κλπ. Μια ανασκόπηση των τεχνικών, που χρησιμοποιούνταν  μέχρι το 2011, παρουσιάζεται στο \en\cite{Aggarwal:2011:HAA:1922649.1922653}\gr.  \par

Τα πρόσφατα αποτελέσματα σε βαθιές αρχιτεκτονικές και ειδικά στον τομέα της ταξινόμηση εικόνας  έδωσε κίνητρο στους ερευνητές να εκπαιδεύσουν δίκτυα \tl{CNN} για
το πρόβλημα της αναγνώρισης δράσης. Η πρώτη σημαντική απόπειρα έγινε από τους \en\cite{6909619}\gr. Σχεδίασαν την αρχιτεκτονική τους με βάση το καλύτερο \tl{CNN}
στον διαγωνισμό  \tl{ImageNet}.
Εξερευνούν διάφορες μεθόδους για τη σύντηξη των χωροχρονικών λειτουργιών χρησιμοποιώντας δυσδιάσατες διαδικασίες κυρίως και τρισδιάστατη συνέλιξη μόνο σε αργή σύντηξη. Οι \en \cite{simonyan2014two} \gr
χρησιμοποιήσαν 2 CNNs, ένα για χωρικές πληροφορίες και ένα για οπτική ροή
και τα συνδύασαν  με τη χρήση της καθυστερημένης σύντηξης. Δείχνουν ότι η εξόρυξη χωρικού περιεχομένου από τα βίντεο και περιεχόμενο κίνησης από την οπτική ροή μπορεί να βελτιώσει
σημαντικά την ακρίβεια της αναγνώρισης της δράσης. Οι \en\cite{DBLP:journals/corr/FeichtenhoferPZ16} \gr επέκτειναν αυτή την προσέγγιση με τη χρήση πρώιμης σύντηξης στο τέλος των \en convolutional
layeers \gr αντί για  καθυστερημένης σύντηξης, η οποία λαμβάνει χώρα στο τελευταίο επίπεδο του δικτύου. Πάνω σ' αυτό, χρησιμοποίησαν ένα δεύτερο δίκτυο για το χρονικό περιεχόμενοοοο το οποίο
συνδέουν με το το άλλο δίκτυο με χρήση της καθυστερημένης σύντηξης. Επιπλέον, οι \en \cite{DBLP:journals/corr/WangXW0LTG16} \gr στήριξαν την μέθοδος τους
σε αυτήν που πρότειναν οι \en \cite{simonyan2014two}\gr. Ασχολούνται με το πρόβλημα του την εύρεσης χρονικού περιεχομένου και εκπαιδεύουν το δίκτυο τους, παρέχοντας του λίγα δείγματα.
Η προσέγγισή τους, την οποία ονόμασαν \en Temporal Segment Network (TSN), \gr διαχωρίζει το βίντεο εισόδου σε K τμήματα  και ένα σύντομο απόσπασμα από κάθε τμήμα επιλέγεται για ανάλυση.
Στη συνέχεια, συνδέουν  το εξαγόμενο χωροχρονικό περιεχόμενο, πραγματοποιώντας τελικά την πρόβλεψή τους. Πιο πρόσφατα, οι \en \cite{DBLP:journals/corr/ZhangWWQW16} \gr και οι \en \cite{DBLP:journals/corr/ZhuLNH17a} \gr χρησιμοποίησαν την \en two-stream \gr, επίσης. Οι \en \cite{DBLP:journals/corr/ZhangWWQW16} \gr αντικατέστησαν την οπτική ροή με ένα διάνυσμα κίνησης που μπορεί να ληφθεί απευθείας
από τα  συμπιεσμένα βίντεο χωρίς επιπλέον υπολογισμό και το τροφοδοτούν στο δίκτυο. Οι \en  \cite{DBLP:journals/corr/ZhuLNH17a} \gr εκπαίδευσαν ένα  \en CNN  \gr για τον υπολογισμό της οπτικής ροής,
καλώντας το \en MotionNet \gr και χρησιμοποίησαν ένα \en CNN \gr ως  χρονικό \en stream \gr για προβάλλουν τις πληροφορίες κίνησης έργου σε κατηγορίες δράσεωνο. Τέλος
χρησιμοποιούν Την καθυστερημένη σύντηξη μέσω της μέσης τιμής με βάρη των σκορ πρόβλεψης των χρονικών και χωρικών \en stream\gr. Από την άλλη πλευρά, μια νέα προσέγγιση εισήχθη από τους
\gr\cite{DBLP:journals/corr/abs-1711-01467}\gr ενσωματώνοντας χάρτες προσοχής με σκοπόν να βελτιώσουν σημαντικά  την απόδοση της αναγνώρισης δράσης. \par

Ορισμένες άλλες μέθοδοι περιλάμβαναν ένα δίκτυο \en RNN \gr  ή  \en LSTM  \gr για την ταξινόμηση κάνουν οι \en\cite{DBLP:journals/corr/DonahueHGRVSD14}\gr, οι \en \cite{DBLP:journals/corr/NgHVVMT15} \gr
και οι \en \cite{DBLP:journals/corr/MaCKA17}\gr. Οι  \en \cite{DBLP:journals/corr/DonahueHGRVSD14}\gr αντιμετωπίζουν τηv  πρόκλησης των μεταβλητών μήκη των ακολουθιών εισόδου και εξόδου,
εκμεταλλευόμενοι τα \en convolutional layers \gr  και μεγάλου εύρους χρονικές αναδρομές (\en recursions\gr). Προτείνουν ένα \en Long-term Recurrent
Convolutional Network (LRCN)\gr, το οποίο είναι ικανό να αντιμετωπίσει  τις εργασίες αναγνώρισης, λεζάντας εικόνας και περιγραφής βίντεο.
Για να ταξινομήσουν μια δεδομένη ακολουθία καρέ, το \en LRCN \gr λαμβάνει αρχικά ως είσοδο ένα καρέ, και πιο συγκεκριμένα τα κανάλια \en RGB \gr και την οπτική ροή του, και προβλέπει μια ετικέτα.
Μετά από αυτό, εξάγει την  κλάση  του βίντεο μέσω του  μέσου όρου των πιθανοτήτων των ετικετών, επιλέγοντας την πιο πιθανή κλάση.
Οι  \en\cite{DBLP:journals/corr/NgHVVMT15}  \gr πρώτα διερευνούν  διάφορες προσεγγίσεις για χρονική ομαδοποίησης (\en temporal pooling\gr) των χαρακτηριστικών.
Αυτές οι τεχνικές περιλαμβάνουν τον χειρισμό καρέ βίντεο ξεχωριστά από 2 αρχιτεκτονικές \en  CNN\gr: είτε απ' το \en AlexNet \gr είτε απ' το \en GoogleNet\gr, και αποτελούνται από 
πρώιμη σύντηξη, καθυστερημένη σύντηξη και  ενός συνδυασμού αυτών. Επιπλέον, προτείνουν ένα \en RNN \gr προκειμένου να εξετάσουν τα βίντεο κλιπ
ως ακολουθίες ενεργοποιήσεων CNN. Το προτεινόμενο \en LSTM \gr λαμβάνει ως είσοδο την έξοδο του τελικού \en CNN layer \gr για κάθε συνεχόμενο καρέ και μετά από 5 \en LSTM layers \gr και χρησιμοποιώντας
έναν \en softmax \gr ταξινομητή, προτείνει μία ετικέτα. Για την ταξινόμηση του βίντεο,  επιστρέφουν μια ετικέτα μετά το τελευταίο βήμα, εφαρμόζουν \en max-pooling \gr στις προβλέψεις στην διάσταση
του χρόνου, αθροίζουν τις προβλέψεις στην διάσταση του χρόνου και επιστρέφουν το μέγιστο ή έναν γραμμικό συνδυασμό με βάρη των προβλέψεων υπό έναν παράγοντα \tl{g}, τα αθροίζουν και επιστρέφουν το μέγιστο.
Έδειξαν ότι όλες οι προσεγγίσεις είναι 1\% διαφορετικές με προκατάληψη για τη χρήση των προβλέψεων με βάρη 
για την υποστήριξη της ιδέας ότι το \en LSTM \gr γίνεται προοδευτικά πιο ενημερωμένο. Τελευταίοι 
αλλά όχι λιγότερο σημαντικοι, οι \en \cite{DBLP:journals/corr/MaCKA17} \gr χρησιμοποίησαν ένα \en two-stream ConvNet \gr για  εξαγωγή χαρακτηριστικών και είτε ένα \en
LSTM \gr ή  \en convolutional \gr πάνω από το χρονικώς κατασκευασμένους πίνακες χαρακτηριστικών για τη σύντηξη χωρικών και χρονικών πληροφοριών. Χρησιμοποιούν ένα \en ResNet-101 \gr
για την εξόρυξη χαρτών ενεργοποίησης  τόσο για χωρικές όσο και για χρονικές ροές. Χωρίζουν το 
βίντεο σε διάφορα τμήματα, όπως έκαναν οι \en \cite{DBLP:journals/corr/WangXW0LTG16}\gr, και χρησιμοποίησαν ένα επίπεδο  \en temporal pooling \gr για την εξαγωγή διακεκριμένων χαρακτηριστικών.
Αφού λάβουν αυτά τα χαρακτηριστικά, το \en LSTM  \gr
εξάγει ενσωματωμένες δυνατότητες από όλα τα τμήματα. \par


Επιπλέον, οι \en \cite{Tran2014LearningSF} \gr  διευρένησαν τα \en  3D Convolutional \gr δίκτυα (\en \cite{pmid:22392705}\gr)
και εισήγαγαν το \en  C3D \gr δίκτυο που έχει \en  3D convolutional layers \gr  με πυρήνες $3 \times  3 \times 3$.
Αυτό το δίκτυο είναι σε θέση να μοντελοποιήσει την  εμφάνιση και την κίνηση ταυτόχρονα χρησιμοποιώντας τρισδιάσττες συνελίξεις και μπορεί να χρησιμοποιηθεί ως
εξαγωγέας χαρακτηριστικών. Συνδυάζοντας την αρχιτεκτονική δύο ροών και τις τρισδιάσττες συνελίξεις οι \cite{DBLP:journals/corr/CarreiraZ17} πρότειναν το  δίκτυο \en I3D\gr.
Πάνω σ' αυτό, oι δημιουργοί τονίζουν τα πλεονεκτήματα της μεταφοράς μάθησης για την εργασία της αναγνώρισης επαναλαμβάνοντας τα δυσδιάστατα προ-εκπαιδευμένα βάρη στην 3η διάσταση.
Οι \en \cite{DBLP:journals/corr/abs-1708-07632} \gr πρότειναν ένα δίκτυο \en 3D ResNet \gr για την αναγνώριση δράσης με βάση τα \en Residual \gr δίκτυα (\en ResNet)(\cite{DBLP:journals/corr/HeZRS15}\gr)
και διερευνούν την απόδοση  των δικτύων \en ResNet \gr με  \en 3D Convolutional \gr πυρήνες. Από την άλλη, οι \en \cite{DBLP:journals/corr/abs-1711-08200} \gr 
βάσισαν  την προσέγγισή τους στα \en DenseNets (\cite{DBLP:journals/corr/HuangLW16a}) \gr και επέκτειναν την αρχιτεκτονική του \en DenseNet \gr χρησιμοποιώντας τρισδιάστατα φίλτρα
και \en pooling \gr πυρήνες αντί για δισδιάστατους, ονομάζοντας αυτή την προσέγγιση ως \en DenseNet3D\gr. Επιπλέον, εισάγουν το \en Layer \gr χρονικής μετάβασης (TTL),
το οποίο συνενώνει χρονικά χάρτες χαρακτηριστικών που εξάγονται  σε διαφορετικά  χρονικά βάθη και αντικαθιστά το επίπεδο μετάβασης του \en  DenseNet\gr.
Παράλληλα, οι \en \cite{DBLP:DibaFSKAYG18} \gr εισήγαγαν  ένα νέο χρονικό \en layer \gr το οποίο μοντελοποιεί  μεταβλητούς χρονικούς  πυρήνες συνέλιξης. Τελευταίοι αλλά εξίσου σημαντικοί, 
oi \en \cite{DBLP:journals/corr/abs-1711-11248}\gr  πειραματίστηκαν  με διάφορες υπόλοιπες αρχιτεκτονικές \en Residual \gr δικτύου χρησιμοποιώντας
συνδυασμούς \en 2D  \gr και \gr 3D convolutional Layer\gr. Σκοπός τους είναι να δείξουν ότι η \en 2D \gr χωρική συνέλιξη ακολουθούμενη από \en 1D \gr χρονική  συνέλιξη επιτυγχάνει \en state-of-the-art \gr
αποτελέσματα, ονομάζοντας αυτού του τύπου το \en layer \gr ως \en R(2 + 1)D\gr. Πρόσφατα οι \en \cite{Guo_2018_ECCV} \gr  πρότειναν ένα \en framework \gr  που μπορεί να μάθει να
 αναγνωρίζει μια προηγουμένως αθέατη \en 3D \gr κλάση δράσης  με λίγα μόνο παραδείγματα εκμεταλλευόμενο την εγγενή δομή των \en 3D \gr δεδομένων μέσω μιας γραφικής αναπαράστασης.
Ακόμα πιο λεπτομερή  παρουσίαση των τεχνικών αναγνώρισης δράσης που χρησιμοποιήθηκαν μέχρι το 2018 πραγματοποιήθηκε από τους \en\cite{DBLP:journals/corr/abs-1806-11230}\gr.

\subsection{Εντοπισμός Δραστηριότητας}

Όπως προαναφέρθηκε, ο εντοπισμός δράσης μπορεί να θεωρηθεί ως προέκταση του
προβλήματος εντοπισμού αντικειμένων. Αντί να εξάγουμε δισδιάστατα πλαίσια οριοθέτησης  σε μία μόνο
εικόνας, ο στόχος των συστημάτων εντοπισμού δράσης είναι να εξάγουν \en action tubes, \gr τα οποία 
είναι ακολουθίες πλαισίων οριοθέτησης που περιέχουν μια ενέργεια που εκτελέστηκε. Έτσι, υπάρχουν
διάφορες προσεγγίσεις, συμπεριλαμβανομένου συνήθως ενός δικτύου ανιχνευτή αντικειμένων  και ενός ταξινομητή. \par

Οι πρώτες προσεγγίσεις ανίχνευσης αντικειμένων περιλάμβαναν την επέκταση ενός αλγορίθμου  πρότασης αντικειμένων
σε 3 διαστάσεις. Οι \en\cite{6619185} \gr επέκτείναν τα παραμορφώσιμα (\en deformable ) \gr μοντέλα (\en\cite{5255236}\gr) με το να αντιμετωπίζουν τις δράσεις  ως
xωροχρονικά μοτίβα και  δημιούργησαν ένα παραμορφώσιμου μοντέλο για κάθε δράση. Oi \en \cite{6909495} \gr εισήγαγαν την έννοια των \en tubelets\gr, γνωστά και ως ακολουθίες
πλαισίων οριοθέτησης και βάσισαν τη μέθοδό τους σε επιλεκτικό αλγόριθμο αναζήτησης (\en\cite{Uijlings13}\gr), επεκτείνοντας τα \en superpixels \gr σε \en supervoxels \gr
για την παραγωγή χωροχρονικών σχημάτων. Απ' την άλλη, οι \en \cite{Oneata} \gr επέκτειναν  μια τυχαιοποιημένη διαδικασία συγχώνευσης \en superpixels \gr που χρησιμοποιούταν για
που χρησιμοποιούνταν  για προτάσεις  αντικειμένων, όπως παρουσιάστηκαν απ' τους \en \cite{Manen:2013:POP:2586117.2587333}\gr.
Οι \en\cite{7298735} \gr  πρώτα προτείνουν πλαίσια οριοθέτησης για κάθε καρέ με χρήση ενός ανιχνευτή ανθρώπου  και κίνησης, ενώ, στη συνέχεια, με τη επιλογή των καλύτερων σε σκορ κουτιών,
πρότειναν έναν άπληστο συνδετικό αλγόριθμο με τη διατύπωση την εργασίας σύνδεσης ως  πρόβλημα  μέγιστης κάλυψης. Οι \en \cite{BMVC2015_177} \gr παράγουν  χωροχρονικές προτάσεις κατευθείαν
από τις πυκνές τροχειές, οι οποίες επίσης χρησιμοποιήθηκαν για ταξινόμηση.
Οι \en\cite{7410734} \gr δημιουργούν ένα γράφημα χωροχρονικής τροχιάς και επιλέγουν προτάσεις δράσεων που βασίζονται μόνο στην εσκεμμένη κίνηση που εξάγεται από το γράφημα.
Οι \en\cite{7410732} \gr διαχωρίζουν  τα τμήματα βίντεο σε \en supervoxels \gr και  χρησιμοποιούν το περιεχόμενο τους ως χωρική σχέση μεταξύ των \en supervoxels \gr σε σχέση με την δράση του προσκηνίου.
Δημιουργούν ένα γράφημα για κάθε βίντεο, όπου τα supervoxels σχηματίζουν τους κόμβους και οι  κατευθυνμένες άκρες απεικονίζουν  τις χωρικές σχέσεις μεταξύ τους. Κατά τη διάρκεια των δοκιμών,
κάνουν μια βόλτα στο περιβάλλον, όπου κάθε βήμα καθοδηγείται από τις σχέσεις περιβάλλοντος κατά τη διάρκεια της εκπαίδευσης, με αποτέλεσμα μια κατανομή  πιθανότητας μιας δράσης για όλα τα supervoxels.
Οι \en\cite{DBLP:journals/corr/MettesGS16} \gr αντί για τοποθέτηση  πλαισίων σε όλα τα καρέ των βίντεο, σχολίασαν σημεία σε ένα αραιό  υποσύνολο καρέ του  βίντεο και χρησιμοποίησαν
προτάσεις που λαμβάνονται μέσω ενός μέτρου επικάλυψης μεταξύ των προτάσεων δράσης και των σημείων.
Οι \en\cite{DBLP:journals/corr/BehlSSSCT17} \gr ασχολούνται με  την ανίχνευση και τον εντοπισμό  ενεργειών σε πραγματικό χρόνο μέσω της λήψης προτάσεων δράσης ανά καρέ και την πρόταση ενός αλγορίθμου
σύνδεσης που είναι σε θέση να κατασκευάσει και να ενημερώνει τα \en action tubes \gr ανά καρέ. Πιο πρόσφατα, οι  \en\cite{8237344} \gr προσπάθησαν  να ασχοληθούν  με το πρόβλημα της ανίχνευσης και
τον εντοπισμό δράσης χωρίς επίβλεψη. Η προσέγγισή τους περιελάμβανε αρχικά την εξόρυξη  κατακερματισμένων \en supervoxel \gr
και στη συνέχεια την ανάθεση ένα βάρους σε κάθε \en supervoxel\gr. Με την εξαγωγή \en supervoxels\gr, δημιουργούν ένα γράφημα και
στη συνέχεια χρησιμοποιούν μια διακριτική \en clustering \gr προσέγγιση εκπαιδεύεται ένας ταξινομητής.

Η εισαγωγή του \en R-CNN (\cite{DBLP:journals/corr/GirshickDDM13}) \gr κατάφερε σημαντικές βελτιώσεις
στην  απόδοση των δικτύων εντοπισμού αντικειμένων. Αυτή η αρχιτεκτονική,
πρώτον, προτείνει περιοχές στην εικόνα που είναι πιθανό να περιέχουν κάποιο αντικείμενο και
στη συνέχεια, τα ταξινομεί χρησιμοποιώντας ένα \en SVM\gr. Εμπνευσμένοι από αυτή την αρχιτεκτονική,
οι \en \cite{DBLP:journals/corr/GkioxariM14} \gr σχεδιάσαν ένα δίκτυο \en RCNN 2-stream \gr για να προτείνει 
προτάσεις δράσεων  για κάθε καρέ, ένα \en stream \gr για το επίπεδο καρέ και ένα για την 
οπτική ροή. Στη συνέχεια, τα συνδέουν χρησιμοποιώντας τον αλγόριθμο σύνδεσης \en Viterbi\gr.
Οι \en \cite{DBLP:journals/corr/WeinzaepfelHS15} \gr επεκτείνουν αυτή την προσέγγιση, εκτελώντας
προτάσεις στο επίπεδο καρέ  και χρησιμοποιώντας ένα tracker για τη σύνδεση των προτάσεων αυτών
χαρακτηριστικά της χωρικής και οπτικής ροής. Επίσης, η μέθοδός τους εκτελεί χρονικό εντοπισμό μέσω της χρήσης
ενός συρόμενου παράθυρου πάνω από τα εντοπισμένα \en tubes\gr.   \par

Η εισαγωγή του \en Faster RCNN (\cite{Ren:2015:FRT:2969239.2969250}) \gr συνήσφερε  πολύ
τη βελτίωση της απόδοσης των δικτύων εντοπισμού δράσης. Οι \en\cite{peng:hal-01349107} \gr και \en\cite{DBLP:journals/corr/SahaSSTC16} \gr 
χρησιμοποιούν το \en Faster R-CNN \gr αντί για το \en  RCNN \gr  για προτάσεις σε επίπεδο καρέ, χρησιμοποιώντας το \en RPN \gr για εικόνες \en RGB \gr και οπτικής ροής.
Αφού λάβουν χωρικές προτάσεις και προτάσεις κίνησης, οι \en \cite{peng:hal-01349107} \gr τις συγχωνεύουν και από κάθε προτεινόμενη \en ROI\gr, παράγουν 4 \en ROIs \gr για να επικεντρωθούν
σε συγκεκριμένο μέρος του σώματος του δρώντα. Μετά από αυτό, συνδέουν την πρόταση χρησιμοποιώντας τον αλγόριθμο \en Viterbi \gr για κάθε κλάση και εκτελούν χρονικό εντοπισμό
χρησιμοποιώντας ένα συρόμενο παράθυρο, με πολλαπλές χρονικές κλίμακες και διασκελισμό κάνοντας χρήση  μιας μεθόδου  μέγιστης υποσυστοιχίας.
Απ' την άλλη, οι \en \cite{DBLP:journals/corr/SahaSSTC16} \gr εκτελούν, επίσης, ταξινόμηση σε επίπεδο καρέ. Μετά απ' αυτό,  η μέθοδός τους εκτελεί σύντηξη με βάση έναν συνδυασμό
της εμφάνισης και της κίνησης με βάση τις προτάσεις και την βαθμολογία αλληλεπικάλυψης. Τέλος, η χρονική προσαρμογή λαμβάνει χώρα χρησιμοποιώντας δυναμικό προγραμματισμό.
Παράλληλα, οι \en \cite{DBLP:journals/corr/WeinzaepfelMS16} \gr χρησιμοποιούν το \en Faster RCNN \gr
για την εξαγωγή ανθρώπινων \en tubes \gr από βίντεο που εστιάζουν στο πρόβλημα του ασθενώς εποπτευόμενoυ εντοπισμού δράσης.
 Στη συνέχεια, χρησιμοποιώντας πυκνές τροχιές και μια \en multi-fold Multiple  Instance  Learning \gr προσέγγιση \en(\cite{7420739}\gr)  εκπαιδεύουν ένα
 ταξινομητή. Οι \en \cite{DBLP:journals/corr/MettesS17} \gr εισήγαγαν μια μέθοδο για \en zero-shot \gr Εντοπισμού δράσης. Η προσέγγισή τους περιλαμβάνει την βαθμολόγηση των προτεινόμενων
 \en action tubes \gr  σύμφωνα με τις αλληλεπιδράσεις μεταξύ των ατόμων που δρουν  και  αντικειμένων. Χρησιμοποίησαν το \en Faster-RCNN\gr,
στο πρώτο βήμα, για την ανίχνευση τόσο των ανθρώπων που δρουν όσο και των αντικειμένων και μετά, χρησιμοποιώντας χωρικές σχέσεις μεταξύ τους,  συνδέουν τα προτεινόμενα πλαίσια στον άξονα του χρόνου
βασιζόμενοι στην \en zero-shot \gr πιθανότητα της παρουσίας των ατόμων, συναφών αντικειμένων γύρω απ' αυτούς και τις αναμενόμενες χωρικές σχέσεις μεταξύ αντικειμένων και ανθρώπων που δρουν. Επιπλέον
οι \en\cite{DBLP:journals/corr/HeIDM17} \gr πρότειναν  το \en Tube Proposal Network (TPN) \gr για τη δημιουργία ανεξαρτήτου κλάσης προτάσεις \en tubelet\gr, οι οποίες χρησιμοποιούν το \en Faster R-CNN \gr
για να λάβουν δισδιάστατες προτάσεις περιοχών και έναν αλγόριθμο σύνδεση για τη σύνδεση των \en tubelets  \gr με τις  προτάσεις των περιοχών. Πιο πρόσφατα, oi \en \cite{DBLP:journals/corr/abs-1807-10066}
\gr πρότειναν μια μέθοδο για εντοπισμό δράσεων στο σύνολο δεδομένων \en AVA  (\cite{DBLP:journals/corr/GuSVPRTLRSSM17}\gr) συνδυάζοντας τις αρχιτεκτονικές
των \en I3D (\cite{DBLP:journals/corr/CarreiraZ17}\gr) και \en Faster RCNN\gr. Χρησιμοποιούν μπλοκ του \en I3D \gr για την λήψη αναπαράστασης βίντεο και το  \en RPN \gr του \en Faster-RCNN \gr
για να προτείνει προτάσεις «ανθρώπου» για το κεντρικό πλαίσιο. \par

Παράλληλα μ' αυτά, οι \en \cite{singh2016online} \gr και \en\cite{kalogeiton17iccv:hal-01519812} \gr  σχεδίασαν τα δίκτυα τους
με βάση το \en Single Shot Multibox Detector \cite{DBLP:journals/corr/LiuAESR15})\gr. Οι \en\cite{singh2016online} \gr δημιούργησαν
ένα  χωροχρονικό δίκτυο  πραγματικoύ χρόνου. Για να λειτουγεί το δίκτυο τους σε πραγματικού χρόνου, \en  \cite{singh2016online} \gr πρότειναν έναν νέο και αποδοτικό
αλγόριθμο με την προσθήκη πλαισίων σε \en tubes \gr σε κάθε καρέ, εάν επικαλύπτονται περισσότερο από ένα κατώφλιοή, ή  εναλλακτικά, τερματίζουν  το \en action tube \gr εάν για \tl{k} καρέ
δεν προσθέθηκε κανένα πλαίσιο. Οι \en\cite{kalogeiton17iccv:hal-01519812} \gr  σχεδίασαν ένα δίκτυο δύο ροών, το οποίο κάλεσαν
\en ACT-detector \gr, και εισήγαγαν τα κυβικά (\en cuboids) anchors\gr. Για K καρέ, και για τα δύο δίκτυα,
οι \en \cite{kalogeiton17iccv:hal-01519812} \gr εξάγουν  χωρικά χαρακτηριστικά σε επίπεδο καρέ, στη συνέχεια, τα στοιβάζουν.
Τέλος, με τη χρήση των κυβικών \en anchors\gr, το δίκτυο  εξάγει \en tubelets\gr,
με τις αντίστοιχες βαθμολογίες κατάταξης και στόχους παλινδρόμησης. Για τη σύνδεση των τουμπέλετς, oi \en \cite{kalogeiton17iccv:hal-01519812} \gr  ακολουθούν 
τα ίδια βήματα με τους \en \cite{singh2016online}\gr. Για χρονική εντοπισμό, χρησιμοποιούν μιά προσέγγιση χρονικής εξομάλυνσης. \par

Πιο πρόσφατα, το δίκτυο \en YOLO (\cite{DBLP:journals/corr/RedmonDGF15}) \gr έγινε η έμπνευση
για  τους \en \cite{DBLP:journals/corr/abs-1903-00304} \gr και τους \en \cite{DBLP:journals/corr/abs-1802-08362}\gr. Στην προσέγγιση που προτάθηκε από τους
\en  \cite{DBLP:journals/corr/abs-1903-00304}\gr, οι έννοιες της
εξέλιξης και τού ποσοστoύ προόδου εισήχθησαν. Εκτός από την πρόταση πλαισίων οριοθέτησης σε επίπεδο καρέ, χρησιμοποιούν το \en YOLO \gr μαζί με έναν ταξινομητή \en RNN \gr για
να εξάγουν χρονικές πληροφορίες για τις προτάσεις. Με βάση αυτές τις πληροφορίες, δημιουργούν \en action tubes, \gr χωρίζοντας τα  σε κλάσεις. Ορισμένες άλλες προσεγγίσεις περιλαμβάνουν
εκτίμηση πόζας αυτή των \en \cite{DBLP:journals/corr/abs-1802-09232}\gr. Πρότειναν μια μέθοδο
υπολογισμού των δισδιάστατων και τρισδιάστατων πόζων και στη συνέχεια εκτέλεσαν ταξινόμηση δράσεων.
Χρησιμοποιούν το διαφορίσιμο \en Soft-argamax \gr για   την εκτίμηση των \tl{2D} και \tl{3D}
αρθρώσεων, επειδή η συναρτηση \en argmax \gr δεν είναι διαφορίσιμη. Στη συνέχεια, για T παρακείμενες
δημιουργούν μια απεικόνιση εικόνας με το χρόνο και τις $N_j$  αρθρώσεις  ως $x-y$ άξονες, έχοντας  2 κανάλια για την \tl{2D} πόζα  και 3  για την \tl{ 3D} πόζα.
Χρησιμοποιούν \en Convolutional Layers \gr για να παράγουν  χάρτες θερμότητες δράσης   και στη συνέχεια χρησιμοποιώντας \en  max plus min pooling \gr και την συνάρτηση \en softmax \gr
 εκτελούν ταξινόμηση δράσης.
 Οι \en \cite{DBLP:journals/corr/ZolfaghariOSB17} \gr πρότειναν μια αρχιτεκτονική τριών ροών που περιλαμβάνει \en 2D \gr πόζα, οπτική ροή και πληροφορίες \en  RGB\gr. Αυτά τα \en streams \gr ενώνονται
 μέσω του μοντέλου της αλυσίδας Μάρκοφ. Επιπλέον, οι \en \cite{8237881} \gr πρότειναν μια αρχιτεκτονική με τη χρήση ενός χρονικού \en convolutional \gr δικτύου παλινδρόμησης,
 για να πιάνουν την  μακροπρόθεσμη εξάρτηση και πληροφίες  μεταξύ γειτονικών καρέ και ένα χωρικό δίκτυο παλινδρόμησης, για  προτάσεις ανά καρέ. Χρησιμοποιούν μεθόδους παρακολούθησης
και δυναμικoύ προγραμματισμού για τη δημιουργία προτάσεων δράσης. \par

Τα περισσότερα από τα προαναφερθέντα δίκτυα χρησιμοποιούν ανά καρέ χωρικές προτάσεις και εξάγουν 
τις χρονικές τους πληροφορίες υπολογίζοντας την οπτική ροή. Από την άλλη
oi \en \cite{DBLP:journals/corr/SahaSC17} \gr σχεδίασαν μια αρχιτεκτονική η οποία περιλαμβάνει προτάσεις σε επίπεδο τμήματος βίντεο, το οποίο σημαίνει  περισσότερα από ένα
καρέ ταυτόχρονα. Οι \en \cite{DBLP:journals/corr/SahaSC17} \gr  πρότειναν μια \en 3D-RPN \gr αρχιτεκτονική 
που είναι σε θέση να δημιουργήσει και να ταξινομήσει τρισδιάσττες προτάσεις   αποτελoύμενες από 2 συνεχόμενα  καρέ. Επίσης, πρότειναν έναν αλγόριθμο σύνδεσης,
τροποποιώντας αυτόν που πρότειναν oi \en \cite{DBLP:journals/corr/SahaSSTC16}\gr. Πάνω σ' αυτό, οι \en \cite{DBLP:journals/corr/HouCS17} \gr  σχεδιάσαν μια αρχιτεκτονική για τη δημιουργία
προτάσεων δράσης για περισσότερα από 2 καρέ, καλώντας το μοντέλο τους \en Tube CNN (T-CNN)\gr. Στην προσέγγισή τους, το επίπεδο του τμήματος βίντεο σημαίνει
ότι ολόκληρο το βίντεο χωρίζεται κλιπ βίντεο ίδιου αριθμού καρέ  και με τη χρήση
του \en C3D \gr για την εξόρυξη χαρακτηριστικών, επιστρέφουν  χωροχρονικές προτάσεις. Μετά την λήψη των 
προτάσων, oi \cite{DBLP:journals/corr/HouCS17} συνδέουν τις \en tube \gr προτάσεις τους  με έναν αλγόριθμο στηριζόμενος
στην πιθανότητα ύπαρξης δράσης  και την επικάλυψη μεταξύ των \en tubes\gr. Τέλος, η λειτουργία ταξινόμησης λαμβάνει χώρα για τα συνδεδεμένα \en action tubes\gr.


\printbibliography

\end{document}
\documentclass{report}

\usepackage[ english, greek]{babel}
\usepackage[utf8]{inputenc}
\usepackage[LGR, T1]{fontenc}

% % 

\newcommand{\tl}{\textlatin}
\newcommand{\en}{\selectlanguage{english}}
\newcommand{\gr}{\selectlanguage{greek}}

\usepackage{hyperref}  % package for linking figures etc
\usepackage{enumitem}  % package for description with bullets
\usepackage{graphicx}  % package for importing images
\usepackage{mathtools} % package for math equation
\usepackage{mathrsfs}  % package for math font
\usepackage{indentfirst} % package for getting ident after section or paragraph
\usepackage{subcaption} % package for subfigures
\usepackage[export]{adjustbox}
\usepackage{longtable} % package for multi pages tables
\usepackage{multirow}  % package for tables, multirow
\usepackage{amssymb}
\usepackage{esvect}
\usepackage[
backend=bibtex,
citestyle=authoryear,
% citestyle=authoryear-comp,
% citestyle=authoryear-ibid,
bibstyle=numeric,
sorting=ynt,
% style=numeric,
% style=alphabetic ,
]{biblatex}
\addbibresource{References}

\graphicspath{ {./theory/figures/} }       % path for images

\begin{document}
\gr 
\chapter{\tl{Tube Proposal Network}}
\section{ Η αρχιτεκτονική του μοντέλου μας}
Σε αυτό το κεφάλαιο, ασχολούμαστε κυρίως με  το \en Tube Proposal Network (TPN)\gr, ένα από τα βασικά στοιχεία του μοντέλου μας \en ActionNet\gr. Πριν ξεκινήσουμε την
περιγραφή του, παρουσιάζουμε τη δομή όλου του μοντέλου μας. Προτείνουμε ένα δίκτυο παρόμοιο με αυτό των \en\cite{DBLP:journals/corr/HouCS17}\gr.
Η αρχιτεκτονική μας αποτελείται από τα ακόλουθα βασικά στοιχεία:
\begin{itemize}
\item Ένα τρισδιάστατο συνελικτικό μοντέλο \en(3D Convolutional Network)\gr, το οποίο χρησιμοποιείται για την εξαγωγή χαρακτηριστικών.
  Στην υλοποίησή μας χρησιμοποιούμε ένα δίκτυο \en 3D ResNet34\gr, το οποίο λαμβάνεται από τους \en\cite{hara3dcnns}\gr  και βασίζεται
  στα \en ResNet CNNs\gr για  ταξινόμηση εικόνων \en(\cite{DBLP:journals/corr/HeZRS15})\gr.
\item Ένα \en TPN\gr για την εξαγωγή υποψήφιων \en ToIs\gr (βασιζόμενοι στην ιδέα που παρουσίαζουν οι \en \cite{DBLP:journals/corr/HouCS17} \gr).
\item Έναν ταξινομητή για την εύρεση της κλάσης των προτεινόμενων \en action video tubes\gr.
\end{itemize}

Η βασική διαδικασία που ακολουθεί το \en ActionNet\gr είναι:
\begin{enumerate}
\item Δεδομένου ενός βίντεο, το διαχωρίσουμε σε τμήματα βίντεο. Αυτά τα τμήματα βίντεο  σε ορισμένες προσεγγίσεις επικαλύπτονται χρονικά και σε άλλες όχι.
\item Για κάθε τμήμα βίντεο, μετά την εκτέλεση της αλλαγής μεγέθους χωροχρονικά, τροφοδοουμε τα καρέ του στο \en ResNet34\gr για να εξάγουμε τους χωροχρονικούς
  χάρτες του. Αυτοί οι χάρτες ενεργοποίησης, στη συνέχεια, τροφοδοτούνται στο \en TPN\gr για την πρόταση ακολουθιών πλαισίων που πιθανόν περιέχουν κάποια δράση τις
  θα  ονομάσουμε \en Tubes of Interest (ToIs)\gr, όπως κάνουν και οι \en \cite{DBLP:journals/corr/HouCS17}\gr .
\item Μετά την πρόταση \en ToIs\gr για κάθε τμήμα βίντεο, χρησιμοποιώντας έναν αλγόριθμο σύνδεσης, το \en ActionNet\gr βρίσκει τα τελικά υποψήφια \en action tubes\gr.
 Αυτά τα \en action tubes\gr δίνονται ως είσοδο σε έναν ταξινομητή για τον προσδιορισμό της κλάσης τους.
\end{enumerate}

Ένα διάγραμμα του μοντέλου μας \en ActionNet \gr εμφανίζεται στην εικόνα \ref{fig:whole_network_}. 
\begin{figure}[h]
  % \includegraphics[scale=0.7]{convolutional_neural_network_structure} \]
  \centering
  \includegraphics[scale=0.42]{model_prenms}
  \caption{Structure of the whole network}
  \label{fig:whole_network_}
\end{figure}

\section{Εισαγωγή στο \tl{TPN}}
Ο κύριος σκοπός του  \en TPN  \gr  είναι να προτείνει \en\textbf{Tubes of Interest} (TOIs)\gr. Αυτά τα \en  tubes \gr  είναι πιθανό να περιέχουν μια γνωστή δράση και αποτελούνται από μερικά
δυσδιάστατα πλαίσια (1 για κάθε καρέ βίντεο). Το \en  TPN \gr  είναι εμπνευσμένο από το \en  RPN \gr  που εισήχθη από το \en   FasterRCNN (\cite{Ren:2015:FRT:2969239.2969250})\gr ,
αλλά αντί για εικόνες, το \en  TPN \gr  χρησιμοποιείται σε βίντεο όπως κάνουν και οι \en \cite{DBLP:journals/corr/HouCS17}\gr . Σε πλήρη αντιστοιχία με τo \en  RPN \gr , η δομή
του \en  TPN \gr  είναι παρόμοια με αυτή  του \en  RPN\gr . Η μόνη διαφορά, είναι ότι το  \en TPN \gr  χρησιμοποιεί \en  3D Convolutional Layers  \gr και \en  3D anchors \gr  αντί \en  2D \gr .  \par
Σχεδιάσαμε 2 κύριες δομές για τo \en  TPN\gr. Κάθε προσέγγιση έχει διαφορετικό ορισμό για τα χρησιμοποιούμενα τρισδιάστατa \en  anchors\gr. 
Η υπόλοιπη δομή του \en  TPN \gr  είναι κυρίως η ίδια με ορισμένες μικρές διαφορές στο στάδιο του  \en  regression. \gr  \par

\section{Προετοιμασία πριν το \tl{TPN}}

\subsection{Η προετοιμασία των δεδομένων}
Πριν εισαχθεί ένα βίντεο στο  \en ResNet \gr  και στο \en  TPN \gr  για να εξαγάγουμε τα χαρακτηριστικά του και πιθανά \en  ToIs \gr , αυτό το βίντεο πρέπει να προεπερξαστεί.
Η διαδικασία προεπεξεργασίας είναι η ίδια και για τις δύο προσεγγίσεις του \en TPN\gr.
Η αρχιτεκτονική μας λαμβάνει ως είσοδο μια ακολουθία από σταθερό αριθμό καρέ  που έχουν σταθερό πλάτος και ύψος. Ωστόσο, κάθε βίντεο είναι πιθανόν να έχει διαφορετική ανάλυση. Αυτό δημιουργεί
την ανάγκη να αλλάξουμε  το μέγεθος κάθε καρέ και πλασίου πριν εισαχθεί στο \en ActionNet\gr. Όπως αναφέρθηκε στο προηγούμενο κεφάλαιο, το πρώτο στοιχείο του δικτύου μας είναι ένα \en  3D RenNet  \gr
που υλοποιήθηκε από τους \en  \cite{hara3dcnns}\gr. Αυτό το δίκτυο έχει σχεδιαστεί να δέχεται βίντεο  με διαστάσεις (112, 112).
Ως αποτέλεσμα, μεταβάλλουμε  το μέγεθος κάθε καρέ από τα βίντεο των \en dataset \gr σε (112, 112). Για να διατηρήσουμε την αναλογία διαστάσεων, προσθέτουμε μηδενικές τιμές είτε
αριστερά και δεξιά, είτε πάνω και κάτω, ανάλογα με το ποια διάσταση είναι μεγαλύτερη. Στο σχήμα  \ref{fig:Preprocess_example} μπορούμε να δούμε το αρχικό καρέ καθώς και το αναδιαμορφωμένο.
Σε πλήρη αντιστοιχία, αλλάζουμε και το μέγεθος των πραγματικών πλαισίων οριοθέτησης αληθείας για κάθε καρέ (Τα σχήματα \en\ref{fig:original_image_rois}\gr   και \en\ref{fig:trans_image_rois}\gr  το απεικονίζουν).

\begin{figure}[h]
  \en
  \centering
  \begin{subfigure}{0.35\textwidth}
    \includegraphics[width=\textwidth]{./figures/original_image.jpg}
    \caption{}
    \label{fig:original_image}
   \end{subfigure}
  \hfill
  \begin{subfigure}{0.35\textwidth}
    \includegraphics[width=\textwidth]{./figures/original_image_rois.jpg}
    \caption{}
    \label{fig:original_image_rois}
  \end{subfigure}
  \hfill
  \begin{subfigure}{0.35\textwidth}
    \includegraphics[width=\textwidth]{./figures/transformed_image.jpg}
    \caption{}
    \label{fig:trans_image}
  \end{subfigure}
  \hfill
  \begin{subfigure}{0.35\textwidth}
    \includegraphics[width=\textwidth]{./figures/transformed_image_rois.jpg}
    \caption{}
    \label{fig:trans_image_rois}
  \end{subfigure}

  \caption{ At (a), (b) frame is its original size and at (c), (d) same frame after preprocessing part}
  \label{fig:Preprocess_example}
\end{figure}

\subsection{\en 3D ResNet \gr}
Πριν από τη χρήση του \en Tube Proposal Network\gr, εξάγουμε  χωροχρονικά χαρακτηριστικά από το βίντεο. Για να γίνει αυτό, εξάγουμε τα 3 πρώτα στρώματα ενός
προεκπαιδευμένου \en 3D ResNet34\gr. Αυτό το μοντέλο είναι προεκπαιδευμένο στο \en Kinetics dataset (\cite{DBLP:journals/corr/KayCSZHVVGBNSZ17}\gr) για διάρκεια του δείγματος ίση με
16 καρέ και μέγεθος δείγματος ίσο με (112, 122).  \par
Αυτό το δίκτυο συνήθως χρησιμοποιείται για την ταξινόμηση ολόκληρου του βίντεο, οπότε μερικά από τα στρώματα του χρησιμοποιούν \en temporal stride \gr ίσο με 2.
Εμείς, όμως, θέτουμε το \en temporal stride \gr ίσο με 1 γιατί δεν θέλουμε να χάσουμε χρονικές πληροφορίες κατά τη διάρκεια της διαδικασίας.
Έτσι, η έξοδος του τρίτου στρώματος είναι ένα χάρτης χαρακτηριστικών με διαστάσεις (256, 16, 7, 7). Τροφοδοτουμε αυτό τον χάρτη  χαρακτηριστικών στο \en TPN\gr,
το οποίο περιγράφεται στις ακόλουθες ενότητες.

\section{Τα τριασδιάστατα  \tl{anchors} ως  \tl{6-dim} διανύσματα}
\subsection{Πρώτη περιγραφή}
Ξεκινήσαμε να σχεδιάζουμε το \en TPN  \gr εμπνευσμένοι από την δουλειά των \en\cite{DBLP:journals/corr/HouCS17}\gr. Έτσι, θεωρούμε κάθε \en anchor \gr ως ένα
τρισδιάστατο πλαίσιο  το οποίο γράφεται ως \en $(x_1, y_1, t_1, x_2, y_2, t_2)$ \gr  όπου \en $x_1, y_1, t_1$\gr  είναι
οι πάνω μπροστά αριστερές διαστάσεις του κύβου και \en $x_2, y_2, t_2$\gr  είναι οι κάτω πίσω δεξιά όπως φαίνεται και στην εικόνα \ref{fig:anchor_6d}.
\begin{figure}[h]

  \centering
  \includegraphics[scale=0.5]{anchor_6d}
  \caption{\en An example of the anchor $(x_1,y_1,t_1,x_2,y_2,t_2)$}
  \label{fig:anchor_6d}
\end{figure}
\gr

Το κύριο πλεονέκτημα αυτής της προσέγγισης είναι ότι εκτός από τις διαστάσεις \en  x-y \gr , η διάσταση του χρόνου είναι μεταβαλλόμενη. Ως αποτέλεσμα, τα προτεινόμενα \en ToIs \gr
δεν έχουν καθορισμένη χρονική διάρκεια. Αυτό θα μας βοηθήσει να ασχοληθούμε με τα μη-κομμένα \en(untrimmed\gr) βίντεο, επειδή τα προτεινόμενα \en TOIs \gr  θα μπορούν να εξαιρέσουν \en background\gr
καρέ.
Για αυτήν την προσέγγιση, χρησιμοποιούμε \en\textbf{n = 4K = 60} anchors\gr  για κάθε \en  pixel\gr  στους χάρτες ενεργοποίησης του \en TPN\gr. Έχουμε \en k anchors\gr  για κάθε διαφορετική
  διάρκεια \en anchor\gr
 (5 κλίμακες των 1, 2, 4, 8, 16, 3 \en aspect rations 1:1, 1:2, 2:1\gr  και 4 διάρκειες 16, 12, 8, 4 καρέ).
Σύμφωνα με τους \en\cite{DBLP:journals/corr/HouCS17}\gr,  τα \en anchors \gr  του δικτύου ορίζονται σύμφωνα με τα πιο συνηθισμένα \en anchors \gr  του συνόλου δεδομένων. Αυτό, ωστόσο,
δημιουργεί την ανάγκη επανασχεδιασμού του δικτύου για κάθε σύνολο δεδομένων. Στην προσέγγισή μας, χρησιμοποιούμε τα ίδια \en  anchors \gr  και για τα δύο σύνολα δεδομένων, επειδή θέλουμε το δίκτυό
μας να μην να βασίζεται στο σύνολο δεδομένων, αλλά να είναι σε θέση να γενικευσει για διάφορα σύνολα δεδομένων. Ως διάρκεια δειγματοληψίας, επιλέξαμε 16 καρέ ανά τμήμα βίντεο, επειδή
η προ-εκπαιδευμένη έκδοση\en  ResNet\gr  που χρησιμοποιούμε έχει εκπαιδευτεί για βίντεο κλιπ με αυτή τη διάρκεια.
Έτσι, η δομή του \en  TPN \gr  είναι:
\begin{itemize}
\item\en 1 3D Convolutional Layer\gr με \en kernel size = 3, stride = 3 \gr  και \en padding = 1
\item\en 1 classification layer\gr  που εξάγει \en \textit{2n scores}\gr  για το αν υπάρχει ή όχι δράση για \en\textit{n tubes}.
\item\en 1 regression layer\gr  που εξάγει \textit{\tl{6n} διαστάσεις} \en($x_1,y_1,t_1,x_2,y_2,t_2$) \gr  για \en\textit{n tubes}.
\end{itemize}

Η δομή του\en TPN\gr  παρουσιάζεται στην Εικόνα \ref{fig:tpn_1_1}. Το αποτέλεσμα του\en  TPN\gr είναι τα\en  k\gr-καλύτερα κουτιά, τα οποία
είναι τα πιο πιθανά να περιέχουν κάποια δράση.
\begin{figure}[h]
  \includegraphics[width=1.\textwidth]{tpn_1_1}
  \caption{Structure of TPN}
  \label{fig:tpn_1_1}
\end{figure}

\subsection{\tl{Training}}
\gr Όπως προαναφέρθηκε, το \en TPN \gr  εξάγει \en  ToIs \gr  ως 6-διάστατα διανύσματα. Για το λόγο αυτό, τροποποιήσαμε τα πραγματικά πλαίσια ανά καρέ σε πραγματικά \en  Tubes \gr.
Θεωρούμε δεδομένο ότι το άτομο που δρα, δεν μπορεί να κινηθεί πολύ σε 16 καρέ, γι ' αυτό χρησιμοποιούμε τέτοιου είδους \en  Tubes \gr. Όπως φαίνεται 
στο σχήμα \ref{fig:gt_tubes_and_rois}, αυτά τα tubes είναι τρισδιάστατα κουτιά που περιλαμβάνουν όλα τα πραγματικά πλαίσια, τα οποία είναι διαφορετικά
ανά καρέ.
\begin{figure}[h]
  \centering
  \begin{subfigure}{0.15\textwidth}
    \includegraphics[width=\textwidth]{output/img_0.jpg}
  \end{subfigure}
  \begin{subfigure}{0.15\textwidth}
    \includegraphics[width=\textwidth]{output/img_3.jpg}
  \end{subfigure}
  \begin{subfigure}{0.15\textwidth}
    \includegraphics[width=\textwidth]{output/img_5.jpg}
  \end{subfigure}
  \begin{subfigure}{0.15\textwidth}
    \includegraphics[width=\textwidth]{output/img_7.jpg}
  \end{subfigure}
  \begin{subfigure}{0.15\textwidth}
    \includegraphics[width=\textwidth]{output/img_11.jpg}
  \end{subfigure}
  \begin{subfigure}{0.15\textwidth}
    \includegraphics[width=\textwidth]{output/img_15.jpg}
  \end{subfigure}
  \caption{Το πραγματικό tube έχει μπλε χρώμα και το πραγματικό ανά καρέ πλαίσιο έχει χρώμα πράσινο}
  \label{fig:gt_tubes_and_rois}
\end{figure}

Για τη διαδικασία \en training\gr, για κάθε βίντεο, επιλέγουμε τυχαία ένα μέρος του, το οποίο έχει διάρκεια 16 καρέ. Θεωρούμε ένα \en anchor \gr ως πρώτο πλάνο,
αν η βαθμολογία επικάλυψης του με το πραγματικό
\en tube \gr είναι μεγαλύτερη από 0.5 . Διαφορετικά, θεωρείται ως \en anchor  \gr φόντου. Χρησιμοποιούμε έναν \en scoring layer \gr για να ταξινομήσουμε σωστά αυτά τα \en anchors \gr και χρησιμοποιούμε
την \en Cross Entropy Loss \gr ως συνάρτηση κόστους \en (loss function\gr). Έχουμε πολλά \en anchors \gr για να προτείνουμε μια δράση, αλλά μικρό αριθμό δράσεων σε κάθε βίντεο, έτσι επιλέγουμε \en
256 anchors \gr συνολικά για κάθε \en video\gr. Ορίζουμε ότι o μέγιστος αριθμός των \en anchors \gr προσκηνίου να είναι 25\% από τους 256 \en anchors \gr και τα υπόλοιπα είναι \en anchors \gr φόντου.  \par
Η σωστή ταξινόμηση ενός \en anchor \gr δεν είναι αρκετή για να προτείνουμε \en ToIs\gr. Είναι, επίσης,  απαραίτητο τα \en anchors \gr να επικαλύπτονται όσο το δυνατόν περισσότερο με τα πραγματικά \en
tubes\gr.
Αυτός είναι ο λόγος που χρησιμοποιούμε ένα επίπεδο παλινδρόμησης. Αυτό το \en layer \gr «κινεί» τον κύβο στην περιοχή που πιστεύεται ότι είναι πιο κοντά στη δράση.
Για συνάρτηση κόστους παλινδρόμησης χρησιμοποιούμε την συνάρτηση κόστους \en smooth-L1 \gr  όπως παρουσιάζεται από τους \en \cite{DBLP:journals/corr/GirshickDDM13}\gr. Για να υπολογίσομε τους
 στόχους παλινδρόμησης, χρησιμοποιούμε την \en pytorch \gr εφαρμογή του  \en FasterRCNN (\cite{jjfaster2rcnn}\gr) για την παλινδρόμηση του πλαισίου και 
τροποποιούμε τον κώδικα επεκτείνοντας τον για 3 διαστάσεις. % Doobf{για περισσότερες πληροφορίες}
Έτσι έχουμε:
\en
\[ \begin{matrix}
    t_x = (x-x_a)/w_a, & t_y = (y-y_a)/h_a, & t_z= (z-z_a)/d_a, \\
    t_w= log(w/w_a), & t_h= log(h/h_a), & t_d = log(d/d_a), \\
    t^*_x = (x^* - x_a)/w_a, & t^*_y = (y^* - y_a)/h_a, & t^*_z = (z^* - z_a)/d_a, \\
    t^*_w = log(w^* /w_a), & t^*_h = log(h^*/h_a), & t^*_d = log(d^*/d_a),
    % t∗x= (x∗−xa)/wa,  t∗y= (y∗−ya)/ha,t∗w= log(w∗/wa),  t∗h= log(h∗/ha)
  \end{matrix}
\] \gr
όπου τα \en\textit{x, y, z, w, h, d} \gr υποδεικνύουν τις συντεταγμένες του κέντρου του τρισδιάστατου κουτιού καθώς επίσης το πλάτος, το ύψος και τη διάρκειά του. Οι μεταβλητές
\en $x, x_a, $\gr και \en $ x^*$ \gr αφορούν το προβλεπόμενο πλαίσιο, το πλαίσιο του \en anchor \gr και το πραγματικό πλαίσιο αντίστοιχα (ομοίως για \en \textit{y, z, w, h, d}\gr).
Φυσικά, υπολογίζουμε την απώλεια παλινδρόμησης μόνο για τα \en anchors \gr προσκηνίου και όχι αυτά του φόντου, συνεπώς στην χειρότερη θα υπολογίσουμε 64 στόχους για κάθε \en video.\gr \par

Για να συνοψίσουμε στη διαδικασία \en training\gr, εκπαιδεύουμε 2 \en layers \gr για το \en TPN, scoring \gr και \en reggression\gr. Η συνάρτηση κόστους περιλαμβάνει τα
\en training losses \gr που προκύπτουν απ' αυτά τα \en layers \gr και ο τύπος της είναι: \en
% \[ L  = \frac{1}{N_{cls}} \sum_iL_{cls}(p_i, p^*) + \frac{1}{N_{reg}}\sum_ip_i^*L_{reg}(t_i,t_i^*) \]
\[ L  =  \sum_iL_{cls}(p_i, p_i^*) + \sum_ip_i^*L_{reg}(t_i,t_i^*) \]
\gr όπου:
\begin{itemize}
\item \en $L_{cls} $ \gr είναι η \en Cross Entropy loss \gr που χρησιμοποιούμε γαι να εκπαιδεύσουμε τα \en anchors\gr, με \en $p_i$  \gr είναι η προβλεπόμενη κλάση, \en $p_i^*$ \gr
  είναι η πραγματική κλάση και \en   $p_i, p_i^* \in \{0,1\}$. \gr
\item \en $L_{reg} $ \gr είναι η συνάρτηση κόστους \en smooth-L1\gr, η οποία πολλαπλασιάζεται με \en $p_i^*$  \gr προκείμενου να ενεργοποιείται όταν υπάρχει θετικό \en anchor  $(p_i^* = 1)$ \gr
  και να απερνεργοποιείται για τα \en background anchors $(p_i^* = 0)$.
\end{itemize}
\gr
\subsection{\tl{Validation}}

Η διαδικασία \en validation \gr είναι κάπως παρόμοια με τη διαδικασία του \en training\gr.
Επιλέγουμε τυχαία 16 καρέ από ένα βίντεο επικύρωσης και εξετάζουμε αν υπάρχει τουλάχιστον 1 προτεινόμενο \en ToI \gr
που επικαλύπτει $\ge$ 0,5 για κάθε πραγματικό \en tube  \gr και παίρνουμε το \en recall score\gr.
Για να λάβουμε καλές προτάσεις, μετά τη λήψη των \en classification scores \gr και των \en regression targets \gr από το
αντίστοιχα \en layers\gr, χρησιμοποιούμε τον αλγόριθμο \en Non-Maximum Suppresion(NMS)\gr.
Έχουμε ορίσει το κατώφλι του \en NMS \gr ίσο με 0,7 και κρατάμε τους πρώτους 150 κύβους με τη μεγαλύτερη βαθμολογία.
\subsection{\tl{Modified Intersection over Union(mIoU) } }
Κατά τη διάρκεια του \en training\gr, έχουμε πολλά \en anchors\gr. Πρέπει να τα ταξινομήσουμε ως \en anchors  \gr προσκηνίου ή
\en anchors \gr παρασκηνίου. Τα \en anchors \gr προσκηνίου είναι εκείνα που περιέχουν κάποια ενέργεια και, αντίστοιχα, του φόντου
που δεν έχουν. Όπως παρουσιάστηκε προηγουμένως, το \en IoU \gr για τα \en cuboids \gr υπολογίζει το ποσοστό μεταξύ του όγκου της επικάλυψης
και του όγκου των Ένωσης.
Διαισθητικά, αυτό το κριτήριο είναι καλό για την αξιολόγηση του βαθμού επικάλυψης 2 \en tube\gr, αλλά έχει ένα μεγάλο μειονέκτημα:
Θεωρεί ότι οι διαστάσεις \en x-y \gr έχουν την ίδια σημασία με τη χρονική διάσταση, τo οποίo δεν επιθυμούμε. Κι αυτό  διότι
πρώτον, μας ενδιαφέρει να είμαστε ακριβείς στη χρονική διάσταση, και στη συνέχεια μπορούμε να διορθώσουμε τον τομέα \en x-y\gr.
Ως αποτέλεσμα, αλλάζουμε τον τρόπο με τον οποίο υπολογίζουμε το \en Intersection over Union\gr. Υπολογίζουμε ξεχωριστά
το \en IoU \gr στις διαστάσεις \en x-y (IoU-xy) \gr και στην \en t \gr  διάσταση \en (IoU-t)\gr. Τέλος,  πολλαπλασιάζουμε αυτά τα δύο \en score
 \gr για να πάρουμε το τελικό \en IoU\gr.
Συνεπός ο τύπος για 2 \en tubes ($x_1, y_1, t_1, x_2, y_2, t_2$) \gr  και \en  ($x'_1, y'_1, t'_1, x'_2, y'_2, t'_2$) \gr είναι:\en
\[ IoU_{xy} = \frac{ \text{Area of Overlap in x-y}} { \text{Area of Union in x-y}}  \]
\[ IoU_t = \frac { max(t_1, t'_1) - min(t_2, t'_2)} {min(t_1,t'_1) - max(t_2,t'_2)} \]
\[ IoU = IoU_{xy} \cdot  IoU_t \]
\gr Το παραπάνω κριτήριο μας βοηθά να εξισορροπήσουμε τις επιπτώσεις του χρόνου στο \en IoU score\gr. Για παράδειγμα, ας εξετάσουμε 2 \en anchors\gr:
\en a = (22, 41, 1, 34, 70, 5) \gr και \en  b = (20, 45, 2, 32, 72, 5)\gr. Αυτά τα 2 \en anchor \gr στις διαστάσεις \en  x-y \gr έχουν βαθμολογία \en IoU \gr ίσο με 0,61.
Αλλά δεν είναι ακριβώς επικαλυπτόμενα στην διάσταση του χρόνου. Χρησιμοποιώντας την πρώτη προσέγγιση έχουμε 0,5057 \en IoU \gr βαθμολογία ενώ η
δεύτερη προσέγγιση μας δίνει 0,4889. Έτσι, το δεύτερο κριτήριο θα απέρριπτε αυτό το \en  anchor\gr, διότι υπάρχει μια διαφορά στην χρονική διάρκεια. \par


Για να επιβεβαιώσουμε την ιδέα μας, εκπαιδεύουμε το \en TPN \gr χρησιμοποιώντας τόσο το \en IoU \gr κριτήριο όσο και το \en mIoU \gr  για την επικάλυψη των \en tubes\gr.
Στο πίνακα \ref{table:iou_miou} μπορούμε να δούμε την απόδοση σε κάθε περίπτωση και για τα δύο σύνολα δεδομένων, \en JHMDB \gr και \en UCF\gr. Το \en recall \gr όριο για αυτή
την περίπτωση είναι 0,5 και κατά την διάρκεια του \en validation \gr χρησιμοποιούμε το κανονικό \en IoU \gr για να καθορίσουμε αν 2 \en tubes \gr επικαλύπτονται.
\begin{table}[h]
  \centering
  \en
  \begin{tabular}{|| c | c || c ||}
    \hline
    \textbf{Dataset} & \textbf{Criterion} & \textbf{Recall(0.5)} \\
    \hline  \hline
    \multirow{2}{4em}{JHMDB} & IoU & 0.70525 \\
    \cline{2-3}
    {} & mIoU & 0.7052 \\
    \hline
    \multirow{2}{4em}{UCF} & IoU & 0.4665 \\
    \cline{2-3}
    {} & mIoU & 0.4829 \\
    \hline      
  \end{tabular}
  \caption{\en Recall results for both datasets using IoU and mIoU metrics}
  \label{table:iou_miou}
\end{table}
\gr
Ο πίνακας  \ref{table:iou_miou} μας δείχνει ότι το τροποποιημένο-\en IoU \gr μας δίνει ελαφρώς καλύτερη απόδοση \en recall \gr μόνο στο σύνολο δεδομένων \en UCF\gr.
Αυτό είναι λογικό, επειδή το σύνολο δεδομένων \en JHMDB \gr χρησιμοποιεί κομμένα βίντεο, συνεπώς  η χρονική διάρκεια δεν επηρεάζει πολύ. Έτσι, από τώρα και στο εξής,
κατά τη διάρκεια του \en training \gr χρησιμοποιούμε το \en mIoU  \gr ως επικαλυπτόμενη πολιτική βαθμολογίας.

\subsection{Βελτιώνοντας το  \tl{ TPN score}}
\gr Μετά την πρώτη δοκιμή, μας ήρθε η  ιδέα ότι σε ένα βίντεο που διαρκεί 16 καρέ, στον τομέα του χρόνου, όλα τα είδη των ενεργειών μπορούν να χωρίζονται στις ακόλουθες κατηγορίες:
\begin{enumerate}
\item Η ενέργεια ξεκινά από το \tl{\textit{n}}-ο πλαίσιο και ολοκληρώνεται μετά το \textit{16o} καρέ του βίντεο που έχει υποβληθεί σε δειγματοληψία.
\item Η ενέργεια έχει ήδη ξεκινήσει πριν από το \textit{1o} καρέ του βίντεο και τελειώνει στο \tl{\textit{n}} πλαίσιο.
\item Η ενέργεια έχει ήδη ξεκινήσει πριν από το \textit{1o} καρέ του βίντεο και ολοκληρώνεται μετά το \textit{16o} καρέ βίντεο.
\item Η ενέργεια ξεκινά και τελειώνει σε αυτά τα 16 καρέ του βίντεο.
\end{enumerate}

Επιπλέον, παρατηρήσαμε ότι οι περισσότερες ενέργειες, στα σύνολα δεδομένων μας, διαρκούν περισσότερο από 16 καρέ. Έτσι, ήρθαμε με την ιδέα να προσθέσουμε 1 \en scoring layer \gr και 1
\en reggression Layer \gr που θα προτείνει \en ToIs  \gr με σταθερή διάρκεια ίση με τη διάρκεια του δείγματος (16 καρέ) και θα λάβει υπόψη τις χωρικές πληροφορίες που παράγονται
από τους χάρτες ενεργοποίησης. Η νέα δομή του \en TPN \gr εμφανίζεται στην εικόνα \ref{fig:tpn_1_2}. Αφού λάβουμε τις προτάσεις και από και από τα δύο \en scoling layers\gr,
τις ενώνουμε με ποσοστό 1:1 μεταξύ των \en ToI \gr που εξαχθήκαν από τα δύο υποδίκτυα.
\begin{figure}[h]
  \centering
  \includegraphics[scale=0.5]{tpn_1_2}
  \caption{\en TPN structure after adding 2 new layers, where k = 5n.}
  \label{fig:tpn_1_2}
\end{figure}
\gr Στόχος μας είναι να «συμπιέσουμε» τους χάρτες της χρονικής διάστασης, προκειμένου να προτείνουμε \en ToIs \gr σύμφωνα μόνο με τις χωρικές πληροφορίες.
Έτσι, βρίκαμε με 2 τεχνικές για να πραγματοποιήσουμε κάτι τέτοιο:
\begin{enumerate}
\item Να χρησιμοποιήσουμε \en  3D Convolutional Layers \gr με μέγεθος πυρήνα = (διάρκεια δείγματος, 1, 1), \en stride \gr = 1 και χωρίς \en padding \gr για \en scoring \gr και \en regression\gr.
  Αυτός ο πυρήνας «κοιτάει» μόνο στη χρονική διάσταση των χαρτών ενεργοποίησης και δεν θεωρεί καμία χωρική εξάρτηση.
\item Nα αποκτήσΟυμε τις μέσες τιμές από τη χρονική διάσταση και στη συνέχεια να χρησιμοποιήσουμε ένα \en 2D Convolutional Layer \gr για \en scoring \gr και \en regression\gr.
\end{enumerate}

% \textbf{TODO na perigrapsw oti thelw na exw ola ta xronika features}
\par
Οι διαδικασίες \en training \gr και \en validation \gr παραμένουν ίδιες. Η μόνη μεγάλη διαφορά είναι ότι τώρα έχουμε κόστη  από 2 διαφορά σύστημα που προτείνουν \en TOIs\gr.
Έτσι, κατα την διάρκεια του \en validation\gr, εμείς αρχικά, ενώνουμε τα προτεινόμενα \en ToIs \gr και, στη συνέχεια, ακολουθούμε την ίδια διαδικασία, η οποία είναι να
υπολογίσουμε το \en recall\gr. Για το \en training loss\gr, έχουμε 2 διαφορετικές \en Cross-entropy loss \gr συναρτήσεις και 2 διαφορετικά \en smooth-L1 losses\gr. Έτσι,
η απώλεια εκπαίδευσης, τώρα, ορίζεται ως:

  % \begin{aligned}
 % L  =  \sum_iL_{cls}(p_i, p_i^*) + \sum_iL_{cls}(p_{fixed,i}, p_{fixed,i}^*) +  
 %  \sum_ip_i^*L_{reg}(t_i,t_i^*) + \sum_ip_{fixed,i}^*L_{reg}(t_{fixed,i},t_{fixed,i}^*) 
% L  =2


% \[ L  =  \sum_iL_{cls}(p_i, p_i^*) + \sum_iL_{cls}(p_{fixed,i}, p_{fixed,i}^*) +  \newline
%   \sum_ip_i^*L_{reg}(t_i,t_i^*) + \sum_ip_{fixed,i}^*L_{reg}(t_{fixed,i},t_{fixed,i}^*) \]
% \end{aligned}

\en
\begin{equation} 
\begin{split}
 L  =  \sum_iL_{cls}(p_i, p_i^*) + \sum_iL_{cls}(p_{fixed,i}, p_{fixed,i}^*) + \\
   \sum_ip_i^*L_{reg}(t_i,t_i^*) + \sum_ip_{fixed,i}^*L_{reg}(t_{fixed,i},t_{fixed,i}^*) 
  % \sum_iL_{cls}(p_i, p_i^*) + \sum_iL_{cls}(p_{fixed,i}, p_{fixed,i}^*) 
\end{split}
\end{equation}
\gr
όπου:
\begin{itemize}
\item\en $L_{cls} $ \grείναι η \en Cross Entropy loss \gr που χρησιμοποιούμε γαι να εκπαιδεύσουμε τα \en anchors\gr, με \en $p_i$ \gr είναι η προβλεπόμενη κλάση, \en $p_i^*$ \gr είναι
  η πραγματική κλάση και   \en $p_i, p_i^* \in \{0,1\}$
\item\en $L_{reg} $ \gr είναι η συνάρτηση κόστους \en smooth-L1\gr, η οποία πολλαπλασιάζεται με \en $p_i^*$ \gr προκείμενου να ενεργοποιείται όταν υπάρχει θετικό \en anchor $(p_i^* = 1)$ \gr
  και να απερνεργοποιείται για τα \en background anchors $(p_i^* = 0)$.
\item\en $p_i $ \gr είναι τα \en anchors \gr από τα \en scoring \gr και \en regression layers \gr με μεταβλητή χρονική διάρκιεα και  \en $p_i^*$ \gr είναι η αντίστοιχη πραγρατική τους κλάση.
\item\en $p_{fixed,i} $ \gr είναι τα \en anchors \gr από τα \en scoring \gr και \en regression layers \gr με σταθερή χρονική διάρκεια ίση με 16 καρέ και \en $p_{fixed,i}^*$  \grείναι η
  αντίστοιχη πραγματική του κλάση.

\end{itemize}

Εκπαιδεύουμε το δίκτυο \en TPN \gr χρησιμοποιώντας και τις δύο τεχνικές και η  απόδοση του \en recall \gr εμφανίζεται στον πίνακα  \ref{table:add_16}.

\begin{table}[h]
  \centering
  \en
  \begin{tabular}{||c | c | c || c ||}
    \hline
    \textbf{Dataset} & \textbf{Fix-time anchors} & \textbf{Type} & \textbf{Recall(0.5)} \\
    \hline  \hline
    \multirow{3}{4em}{JHMDB} & No &  - & 0.7052 \\
    \cline{2-4}
    {} & \multirow{2}{*}{Yes} & Kernel & 0.6978 \\
    \cline{3-4}
    {} & {} & Mean & 0.7463 \\
    \hline
    \multirow{3}{4em}{UCF} & No & - & 0.4829 \\
    \cline{2-4}
    {} & \multirow{2}{*}{Yes} & Kernel & 0.4716 \\
    \cline{3-4}
    {} & {} & Mean & 0.4885 \\
    \hline      
  \end{tabular}
  \caption{Recall results after adding fixed time duration anchors}
  \label{table:add_16}
\end{table}
\gr
Όπως μπορούμε να δούμε από τα προηγούμενα αποτελέσματα, τα νέα \en layers \gr αύξησαν σημαντικά την απόδοση του \en recall\gr. Πέρα από αυτό, ο πίνακας \ref{table:add_16} δείχνει ότι
η λήψη των μέσων τιμών από τη χρονική διάσταση μας δίνει τα καλύτερα αποτελέσματα.


\subsection{Προσθήκη \tl{Regressor}}
Το αποτέλεσμα του \en TPN  \gr είναι τα $alpha$-υψηλότερα βαθμολογικά \en anchors \gr που μετακινήθηκαν σύμφωνα με την \en regression \gr πρόβλεψη τους. Μετά από αυτό, πρέπει να μετατρέψουμε τα
προτεινόμενα \en anchors \gr σε \en ToIs\gr.
Για να γίνει αυτό, προσθέτουμε ένα σύστημα παλινδρόμησης που παίρνει ως εισόδο τους χάρτες χαρακτηριστικών των τρισδιάστατων κουτιών και επιστρέφει μια ακολουθία από δυσδιάσταστα κουτιά,
ενα για κάθε καρέ.
Το μόνο πρόβλημα είναι ότι η παλινδρόμηση χρειάζεται ως είσοδο χάρτες χαρακτηριστικών  σταθερού μεγέθους. Αυτό το πρόβλημα είναι ήδη λυμένο από τα \en R-CNNs  \gr που χρησιμοποιούν \en ROI pooling \gr
και \en ROI align \gr προκειμένου να πάρουμε του σταθερό μέγεθος χάρτες ενεργοποίησης από \en ROIs \gr μεταβαλλόμενα μεγέθους. Στην περίπτωση μας, επεκτείνουμε την λειτουργία \en RoI Align\gr ,
που παρουσιάζεται από το \en MaskR-CNN\gr, και εμείς το ονομάζουμε \en\textbf{3D ROI align}.
\paragraph{3D Roi Align}
\en To  \en 3D ROI align \gr είναι μια τροποποίηση του \en ROI Align \gr που παρουσιάστηκε από τo \en MaskR-CNN\gr. Η κύρια διαφορά μεταξύ αυτών των δύο είναι ότι το
\en MaskR-CNN\gr, στο \en RoiAlign\gr,  χρησιμοποιεί διγραμμική παρεμβολή για την εξαγωγή των χαρακτηριστικών των \en ROIs \gr και το δικός μας \en 3D ROI Align \gr χρησιμοποιεί
τριγραμμική παρεμβολή για τον ίδιο λόγο. Και πάλι, η 3η διάσταση είναι χρόνος.
Συνεπώς, έχουμε ως είσοδο ένα χάρτη χαρακτηριστικών που εξάγεται από το \en  ResNet34  \gr με διαστάσεις (64, 16, 28, 28) και έναν τένσορα που περιέχει τα προτεινόμενα \en ToIs\gr.
Για κάθε \en ToI \gr του οποίου ο χάρτης ενεργοποίησης έχει μέγεθος ίσο με (64, 16, 7, 7), έχουμε ως έξοδο ένα χάρτη χαρακτηριστικών με μέγεθος (64, 16, 7, 7).  \par

\subsubsection{\en Regression procedure}
\gr Στην αρχή, για κάθε προτεινόμενο \en ToI\gr, έχουμε τους αντίστοιχους χάρτες ενεργοποίησης χρησιμοποιώντας \en 3D ROI align\gr. Αυτά τα χαρακτηριστικά δίνονται ως είσοδο σε έναν \en Regressor\gr.
Αυτός επιστρέφει 16 $\cdot$ 4 προβλεπόμενες μεταβολές $(\delta_x,\delta_y, \delta_w,\delta_h)$, 4 για κάθε καρέ, όπου $ \delta_x, \delta_y$ καθορίζουν τις συντεταγμένες του κέντρου των προτάσεων και
$\delta_w, \delta_h$ το πλάτος και το ύψος του, όπως ορίζεται από τους \en\cite{DBLP:journals/corr/GirshickDDM13}\gr.
Κρατάμε μόνο τις προβλεπόμενες μεταβολές, για τα καρέ που $\ge t_1$ και $< t_2$ και για υπόλειπα θέτουμε ένα μηδενικό \en 2D \gr κουτί. 
Μετά από αυτό, τροποποιούμε κάθε \en anchor\gr, γραμμένο ως έναν κύβο δηλαδή γραμμένο ως $(x_1,y_1,t_1, x_2, y_2, t_2)$ σε μια ακολουθία πλαισίων \en 2D\gr, όπως: \\
$(0,0,0,0, ..., x_{T_1},y_{T_1},x'_{T_1},y'_{T_1}, ... ,x_{i},y_{i},x'_{i}, ..., x_{T_2},y_{T_2},x'_{T_2},y'_{T_2}, 0,0,0,0, ....)$, \\
όπου:
\begin{itemize}
\item $ T_1 \le i \le T_2$, για $T_1 < t_1 + 1,  T_2 < t_2 \text{ και }T_1,T_2 \in \mathbb{Z} $
\item $ x_i = x_1, y_i= y_1, x'_i = x_2, y'_i = y_2 $.
\end{itemize}

\paragraph{\tl{ Training}}
Για να εκπαιδεύσουμε τον \en Regressor \gr μας, ακολουθούμε τα ίδια βήματα που ακολουθήσαμε προηγουμένως για την προηγούμενη διαδικασία εκπαίδευσης του \en TPN\gr. Αυτό σημαίνει ότι 
επιλέγουμε τυχαία 16  \en ToIs \gr από αυτές που προτείνονται από το \en scoring layer \gr του \en TPN\gr. Απ' αυτά, 4 είναι τα \en anchors \gr προσκηνίου, το οποίο σημαίνει ότι αποτελούν το 25\%
του συνολικού αριθμoύ των \en anchors \gr όπως συνέβη προηγουμένως. Εξάγουμε τα αντίστοιχα χαρακτηριστικά τους χρησιμοποιώντας \en 3D ROI Algin \gr και υπολογίζουμε τους στόχους τους, όπως
κάναμε για το \en regression layer\gr. Τροφοδοτούμε το δίκτυο μας με αυτά τα χαρακτηριστικά και συγκρίνουμε τους προβλεπόμενους στόχους με τούς αναμενόμενους.
Ξανά πάλι, χρησιμοποιούμε \en smooth-L1 loss function \gr για τη συνάρτηση κόστους, υπολογίζoντας την μόνο για \en ToIs \gr που είναι στο προσκήνιο. Έτσι, προσθέτουμε μια άλλη παράμετρο στο
φόρμουλα απώλειας εκπαίδευσης που πλέον ορίζεται ως:
\en
\begin{equation} 
\begin{split}
 L  =  \sum_iL_{cls}(p_i, p_i^*) + \sum_iL_{cls}(p_{fixed,i}, p_{fixed,i}^*) + \\
 \sum_ip_i^*L_{reg}(t_i,t_i^*) + \sum_ip_{fixed,i}^*L_{reg}(t_{fixed,i},t_{fixed,i}^*) + \\
  \sum_iq_i^*L_{reg}(c_{i}, c_{i}^*) + \\
  % \sum_iL_{cls}(p_i, p_i^*) + \sum_iL_{cls}(p_{fixed,i}, p_{fixed,i}^*) 
\end{split}
\end{equation}
\gr όπου εκτός από τις παραμέτρους που καθορίστηκαν προηγουμένως, ορίζουμε $c_{i} $ ως στόχους παλινδρόμησης για τα επιλεγμένα tubes $q _I $.
Αυτά τα \en tubes \gr είναι που επιλέγονται τυχαία από τα προτεινόμενα \en ToIs \gr και  $q_i^*$ είναι τα αντίστοιχοι πραγματικά \en tubes\gr, οι οποίοι είναι τα  πλησιέστερα σε κάθε $q_i$ \en tube\gr.
Και πάλι χρησιμοποιούμε το $q_i^*$ ως παράγοντα, επειδή θεωρούμε ένα \en tube \gr ως φόντο όταν δεν επικαλύπτεται με οποιοδήποτε πραγματικό \en tube \gr περισσότερο από 0,5.

\paragraph{\tl{Validation}}
Χρησιμοποιούμε, ξανά, τη μετρική \en recall \gr για να αξιολογήσουμε την απόδοση τoυ παλινδρομητή. Υπολογίζουμε 3 επιδόσεις \en recall\gr:
\begin{description}
\item [\tl{Cuboid Recall},] που είναι η \en recall \gr μετρική για τα προτεινόμενα τρισδιάστατα κουτιά. Ενδιαφερόμαστε για αυτήν την επίδοση γιατί,
  θέλουμε να μάθουμε πόσο καλές είναι οι προτάσεις μας πριν τις τροποποιήσουμε σε σειρές κουτιών.
\item [\tl{Single frame Recall},] η οποία είναι η επίδοση \en recall \gr για τις προτεινόμενες ακολουθίες κουτιών σε σχέση με τις πραγματικές.
\item[\tl{Follow-up Single Frame Recall},] που είναι η απόδοση της ανάκλησης μόνο για τα τρισδιάστατα Κουτιά που ήταν πάνω από το όριο επικάλυψης μεταξύ
  των προτεινόμενων κυβων και των πραγματικών κύβων. Χρησιμοποιούμε αυτή τη μετρική για να γνωρίζουμε πόσα από τους προτεινόμενα τρισδιάστατα κουτιά
  κατέληξαν να είναι καλές προτάσεις.
\end{description}


\subsubsection{Αρχιτεκτονικές για τον \en Regressor \gr} 
Σχεδιάσαμε 2 προσεγγισεις για την υλοποίηση του \en Regressor\gr. Aυτές απεικονίζονται στις Εικόνες \ref{fig:regressor_3d} και \ref{fig:reg_1_2}, με την πρώτη προσέγγιση
να αποτελείται από ένα \en 3D Convolutional Layer \gr σε αντίθεση με την δεύτερη προσέγγιση που έχει ένα \en 2D Convolutional Layer\gr.

\begin{figure}[h]
  \centering
  \includegraphics[scale=0.48]{regressor_1_1}
  \caption{Structure of Regressor}
  \label{fig:regressor_3d}
\end{figure}

\begin{figure}[h]

  \centering
  \includegraphics[scale=0.48]{regressor_1_2}
  \caption{Structure of Regressor}
  \label{fig:reg_1_2}
\end{figure}

Η διαδικασία που ακολουθούν και οι δύο \en Regressors \gr περιγράφεται παρακάτω:
\begin{enumerate}
\item Αρχικά εξάγουμε τα αντίστοιχα \en feature maps \gr για κάθε \en ToI \gr χρησιμοποιώντας \en 3D Roi Align\gr, και ακολούθως τα κανονικοποιούμε. Μετά, στην πρώτη προσέγγιση
τροφοδοτούμε ένα \en 3D Convolutional Layer \gr με \en  kernel \gr ίσο με 1, \en stride \gr ίσο με 1 και χωρίς \en padding\gr. Στην συνέχεια, εφαρμόζουμε μια
\en pooling \grδιαδικασία στην διάσταση του χρόνου (είτε \en avg \gr είτε \en max pooling\gr). Απ'   την άλλη, στην δεύτερη προσέγγιση, πρώτα εφαρμόζουμε \en avg/max pooling \gr στην
διάσταση του χρόνου και στην συνέχεια τροφοδοτούμε ένα \en 2D Convolutional Layer \gr με \en kernel = 1, stride = 1 \gr και χωρίς \en padding\gr.
\item Και στις δύο περιπτώσεις λαμβάνουμε ως έξοδο ένα \en feature map \gr με διαστάσεις (64,7,7) το οποίο περνάμε διαδοχικά από 1 γραμμικό  \en layer, 1 Relu Layer, 1 Dropout Layer, \gr
  άλλο ένα γραμμικό \en Layer\gr, άλλο ένα \en ReLu layer \gr και ένα τελικό γραμμικό \en layer\gr. Το τελευταίο \en layer \gr μας δίνει ως έξοδο 64 στόχους, 4 $\cdot$ 16 μετακινήσεις.
\end{enumerate}


\begin{table}[h]
  \centering
  \en
  \begin{tabular} {||c | c || c | c | c ||}
    \hline
    \textbf{Dataset} & \textbf{Pooling} & \textbf{Cuboid} & \textbf{Singl. Fr. } &  \textbf{Follow-up S.F.}\\
    \hline                
    \multirow{2}{*}{JHMDB} & avg & 0.8545 & 0.7649 & 0.7183 \\
    \cline{2-5}
    {} & max & 0.8396 & 0.7761 & 0.5783 \\
    \cline{1-5}
    \multirow{2}{*}{UCF} & avg & 0.5319 & 0.4694 & 0.5754 \\
    \cline{2-5}
    {} & max & 0.5190 & 0.5021 & 0.5972 \\
    \cline{1-5}
                                   
  \end{tabular}
  \caption{Recall results after convertying cuboids into sequences of frames}
  \label{table:reg_1_1}
\end{table}

\begin{table}[h]
  \centering
  \en
  \begin{tabular}{||c | c | c || c  c  c ||}
    \hline
    \textbf{Dataset} & \textbf{Pooling} & \textbf{F. Map} & \textbf{Recall} &  \textbf{ Recall SR}  &  \textbf{Recall SRF} \\
    \hline
    \multirow{6}{*}{JHMDB} & \multirow{3}{*}{mean} & 64 &  0.6828  & 0.5112  & 0.7610 \\
    \cline{3-6}
    {} & {} & 128 & 0.8694 & 0.7799 & 0.6756 \\
    \cline{3-6}
    {} & {} & 256 & 0.8396 & 0.7687 & 0.7029 \\
    \cline{2-6}
    {} & \multirow{3}{*}{max} & 64 &  0.8582 & 0.7985 & 0.5914\\
    \cline{3-6}
    {} & {} & 128 & 0.8358 & 0.7724 & 0.8118 \\
    \cline{3-6}
    {} & {} & 256 & 0.8657 & 0.8022 & 0.7996 \\
    \hline
    \multirow{6}{*}{UCF} & \multirow{3}{*}{mean} & 64 & 0.5055 & 0.4286 & 0.5889 \\
    \cline{3-6}
    {} & {} & 128 & 0.5335 & 0.4894 & 0.5893 \\
    \cline{3-6}
    {} & {} & 256 & 0.5304 & 0.4990 & 0.6012 \\
    \cline{2-6}
    {} & \multirow{3}{*}{max} & 64 & 0.5186 & 0.4990 & 0.5708 \\
    \cline{3-6}
    {} & {} & 128 & 0.5260 & 0.4693 & 0.5513 \\
    \cline{3-6}
    {} & {} & 256 & 0.5176 & 0.4878 & 0.6399 \\
    \hline

  \end{tabular}
  \caption{Recall performance using 3 different feature maps as Regressor's input and 2 pooling methods}
  \label{table:reg_1_2}
\end{table}
\gr 
Οι πίνακες \ref{table:reg_1_1} και \ref{table:reg_1_2} περιέχουν τα αποτελέσματα για την πρώτη και δεύτερη προσέγγιση. Μάλιστα, για την δεύτερη προσέγγιση ελεγξαμε 3 διαφορετικούς χάρτες
χαρακτηριστικών, ενώ και στις 2 περιπτώσεις πειραματιστήκαμε χρησιμοποιοώντας και τις 2 προαναφερθέντες \en pooling \gr μεθόδους για τα σύνολα δεδομένων \en JHMDB \gr και \en UCF-101\gr.
Παρατηρούμε ότι, με βάση τα παραπάνω αποτελέσματα, λαμβάνουμε σχετικά χαμηλή \en recall \gr απόδοση για το \en dataset UCF \gr ενώ για το \en JHMDB \gr  τα αποτελέσματα είναι κάπως καλύτερα.
Πιο συγκεκριμένα, με βάση την πρώτη προσέγγιση λαμβάνουμε τελικά \en recall \gr απόδοση ίση με 76-77\% για το \en JHMDB \gr και 46-50\% για το \en UCF\gr. Με βάση την δεύτερη προεγγιση,
στην καλύτερη περίπτωση λαμβάνουμε 80\% απόδοση \en recall  \gr για το \en JHMDB \gr ενώ για το \en UCF \gr παραμένουμε στα ίδια περίπου αποτελέσματα.
Παράλληλα παρατηρούμε ότι χάνουμε περίπου 30-40\% από τις καλές \en cuboid \gr προτάσεις και στις 2 περιπτώσεις, το οποίο αποτελεί μεγάλο πρόβλημα και των δύο προσεγγίσεων.
Όλ' αυτά μας κάνουν να ξανασκεφτούμε τον τρόπο που σχεδιάσαμε το \en TPN \gr και μας οδήγησε στο να σχεδιάσουμε ένα νέο μοντέλο.


\section{Τα τριασδιάστατα anchors ως \tl{4k} διανύσματα}
Σε αυτή την προσέγγιση, ορίζουμε τους \en 3D anchors \gr ως διανύσματα με \en 4k \gr συντεταγμένες (\tl{k} = 16 καρέ = διάρκεια δείγματος).
Έτσι ένα τυπικό \en anchor \gr γράφεται ως ($x_1, y_1, x'_1, y'_1, x_2, y_2, ...$)
όπου $x_1, y_1, x'_1, y'_1 $ είναι οι συντεταγμένες για 1ο καρέ, $x_2, y_2, x'_2, y'_2$ για το 2ο καρέ κλπ, όπως παρουσιάστηκε από τους \cite{DBLP:journals/corr/abs-1712-09184}. 
Η εικόνα \ref{fig:anchor_4k} απεικονίζει ένα τέτοιου τύπου \en anchor\gr.

\begin{figure}[h]
  \centering
  \includegraphics[scale=0.5]{anchor_4k}
  \caption{An example of the anchor $(x_1,y_1,x'_1,y'_1,x_2,y_2, ...)$}
  \label{fig:anchor_4k}
\end{figure}

Το κύριο πλεονέκτημα αυτής της προσέγγισης είναι ότι δεν χρειάζεται να μεταφράσουμε τα \en 3D anchors \gr σε 2D κουτιά, γεγονός που προκάλεσε πολλά προβλήματα στην προηγούμενη προσέγγιση.
Ωστόσο,αυτή η προσέγγιση έχει ένα μεγάλο μειονέκτημα, το οποίο είναι το γεγονός ότι αυτός ο τύπος \en anchor \gr έχει σταθερή χρονική διάρκεια.
Για να αντιμετωπίσουμε αυτό το πρόβλημα, έχουμε ορίσει \en anchors \gr με διαφορετικές χρονικές διάρκειες, οι οποίες είναι 16, 12, 8 και 4 καρέ.
\en Anchors \gr με διάρκεια $<$  διάρκεια του δείγματος (16 καρέ) μπορούν  να γραφτούν ως διάνυσμα \en 4k \gr με μηδενικές συντεταγμένες στα καρέ μεγαλύτερα από τη χρονική διάρκεια.
Για παράδειγμα, ένα \en anchor \gr με 2 καρέ διάρκεια, ξεκινώντας από το 2ο καρέ και τερματίζοντας στον 3ο μπορεί να γραφεί ως
(0, 0, 0, 0, $x_1, y_1, x'_1, y'_1, x_2, y_2, x'_2, y'_2$, 0, 0, 0, 0)  εάν η διάρκεια δείγματος
 είναι 4 καρέ. 

\begin{figure}[h]
  \centering
  % \includegraphics[width=1.\textwidth]{tpn_2}
  \includegraphics[scale=0.4]{tpn_2}
  \caption{The structure of TPN according to new approach}
  \label{fig:New_structure}
\end{figure}
Αυτή η νέα προσέγγιση μας οδήγησε στο να αλλάξουμε την δομή του \en TPN\gr. Η νέο δομή του απεικονίζεται στην εικόνα  \ref{fig:New_structure}. Όπως
μπορούμε να δούμε, προσθέσαμε \en scoring \gr και \en regression layers \gr για κάθε διάρκεια. Έτσι, το \en TPN \gr ακολουθεί τα επόμενα βήματα για να παράγει \en ToIs\gr.
\begin{enumerate}
\item Στην αρχή, τροφοδοτούμε τον χάρτη χαρακτηριστικών, που εξάγεται από το \en 3D ResNet34\gr, ως είσοδο σε ένα \en 3D Convolutional Layer \gr με μέγεθος πυρήνα = 1,
  \en stride = 1 \gr και χωρίς \en padding\gr.
\item Από το \en Convolutional Layer\gr, έχουμε ως έξοδο ένα χάρτη ενεργοποίησης με διαστάσεις (256, 16, 7, 7). Για τη μείωση της χρονικής διάστασης, χρησιμοποιούμε 4 \en pooling layers\gr,
  ένα για κάθε δείγμα διάρκειας με μεγέθη πυρήνα  \textit{(16, 1, 1), (12, 1, 1,), (8, 1, 1) και (4, 1, 1)} και \en stride = 1\gr, για τη διάρκεια του δείγματος 16, 12, 8 και 4 καρέ αντίστοιχα.
  Έτσι, έχουμε χάρτες ενεργοποίησης με διαστάσεις \textit{(256, 1, 7, 7), (256, 5, 7, 7), (256, 9, 7, 7) και (256, 13, 7, 7)}, στης oποίες η δεύτερη διάσταση είναι ο αριθμός των πιθανών
  χρονικών διακυμάνσεων. Για παράδειγμα, σε  χάρτη χαρακτηριστικών με διαστάσεις $(256, 5, 7, 7)$, το οποίο σχετίζεται με \en anchors \gr με διάρκεια 12 καρέ, μπορούμε να έχουμε 5 πιθανές περιπτώσεις,
  από το καρέ 0 μέχρι το καρέ 11, από το καρέ 1 μέχρι το καρέ 12 κλπ.
  
\item Ξανά, όπως και στην προηγούμενη προσέγγιση, για κάθε \en pixel \gr του χάρτη ενεργοποίησης αντιστοιχούμε \en\textbf{n = k = 15}
  anchors \gr(5 κλίμακες από 1, 2, 4, 8, 16, και 3 \en aspect ratios \gr 1:1, 1:2, 2:1). Φυσικά, έχουμε 4 διαφορετικούς χάρτες ενεργοποίησης, με 1, 5, 9 και 13
  διαφορετικές περιπτώσεις και  $7  \times 7$ διαστάσεις σε κάθε φίλτρο. Έτσι, συνολικά έχουμε $28  \cdot 15 \cdot 49 = 20580$ διαφορετικά \en anchors\gr.
  Αντίστοιχα, έχουμε 20580 διαφορετικούς στόχους \en regression\gr.

\end{enumerate}

\subsection{\tl{Training}}
Η διαδικασία training παραμένει σχεδόν η ίδια όπως και στην προηγούμενη προσέγγιση. Έτσι, και πάλι, εμείς τυχαία επιλέγουμε ένα τμήμα βίντεο και τα αντίστοιχα πραγματικά \en tubes\gr. Όμως,
θεωρούμε τα \en anchors \gr ως προσκήνιο όταν έχουν επικάλυψη  μεγαλύτερη από 0,8 με οποιαδήποτε πραγματικό \en tube\gr, ενώ θεωρούμε \en anchors \gr παρασκηνίου αυτά  των οποίων η επικάλυψη
είναι μεγαλύτερη που 0,1 και μικρότερη από 0,3. Δεν ασχολούμαστε με τα υπόλοιπα \en anchors\gr. 

\begin{table}[h]
  \centering
  \en
  \begin{tabular}{||c | c || c  c c||}
    \hline
    \textbf{Dataset} & \textbf{Pooling} &  \textbf{Recall(0.5)} & \textbf{Recall(0.4)} & \textbf{Recall(0.3)} \\
    \hline  \hline
    \multirow{2}{4em}{JHMDB} & mean & 0.6866 & 0.7687 & 0.8582 \\
    \cline{2-5}
    {} & max &  0.8134 & 0.8694 & 0.9216 \\
    \hline
    \multirow{2}{4em}{UCF} & avg &  0.5435 & 0.6326 & 0.7075 \\
    \cline{2-5}
    {} & max & 0.6418 & 0.7255 & 0.7898 \\
    \hline
  \end{tabular}
  \caption{Recall results using 2nd approach for anchors}
  \label{table:tpn_2_1}
\end{table}

Όπως δείχνει ο πίνακας \ref{table:tpn_2_1}, είναι προφανές ότι έχουμε καλύτερες επιδόσεις \en recall \gr σε σύγκριση με την προηγούμενη προσέγγιση.
Επιπλέον, μπορούμε να δούμε ότι το \en 3D max pooling \gr  αποδίδει καλύτερα από το \en 3D avg pooling\gr. Η διαφορά
μεταξύ των δύο είναι περίπου 10\%, η οποία είναι αρκετά μεγάλη για να μας κάνει να επιλέξουμε το \en max pooling \gr ως λειτουργία ομαδοποίησης πριν από τη
λήψη των \en scores \gr και \en regression targets \gr των \en anchors\gr.

\subsection{Προσθήκη \tl{Regressor}}

Ακόμα και αν, το  \en TPN \gr μας εξάγει κουτιά σε επίπεδο καρέ, πρέπει να βελτιώσουμε αυτές τις προβλέψεις προκειμένου να αλληλοεπικαλύπτονται
με τα πραγματικά κουτιά  όσο το δυνατόν καλύτερα.
Έτσι, σε πλήρη αντιστοιχία με την προηγούμενη προσέγγιση, προσθέσαμε έναν \en Regressor \gr για να προσπαθήσουμε να πάρουμε καλύτερα αποτελέσματα \en recall\gr.

\paragraph{\tl{3D Roi align}}
Σε αυτή την προσέγγιση, γνωρίζουμε ήδη τις χωρικές συντεταγμένες. Έτσι, μπορούμε να χρησιμοποιήσουμε τη μέθοδο που προτείνεται από τους \en\cite{DBLP:journals/corr/abs-1712-09184}\gr. Αποτελεί
επέκταση του \en RoiAlign \gr χωρίζοντας το \en tube \gr σε $T$ δυσδιάστατα κουτιά. Στη συνέχεια, χρησιμοποιείται το  κλασικό \en RoiAlign \gr για να εξαχθεί  μια περιοχή από κάθε μίας από τα 
χρονικά \en slices \gr στον χάρτη ενεργοποίησης. Μετά από αυτό, τα \en feature maps \gr που προέκυψαν μέσω του \en RoiAlign \gr συνδέονται στην διάσταστη του χρόνου, ώστε να προκύψει
χάρτης χαρακτηριστικών με διαστάσεις $T \times R \times R$,  όπου $R $ είναι η ανάλυση εξόδου του \en RoiAlign\gr, το οποίο είναι 7 στην περίπτωσή μας.  \par


Εκπαιδεύουμε τον \en Regressor \gr μας χρησιμοποιώντας την ίδια \en loss function \gr όπως ο τύπος της προηγούμενης προσέγγισης που είναι:
\begin{equation*} 
\begin{split}
 L  =  \sum_iL_{cls}(p_i, p_i^*) + \sum_iL_{cls}(p_{fixed,i}, p_{fixed,i}^*) + \\
 \sum_ip_i^*L_{reg}(t_i,t_i^*) + \sum_ip_{fixed,i}^*L_{reg}(t_{fixed,i},t_{fixed,i}^*) + \\
  \sum_iq_i^*L_{reg}(c_{i}, c_{i}^*) + \\
  % \sum_iL_{cls}(p_i, p_i^*) + \sum_iL_{cls}(p_{fixed,i}, p_{fixed,i}^*) 
\end{split}
\end{equation*}

Και σ' αυτήν την προσέγγιση χρησιμοποιούμε 2 διαφορετικές αρχιτεκτονικές για την υλοποίηση του \en Regressor\gr.
Ως πρώτη προσέγγιση, χρησιμοποιούμε ένα \en 3D Convolutional Layer \gr ακολουθούμενο από 2 γραμμικά \en Layer\gr. Aντίστοιχα, στην δεύτερη προσέγγιση χρησιμοποιούμε ένα \en 2D
Convolution Layer \gr ακολουθούμενο, και αυτό, από 2 γραμμικά \en Layer\gr. H πρώτη προσέγγιση είναι ακριβώς η ίδια με πριν. Ωστόσο, η δεύτερη προσέγγιση
αντιμετοπίζει τα \en feature maps \gr σαν να μην υπάρχουν χρονικές εξαρτήσεις μεταξύ τους. Δηλαδή:
\begin{enumerate}
\item Στην αρχή, χρησιμοποιούμε το \en 3D Roi Align \gr για να εξάγουμε τα \en feature maps \gr Και μετά τα κανονικοποιούμε. Έστω λοιπόν ότι προκύπτουν \tl{k} χάρτες ενεργοποιήσης
  με διαστάσεις (\tl{k}, 256, 16, 7, 7)
\item Χωρίζουμε τα υποψήφια \en ToIs \gr σε \en T 2D \gr κουτιά, οπότε οι διαστάσεις του τένσορα που περιέχει τις συντεταγμένες των \en ToIs \gr γίνονται από $(k,4\cdot sample duration)$
  σε $(k,sample duration, 4)$. Διαφοροποιούμε τον τένσορα ώστε να πάρει τις διαστάσεις  $(k\cdot sample duration, 4)$, όπου οι πρώτες \tl{k} συντεταγμένες αναφέρονται
  στο πρώτο καρέ, οι επόμενες \tl{k} στο δεύτερο κλπ.
\item Αντίστοιχα διαφοροποιούμε και τους χάρτες ενεργοποίησης όπου από διαστάσεις $(k, 64, sample duration, 7, 7)$ καταλήγομε σε χάρτες με διαστάσεις $(k\cdot sample duration, 64, 7, 7)$.
  Πλέον λοιπόν επεξεργαζόμαστε του χάρτες χαρακτηριστικών σαν να είναι δυσδιάστατοι. Έτσι τροφοδοτούμε το \en 2D Convolutional Layer \gr και ακολουθούμενο από τα άλλα δύο γραμμικά \en Layer\gr.
  \item Τα εξαγώμενα \en targets\gr, φυσικά θα είναι μόνο 4 όσο δηλαδή για 1 καρέ.
\end{enumerate}

\begin{table}[h]
  \centering
  \en
  \begin{tabular}{||c | c || c  c  c||}
    \hline
    \textbf{Dataset} & \textbf{Feat. Map} & \textbf{Recall(0.5)} & \textbf{Recall(0.4)} & \textbf{Recall(0.3)}\\
    \hline
    \multirow{2}{*}{JHMDB} &  64 & 0.7985 & 0.903 & 0.9552 \\
    \cline{2-5}
    {} & 128 & 0.7836 & 0.8881 & 0.944\\
    \hline
    \multirow{2}{*}{UCF}  & 64 & 0.5794 & 0.7206 & 0.8134 \\
    \cline{2-5}
    {} & 128 & 0.5622 & 0.7204 & 0.799 \\
    \hline

  \end{tabular}
  \caption{Recall performance when using a 3D Convolutional Layer in Regressor's architecture}
  \label{table:reg_2_1}
\end{table}

\begin{table}[h]
  \centering
  \en
  \begin{tabular}{||c | c || c  c  c||}
    \hline
    \textbf{Dataset}  & \textbf{Feat. Map} & \textbf{Recall(0.5)} & \textbf{Recall(0.4)} & \textbf{Recall(0.3)}\\
    \hline
    \multirow{3}{*}{JHMDB} & 64 & 0.8358 & 0.9216 & 0.9739\\
    \cline{2-5}
    {} & 128 & 0.8172 & 0.9142 & 0.9627 \\
    \cline{2-5}
    {} & 256 & 0.7724 & 0.8731 & 0.9328 \\
    \hline
    \multirow{3}{*}{UCF} & 64 & 0.6368 & 0.7346 & 0.7737 \\ 
    \cline{2-5}
    {} & 128 & 0.6363 & 0.7133 & 0.7822 \\
    \cline{2-5}
    {} & 256 &  0.6363 & 0.7295 & 0.7822 \\
    \hline

  \end{tabular}
  \caption{Recall performance when using a 2D Convolutional Layer instead of 3D in Regressor's model}
  \label{table:reg_2_2}
\end{table}
\gr Με βάση τους παραπάνω πίνακες \ref{table:reg_2_1} και \ref{table:reg_2_2}, η καλύτερη επίδοση προκύπτει για τους χάρτες χαρακτηριστικών μεγέθους $(64,16,7,7)$. Αυτό το
αποτέλεσμα ήταν αναμενόμενο γιατί αυτοί οι χάρτες ενεργοποίησης βρίσκονται «πιο κοντά» στα πραγρατικά χαρακτηριστικά. Συγκρίνοντας
τις δύο προτεινόμενες μεθόδους παρατηρούμε ότι η δεύτερη, αυτή δηλαδή που χρησιμοποιεί \en 2D Convolutional Layer \gr έχει τα καλύτερα
αποτελέσματα. Αν και βελτιώσαμε την απόδοση του \en TPN \gr ακόμα δεν έχουμε καταφέρει να έχουμε σχεδόν σε όλες τις περιπτώσεις
καλές προτάσεις \en tube\gr.
\subsection{Μείωση της διάρκειας του δείγματος}
Προκειμένου να πετύχουμε καλύτερα αποτελέσματα αποφασίσαμε να μειώσουμε την διάρκεια του δείγματος από τα 16 καρέ σε 8 και 4 αντίστοιχα. Κι
αυτό γιατί με αυτόν τον τρόπο θα μειωθούν παράλληλα με τον αριθμός τους οι διαστάσεις των \en anchors, \gr  άρα και οι διαστάσεις των στόχων παλιδρόμησης και γενικά ο αριθμός των παραμέτρων που πρέπει να εκπαιδευτούν από το σύστημα. \par
Εκπαιδεύομαι το \en TPN \gr με και χωρίς \en Regressor \gr προκειμένου να βρούμε την κατάλληλη αρχιτεκτονική. Τα αποτελέσματα χωρίς \en Regressor \gr
παρουσιάζονται στον πίνακα \ref{table:new_sample} ενώ τα αποτελέσματα της αρχιτεκτονικής με \en Regressor \gr στον πίνακα \ref{table:new_sample_reg}.

\begin{table}[h]
  \centering
  \en
  \begin{tabular}{|c | c || c c c|}
    \hline
    \textbf{Dataset} & \textbf{Sample dur} & \textbf{Recall(0.5)} &  \textbf{Recall(0.4)} &  \textbf{Recall(0.3)} \\
    \hline
    \multirow{3}{*}{JHMDB} & 16 & 0.8134 & 0.8694 & 0.9216 \\
    \cline{2-5}
    {} & 8 & 0.9515 & 0.9888 & 1.0000 \\
    \cline{2-5}
    {} & 4 & 0.8843 & 0.9627 & 0.9888 \\
    \hline
    \multirow{3}{*}{UCF} & 16 & 0.6418 & 0.7255 & 0.7898 \\
    \cline{2-5}
    {} & 8 & 0.7942 & 0.8877 & 0.9324\\
    \cline{2-5}
    {} & 4 & 0.7879 & 0.8924 & 0.9462 \\
    \hline
    
  \end{tabular}
  \caption{Recall results when reducing sample duration to 4 and 8 frames per video segment}
  \label{table:new_sample}
\end{table}

\begin{table}[h]
  \centering
  \en
  \begin{tabular}{|c | c | c || c c c|}

    \hline
    \textbf{Dataset} & \textbf{Sample dur} & \textbf{Type} & \textbf{Recall(0.5)} &  \textbf{Recall(0.4)} &  \textbf{Recall(0.3)} \\
    \hline
    \multirow{4}{*}{UCF} & \multirow{2}{*}{8} & 2D & 0.8078 & 0.8870 & 0.9419 \\
    \cline{3-6}
    {} & {} & 3D & 0.8193 & 0.8930 & 0.9487 \\
    \cline{2-6}
    {} & \multirow{2}{*}{4}& 2D & 0.7785 & 0.8914 & 0.9457 \\
    \cline{3-6}
    {} & {} & 3D & 0.7449 & 0.8605 & 0.9362 \\
    \hline
    \multirow{4}{*}{JHDMBD} & \multirow{2}{*}{8} & 2D &  0.9366 & 0.9851 & 0.9925  \\
    \cline{3-6}
    {} & {} & 3D & 0.8918 & 0.9776 & 0.9963  \\ 
    \cline{2-6}
    {} & \multirow{2}{*}{4}& 2D & 0.9552 & 0.9963 & 1.0000 \\
    \cline{3-6}
    {} & {} & 3D & 0.9142 & 0.9701 & 0.9888  \\
    \hline
    
  \end{tabular}
  \caption{Recall results when a regressor and sample duration equal with 4 or 8 frames per video segment}
  \label{table:new_sample_reg}

\end{table}

Το πρώτο συμπέρασμα που προκύπτει από τους πίνακες \ref{table:new_sample} και  \ref{table:new_sample_reg}  είναι ότι όντως μειώνοντας την
διάρκεια του δείγματος πετυχαίνομε καλύτερα αποτελέσματα. Παράλληλα, με βάση τον πίνακα \ref{table:new_sample_reg} και δεδομένων των
προηγούμενων αποτελεσμάτων (Πίνακες \ref{table:reg_2_1} και \ref{table:reg_2_2}) γίνεται ξεκάθαρο ότι η προσέγγιση που περιλαμβάνει
ένα \en 2D Convolutional Layer \gr υπερέχει αυτής με το \en 3D Convolutional Layer\gr. Σ' οτι αφορά τα σύνολα δεδομένων παρατηρούμε ότι
η προσέγγιση με διάρκεια δείγματος 8 καρέ υπερέχει σχεδόν σε όλες τις περιπτώσεις. Συνεπώς, για τα επόμενα κεφάλαια θα προτιμάται
να χρησιμοποιείται αυτή.
\end{document}
\documentclass{report}

% \usepackage[greek,english]{babel}
% % 
\usepackage{polyglossia}


\setmainfont[Mapping=tex-text]{Liberation Sans } % choose a font that supports greek characters
% \setmainfont[Mapping=tex-text]{GFS Bodoni } % choose a font that supports greek characters Auti tha valw

\setdefaultlanguage{english}
\setotherlanguage[variant=modern]{greek}

% \newcommand{\tl}{\textlatin}
% \newcommand{\en}{\selectlanguage{english}}
% \newcommand{\gr}{\selectlanguage{greek}}

\usepackage{hyperref}  % package for linking figures etc
\usepackage{enumitem}  % package for description with bullets
\usepackage{graphicx}  % package for importing images
\usepackage{mathtools} % package for math equation
\usepackage{mathrsfs}  % package for math font
\usepackage{indentfirst} % package for getting ident after section or paragraph
\usepackage{subcaption} % package for subfigures
\usepackage[export]{adjustbox}
\usepackage{longtable} % package for multi pages tables
\usepackage{multirow}  % package for tables, multirow
\usepackage{amssymb}
\usepackage{esvect}
\usepackage[
backend=bibtex,
citestyle=authoryear,
% citestyle=authoryear-comp,
% citestyle=authoryear-ibid,
bibstyle=numeric,
sorting=ynt,
% style=numeric,
% style=alphabetic ,
]{biblatex}
\addbibresource{References}

\graphicspath{ {./theory/figures/} }       % path for images

\begin{document}
 
\chapter{Αλγόριθμος σύνδεσης των tubes}

In previous chapter we described methods for generating candidate action tubes given a small video segment lasting 8 or 16 frames. However, actual videos
and actual human actions, in wild, last more than 16 frames most of the times. Current networks are unable to process a whole video at once, in order to generate action tubes
due to memory and computing power issues.  As mentioned in chapter 2, a lot action localization approaches deal with this situation by given a video either
propose candidate areas in frame-level and then they connect these in order to generate action tube proposal either, seperate it into video segments,
proposing  sequences of bounding boxes for each video segment and then link them in order to generate action proposal. Both aforementioned techniques makes
the suitable choice of linking method an important factor for the performance of the network. That's because, even though frame-level or video segment-level
might be very good, if the proposed connection algorithm doesn't work well, final action tube proposals won't be efficient, so the final model will never
be able to achieve high classification performance. In other words, if connecting algorithm doesn't generate action proposals with great recall and MABO performance,
the model's classifier won't be able to perform suitable classification, because propably it would be given action tubes without any context.
In this chapter, we present 3 different approaches used for linking proposed ToIs generated from TPN in the previous chapter.

\section{First approach: combine overlap and actioness}
Our algorithm is inspired by \cite{DBLP:journals/corr/HouCS17}, which calculates all possible sequences of ToIs. In order find the best candidates,
it uses a score which tells us how likely a sequence of TOIs is  to contain an action. This score is a combination of 2 metrics:
\begin{description}
\item[ Actioness,  ] which is the TOI's possibility to contain an action. This score is produced by TPN's scoring layers.
\item [ TOIs' overlapping, ] which is the IoU of the last frames of the first TOI and the first frames of the second TOI.
\end{description}

The above scoring policy can be described by the following formula:
\[ S = \frac{1}{m} \sum_ {i=1}^{m} Actioness_i + \frac{1}{m-1} \sum_{j=1}^{m-1} Overlap_{j,j+1} \]

For every possible combination of TOIs we calculate their score as show in figure \ref{fig:connection_algo}.

\begin{figure}[h]
  \centering
  \includegraphics[scale=0.225]{connection_algo}
  \caption{An example of calculating connection score for 3 random TOIs taken from \cite{DBLP:journals/corr/HouCS17}}
  \label{fig:connection_algo}
\end{figure}

The above approach, however, needs too much memory for all needed calculations, so a memory usage  problem is
appeared. The reason is, for every new video segments we propose \textit{k TOIs} (16 during training and 150 during validation).
As a result, for a small video seperated in  \textbf{10 segments}, we need to calculate 
\textbf{  150\textsuperscript{10} scores} during validation stage. This causes our system to overload and it takes too much time to process
just one video. \par

In order to deal with this problem, we create a greedy algorithm in order to find the candidates tubes. Inituitively, this algorithm after
a new video segment keeps tubes with score higher than a threshold, and deletes the rest. So, we don't need to calculate combinations with
very low score. We wrote code for calculating tubes' scores in CUDA language, which has the ability to
parallel process the same code using different data. Our algorithm is described below:

\begin{enumerate}
\item Firstly,  initialize empty lists for the final tubes, final tubes' duratio, their scores, active tubes, their correnspondig duration,
  active tubes' overlapping sum and actioness sum where:
  \begin{itemize}
  \item Final tubes list contains all tubes which are the most likely to contain an action, and their score list contains their
    corresponding scores. We refer to each tube by its index which is related a tensor, in which we saved all the ToIs proposed
    from TPN for each video segment.
  \item Active tubes list contains all tubes that will be match with the new TOIs. Their overlapping sum list and actioness sum list
    contain their sums in order to avoid calculating then for each loop. 
  \end{itemize}
Also, we initialize threshold equal to 0.5 .
\item For the first video segment, we add all the TOIs to both active tubes and final tubes. Their scores are only their actioness because
  there are no tubes for calculating their overlapping score. So, we set their overlaping sum equal to 0.
\item For each next video, after getting the proposed ToIs, firstly we calculate their overlapping score with each active tube. Then, we
  empty active tubes, active tubes' duration, overlapping sum and actioness score lists.  For each new tube that has score higher than the threshold
  we add both to final tubes and to active tubes, and we increase their duration.
\item If the number of active tubes is higher than a threshold, we set the threshold equal to the score of
  the 100th higher score. On top of that, we update the final tubes list, removing all tubes that have score lower than the threshold.
\item After that, we add in active tubes, the current video segment's proposed TOIs. Also their actioness scores in actioness sum list and
  zero values in corrensponding positions in overlaps sum list (such as in the 1st step).
\item We repeat the previous 3 steps until there is no video segment left.
\item Finally, as we mentioned before, we have a list which contains the indexes of the saved tubes. So, we modify them in order to have
  the final bounding boxes. However, 2 succeeding ToIs do not have extactly the same bounding boxes in the frames that overlap. For example,
  ToIs from the $1^{st}$ video segment start from frame 1 to frame 16. If we have video step equal with 8, it overlaps with the ToIs from the
  succeeding video segment in frames 8-16. In those frames, in final tube, we choose the area that contains both bounding boxes which is
  denoted as $(min(x_1,x'_1), min(y_1,y'_1), max(x_2,x'_2), max(y_2,y'_2))$ for bounding boxes $(x_1,y_1,x_2,y_2)$ and $(x'_1,y'_1,x'_2,y'_2)$.
\end{enumerate}

Μερικά ενδεικτικά αποτελέσματα
sample duration 16
sample duration 8

\subsection {UCF Dataset}
enas pinakas me ola

\section{second approach}

\section{third approach}
enas pinakas


\end{document}
\documentclass{report}

% \usepackage[greek,english]{babel}
% % 
\usepackage{polyglossia}


\setmainfont[Mapping=tex-text]{Liberation Sans } % choose a font that supports greek characters
% \setmainfont[Mapping=tex-text]{GFS Bodoni } % choose a font that supports greek characters Auti tha valw

\setdefaultlanguage{english}
\setotherlanguage[variant=modern]{greek}

% \newcommand{\tl}{\textlatin}
% \newcommand{\en}{\selectlanguage{english}}
% \newcommand{\gr}{\selectlanguage{greek}}

\usepackage{hyperref}  % package for linking figures etc
\usepackage{enumitem}  % package for description with bullets
\usepackage{graphicx}  % package for importing images
\usepackage{mathtools} % package for math equation
\usepackage{mathrsfs}  % package for math font
\usepackage{indentfirst} % package for getting ident after section or paragraph
\usepackage{subcaption} % package for subfigures
\usepackage[export]{adjustbox}
\usepackage{longtable} % package for multi pages tables
\usepackage{multirow}  % package for tables, multirow
\usepackage{amssymb}
\usepackage{esvect}
\usepackage[
backend=bibtex,
citestyle=authoryear,
% citestyle=authoryear-comp,
% citestyle=authoryear-ibid,
bibstyle=numeric,
sorting=ynt,
% style=numeric,
% style=alphabetic ,
]{biblatex}
\addbibresource{References}

\graphicspath{ {./theory/figures/} }       % path for images

\begin{document}
 
\chapter{Classification stage}


\section{JHDMB dataset}
\subsection{Linear-RNN-MLP-SVM}

\section{UCF dataset}
\subsection{temporal localization}
\end{document}
% \documentclass{report}

% \usepackage[ english, greek]{babel}
% \usepackage[utf8]{inputenc}
% \usepackage[LGR, T1]{fontenc}

% % % 

% \newcommand{\tl}{\textlatin}
% \newcommand{\en}{\selectlanguage{english}}
% \newcommand{\gr}{\selectlanguage{greek}}

% \usepackage{hyperref}  % package for linking figures etc
% \usepackage{enumitem}  % package for description with bullets
% \usepackage{graphicx}  % package for importing images
% \usepackage{mathtools} % package for math equation
% \usepackage{mathrsfs}  % package for math font
% \usepackage{indentfirst} % package for getting ident after section or paragraph
% \usepackage{subcaption} % package for subfigures
% \usepackage[export]{adjustbox}
% \usepackage{longtable} % package for multi pages tables
% \usepackage{multirow}  % package for tables, multirow
% \usepackage{amssymb}
% \usepackage{esvect}
% \usepackage[
% backend=bibtex,
% citestyle=authoryear,
% % citestyle=authoryear-comp,
% % citestyle=authoryear-ibid,
% bibstyle=numeric,
% sorting=ynt,
% % style=numeric,
% % style=alphabetic ,
% ]{biblatex}
% \addbibresource{References}

% \graphicspath{ {./theory/figures/} }       % path for images

% \begin{document}
\gr 

\chapter{Επίλογος - Μελλοντικές επεκτάσεις}
\section{Επίλογος}
Σε αυτή τη διατριβή εξερευνούμε το πρόβλημα της αναγνώρισης και του εντοπισμού της δράσης.
Σχεδιάζουμε ένα δίκτυο βασισμένο στην προσέγγιση των  \cite{DBLP:journals/corr/HouCS17} σε συνδυασμό με ορισμένα στοιχεία από τους \en\cite{DBLP:journals/corr/abs-1712-09184},
\cite{Ren:2015:FRT:2969239.2969250}, \cite{Girshick:2015:FR:2919332.2920125}, \cite{DBLP:journals/corr/abs-1903-00304} \gr
 και \en \cite{hara3dcnns}\gr. \par

 Γράψαμε μια \en pytorch \gr υλοποίηση παίρνοντας κώδικα  μόνο από το \en \cite{jjfaster2rcnn}\gr. Επιπλέον, γράψαμε τον δικό μας κώδικα χρησιμοποιώντας
 μερικές λειτουργίες της γλώσσας \en CUDA \gr που έχουν σχεδιαστεί από εμάς (όπως
υπολογισμός των βαθμολογιών σύνδεσης, τροποποίηση σωλήνων κλπ). \par

Προσπαθήσαμε να σχεδιάσουμε   το \en TPN \gr, ένα δίκτυο  για την προτάσεις \en ToIs, \gr ακολουθίες πλαισίων δηλαδή 
σε δεδομένο βίντεο, εμπνευσμένο από το \en  RPN \gr  του \en Faster R-CNN\gr. To σχεδιάσαμε
χρησιμοποιώντας γενικευμένα \en anchors \gr  και όχι συγκεκριμένα για κάθε σύνολο δεδομένων. Προσπαθούμε δηλαδή
να γενικεύσουμε την προσέγγισή μας για διάφορα σύνολα δεδομένων, αντίθετα με την προσέγγιση
που προτείνεται από τους \en \cite{DBLP:journals/corr/abs-1712-09184}\gr, στην οποία χρησιμοποιoύνται τα πιο συχνά εμφανιζόμενα
\en anchors \gr για κάθε σύνολο δεδομένων.

Επιπροσθέτως, σχεδιάσαμε έναν αφελή αλγόριθμο σύνδεσης για τη σύνδεση
των προτεινόμενων \en ToIs \gr  με βάση αυτόν που προτάθηκε απ' τους \en \cite{DBLP:journals/corr/abs-1712-09184}\gr.
Στην προσέγγισή μας, χρησιμοποιούμε την ίδια πολιτική βαθμολόγησης, η οποία είναι ένας συνδυασμός των βαθμολογιών της πιθανότητας δράσης και του σκορ επικάλυψης.
Η κύρια διαφορά είναι ότι αποφεύγουμε να υπολογίζουμε
πιθανούς συνδυασμούς χρησιμοποιώντας ένα όριο ενημέρωσης, που μπορεί να ανανεώνεται. Επίσης, δοκιμάσαμε κι άλλον έναν
αλγόριθμο σύνδεσης εμπνευσμένος απ' τους \cite{DBLP:journals/corr/abs-1903-00304}.
Ωστόσο, η εφαρμογή μας δεν ήταν πολύ καλή, συνεπώς δεν εξερευνήσαμε όλες τις δυνατότητες του.

Τέλος, διερευνήσαμε αρκετούς ταξινομητές  για το στάδιο ταξινόμησης του
δικτύου. Αυτοί είναι: ένα \en RNN\gr, ένας Γραμμικός ταξινομητής, ένα \en SVM \gr  και ένα \en MLP\gr.
Χρησιμοποιήσαμε μια εφαρμογή απ' το  Fast RCNN για τον ταξινομητή  \en SVM\gr, η οποία περιελάμβανε την διαδικασία
εκπαίδευσης μέσω σκληρών αρνητικών. Εξετάσαμε μερικές τεχνικές εκπαίδευσης για
βέλτιστη απόδοση ταξινόμησης και 2 εκπαιδευτικές προσεγγίσεις για τον ταξινομητή \en MLP\gr, την κλασσική και μία που
χρησιμοποιούμε προεξαγόμενα χαρακτηριστικά.

\section{Μελλοντικές επεκτάσεις}

Υπάρχουν πολλά περιθώρια βελτίωσης για το δίκτυό μας, προκειμένου να επιτευχθεί
τελευταίας τεχνολογίας αποτελέσματα. Οι σημαντικότερες περιγράφονται στις επόμενες παραγράφους.

\paragraph{Βελτίωση των προτάσεων του \en TPN\gr}

Υλοποιήσαμε 2 δίκτυα για την πρόταση ακολουθιών από πλαίσια σε ένα τμήμα βίντεο. Πετύχαμε περίπου 63\% βαθμολογία \en recall \gr
για τη διάρκεια του δείγματος ίση με 16 καρέ και περίπου 80\% \en recall \gr για τη διάρκεια του δείγματος ίση με 8. Αυτά τα σκορ 
δείχνουν ότι υπάρχει αρκετός χώρος για βελτίωση ειδικά για την περίπτωση με δείγμα 16 καρέ.
Παρόλο που έχουν διερευνηθεί πολλές αρχιτεκτονικές δικτύων για
παλινδρόμηση, μια καλή ιδέα θα ήταν να δοκιμάσουμε άλλα δίκτυα, δεν είναι απαραίτητα εμπνευσμένη
από δίκτυα εντοπισμού αντικειμένων όπως κάναμε εμείς. 
Επιπλέον, προσθέτοντας έναν παράγοντα \textit{λ} στον τύπο του \en training loss \gr
 θα ήταν μια καλή ιδέα και θα διερευνούσε ποια είναι η καλύτερη προσέγγιση αυτού.
Έτσι, η απώλεια εκπαίδευσης θα μπορούσε να οριστεί ως:
\begin{equation} 
\begin{split}
 L  =  \sum_iL_{cls}(p_i, p_i^*) + \lambda_1 \sum_ip_i^*L_{reg}(t_i,t_i^*) + \lambda_2  \sum_iq_i^*L_{reg}(c_{i}, c_{i}^*) \\
\end{split}
\end{equation}
Επιπλέον, θα ήταν μια καλή ιδέα να χρησιμοποιήσουμε την μέθοδο του \en SSD  (\cite{DBLP:journals/corr/LiuAESR15}) \gr που προτείνει \en RoIs \gr 
αντί για το \en RPN\gr, για να συγκρίνετε το αποτέλεσμα.  Τέλος, θα μπορούσαμε να πειραματιστούμε χρησιμοποιώντας τα δίκτυα \en Feature Pyramid (\cite{8099589}), \gr
τα οποία θα μπορούσαν να επεκταθούν σε 3 διαστάσεις ως ένα άλλο δίκτυο εξαγωγής χαρακτηριστικών ή κάποιο άλλο είδος \en 3D ResNet\gr.

\paragraph{Αλλαγή του αλγορίθμου σύνδεσης}
Σε αυτή τη διατριβή, μια άλλη πρόκληση που αντιμετωπίσαμε ήταν η σύνδεση των προτεινόμενων \en ToIs \gr για την πρόταση \en action tubes\gr. Υλοποιήσαμε έναν πολύ αφελή αλγόριθμο,
που δεν ήταν σε θέση να μπορεί να μας δώσει πολύ καλές προτάσεις παρά τις αλλαγές που προσπαθήσαμε να κάνουμε. Υλοποιήσαμε έναν άλλο αλγόριθμο σύνδεσης που ήταν βασισμένος στην εκτίμηση της χρονικής
πρόοδο ενός \en action tube \gr και την αλληλεπίδραση του με άλλα. Αν και δεν μας έδωσε και πολύ καλές προτάσεις, πιστεύουμε ότι πρέπει να εξερευνήσουμε τις δυνατότητες αυτού του αλγορίθμου. Κι αυτό 
επειδή είναι  εκμεταλλεύεται την πρόοδο της ενέργειας, την οποία δεν είχε ο προηγούμενος αλγόριθμος.

\paragraph{Εξερεύνηση άλλων τεχνικών ταξινόμησης}

Για το στάδιο ταξινόμησης, πειραματιστήκαμε κυρίως πάνω σε έναν ταξινομητή \en SVM  \gr για το σύνολο δεδομένων \en JHMDB \gr και δεν ασχοληθήκαμε καθόλου  με το σύνολο δεδομένων \en UCF\gr. Ο πρώτος μας
στόχος είναι να είμαστε σε θέση να εξάγουμε καλά αποτελέσματα ταξινόμησης για το σύνολο δεδομένων \en UCF\gr.  Πιστεύουμε ότι θα πρέπει να διερευνήσουμε τους χάρτες χαρακτηριστικών του \en UCF \grκαι  τεχνικές  που εφαρμόζονται στους χάρτες χαρακτηριστικών πριν από την ταξινόμηση. Επιπλέον, θα μπορούσαμε να δοκιμάσουμε άλλες τεχνικές ταξινόμησης όπως \en Random Forests \gr ή να πειραματιστούμε περισσότερο με τον ταξινομητή \en RNN \gr για το σύνολο δεδομένων \en UCF\gr.
Τέλος, μια άλλη διαδικασία ταξινόμησης θα ήταν μια καλή ιδέα, όπως η εξαγωγή πρώτα όλων των πιθανών \en action tubes \gr και, στη συνέχεια, η χρήση άλλων δικτύων για εξαγωγή χαρακτηριστικών προκειμένου
να ταξινομήσουμε τα \en action tubes\gr.

\en
% \end{document}
\en
% \documentclass{report}

% \usepackage{subcaption} % package for subfigures
% \usepackage{hyperref}  % package for linking figures etc
% \usepackage{enumitem}  % package for description with bullets
% \usepackage{graphicx}  % package for importing images
% \usepackage{mathtools} % package for math equation
% \usepackage{mathrsfs}  % package for math font
% \usepackage{indentfirst} % package for getting ident after section or paragraph
% \usepackage[export]{adjustbox}
% % \usepackage{amsmath}

% \setlength{\parindent}{2em} % how much indent to use when we start a paragraph

% \graphicspath{ {./theory/figures/} }       % path for images

% \begin{document}

\chapter{Introduction}
Nowadays, the enormous increase of computing power help us deal with a lot of difficult situations appeared in our daily life.
A lot of areas of science have managed to tackle with problems, which were consided non trivial 20 years ago. One of
these area is Computer Vision and an important problem is human action recognition and localization.
\section{Problem statement}
The area of human action recognition and locatization has 2 main goals:
\begin{enumerate}
\item Automatically detect and classify any human activity, which appears in a video.
\item Automatically locate in the video, where the previous action is performed.
\end{enumerate}

\subsection{Human Action Recognition}
Considering human action recognition, a video may be consisted of only by 1 person doing something. However, this is a ideal
situation. In most cases, videos contain multiple people, who perform multiple actions or may not act at all in some segments.
So, our goal is not only to classify an action, but to dertemine the temporal boundaries of each action.
\subsection{Human Action Localization}
Alongside with Human Action Recognition, another problem is to present spatial boundaries of each action. Usually, this means
presenting a 2D bounding box for each video frame, which contains the actor. Of course, this bounding box moves alongside with
the actor.

\section{Applications}
The field of Human Action Recognition and Localization has a lot of applications which include 
 content based video analysis,automated video segmentation, security and surveillance systems,
human-computer interaction.

The huge availability of data (especially of videos) create the  necessity to find ways to take advantage of them.
About 2.5 billion images are uploaded at Facebook database every month, more than 34K hours of video in YouTube and
about 5K images every minute. On top of that, there are about 30 million surveillance cameras in US, which means
about 700K video hours per day. All those data need to be seperated in categories according to their content in
order to search them more easily. This process takes place by hand, by a user who attaches
keywords or tags to each video. However, most users avoid doing that, so many videos end up without any tagging information.
This situation creates the need to create algorithms for automated indexing based on the content of the video.

Another application is video summury. This area take place usually in movies or sports events. In movies,
video analysis algorithms can create a small video containing all the important moments of the movie. This
can be achieved by choosing video segments which an important action takes place such as killing the villain
of the movie. In sports events, video summury applications include creating highlight videos automatically, like
a video containing all achieved goals in football match.

On top of that, human action recognition can replace human operators in surveillance systems. Until now,
security systems include a system of multiple cameras handled by a human operator, who judges if a person
is acting normally or not. Automatic action classification systems can act like human, and immediately
judge if there is any human behavioral anomaly.

Last but not least, another field of application is related with human-computer interaction. Robotic applications
help elderly people deal with their daily needs. Also, gaming applications using Kinect create new kinds of
gaming experience without the need of a physical game controller.

\section{Challenges and Datasets}
There are various types of human activities. Depending on their complexity, weconceptually categorize human activities into four different
levels: gestures, actions, interactions, and group activities. Gestures are elementary movements of a person’s body part, and are the atomic
components describing the meaningful motion of a person. ``Stretching an arm'' and ``raising a leg'' are good examples of gestures.
Actions are single person activities that may be composed of multiple gestures organized temporally, such as ``walking'', ``waving'', and
``punching''. Interactions are human activities that involve two or more persons and/or objects. For example, ``two persons fighting'' is
an interaction between two humans and ``a person stealing a suitcase from another'' is a human-object interaction involving two humans and one
object. Finally, group activities are the activities performed by conceptual groups composed of multiple persons and/or objects. ``A group of persons marching'', ``a group having a meeting'', and ``two groups fighting'' are typical examples of them.
The wide variety of human activities and applications creates a lot of challenges which involve action recognition systems.
The most important challenges include large variations in appearence of the actors, camera view-point changes, occlusions,
non-rigid camera motions etc. On top of that, a big problem is that there are too many action classes which means
that manual collection of training sample is prohibitive. Also, some times, action vocabulary is not well defined.
As figure \ref{fig:open_example} shows, ``Open'' action can include a lot of kinds of actions, so we must carefully
choose which granularity of the action we will consider.

\begin{figure}[h]
  \centering
  \includegraphics[scale=0.3]{open_example}
  \caption{Examples of ``Open'' action}
  \label{fig:open_example}

\end{figure}

In order to deal with those challenges, several standard action datasets have been created in order to delevop
robust human action recognition systems and detection algorithms.
The first datasets included 1 actor performing using a static camera over homogeneous backgrounds.
Even though, those datasets helped us design the first action recognition algorithms, they were not able to deal with the above
challenges.
This lead us to design datasets containing more ambigious videos, such as Joint-annotated Human Motion Database(JHMDB) (\cite{Kuehne11})
and UCF-101 (\cite{soomro2012ucf101}). These datasets contain only human actions, the second category presented above.

\subsection{JHMDB Dataset}
The JHMDB dataset (\cite{Jhuang:ICCV:2013}) is a fully annotated dataset for human actions and human poses. It is consisted of 21 action categories and 928
clips extracted from Human Motion Database (HMDB51) \cite{Kuehne11}. This dataset contains trimmed videos with duration between
15 to 40 frames. Each clip is annotated for each frame using a 2D pose and contains only 1 action.
In order to train our model for action localization, we modify 2D poses into 2D boxes containing the whole pose in each frame.
There are available 3 different splits for training data, proposed by the authors. We chose the first split which contains 660
videos for training set and 268 for validation . 

\subsection{UCF-101 Dataset}
The UCF-101 dataset (\cite{soomro2012ucf101}) contains 13320 videos from 101 action categories.
From those, for 24 classes and 3194 video spatio-temporal annotations are included. This means that there is a 2D bounding box surrounding the actor for each frame in which an action is taking place.
We seperate them in 2284 videos for training set and 910 for validation test according to the
first proposed training split. For training data, there are videos up to 641 frames, while in validation data max number of frames is 900.
Each video, both training and validation, is untrimmed, including sometimes more than 1 actions taking place simultaneously.
We took annotations from
\cite{singh2016online} because the by the authors proposed annotations contain some mistakes.

\section{Motivation ans Contibutions}
The current achievements in Object Recognition Networks and in 3D Convolution Networks for Action Recognition have triggered us to try
to combine them in order to achieve state-of-the-art results for action localization. We introduce a new network structure inspired by
\cite{DBLP:journals/corr/HouCS17}, \cite{DBLP:journals/corr/abs-1712-09184},\cite{Ren:2015:FRT:2969239.2969250} and for implementation
by \cite{jjfaster2rcnn}.

Our contributions are the following:
\begin{enumerate}
\item We create a new framework for action localization extending the code taken from faster RCNN implementation. Based on the structure
  proposed by \cite{DBLP:journals/corr/HouCS17}, we modified it, using a 3D Resnet34 instead of C3D, which previous approach used.

\item Furthermore, we proposed our own TPN Network, a Network for proposing candidate action tubes give a small video segment.
  Following the approach \cite{DBLP:journals/corr/HouCS17} proposed, we firstly implement an architecture which uses
  cubboids as anchors, which then using a regressor it becomes a sequence of bounding boxes, likely to contain an action.
  We experiment with two candidate regressor's architecture and proposed and implement a 3D RoiAlign which uses trilinear
  interpolation for extracting each proposed action tube's activation maps. 
  Inspired by \cite{DBLP:journals/corr/abs-1712-09184}, we proposed and implement a TPN which uses predefined sequences of bounding
  boxes as 3D anchors. We proposed anchors that last equal with and less than video segment's duration in order our architecture to be able to
  perform temporal localization.  we explore two different regressors' architectures for better spatial precision using activation
  maps extracted from 2D RoiAlign, treating each frame seperatly.
\item Inspired by liniking algorithm proposed by \cite{DBLP:journals/corr/HouCS17}, we introduce our own linking algorithm, which
  uses a combination of actioness and overlap scores in order to decide if 2 proposed action tubes would connect or not and some updatable lists.
  Our approach includes gathering all candidate action tubes whose score is bigger than a threshold, and use them as active action tubes for
  new possible connections. When the number of gathered active action tube is bigger than a threshold, we keep the k-best scoring action tubes
  and remove the rest.  We implement this algorithm using, also, CUDA code in order to calculate connection score faster. We proposed 3 versions of this algorith:
  \begin{enumerate}
  \item An approach which uses an updatable scoring threshold, in order not to calculate unnecessary connection scores
  \item An approach which doesn't use an updatable scoring threshold, but it just updateds ``active'' action tube more frequently.
  \item An approach which, also, uses NMS or softmax-NMS algorithms for getting wider action tube proposals.
  \end{enumerate}
  Also, we implement, from scratch, another connection algorithm proposed by \cite{DBLP:journals/corr/abs-1903-00304} and extending it in order to work for ToIs instead of frames, which they proposed.
  We modified our TPN structure in order to calculate progression and progress rate scores in order to calculate connection scores and generate candidate action tubes.
\item We experiment using several classifier in order to find the most suitable. We considered 2 feature maps extracted using 3D RoiAlign and proposed action tubes, without any other
  modification. Also, we explore the different ratios and number of groundtruth foreground tubes that should be used during training
  stage. Finally, we tried to perform only temporal localization using temporal information generated from proposed action tubes.
\end{enumerate}

\section{Thesis structure}
The rest of Thesis is organized as follows. Chapter 2 provides an general introduction to Machine Learning techniques currently used.
After that, we present the basic elements of object recognition systems and alongside with loss functions and evaluation metrics that
we used. Also, Chapter 2 presents an brief overview of literature on human action recognition and localization. Chapter 3 introduces the first basic element of our network, Tube Proposal Network (TPN), a network which proposes Tubes of Interest (ToIs), which are sequences of bounding boxes, with are likely to contain a performed action. Furthermore, it contains all the proposed architectures for achieving this.
Chapter 4 proposes algorithms for linking the proposed TOIs from every video segment and proposal performance is presented.
In Chapter 5, we present all the classification approaches we used for designing our architecture and some classification results.
Chapter 6 is used for conclusions, summary of our contribution alongside with possible future work.

% \end{document}

% \documentclass{report}

% \usepackage{subcaption} % package for subfigures
% \usepackage{hyperref}  % package for linking figures etc
% \usepackage{enumitem}  % package for description with bullets
% \usepackage{graphicx}  % package for importing images
% \usepackage{mathtools} % package for math equation
% \usepackage{mathrsfs}  % package for math font
% \usepackage{indentfirst} % package for getting ident after section or paragraph
% \usepackage{amssymb}
% % \usepackage{amsmath}

% \setlength{\parindent}{2em} % how much indent to use when we start a paragraph

% \graphicspath{ {./theory/figures/} }       % path for images

% \begin{document}

\chapter{Background}


\section{Machine Learning}

\subsection{Introduction}
Machine Learning (ML) is a field which is raised out of Artificial Intelligence (AI). Applying AI, we wanted to build better and intelligent
machines. But except for few mere tasks such as finding the shortest path between point A and B, we were unable to program more complex
and constantly evolving challenges. There was a realisation that the only way to be able to achieve this task was to let machine learn
from itself. This sounds similar to a child learning from its self. So machine learning was developed as a new capability for computers.
And now machine learning is present in so many segments of technology, that we don’t even realise it while using it. \par
Finding patterns in data on planet earth is possible only for human brains. The data being very massive, the time taken to compute is
increased, and this is where Machine Learning comes into action, to help people with large data in minimum time. \\
There are three kinds of Machine Learning Algorithms :
 
\begin{enumerate}
\item Supervised Learning
\item Unsupervised Learning
\item Reinforcement Learning
\end{enumerate}

\subsubsection{Supervised Learning}
A majority of practical machine learning uses supervised learning. In supervised learning, the system tries to learn from the previous examples that are given. Speaking mathematically, supervised learning is where you have both input variables (\textit{x}) and output variables (\textit{Y}) and can use an algorithm to derive the mapping function from the input to the output. The mapping function is expressed as
$Y = f(x)$.
\begin{figure}[h]
  \centering
  \includegraphics[scale=0.3]{supervised_leaning_example2}
  \caption{Example of supervised Learning}
  \label{fig:supervised_learning_example}
\end{figure}

As shown in Figure \ref{fig:supervised_learning_example}, we have initially taken some data and marked them as ‘Spam’ or ‘Not Spam’. This
labeled data is used by the training supervised model, in order to train the model. Once it is trained, we can test our model by testing it with some test new mails and checking of the model is able to predict the right output. 

Supervised learning problems can be further divided into two parts, namely \textbf{classification}, and \textbf{regression}.
\begin{description}
\item[ Classification] : A classification problem is when the output variable is a category or a group, such as “black” or “white” or “spam” and “no spam”.
\item[ Regression ] :  A regression problem is when the output variable is a real value, such as “Rupees” or “height.”
\end{description}
Some Supervised learning algorithms include:
\begin{itemize}
\item Decision trees
\item Support-vector machine
\item Naive Bayes classifier
\item k-nearest neighbors
\item linear regression
\end{itemize}

\subsubsection{Unsupervised Learning}

In unsupervised learning, the algorithms are left to themselves to discover interesting structures in the data. Mathematically, unsupervised
learning is when you only have input data (\textit{X}) and no corresponding output variables. This is called unsupervised learning because
unlike supervised learning above, there are no given correct answers and the machine itself finds the answers.
\begin{figure}[h]
  \centering
  \includegraphics[scale=0.3]{unsupervised_learning_example}
  \caption{Example of unsupervised Learning}
  \label{fig:unsupervised_learning_example}
\end{figure}
In Figure \ref{fig:unsupervised_learning_example}, we have given some characters to our model which are ‘Ducks’ and ‘Not Ducks’. In our
training data, we don’t provide any label to the corresponding data. The unsupervised model is able to separate both the characters by
looking at the type of data and models the underlying structure or distribution in the data in order to learn more about it. Unsupervised
learning problems can be further divided into \textbf{association} and \textbf{clustering} problems.

\begin{description}
\item[ Association] : An association rule learning problem is where you want to discover rules that describe large portions of your data, such as “people that buy X also tend to buy Y”.
\item[ Clustering] : A clustering problem is where you want to discover the inherent groupings in the data, such as grouping customers by purchasing behaviour.
\end{description}

\subsubsection{Reinforcement Learning}
A computer program will interact with a dynamic environment in which it must perform a particular goal (such as playing a game with an
opponent or driving a car). The program is provided feedback in terms of rewards and punishments as it navigates its problem space. 
Using this algorithm, the machine is trained to make specific decisions. It works this way: the machine is exposed to an environment
where it continuously trains itself using trial and error method.
\begin{figure}[h]
  \centering
  \includegraphics[scale=0.3]{reinforment_learning_example}
  \caption{Example of Reinforcement Learning}
  \label{fig:reinforment_learning_example}
\end{figure}
In Figure \ref{fig:reinforment_learning_example}, we can see that the agent is given 2 options i.e. a path with water or a path with fire. A reinforcement algorithm
works on reward a system i.e. if the agent uses the fire path then the rewards are subtracted and agent tries to learn that it should avoid
the fire path. If it had chosen the water path or the safe path then some points would have been added to the reward points, the agent then
would try to learn what path is safe and what path isn’t

\subsection{Neural Networks}
Neural Networks are a class of models within the general machine learning literature. Neural networks are a specific set of algorithms that
have revolutionized the field of machine learning. They are inspired by biological neural networks and the current so called deep neural
networks have proven to work quite very well. Neural Networks are themselves general function approximations, that is why they can be applied
to literally almost any machine learning problem where the problem is about learning a complex mapping from the input to the output space.

\subsection{A single Neuron}
The basic unit of computation in a neural network is the neuron, often called a \textbf{node} or \textbf{unit}. It receives input from some
other nodes, or from an external source and computes an output. In purely mathematical terms, a neuron in the machine learning world is a
placeholder for a mathematical function, and its only job is to provide an output by applying the function on the inputs provided.
Each input has an associated weight (\textit{w}), which is assigned on the basis of its relative importance to other inputs. The node applies
a function \textit{f  (defined below)} to the weighted sum of its inputs as shown in Figure \ref{fig:Perceptron}.
\begin{figure}[h]
  \centering
  \includegraphics[scale=1.0]{perceptron}
  \caption{An example of a single Neuron}
  \label{fig:Perceptron}
\end{figure}
The network takes numerical inputs \textit{X1} and \textit{X2} and has weights \textit{w1} and \textit{w2} associated with those inputs.
Additionally, there is another \textit{input 1} with weight \textit{b} (called \textit{Bias}) associated with it. The main function of Bias is to provide every node with a trainable constant value (in addition to the normal inputs that the node receives). The output Y from the neuron
is computed as shown in the Figure \ref{fig:Perceptron}. The function \textit{f} is non-linear and is called \textbf{Activation Function}. The
purpose of the activation function is to introduce non-linearity into the output of a neuron. This is important because most real world data
are non linear and we want neurons to learn these non-linear representations.
\subsubsection{Activation Functions} 
Every activation function (or non-linearity) takes a single number and performs a certain fixed mathematical operation on it. There are
several activation functions:
\begin{description}
\item[ Sigmoid ] : takes a real-valued input and squashes it to range between 0 and 1. Its formula is:
  \[ \sigma(x) = \frac{1}{1 + e^{-x} } \]
  It is easy to understand and apply but it has major reasons which have made it fall out of popularity:
  \begin{itemize}
  \item Vanishing gradient problem
  \item Its output isn’t zero centered. It makes the gradient updates go too far in different directions.
  \item Sigmoids saturate and kill gradients.
  \item Sigmoids have slow convergence.
  \end{itemize}

\item [ Tanh ] : takes a real-valued input and squashes it to the range [-1, 1]. Its formula is:
  \[ tanh(x) = 2 \sigma(2x) -1 \]
  Now it’s output is zero centered because its range in between -1 to 1. Hence optimization is easier in this method and  in practice it is always preferred over Sigmoid function . But still it suffers from Vanishing gradient problem.

\item[ ReLU ]: ReLU stands for \textit{Rectified Linear Unit}. It takes a real-valued input and thresholds it at zero (replaces negative values with zero). So its formula is:
  \[ f(x) = max(0,x) \]
  It has become very popular in the past couple of years. It was recently proved that it had 6 times improvement in convergence from Tanh
  function. Seeing the mathamatical form of this function we can see that it is very simple and efficinent . A lot of times in Machine
  learning and computer science we notice that most simple and consistent techniques and methods are only preferred and are best.
  Hence it avoids and rectifies vanishing gradient problem . Almost all deep learning Models use ReLu nowadays.
\end{description}

Figure \ref{fig:Activation}  show each of the above activation functions.
\begin{figure}[h]
  \centering
  \includegraphics[scale=1.0]{activation_figures}
  \caption{Plots of Activation functions}
  \label{fig:Activation}
\end{figure}

\subsubsection{Feedforward Neural Network}
Till now we have covered neuron and activation functions which together for the basic building blocks of any neural network. The feedforward
neural network was the first and simplest type of artificial neural network devised. It contains multiple neurons (nodes) arranged in layers.
A layer is nothing but a collection of neurons which take in an input and provide an output. Inputs to each of these neurons are processed
through the activation functions assigned to the neurons. Nodes from adjacent layers have connections or edges between them. All these connections have weights associated with them.
\begin{figure}[h]
  \centering
  \includegraphics[scale=1.0]{feedforwardnetwork}
  \caption{An example of a Feedforward Neural Network}
  \label{fig:feedforwardnetwork}
\end{figure}
An example of a feedforward neural network is shown in Figure \ref{fig:feedforwardnetwork}. A feedforward neural network can consist of three types of nodes:
\begin{description}
\item[ Input Nodes ] The Input nodes provide information from the outside world to the network and are together referred to as the
  “Input Layer”. No computation is performed in any of the Input nodes – they just pass on the information to the hidden nodes.
\item[  Hidden Nodes ]  The Hidden nodes have no direct connection with the outside world (hence the name “hidden”). They perform
  computations and transfer information from the input nodes to the output nodes. A collection of hidden nodes forms a “Hidden Layer”.
  While a feedforward network will only have a single input layer and a single output layer, it can have zero or multiple Hidden Layers.
\item [ Output Nodes ] The Output nodes are collectively referred to as the “Output Layer” and are responsible for computations and
  transferring information from the network to the outside world.
\end{description}

In a feedforward network, the information moves in only one direction – forward – from the input nodes, through the hidden nodes (if any)
and to the output nodes. There are no cycles or loops in the network (this property of feed forward networks is different from Recurrent
Neural Networks in which the connections between the nodes form a cycle). Another important point to note here is that each of the hidden
layers can have a different activation function, for instance, hidden layer1 may use a sigmoid function and hidden layer2 may use a ReLU,
followed by a Tanh in hidden layer3 all in the same neural network. Choice of the activation function to be used again depends on the
problem in question and the type of data being used.

\subsection {2D Convolutional Neural Network}
A Convolutional Neural Network (ConvNet/CNN) is one of the variants of neural networks used heavily in the field of Computer Vision. It
derives its name from the type of hidden layers it consists of. The hidden layers of a CNN typically consist of convolutional layers, pooling
layers, fully connected layers, and normalization layers. Here it simply means that instead of using the normal activation functions defined
above, convolution and pooling functions are used as activation functions.
It can take in an input image, assing importance (learning weights and biases) to various aspects/objects in the image and be able to
differentiate one from the other. The pre-processing required in a ConvNet is much lower as compared to the other classification algorithms.
While in primitive method filters are hand-engineered, with enough training, ConvNets have the ability to learn these filters/characteristics.

The architecture of a ConvNet is analogous to that of the connectivity pattern of Neurons in the Human Brain and was inspired by the
structure of the Visual Cortex. However, most ConvNets costist mainly in 2 parts:
\begin{description}[font=$\bullet$\scshape\bfseries]
\item [ Feature extractor] : \\
  This part of the network takes as input the image and extract features that are meaningful for its classification. It amplifies aspects
  of the input that are important for discrimination and suppresses irrelevant variations. Usually, the feature extractor cosists of
  several layers. For instance, an image which could be seen as an array of pixel values. The first layer often learns reprensations
  that represent the presence or absence of edges at particular orientations and locations in the image. The second layer typically
  detects motifs by spotting particular arrangements of edges, regardeless of small variations in the edge positions. Finally, the third
  may assemble motifs into larger combinations that correspond to paths of familiar objects, and subsequent layers would detect objects
  as combinations of these parts.
  
\item [ Classifier ] : \\
  This part of the network takes as input the previously computed features and use them to predict the correct label.
\end{description}

\begin{figure}[h]
  % \includegraphics[scale=0.7]{convolutional_neural_network_structure} \]
  \centering
  \includegraphics[scale=0.7]{convolutional_neural_network_structure}
  \caption{Typical structure of a ConvNet}
\end{figure}
\paragraph{Convolutional Layers}
In order to extract such features, ConvNets use 2D convolution operations. These operations take place in convolutional layers. Convolutional
layers consist of a set of learnable filters. Every filter is small spatially (along widht and height), but extends through the full depth of
input. During forword pass, we slide (more precisely, convolve) each filter across the width and height of the input volume and compute dot
products between the entries of the filter and the input at any position (as Figure \ref{fig:conv_example} shows). The objective of the
Convolution Operation is to extract the high-level features such as edges, from the input image. ConvNets need not be limited to only one
Convolutional Layer. Conventionally, the first ConvLayer is responsible for capturing the Low-Level features such as edges, color, gradient
orientation, etc. With added layers, the architecture adapts to the High-Level features as well, giving us a network which has the wholesome
understanding of images in the dataset, similar to how we would.

\begin{figure}[h]
  \centering
  \begin{minipage}[b]{0.4\textwidth}
    \includegraphics[width=\textwidth]{conv_1}
  \end{minipage}
  \hfill
  \begin{minipage}[b]{0.4\textwidth}
    \includegraphics[width=\textwidth]{conv_2}
  \end{minipage}
  \caption{Convolution with kernel of 3, stride of 2 and padding of 1}
  \label{fig:conv_example}
\end{figure}
\paragraph{Pooling Layers}
Pooling Layers are also refered as downsampling layers and are used to reduce the spatial dimensions, but not depth, on a convolution neural network.
The intuitive reasoning behind this layer is that once we know that a specific feature is in the original input volume (there will be a high activation value),
its exact location is not as important as its relative location to the other features. The main advantages of pooling layer are:

\begin{itemize}
\item We gain computation performance since the amount of parameters is reduce.
\item Less parameters also means We deal with overfitting situations.
\end{itemize}

The pooling operation is specified, rather than learned. Two common functions used in the pooling operation are:
\begin{description}
\item[ Average Pooling ] Calculate the average value for each patch on the feature map.
\item[ Maximum Pooling (or Max Pooling) ] Calculate the maximum value for each patch of the feature map.
\end{description}

\begin{figure}[h]
  \centering
  \includegraphics[scale=0.35]{max_pooling}
  \caption{Example of Max pooling operation with a 2x2 filter  and stride of 2}
  \label{fig:pooling_eg}
\end{figure}

\subsection{3D Convolutional Neural Network}
Traditionally, ConvNets are targeting RGB images (3 channels). The goal of 3D CNN is to take as input a video and extract features from it.
When ConvNets extract the graphical characteristics of a single image and put them in a vector (a low-level representation), 3D ConvNets
extract the graphical characteristics of a set of images. 3D CNNs takes in to account a temporal dimension (the order of the images in the
video). From a set of images, 3D CNNs find a low-level representation of a set of images, and this representation is useful to find the
right label of the video (a given action is performed). \\
In order to extract such features, 3D ConvNets  use 3D convolution operations.
\begin{figure}[h]
  % \includegraphics[scale=0.7]{convolutional_neural_network_structure} \]
  \centering
  \includegraphics[scale=0.3]{3d_conv}
  \caption{3D Convolution operation}
\end{figure}

There are several existing approaches to tackle the video classification. This is a nonexaustive list of existing approaches:
\begin{description} [font=$\bullet$\scshape\bfseries]
\item[ ConvNets + LSTM cell] : Extract features from each frame with a ConvNet, passing the sequence to an RNN
\item[ Temporal Relation Networks] : Extract features from each frame with a ConvNet and pass the sequence to an MLP
\item[ Two-Stream Convolutional Networks] : Use 2 CNN, 1 spatial stream ConvNet which process one single frame at a time, and 1 Temporal stream ConvNet which process multi-frame optical flow
\end{description}

\section{Object Detection}
Within the field of Deep Learning, the sub-discipline called “Object Detection” involves processes such as identifying the objects through a picture, video or a webcam feed.
Object Detection is used almost everywhere these days. The use cases are endless such as Tracking objects, Video surveillance, Pedestrian detection etc. 
An object detection model is trained to detect the presence and location of multiple classes of objects. For example, a model might be trained with images that
contain various pieces of fruit, along with a label that specifies the class of fruit they represent (e.g. an apple, a banana, or a strawberry),
and data specifying where each object appears in the image.

The main process followed by most of CNN for Object Detection is:
\begin{enumerate}
\item Fistly, we do feature extraction using as backbone network, the first Convolutional Layers of a known pre-trained CNN such
  as AlexNet, VGG, ResNet etc.
\item Then, we propose regions of interest (ROI) in the image. These regions contain possibly an object, which we are looking for.
\item Finally, we classify each proposed ROI.
\end{enumerate}

\subsection{ Region Proposal Network}

From the 3 above steps, the 2nd step is considered to be very important. That is because, in this step, we should choose regions of
the image, which will be classified. Poor choice of ROIs means that the CNN will pass by some object that are located in the image,
because, they were not be proposed to be classified.

\begin{figure}[h]
  \centering
  \includegraphics[scale=0.3]{RPN_structure}
  \caption{ Region Proposal Network's structure}
  \label{fig:rpn_structure}
\end{figure}

The first Object-Detection CNNs use several algorithms for proposing ROIs. For example, R-CNN\cite{DBLP:journals/corr/GirshickDDM13},
and Fast R-CNN\cite{Girshick:2015:FR:2919332.2920125} used Selective Search Algorithm for extracting ROIs.
One of noverlties introduced by the Faster R-CNN\cite{Ren:2015:FRT:2969239.2969250} is \textbf{Region Proposal Network} (RPN). Its
Function is to propose ROIs and its structure can be shown in \ref{fig:rpn_structure}. As we can see, RPN is consisted of:
\begin{itemize}
\item 1 2D Convolutional Layer
\item 1 score layer 
\item 1 regression layer
\end{itemize}

Another basic element of RPN is the \textbf{anchors}. Anchors are predefined boxes used for extracting ROIs. In figure \ref{fig:anchors} is
depicted an exaple of some anchors
\begin{figure}[h]
  \centering
  \includegraphics[scale=0.3]{anchors}
  \caption{ Anchors for pixel (320,320) of an image (600,800) }
  \label{fig:anchors}
\end{figure}

For each feature map's pixel corresponds \textbf{k (k=9)} anchors (3 different scales and 3 different ratios 1:1, 1:2, 2:1). \par
As a result, RPN gets as input feature maps extracted from the backbone CNN. Then performs 2D convolution over this input and passes
the output to its scoring layer and regression layer. Those scores represent the confidence of existing an object in this specific position.
On top of that, regression layer outputs 4k displacements, 4 for each anchor. Finally, we keep as output only the \textit{ n-best scoring} anchors.

\subsection{Roi Align}

The biggest problem facing Object Detection Networks is the need for fixed input size. Classification networks require a fixed input size, which
is easy for image classification because it is handle by resizing the input image. However, in object recognition, each proposal has a different size
and shape. This creates the need for converting all proposals to a fixed shape. At Fast-RCNN(\cite{Girshick:2015:FR:2919332.2920125}) and
Faster-RCNN(\cite{Ren:2015:FRT:2969239.2969250}), this operation happens by applying Roi Pooling. However, this wrapping is digitalized because
the cell boundaries of the target feature map are forced to realign with the boundary of the input feature maps as shown in Figure \ref{fig:roi_pooling_align_1}
at the top left diagram. As a result, each target cells may not be in the same size (Figure \ref{fig:roi_pooling_align_2}).

\begin {figure}[h]
  \centering
  \begin{subfigure}{0.35\textwidth}
    \includegraphics[scale=0.3]{roi_align_1}
    \caption{}
    \label{fig:roi_pooling_align_1}
  \end{subfigure}
  \hfill
    \begin{subfigure}{0.35\textwidth}
    \includegraphics[scale=0.3]{roi_al_3}
    \caption{}
    \label{fig:roi_pooling_align_3}
  \end{subfigure}
  \hfill
    \begin{subfigure}{0.35\textwidth}
    \includegraphics[scale=0.3]{roi_al_2}
    \caption{}
    \label{fig:roi_pooling_align_2}
  \end{subfigure}
  \hfill
    \begin{subfigure}{0.35\textwidth}
    \includegraphics[scale=0.3]{roi_al_4}
    \caption{}
    \label{fig:roi_pooling_align_4}
  \end{subfigure}
  \caption{}
  \label{fig:roi_pooling_align}
\end{figure}

On the other hand, Mask-RCNN (\cite{DBLP:journals/corr/HeGDG17}) introduced Roi Align operation. Roi Align avoids digitalizing
the boundary of the cells as shown in Figure \ref{fig:roi_pooling_align_3}, and achieves to make every target cell to have the same size
according to Figure \ref{fig:roi_pooling_align_4}. In order to calculate feature maps values, Roi Align uses bilinear interpolation as
shown in Figure \ref{fig:roi_align_2}. This means that we calculate the value of the desired bins according to their neighbours'.

\begin{figure}[h]
  \centering
  \includegraphics[scale=0.35]{roi_align_2}
  \caption{}
  \label{fig:roi_align_2}
\end{figure}

% In \ref{fig:roi_align_2}, the $ 5 \times 5$ feature map from the last convolution layer is shown with the dotted lines and our proposed
% Roi is shown with black lines. This rectangular proposal is overlaid with the feature map at a certain position. According to \ref{fig:roi_align_2},
% our desired output has dimensions $2 \times 2$. So, the rectangular proposal is devided into a $2 x 2$ grid. From each grid cell we sample
% four regions as shown with the blue dots, 

\subsection{Non-maximum suppression (NMS) algorithm}

That's because the neighbour windows have similar scores to some extent. This 
Another proble that object detection networks face is that neighbourd bounding boxes have similar scores to some extent. Most object
detection systems employ a sliding window or a Region Proposal Network for proposing areas in the image that is likely to contain an
object. These techniques,which return several areas in the images, achieve high recall performance. However, in these approaches,
more that 1 proposal may be related with only one groundtruth object coordinates. This situation creates
the need for choosing the best proposals, because, alternatively, hundreds of unnecesssary proposals will be classified. For that
reason, Non-Maximum Suppression (NMS) algorithm was proposed for filtering these proposals base on some criteria. NMS gets as input
a list of proposal bounding boxes B, their corresponding confidence score S and an overlap threshold N and return as output
a list of the final filtered proposals D. NMS algorith's steps are:
\begin{enumerate}
  
\item Initialize an empty list D. Select the proposal with the highest confidence score, remove it from B and add it to D.
\item Calculate the overlap score between this proposal and all the other proposals. For all the proposals that their overlap
  score is bigger than N, remove from B.
\item From the remaining proposals, picked again the one with the highest score and remove it from B.
\item Repeat steps 2 and 3 until no more proposals are left in list B.
\end{enumerate}

The aforementioned algorithm shows that the whole process depends mostly on a sigle threshold value.  So that makes the selection
of threshold a crucial factor for the perfomance of the model. In some situations, bad choice of the threshold may make the network
to remove bounding boxes with good confidence score, if there are side by side. Figure \ref{nms_eg} shows a situation like this,
where red and blue boxes will be removed because of the presence of the black box.
\begin{figure}[h]
  \centering
  \includegraphics[scale=0.2]{soft_nms.jpeg}
  \caption{Example of a situation where NMS algorithm will remove good proposals}
  \label{fig:nms_eg}
\end{figure}

\subsubsection{Soft NMS}
A simple and efficinet way to deal the aforementioned situation is to use Soft-NMS algorithm, which is presented in \cite{DBLP:journals/corr/BodlaSCD17}.
Soft-NMS algorithm is based on the idea of
reducing condifence score of the proposals propotional to their overlap score, instead of completely removing them.
The score calculation follows the formula 
\[ s_i = \begin{dcases}
    s_i, & overlapscore(M, b_i) < N_t \\
    s_i(1-overlapscore(M, b_i)), & overlapscore(M, b_i) \ge N_t
    \end{dcases}
    \]

    where $s_i$ is the score of proposal $i$, $b_i$ is the box coordinates of proposal $i$, $M$ is the coordinates of
    the box with the maximum confidence and $N_t$ is the overlap threshold. So steps of soft-NMS algorithm are:

\begin{enumerate}
  
\item Select the proposal with the highest confidence score, remove it from B and add it to D.
\item Calculate the overlap score between this proposal and all the other proposals. For all the proposals that their overlap
  score is bigger than N, recalculate their confidence score according to previous formula.
\item From the remaining proposals, picked again the one with the highest score and remove it from B.
\item Repeat steps 2 and 3 until no more proposals are left in list B.
\end{enumerate}
    
    
\section{Losses and Metrics}
In order to train our model and check its performance, we use some known Loss functions and Metrics used in Object Detection systems.
\subsection{Losses}
For training our network, we use \textbf{Cross Entropy Loss} for classification layers and \textbf{smooth L1-loss} for bounding box regression
in each frame and their diagram is show at Figure \ref{fig:cross_l1}.
\subsubsection{Cross Entropy Loss}
Cross-entropy loss, or log loss, measures the performance of a classification model whose output is a probability value between 0 and 1. \par
Entropy is the measure of uncertainty assosiated with a given distribution \textit{q(y)} and its formula is: 
\[ H = -\sum_{i=1}^n p_i \cdot \log p_i \]
Intuitively, entropy tells us how ``surprised'' we are when some event E happened. When we are sure about an event E to happened ($ p_E = 1$)
we have 0 entropy (we are not surprised) and vise versa. \par
On top of that, let's assume that we have 2 distributions, one known (our network's distribution) \textit{p(y)} and one unknown (the actual
data's distribution) \textit{q(y)}. Cross-entropy tells us how accurate is our known distribution in predicting the unknown distribution's
results. Respectively, Cross-entropy measures how accurate is our model in predicting the test data. Its formula is:
\[ H_p(q) = - \sum_{c=1}^C q(y_c) \cdot log(p(y_c)) \]

\subsubsection{Smooth L1-loss}

Smooth L1-loss can be interpreted as a combination of L1-loss and L2-loss. It behaves as L1-loss when the absolute
value of the argument is high, and it behaves like L2-loss when the absolute value of the argument is close to zero.
It is usually used for doing box regression on some object detection systems like Fast-RCNN(\cite{Girshick:2015:FR:2919332.2920125}),
Faster-RCNN(\cite{Ren:2015:FRT:2969239.2969250}) and it is less sensitive to outliers according according to \cite{Girshick:2015:FR:2919332.2920125}.
As showm in \cite{Girshick:2015:FR:2919332.2920125}, its formula is:

\[ smooth_{L1}(x) = \begin{dcases}
    0.5x^2 & \text{ if } x < 1 \\
    |x| - 0.5 & \text{otherwise}
  \end{dcases}\]

It is similar to Huber loss whose formula is:

\[
L_{\delta}(x) = \begin{dcases}
    \frac{1}{2}a^2 & \text{ for } |a| \le \delta \\
\delta(|a| - \frac{1}{2}\delta), & \text{otherwise}
\end{dcases}
\]

if we set $\delta $ parameter equal 1. \par

Smooth L1-loss combines the advantages of L1-loss (steady gradients for large values of \textit{x}) and L2-loss (less oscillations during
updates when \textit{x} is small). Figure \ref{fig:cross_l1} \subref{fig:smooth_l1} shows a comparison between L1-norm, L2-norm  and smooth-L1 .

\begin{figure}[h]
  \centering
  \begin{subfigure}{0.49\textwidth}
    % \fbox{\includegraphics[width=\textwidth]{c_entropy}}
    % \includegraphics[width=\textwidth]{c_entropy}
    \includegraphics [width=7.0cm,height=4.5cm]{c_entropy}
      \caption{}
      \label{fig:cross_entropy}

  \end{subfigure}
  \hfill
  \begin{subfigure}{0.49\textwidth}
% \fbox{    \includegraphics [width=5.5cm,height=3.8cm]{smooth_l1}}
    \includegraphics[width=\textwidth]{smooth_l1}
    % \includegraphics [width=6.0cm,height=4cm]{smooth_l1}
    % \includegraphics[width=\textwidth]{smooth_l1}
      \caption{}
      \label{fig:smooth_l1}
  \end{subfigure}

  \caption{ (\subref{fig:cross_entropy}) and (\subref{fig:smooth_l1})  show the behavior of cross-entropy loss and smooth-L1 respectively.}
  \label{fig:cross_l1}%
\end{figure}

\subsection{Metrics}
Evaluating our machine learning algorithm is an essential part of any project. The way we choose our metrics influences how the performance
of machine learning algorithms is measured and compared.
They influence how to weight the importance of different characteristics in the results and finally,
the ultimate choice of which algorithm to choose. Most of the times we use classification accuracy
to measure the performance of our model, however it is not enough to truly judge our model. \par
At first, we introduce some basic evaluation metrics in order,then, to present those we use for our assesment.

\subsubsection{Intersection over Union}
The first and most important metric that we use is Intersection over Union (IoU). IoU  measures the overlap between 2 boundaries.
It is usually used in Object Recognition Networks in order to define how good overlap a predicted bounding box  with the actuall
bounding box as shown in Figure \ref{fig:iou_fig}. We predefine an IoU threshold (say 0.5) in classifying whether the prediction is a true positive or a false positive.
\begin{figure}[h]
  \centering
  \includegraphics[scale=0.55]{iou}
  \caption{Example of IoU scoring policy}
  \label{fig:iou_fig}
\end{figure}

Intersection over Union is defined as:
\[ IoU = \frac{\text{area of overlap}}{\text{area of union}} \]
In Figure \ref{fig:iou_fig}, spatial IoU between 2 bounding boxes, $(x_1,y_1,x_2,y_2)$ and $(x_3,y_3,x_4,y_4)$,is presented, which means IoU metric is implemented for x-y dimensions.
Area of overlap and area of union can be defined as:
\[\text{ Area of overlap } = \lVert( min(x_2,x_4) - max(x_1,x_3), min(y_2,y_4)- max(y_1,y_3) )\rVert\]
\[\text{ Area of union  }  = \lVert( max(x_2,x_4) - min(x_1,x_3), max(y_2,y_4) - min(y_1,y_3) )\rVert\]
On top of that, we can implement IoU for 1 dimension and for 3 dimensions.
\paragraph{1D IoU} We can name 1D IoU as temporal overlap. Let's consider 2 temporal segments $(t_1,t_2)$ and $(t_3,t_4)$, between which we want to
estimate their overlap score. Their IoU can be described as:
\[ \text { Length of overlap } = \lVert(min(t_2,t_4) - max(t_1,t_3))\rVert  \]
\[ \text { Length of union } = \lVert(max(t_2,t_4) - min(t_1,t_3))\rVert  \]
\paragraph{3D IoU} 3-dimentional Intersection over Union which can, also, be named as spatiotemporal IoU, can be defined by 2 ways:
% categories: whether the $3^{rd}$ dimension is continuous or discrete.
\begin{description}
\item[3D boxes are cuboids] In this case, 3D boxes can be written  as $(x,y,z,x',y',z')$.
So, the IoU overlap between 2 boxes, $(x_1,y_1,z_1,x_2,y_2,z_2)$
  and $(x_3,y_3,z_3,x_4,y_4,z_4)$, is defined as:
\begin{equation*} 
\begin{split}
  \text { Volume of overlap } = \lVert(min(x_2,x_4) - max(x_1,x_3), \\
  min(y_2,y_4) -  max(y_1,y_3), min(z_2,z_4) -  max(z_1,z_3) ) \rVert  \\
\text { Volume of union } = \lVert(max(x_2,x_4) - min(x_1,x_3), \\
 max(y_2,y_4) - min(y_1,y_3), max(z_2,z_4) - min(z_1,z_3)
\end{split}
\end{equation*}

\item[x-y are continuous and z discrete] In this case 3D boxes is defined as a sequence of 2D boxes $(x,y,x',y')$. For this definition, z-dimension is
  discrete, and IoU can be defined with 2 ways, which both result in the same overlapping score. Let's cosider 2 sequences of boxes, with temporal limits
  $(t_1,t_2)$ and $(t_3,t_4)$. We calculate their IoU following one of the following methods:
  \begin{enumerate}
\item IoU is the product between temporal-IoU and the average spatial-IoU between 2D boxes in the overlapping temporal area and it is described as:
\[ IoU =  IoU((t_1,t_2),(t_3,t_4)) \cdot \frac{1}{K_2-K_1} \sum_{i=K_1}^{K_2} IoU(X_1^i, X_2^i) \]
  where \begin{itemize}
  \item$K_1 = min(t_2,t_4)$
  \item$K_2 = max(t_1,t_3)$
  \item$ X_1^i =  (x_1^i,y_1^i,x_2^i,y_2^i)$ and $X_2^i =  (x_3^i,y_3^i,x_4^i,y_4^i) $
  \end{itemize}
\item IoU is the average spatial-IoU if we consider 2D boxes as $(0,0,0,0)$ if $ t \notin [t_{start},t_{finish}]$ and it is written as:
  \[ IoU = \frac{1}{K} \sum_{i = min(t_1,t_3}^{max(t_2,t_4)} IoU(X_1^i,X_2^i) \]
  \begin{itemize}
  \item $K = max(t_2,t_4) - min(t_1,t_3)$
  \item $ X_1 = (x_1,y_1,x_2,y_2)  \text{ if } i \in [t_1,t_2] \text{ or }
        (0,0,0,0)  \text{ if } i \notin [t_1,t_2]  $
  \item $ X_2 = (x_3,y_3,x_4,y_4)  \text{ if } i \in [t_3,t_4] \text{ or }
        (0,0,0,0)  \text{ if } i \notin [t_3,t_4]  $
    \end{itemize}
  
\end{enumerate}
\end{description}
From above implementations, we are involve mostly with temporal and spatiotemporal IoU.

\subsubsection{Precision  \& Recall }

In order to describe \textbf{precision} and \textbf{recall} metrics, we will use an example. Let's consider a group of people in which, some of them are sick and
the others are not. We use a network which is able to predict if a person is sick or not if we give it some data as input.
\begin{description}
\item[ Precision ]  measures how accurate are our model's predictions. i.e. the percentage of  predictions that are correct.
  In our case, how accurate is our model when it predicts that a person is sick.
\item[ Recall ] measures how good we found all the sick people. In our case, how many of the acctual sick people we managed to find.
\end{description}
Their definisions are:
\[ Precision = \frac{TP}{TP + FP} \]  
\[  Recall = \frac{TP}{TP + FN} \] 
where \begin{itemize}
\item TP = True positive, which means that we predict a person to be sick and he is actually sick.
\item TN = True negative, we predict that a person isn't sick and he isn't.
\item FP = False positive, we predict a person to be sick but he isn't actually.
\item FN = False negative, we predict a person not to be sick but he actually is.
\end{itemize}

From these 2 metrics we use recall metric in order to evaluate our networks performance, and more specifically, its performance on finding good action tube proposals.
We consider a groundtruth action as true positive when there is at least 1 proposed action tube that its IoU overlap score is bigger that
a predifed threshold. If there is no such action tube, then we consider this groundtruth action tube as false negative.


\subsubsection{mAP }

Precision and recall are single-value metrics based on the whole list of predictions. By looking their formulas, we can see that there is
a trade-off between precision and recall performance. This trade-off can be adjusted by the softmax threshold, used in model's final layer.
In order to have high precision performance, we need to decrease the number of FP. But this will lead to decrease recall performance and
vice-versa. \par
As a result, these metrics fail to determine if a model is performing well in object detection tasks as well as action detection tasks. For
that reason, we use mean Average Precision (mAP) metric.
\paragraph{AP (Average precision)} Before defining mAP metric, we will define Average Precision metric (AP). AP is a popular metric in measuring the accuracy of
object detectors like Faster R-CNN, SSD, etc. Average precision computes the average precision value for recall value over 0 to 1.
As mention before, during classification stage, our prediction result in True positive(TP), False positive(FP), True Negative(TN) and False Negative(FN). For object recognition and action
localization networks, we don't care about TN. We consider a prediction as True positive when our prediction (an bounding box for object detection network or a sequence of bounding boxes in
action localization networks) overlaps with the groundtruth bounding box/action tube over a predefined threshold, and our predicted class is the same as groundtruth's. In addition, we
consider a False Negative when either no detection overlaps with the groundtruth bounding box/action tube or our prediction was incorrect. We consider a prediction as False positive, when more that
one predictions overlap with the groundtruth. In this situation, we consider the prediction with the biggest confidence score as TP and the rest as FP.\par

For a class, we need to calculate, first, its precision and recall scores in order to calculate its AP score. We sort our predictions according to their confidence score and for each new prediction we
calculate precision and recall values. An example, for a class containing 4 TP and 8 predictions  is shown in Table \ref{table:map_1}. Precision and recall
are calculated according to the number of elements that are above in the order. So, for rank \#3, Precision is calculated as the proposion of TP = 2/3 = 0.67
and Recall as the proportion of TP out of all the possible TP = 2/4 = 0.5.

\begin{table}[h]
  \centering
  \begin{tabular}{|c | c | c | c |}
    \hline
    \textbf{Rank} & \text{Prediction} & \textbf{Precision} & \textbf{Recall} \\
    \hline
    1 & Correct & 1.0  & 0.25 \\
    \hline
    2 & Correct & 1.0  & 0.5 \\
    \hline
    3 & False   & 0.67 & 0.5 \\
    \hline
    4 & False   & 0.5  & 0.5 \\
    \hline
    5 & False   & 0.4  & 0.5 \\
    \hline
    6 & Correct & 0.5  & 0.75 \\
    \hline
    7 & False   & 0.42 & 0.75 \\
    \hline
    8 & Correct & 0.5  & 1   \\
    \hline
  \end{tabular}
  \caption{Ordered by confidence predictions and their precision and recall values}
  \label{table:map_1}
\end{table}

\begin{figure}[h]
  \centering
  \includegraphics[scale=0.4]{pr_curve}
  \caption{Precision/Recall curve}
  \label{fig:pr_curve}
\end{figure}

We plot Precision against Recall and their curve is shown in Figure \ref{fig:pr_curve}.  The general definition for Average Precision(AP) is finding
the area under the precision-recall curve and its formula is:
\[ AP = \int_{0}^{1} p(r)dr \]
Precision and Recall values $ \in [0,1] $, so AP $\in [0,1]$, too. This integral can be replace with a finite, as we have a finite number of predictions.
So its formula is:
\[ AP = \sum_{k=1}^{n} P(k)\Delta r(k) \]
where $P(k)$ is the precision until prediction $k$ and $\Delta r$ is the change in recall from $k-1$ to $k$.

\paragraph{Interpolated Precision}

At we can see in \ref{fig:pr_curve}, P-R curve has a zigzag pattern as it goes down with false predictions, and goes up with correct.
So, before calculation AP, we need to smooth out this zigzag pattern using Intenpolated precision, as introduced in \cite{Everingham:2010:PVO:1747084.1747104}.
Interpolated precision is calculated at each recall level $r$ by taking the maximum precision measured for that $r$ and it is defined as:
\[ p_{interp}(r) = \max_{\tilde{r}:\tilde{r}\ge r} p(\tilde{r}) \]

where $p(\tilde{r})$ is the measured precision at recall $\tilde{r}$. Graphically, at each recall level, we replace each precision value
with the maximum precision value to the right of that recall level.  At Figure \ref{fig:pr_curve2} are shown both P-R curvers. The previous P-R curve has blue
colour and the interpolated has red color.

\begin{figure}[h]
  \centering
  \includegraphics[scale=0.4]{pr_curve2}
  \caption{Both P-R curves. }
  \label{fig:pr_curve2}
\end{figure}

In order to calculate AP, we sample the curve at all unique recalls values, whenever the maximum precision drops. On top of that, we define mean Average Precision (mAP)
the mean of the AP for each class. So, AP and mAP are defined as
\[  AP = \sum (r{n+1} - r_n) p_{interp} (r_{n+1}) \]
\[ p_{interp}(r_{n+1}) = \max_{\tilde{r} \ge r_{n+1} p(\tilde{r})}p(\tilde{r}) \]
\[ mAP = \frac{1}{N} \sum_{i=1}^N AP_i \]


% In our case, we use the metrics presented in \cite{DBLP:journals/corr/GkioxariM14} in
% order to quantify our results:
% \begin{description}
% % \item[ frame-AP ] measures the area under the precision-recallcurve of the detections for each frame (similar to the PASCAL  VOC  detection  challenge  \cite{Everingham10}).   A  detection is correct
% %   if the intersection-over-union with theground truth at that frame is greater than and the ac-tion label is correctly predicted.
% \item [ video-AP ] measures the area under the precision-recallcurve of the action tubes predictions. A tube is correctif the mean per frame intersection-over-union with theground truth across the
%   frames of the video is greaterthan and the action label is correctly predicted.
% % \item [ AUC ] measures the area under the ROC curve, a metricpreviously used on this task.  An action tube is correctunder the same conditions as invideo-AP. Following
% %   \cite{Tian:2013:SDP:2514950.2515975} , the ROC curve is plotted until a false positive rateof 0.6, while keeping the top-3 detections per class andper video. Consequently,
% %   the best possible AUC scoreis 60\%.
% \end{description}

\subsubsection{Mean Average Best Overlap - MABO}

In order to evaluate more the quality of our proposals, both during TPN and connecting tube stages, recall metric isn't enough.
That's because recall metric tells us only in how many known objects there was at least 1 proposal that satified the detection
criterion. However, it doen't tells us how close the proposals are to the groundtruth tubes. In order to quantify this performance,
Mean Average Best Overlap (MABO) was introduced by \cite{DBLP:journals/corr/WinschelLE16}. The importance of MABO can be clarified 
we consider figure \ref{fig:mabo_fig}.

\begin{figure}[h]
  \centering
  \includegraphics[scale=0.25]{mabo}
  \caption{Recall versus MABO exaple}
  \label{fig:mabo_fig}
\end{figure}

As we can see, recall performance is almost perfect, but, MABO perfonmance, which tells us where most proposal overlap scores are gather,
is just fine. In order to define MABO, we need first to define Average Best Overlap. Let $ c \in C$ denote a class $c$ from the set of all
classes $C$ and $G^c$ the set of ground truth annotations of this class in all images; let $L$ be the set of all generated object proposals
for all images. Average Best Overlap is defined as the average value of the maximum overlap score, (in our situation, we use intersection
over union) of $L$ with each groundtruth annotation $g \in G^c$. The Mean Average Best Overlap (MABO) is defined as the average value of all
class ABO values :
\[ MABO = \frac{1}{|C|} \sum_{c \in C} [ \frac {1}{|G^C|} \sum_{g \in G^c} \max_{l \in L} IoU(g,l)]  \]
In order our situation, we consider only one class, for defining proposals, so, MABO identifies with ABO.

% \end{document}
\documentclass{report}

\usepackage{subcaption} % package for subfigures
\usepackage{hyperref}  % package for linking figures etc
\usepackage{enumitem}  % package for description with bullets
\usepackage{graphicx}  % package for importing images
\usepackage{mathtools} % package for math equation
\usepackage{mathrsfs}  % package for math font
\usepackage{indentfirst} % package for getting ident after section or paragraph
\usepackage[export]{adjustbox}
\usepackage{multirow}  % package for tables, multir
\usepackage{amssymb}
% \usepackage{tabu}   % for tables 
\setlength{\parindent}{2em} % how much indent to use when we start a paragraph

\graphicspath{ {./theory/figures/} }       % path for images

\begin{document}

\chapter{Tube Proposal Network}

One of the basic elements of ActionNet is \textbf{Tube Proposal Network}(TPN). The main purpose of this network is to propose
\textbf{Tube of Interest}(TOIs). These tubes are likely to contain an known action and are consisted of some 2D boxes
(1 for each frame). TPN is inspired from RPN introduced by FasterRCNN (\cite{Ren:2015:FRT:2969239.2969250}), but instead of images, TPN
is used in videos as show in \cite{DBLP:journals/corr/HouCS17}. In full correspondence with RPN, the structure
of TPN is similar to RPN. The only difference, is that TPN uses 3D Convolutional Layers and 3D anchors instead of 2D. \par
We designed 2 main structures for TPN. Each approach has a different definition of the used 3D anchors.
The rest structure of the TPN is mainly the same with some little differences in the regression layer. \par
Before describing TPN, we present the preprocess procedure which is the same for both approaches.

\section{Preparation for TPN}

\subsection{Preparing data}

Our architecture gets as input a sequnece of frames which has a fixed size in widht, height and duration. However, each video has different resolution. That's creates the
need to resize each frame before. As mentioned in previous chapter, the first element of our network is a 3D RenNet taken from \cite{hara3dcnns}. This network is designed to
get images with dimensions (112,112). As a result, we resize each frame from datasets' videos into (112,112) frames. In order to keep aspect ratio, we pad each frame either
left and right, either above and bellow depending which dimension is bigger. In figure  \ref{fig:Preprocess_example} we can see the original frame and the resize and padded one.
In full correspondance, we resize the groundtruth bounding boxes for each frame (figure \ref{fig:original_image_rois} and \ref{fig:trans_image_rois} show that).

\begin{figure}[h]
  \centering
  \begin{subfigure}{0.35\textwidth}
    \includegraphics[width=\textwidth]{./figures/original_image.jpg}
    \caption{}
    \label{fig:original_image}
  \end{subfigure}
  \hfill
  \begin{subfigure}{0.35\textwidth}
    \includegraphics[width=\textwidth]{./figures/original_image_rois.jpg}
    \caption{}
    \label{fig:original_image_rois}
  \end{subfigure}
  \hfill
  \begin{subfigure}{0.35\textwidth}
    \includegraphics[width=\textwidth]{./figures/transformed_image.jpg}
    \caption{}
    \label{fig:trans_image}
  \end{subfigure}
  \hfill
  \begin{subfigure}{0.35\textwidth}
    \includegraphics[width=\textwidth]{./figures/transformed_image_rois.jpg}
    \caption{}
    \label{fig:trans_image_rois}
  \end{subfigure}

  \caption{ At (a), (b) frame is its original size and at (c), (d) same frame after preprocessing part}
  \label{fig:Preprocess_example}
\end{figure}


\subsection{3D ResNet}
Before using Tube Proposal Network, we spatio-temporal features from the video. In order to do so, we extract the 3 first Layers of a
pretrained 3D ResNet. It is pretrained in Kinetics dataset \cite{DBLP:journals/corr/KayCSZHVVGBNSZ17} for sample duration = 16 and
sample size = (112,122). \par
This network normally is used for classifying the whole video, so some of its layers use temporal stride = 2.
We set their temporal stride equal to 1 because we don't want to miss any temporal information during the process.
So, the output of the third layer is a feature maps with dimesions (256,16,7,7). We feed this feature map to TPN, which is described
in following sections.

\section{ 3D anchors as 6-dim vector}
\subsection{First Description}
We started desinging our TPN inspired by \cite{DBLP:journals/corr/HouCS17}. We consider each anchor as a 3D bounding box written as
$(x_1, y_1, t_1, x_2, y_2, t_2)$ where $x_1, y_1, t_1$
are the upper front left coordinates of the 3D and $x_2, y_2, t_2$ are the lower back left as shown in figure \ref{fig:anchor_6d}.
\begin{figure}[h]
  \centering
  \includegraphics[scale=0.5]{anchor_6d}
  \caption{An example of the anchor $(x_1,y_1,t_1,x_2,y_2,t_2)$}
  \label{fig:anchor_6d}
\end{figure}

The main advantage of this approach is that except from x-y dims, dimension of time is mutable. As a result, the proposed TOIs have
no fixed time duration. This will help us deal with untrimmed videos, because proposed TOIs would exclude background frames.
For this approach, we use \textbf{n=4k=60} anchors for each pixel in the feature map of TPN. We have k anchors for each sample
duration( 5 scales of 1, 2, 4, 8, 16, 3 aspect ratios of 1:1, 1:2, 2:1 and 4 durations of 16,12,8,4 frames).
In \cite{DBLP:journals/corr/HouCS17},  network's anchors are defined according to the dataset most common anchors. This, however,
creates the need to redesign the network for each dataset. In our approach, we use the same anchors for both datasets, because we want our network not
to be dataset-specific but to be able to generalize for several datasets. As sample duration, we chose 16 frames per video segment because
our pre-trained ResNet is trained for video clips with that duration.
So the structure of TPN is:
\begin{itemize}
\item 1 3D Convolutional Layer with kernel size = 3, stride = 3 and padding = 1
\item 1 classification layer outputs \textit{2n scores} whether there is an action or not for \textit{n tubes}.
\item 1 regression layer outputs \textit{6n coordinates} ($x_1,y_1,t_1,x_2,y_2,t_2$) for \textit{n tubes}.
\end{itemize}

which is shown in figure \ref{fig:tpn_1_1}
\begin{figure}[h]

  \includegraphics[width=1.\textwidth]{tpn_1_1}
  \caption{Structure of TPN}
  \label{fig:tpn_1_1}
\end{figure}

The output of TPN is the k-best scoring cuboid, in which it is likely to contain an action.
% \subsection{Preparing Data for training}


\subsection{Training}
As mentioned before, TPN extracts TOIs as 6-dim vectors. For that reason, we modify out groundtruth ROIs to groundtruth Tubes.
We take for granted that the actor cannot move a lot during 16 frames, so that's why we use this kind of tubes. As shown 
in figure \ref{fig:gt_tubes_and_rois}, these tubes are 3D boxes which include all the groundtruth rois, which are different
for each frame.

\begin{figure}[h]
  \centering
  \begin{subfigure}{0.15\textwidth}
    \includegraphics[width=\textwidth]{output/img_0.jpg}
  \end{subfigure}
  \begin{subfigure}{0.15\textwidth}
    \includegraphics[width=\textwidth]{output/img_3.jpg}
  \end{subfigure}
  \begin{subfigure}{0.15\textwidth}
    \includegraphics[width=\textwidth]{output/img_5.jpg}
  \end{subfigure}
  \begin{subfigure}{0.15\textwidth}
    \includegraphics[width=\textwidth]{output/img_7.jpg}
  \end{subfigure}
  \begin{subfigure}{0.15\textwidth}
    \includegraphics[width=\textwidth]{output/img_11.jpg}
  \end{subfigure}
  \begin{subfigure}{0.15\textwidth}
    \includegraphics[width=\textwidth]{output/img_15.jpg}
  \end{subfigure}
  \caption{Groundtruth tube is coloured with blue and groundtruth rois with colour green}
  \label{fig:gt_tubes_and_rois}
\end{figure}

For training procedure, for each video, we randomly select a part of it which has duration 16 frames. For each video, we train TPN in
order to score all the anchors using IoU criterion (as explained in chapter 2 when 3D boxes are cubboids) and we use Cross Entropy Loss as a loss function.
For regression, we use smooth-L1 loss. For regression targets, we use pytorch FasterRCNN implementation (\cite{jjfaster2rcnn}) for bounding box regression.
We modified the code in order to extend it for 3 dimensions. \textbf{TODO more details}
So we have:
\[ \begin{matrix}
    t_x = (x-x_a)/w_a, & t_y = (y-y_a)/h_a, & t_z= (z-z_a)/d_a, \\
    t_w= log(w/w_a), & t_h= log(h/h_a), & t_d = log(d/d_a), \\
    t^*_x = (x^* - x_a)/w_a, & t^*_y = (y^* - y_a)/h_a, & t^*_z = (z^* - z_a)/d_a, \\
    t^*_w = log(w^* /w_a), & t^*_h = log(h^*/h_a), & t^*_d = log(d^*/d_a),
    % t∗x= (x∗−xa)/wa,  t∗y= (y∗−ya)/ha,t∗w= log(w∗/wa),  t∗h= log(h∗/ha)
  \end{matrix}
\]
where \textit{x, y, z, w, h, d} denote the 3D box's center coordinates and its widht, height and duration. Variables $x, x_a, \text{ and } x^*$
are for the predicted box, anchor box, and groundthruth box respectively (likewise for \textit{y, z, w, h, d}). \par

\textbf{TODO Training Loss formula}

\subsection{Validation}

Validation procedure is a bit similar to training procedure.
We randomly select 16 frames from a validation video and we examine if there is at least 1 proposed TOI
which overlaps $\ge$ 0.5 with each groundtruth action tube and we get recall score. 
In order to get good proposals, after TPN we use Non-Maximum Suppresion (NMS) algorith. We set NMS threshold equal with 0.7, so we reject
thems with overlapping score > 0.7 .

\subsection{Modified Intersection over Union(mIoU) TODO check again} 
During training, we get numerous anchors. We have to classify them as foreground anchors or
background anchors. Foreground anchors are those which contain some action, and, respectively, background
don't. As presented before, IoU for cuboids calculates the ratio between volume of overlap and volume of
union.
Intuitively, this criterion is good for evaluating 2 tubes if they overlap but it has one big drawback:
it considers x-y dimesions to have same importance with time dimension, which we do not desire. That's becase
firstly we care to be accurate in time dimension, and then we can fix x-y domain.
As a result, we change the way we calculate the Intesection Over Union. We calculate seperately
the IoU in x-y domain (IoU-xy) and in t-domain (IoU-t). Finally, we multiply them in order to get the final IoU.
So the formula for 2 tubes ($x_1, y_1, t_1, x_2, y_2, t_2$) and ($x'_1, y'_1, t'_1, x'_2, y'_2, t'_2$) is:
\[ IoU_{xy} = \frac{ \text{Area of Overlap in x-y}} { \text{Area of Union in x-y}}  \]
\[ IoU_t = \frac { max(t_1, t'_1) - min(t_2, t'_2)} {min(t_1,t'_1) - max(t_2,t'_2)} \]
\[ IoU = IoU_{xy} \cdot  IoU_t \]
The above criterion help us balance the impact of time domain in IoU. For example, let us consider 2 anchors:
a = (22, 41, 1, 34, 70, 5) and b = (20, 45, 2, 32, 72, 5). These 2 anchors in x-y domain have IoU score equal to 0.61.
But they are not exactly overlaped in time dim. Using the first approach we get 0.5057 IoU score and using the
second approach we get 0.4889. So, the second criterion would reject this anchor, because there is a difference in time
duration.  \par

In order to verify our idea, we train TPN using both IoU and mIoU criterion for tube-overlapping. At Table \ref{table:iou_miou}
we can see the performance in each case for both datasets, JHMDB and UCF. Recall threshold for this case is 0.5 and during validation,
we use regular IoU for defining if 2 tubes overlap.
\begin{table}[h]
\centering
  \begin{tabular}{|| c | c | c ||}
    \hline
    \textbf{Dataset} & \textbf{Criterion} & \textbf{Recall(0.5)} \\
    \hline  \hline
    \multirow{2}{4em}{JHMDB} & IoU & TODO \\
    \cline{2-3}
    {} & mIoU & TODO \\
    \hline
    \multirow{2}{4em}{UCF} & IoU & TODO \\
    \cline{2-3}
    {} & mIoU & TODO \\
    \hline      
  \end{tabular}
  \caption{Recall results for both datasets using IoU and mIoU metrics}
  \label{table:iou_miou}
\end{table}

Table \ref{table:iou_miou} shows that modified-IoU give us slightly better recall performance, so that's the overlapping scoring policy
for now on, during Training mode.

\subsection{Improving TPN score}
After first test, we came with the idea that in a video lasting 16 frames, in time domain, all kind of actions can be seperated in the following categories:
\begin{enumerate}
\item Action starts in the n-th  frame and finishes after the 16th frame of the sampled video.
\item Action has already begun before the 1st frame of the video and ends in the n-th frame.
\item Action has already begun before the 1st frame of the video and finishes after the 16th video frame.
\item Action starts and ends in that 16 frames of the video.
\end{enumerate}

On top of that, we noticed that most of actions, in our datasets, last more that 16 frame. So, we added 1 scoring layer and 1 reggression layer as shown in 
figure \ref{fig:tpn_1_2}. These two layers have anchors with fixed time duration. Their purpose is to be trained only in x-y domain, keeping time duration
steady. 
\begin{figure}[h]
  \centering
  \includegraphics[scale=0.5]{tpn_1_2}
  \caption{TPN structure after adding 2 new layers, where k = 5n.}
  \label{fig:tpn_1_2}
\end{figure}

Training and Validation procedures remain the same. The only big difference is that now we have from 2 difference system proposed TOIs. So, we first concate
them and, then, we follow the same procedure. For training loss, we have 2 different cross-entropy losses and 2 different smooth-L1 losses, each for every
layer correspondly. \\

\textbf{TODO Training Loss formula}

\paragraph {2 approaches}

\textbf{Kernel vs mean}

\begin{table}[h]
  \centering
  \begin{tabular}{||c | c | c | c ||}
    \hline
    \textbf{Dataset} & \textbf{Fix-time anchors} & \textbf{Type} & \textbf{Recall(0.5)} \\
    \hline  \hline
    \multirow{3}{4em}{JHMDB} & No &  - & TODO \\
    \cline{2-4}
    {} & \multirow{2}{*}{Yes} & Kernel & TODO \\
    \cline{3-4}
    {} & {} & Mean & TODO \\
    \hline
    \multirow{3}{4em}{UCF} & No & - & TODO \\
    \cline{2-4}
    {} & \multirow{2}{*}{Yes} & Kernel & TODO \\
    \cline{3-4}
    {} & {} & Mean & TODO \\
    \hline      
  \end{tabular}
  \caption{Recall results after adding fixed time duration anchors}
  \label{table:add_16}
\end{table}

AS we can see from the previous results, the new layers increased recall performance significantly.

\subsection{Adding regressor}
The output of TPN is $\alpha$-highest scoring anchors moved according to their regression prediction. After that, we have to translate the anchor into tubes.
In order to do so, we add a regressor system which gets as input TOIs' feature maps and returns a sequence of 2D boxes, each for every frame.
The only problem is that the regressor needs a fixed input size of featuremaps. This problem is already solven by R-CNNs which use roi pooling and roi align
in order to get fixed size feature maps from ROIs with changing sizes. In our situation, we extend roi align operation, presented by Mask R-CNN, and we
call it \textbf{3D Roi Align}.

\paragraph{3D Roi Align}
3D Roi align is a modification of roi align presented by Mask R-CNN (\cite{DBLP:journals/corr/HeGDG17}). The main difference between those two is that Mask R-CNN's roi align uses
bilinear interpolation for extracting ROI's features and ours 3D roi align uses trilinear interpolation for the same reason. Again, the 3rd dimension is
time.
So, we have as input a feature map extracted from ResNet34 with dimensions (64,16,28,28) and a tensor containing the proposed TOIs.
For each TOI whose actiovation map whose size is (64,16,7,7), we get as output a feature map with size (64, 16, 7, 7). \par
\subsubsection{Regression procedure}
At first, for each proposed ToI, we get its corresponding activation maps using 3D Roi Align. These features are given as input to a regressor. This regressor returns 16 $\cdot$ 4 predicted
transforms $(\delta_x,\delta_y, \delta_w,\delta_h)$, 4 for each frame, where $ \delta_x, \delta_y$ specify the coordinates of proposal's center and $\delta_w, \delta_h$ its width and height, as specified
in \cite{DBLP:journals/corr/GirshickDDM13}.  We keep only the predicted translations, for the frames that are $\ge t_1$ and $< t_2$ and for the other frames, we set a zero-ed 2D box. 
After that, we modify each anchor from a cuboid written like $(x_1,y_1,t_1, x_2, y_2, t_2)$ to a sequence of 2D boxes, like: \\
$(0,0,0,0, ..., x_{T_1},y_{T_1},x'_{T_1},y'_{T_1}, ... ,x_{i},y_{i},x'_{i}, ..., x_{T_2},y_{T_2},x'_{T_2},y'_{T_2}, 0,0,0,0, ....)$, where
\begin{itemize}
\item $ T_1 \le i \le T_2$, for $T_1 < t_1 + 1,  T_2 < t_2 \text{ and }T_1,T_2 \in \mathbb{Z} $
\item $ x_i = x_1, y_i= y_1, x'_i = x_2, y'_i = y_2 $.
\end{itemize}

\paragraph{ Training}
mIou, 16mean
After getting proposed TOIs from TPN, we pick, randomly, 16 tubes which will be input in the regressor.
Finally, we find the traslation for each rois and, again, we use smooth-L1 loss for loss function.\par

\textbf{TODO Training Loss formula}

\subsubsection{First regression Network} \textbf{Na pw gia to normalize kai to pooling kernel}\par
The architecture of reggression network is show in Figure \ref{fig:regressor_3d}, and it is described below:
\begin{figure}[h]
  \centering
  \includegraphics[scale=0.48]{regressor_1_1}
  \caption{Structure of Regressor}
  \label{fig:regressor_3d}
\end{figure}

\begin{enumerate}
\item Regressor is consisted, at first, with a 3D convolutional layer with kernel = 1, stride = 1 and no padding. This layer gets as input ToI's activation map extracted by 3D Roi Align.
\item After that, we calculate the average value in time domain, so from a feature map with dimensions (64,16,7,7), we get as output a feature map (64,7,7).
\item These feature maps are given as input to a Linear layer followerd by a Relu Layer, a Dropout Layer, another Linear Layer and Relu Layer and a final Linear.
\end{enumerate}

\textbf{NA DW TA FEATURE MAPS POIA XRISIMOPOIISA}
\begin{table}[h]
  \centering
  \begin{tabular} {||c | c | c ||}
    \hline
    \textbf{Dataset} & \textbf{Pooling} & Recall \\
    \hline                
    \multirow{2}{*}{JHMDB} & avg & TODO \\
    \cline{2-3}
    {} & mean & TODO \\
    \cline{1-3}
    \multirow{2}{*}{UCF} & avg & TODO \\
    \cline{2-3}
    {} & mean & TODO \\
    \cline{1-3}
                                   
  \end{tabular}
  \caption{}
  \label{table:reg_1_1}
\end{table}

\textbf{TODO sxoliasmos apotelesmatwn}
As the above results show, when we translate a TOI into a sequence of ROIs, recall reduces about 20-30\%, which is a big problem. 
 % As a result, we should find a solution, in order to
\subsection{Changing Regressor - from 3D to 2d}
After getting first recall results, we experiment using another architecture for the regressor network. Instead of having a 3D Convolutional
Layer, we will use a 2D Convolutional Layer in order to treat the whole time dimension as one during convolution operation. So, as shown in Figure \ref{fig:reg_1_2},
the $2^{nd}$ Regression Network is about the same with first one, with 2 big differences:
\begin{enumerate}
\item We performing a pooling operation at the feature maps extracted by 3D Roi Align operation
\item Instead of a 3D Convolutional Layer, we have a 2D Convolutional Layer with kernel size = 1, stride = 1 and no padding.
\end{enumerate}

\begin{figure}[h]

  \centering
  \includegraphics[scale=0.48]{regressor_1_2}
  \caption{Structure of Regressor}
  \label{fig:regressor_2d}
\end{figure}

On top of that, we tried to determine which feature map is the most suitable  for getting best-scorig recall performance. This feature map will be given as
input to Roi Algin operation.  At Table \ref{table:reg_1_2}, we can see the recall performance for different feature maps and different pooling methods.

\begin{table}[h]
  \centering
  \begin{tabular}{||c | c | c || c||}
    \textbf{Dataset} & \textbf{Pooling} & \textbf{Feature Map} & \textbf{Recall (0.5)} \\
    \hline
    \multirow{6}{*}{JHMDB} & \multirow{3}{*}{mean} & 64 & TODO \\
    \cline{3-4}
    {} & {} & 128 & TODO \\
    \cline{3-4}
    {} & {} & 256 & TODO \\
    \cline{2-4}
    {} & \multirow{3}{*}{max} & 64 & TODO \\
    \cline{3-4}
    {} & {} & 128 & TODO \\
    \cline{3-4}
    {} & {} & 256 & TODO \\
    \hline
    \multirow{6}{*}{UCF} & \multirow{3}{*}{mean} & 64 & TODO \\
    \cline{3-4}
    {} & {} & 128 & TODO \\
    \cline{3-4}
    {} & {} & 256 & TODO \\
    \cline{2-4}
    {} & \multirow{3}{*}{max} & 64 & TODO \\
    \cline{3-4}
    {} & {} & 128 & TODO \\
    \cline{3-4}
    {} & {} & 256 & TODO \\
    \hline


  \end{tabular}
  \caption{}
  \label{table:reg_1_2}
\end{table}

\textbf{TODO add more comment}
As we noticed from the above results, our system has difficulty in translating 3D boxes into 2D sequence of ROIs. So, that makes us rethink the way we designed
our TPN.

\section{ 3D anchors as 4k-dim vector}
In this approach, we set 3D anchors as 4k coordinates (k = 16 frames = sample duration). So a typical anchor is written as ($x_1, y_1, x'_1, y'_1, x_2, y_2, ...$)
where $x_1, y_1, x'_1, y'_1 $ are the coordinates for the 1st frame, $x_2, y_2, x'_2, y'_2$ are the coordinates for the 2nd frame etc as presented in \cite{DBLP:journals/corr/abs-1712-09184}.
In figure \ref{fig:anchor_4k} we can an example of this type of anchor.

\begin{figure}[h]
  \centering
  \includegraphics[scale=0.5]{anchor_4k}
  \caption{An example of the anchor $(x_1,y_1,x'_1,y'_1,x_2,y_2, ...)$}
  \label{fig:anchor_4k}
\end{figure}

The main advantage of this approach is that we don't need to translate the 3D anchors into 2D boxes. However, it has a big drawback, which is the fact that this type of anchors
has fixed time duration. In order to deal with this problem, we set anchors with different time durations, which are 16, 12, 8 and 4. Anchors with duration $ < $
sample duration (16 frames) can be written as 4k vector with zeroed coordinateds in the frames bigger that the time duration. For example, an anchor with
2 frames duration, starting from the 2nd frame and ending at the 3rd can be written as (0, 0, 0, 0, $x_1, y_1, x'_1, y'_1, x_2, y_2, x'_2, y'_2$, 0, 0, 0, 0) if sample
duration is 4 frames. 

\begin{figure}[h]
  \centering
  \includegraphics[width=1.\textwidth]{tpn_2}
  % \includegraphics[scale=0.35]{tpn_2}
  \caption{The structure of TPN according to new approach}
  \label{fig:New_structure}
\end{figure}

This new approach led us to change the structure of TPN. The new one can is presented in figure \ref{fig:New_structure}. As we can see, we \
added scoring and regression layers for each duration.

\subsection{Training}
In following figures we can see recall performance for sample duration = 16 when using max or avg pooling. 
% \begin{center}
%   % \begin{tabu} to 0.8\textwidth { | X[l] | X[c] | X[r] | }
%   \begin{tabular} { | c | c | c | c |}
%     \hline
%     \textbf{Duration} & \textbf{Type of Pooling} & \textbf{Dataset}  &  \textbf{Threshold } & \textbf{recall}\\
%     \hline
%     16  & Avg  & jHMDB  &  0.5 & \\
%     \hline
%     % \end{tabu}
%   \end{tabular}
% \end{center}
From the above results, we can see that using max pooling achieves better results.

\begin{table}[h]
  \centering
  \begin{tabular}{||c | c | c | c ||}
    \hline
    \textbf{Dataset} & \textbf{Pooling} &  \textbf{Recall(0.5)} \\
    \hline  \hline
    \multirow{2}{4em}{JHMDB} & mean &  TODO \\
    \cline{2-3}
    {} & max & TODO \\
    \hline
    \multirow{2}{4em}{UCF} & mean &  TODO \\
    \cline{2-3}
    {} & max & TODO \\
    \hline
  \end{tabular}
  \caption{Recall results after adding fixed time duration anchors}
  \label{table:tpn_2_1}
\end{table}

\subsection{Adding regressor}
In full correspondance with the previous approach, we added an regressor for trying to find better results. We 

\begin{table}[h]
  \centering
  \begin{tabular}{||c | c | c || c||}
    \textbf{Dataset} & \textbf{Pooling} & \textbf{Feature Map} & \textbf{Recall (0.5)} \\
    \hline
    \multirow{6}{*}{JHMDB} & \multirow{3}{*}{mean} & 64 & TODO \\
    \cline{3-4}
    {} & {} & 128 & TODO \\
    \cline{3-4}
    {} & {} & 256 & TODO \\
    \cline{2-4}
    {} & \multirow{3}{*}{max} & 64 & TODO \\
    \cline{3-4}
    {} & {} & 128 & TODO \\
    \cline{3-4}
    {} & {} & 256 & TODO \\
    \hline
    \multirow{6}{*}{UCF} & \multirow{3}{*}{mean} & 64 & TODO \\
    \cline{3-4}
    {} & {} & 128 & TODO \\
    \cline{3-4}
    {} & {} & 256 & TODO \\
    \cline{2-4}
    {} & \multirow{3}{*}{max} & 64 & TODO \\
    \cline{3-4}
    {} & {} & 128 & TODO \\
    \cline{3-4}
    {} & {} & 256 & TODO \\
    \hline


  \end{tabular}
  \caption{}
  \label{table:reg_2_1}
\end{table}

% \subsection{Changing Regressor - from 3D to 2d}
% After getting first recall results, we experiment using another architecture for the regressor network. Our idea was that we can treat feature maps like not having
% temporal dependencies between their frames. So, at first, we get from Roi Align operation activation maps with dimensions \textit{(k,64,16,7,7)} where k is the number
% of ToIs proposed for this video segment. We reshape this feature map to \textit{ (k*16,64,7,7) }, and in same time, we reshape their proposed action tubes from \textit{(k,16,4)}
% to \textit{(k*16,4)}. So, the new regression Network is consisted with:
% % \begin{enumerate}
% % \item a 2D Convolutional Network
% \paragraph{From 3D to 2d}
% The first idea we thought, was to change the first Convolutional layer from 3D to 2D. This means that we consider  features  not to have temporal dependencies for
% each frame. As we can see in the figure \ref{}, we got worse results, so, we rejected this idea.

% \subsection{Trying to  improve performance}
% TODO
% \subsection{Changing training procedure}
% TODO
\subsection{Changing sample duration}
After trying all the previous version, we noticed that we get about the same recall performances. So, we thought that we could try
to reduce the sample duration. On top of that, we trained our network for sample duration = 8 and 4 frames.

\begin{table}
  \centering
  \begin{tabular}{|c | c| c|}
    \hline
    \textbf{Dataset} & \textbf{Sample dur} & \textbf{Recall} \\
    \hline
    \multirow{2}{*}{JHMDB} & 8 & TODO \\
    \cline{2-3}
    {} & 4 & TODO \\
    \hline
    \multirow{2}{*}{UCF} & 8 & TODO \\
    \cline{2-3}
    {} & 4 & TODO \\
    \hline
    
  \end{tabular}
  \caption{}
  \label{table:reg_2_sample}
\end{table}
                                             
                                             
    

    


\end{document}


% \documentclass{report}

% \usepackage{subcaption} % package for subfigures
% \usepackage{hyperref}  % package for linking figures etc
% \usepackage{enumitem}  % package for description with bullets
% \usepackage{graphicx}  % package for importing images
% \usepackage{mathtools} % package for math equation
% \usepackage{mathrsfs}  % package for math font
% \usepackage{indentfirst} % package for getting ident after section or paragraph
% \usepackage{multirow}  % package for tables, multirow
% \usepackage{longtable} % package for multi pages tables


% \usepackage[export]{adjustbox}
% % \usepackage{amsmath}

% \setlength{\parindent}{2em} % how much indent to use when we start a paragraph

% \graphicspath{ {./theory/figures/} }       % path for images

% \begin{document}

\chapter{Connecting Tubes}
\section{Introduction}
In the previous chapter, we described methods for generating candidate action tubes given a small video segment lasting 8 or 16 frames. However, actual videos
and actual human actions, in the  wild, last more than 16 frames most of the times. Current networks are unable to process a whole video at once, in order to generate action tubes
due to memory and computing power issues.  As mentioned in chapter 2, a lot action localization approaches deal with this situation given a video, either by 
proposing candidate areas in the frame-level and then  connecting them in order to generate action tube proposal either, separating the video into video segments,
proposing  sequences of bounding boxes for each video segment and then linking them in order to generate action proposals. Both aforementioned techniques make
the suitable choice of linking method an important factor for the performance of the network. That's because, even though frame-level or video segment-level proposals
might be very good, if the proposed connection algorithm doesn't work well, final action tube proposals won't be efficient, so the final model will never
be able to achieve high classification performance. In other words, if connecting algorithm doesn't generate action proposals with great recall and MABO performance,
the model's classifier won't be able to perform suitable classification, because probably it would be given action tubes without any context.
In this chapter, we present 3 different approaches used for linking proposed ToIs generated from TPN in the previous chapter.

\section{First approach: combine overlap and actioness}
Our algorithm is inspired by \cite{DBLP:journals/corr/HouCS17}, which calculates all possible sequences of ToIs. In order to find the best candidate action tubes,
it uses a score, which tells us how likely a sequence of ToIs is  to contain an action. This score is a combination of 2 metrics:
\begin{description}
\item[ Actioness,  ] which is the ToI's possibility to contain an action. This score is produced by TPN's scoring layers.
\item [ ToIs' overlapping, ] which is the IoU of the last bounding boxes of the first ToI and the first frames of the second ToI.
\end{description}

The above scoring policy can be described by the following formula:
\[ S = \frac{1}{m} \sum_ {i=1}^{m} Actioness_i + \frac{1}{m-1} \sum_{j=1}^{m-1} Overlap_{j,j+1} \]

For every possible combination of ToIs we calculate their score as shown in figure \ref{fig:connection_algo}.

\begin{figure}[h]
  \centering
  \includegraphics[scale=0.225]{connection_algo}
  \caption{An example of calculating connection score for 3 random ToIs taken from \cite{DBLP:journals/corr/HouCS17}}
  \label{fig:connection_algo}
\end{figure}

The above approach, however, needs too much memory for all needed calculations, so a memory usage  problem is
appeared. The reason is, for every new video segment, we propose \textit{k ToIs} (16 during training and 150 during validation).
As a result, for a small video separated into  \textbf{10 segments}, we need to calculate 
\textbf{  150\textsuperscript{10} scores} during validation stage. This causes our system to overload and it takes too much time to process
just one video. \par

In order to deal with this problem, we create a greedy algorithm in order to find the candidates tubes. Intuitively, this algorithm after
getting  a new video segment, keeps tubes with a score higher than a threshold, and deletes the rest. So, we don't need to calculate combinations with a
very low score. We wrote code for calculating tubes' scores in CUDA language, which has the ability to
parallel process the same code using different data. Our algorithm is described below:

\begin{enumerate}
\item Firstly,  initialize empty lists for the final tubes, final tubes' duration, their scores, active tubes, their corresponding duration,
  active tubes' overlapping sum and actioness sum where:
  \begin{itemize}
  \item Final tubes list contains all tubes which are the most likely to contain an action, and their score list contains their
    corresponding scores. We refer to each tube by its index, which is related a tensor, in which we saved all the ToIs proposed
    from TPN for each video segment.
  \item Active tubes list contains all tubes that will be matched with the new ToIs. Their overlapping sum list and actioness sum list
    contain their sums in order to avoid calculating then for each loop. 
  \end{itemize}
Also, we initialize  connection threshold equal to 0.5 .
\item For the first video segment, we add all the ToIs to both active tubes and final tubes. Their scores are only their actioness because
  there are no tubes for calculating their overlapping score. So, we set their overlaping sum equal to 0.
\item For each next video, after getting the proposed ToIs, firstly we calculate their overlapping score with each active tube. Then, we
  empty active tubes, active tubes' duration, overlapping sum and actioness score lists.  For each new tube that has a score higher than the
  connection threshold,  we add both to final tubes and to active tubes and their corresponding lists, and we increase their duration.
\item If the number of active tubes is higher than a threshold, we set connection threshold equal to the score of
  the 100th higher score. On top of that, we update the final tubes list, removing all tubes that have score lower than the threshold.
\item After that, we add in active tubes, the current video segment's proposed ToIs, alongside with their actioness scores in the  actioness sum list and
  zero values in corresponding positions in the overlaps sum list (such as in the 1st step).
\item We repeat the previous 3 steps until there is no video segment left.
\item Finally, as we mentioned before, we have a list which contains the indexes of the saved tubes. So, we modify them in order to have
  the corresponding bounding boxes. However, 2 succeeding ToIs do not, always, have exactly the same bounding boxes in the frames that overlap. For example,
  ToIs from the $1^{st}$ video segment start from frame 1 to frame 16. If we have video step equal to 8, these ToIs overlap temporally
  with the ToIs from the  succeeding video segment in frames 8-16. In those frames, in final tube, we choose the area that contains both bounding boxes which is
  denoted as $(min(x_1,x'_1), min(y_1,y'_1), max(x_2,x'_2), max(y_2,y'_2))$ for bounding boxes $(x_1,y_1,x_2,y_2)$ and $(x'_1,y'_1,x'_2,y'_2)$.
\end{enumerate}
% We implement this algorithm using CUDA language for counting the scores. In \cite{DBLP:journals/corr/HouCS17}, they use temporal \textit{stride = sample duration} during testing. We want 
\subsection{JHMDB Dataset}

In order to validate our algorithm, we firstly experiment in JHMDB dataset's videos in order to define the best overlapping policy and
the video overlapping step. Again, we use recall as evaluation metric. A groundtruth action tube is considered to be found, as well as positive,
if there is at least 1 video tube which overlaps with it over a predefined threshold, otherwise it is considered not found.  These thresholds are again 0.5, 0.4 and 0.3.
We set TPN to return 30 ToIs per video segment.
We chose to update threshold when active tubes are more than 500 and to keep the first 100 tubes as active. We did so, because, a big part of the
code is performing in the CPU. That's because, we use lists, which are very easy to handle for adding and removing elements. So, if we use bigger update
limits, it takes much more time to process them.

\paragraph{Sample duration equal to 16 frames} At first, we use as sample duration = 16 frames and video step = 8. As overlapping frames we count frames
\textit{(8...15)} so we have overlapping scores from \#8 frames, which we calculate their average value. Also, we use only \#4 frames with combinations \textit{(8...11), (10...13) and (12...15)} and 
\#2 frames with combinations \textit{(8,9), (10,11), (12,13), and (14,15)}. The results are shown in Table \ref{table:step8_16} (in bold are
the frames with which we calculate the overlap score).

\newpage
\begin{center}
\begin{longtable}{||c||c c c||}

  \hline
  \multirow{2}{5em}{combination} & {} &overlap thresh & {} \\
                                    &  0.5  &  0.4 &  0.3 \\         
  \hline  \hline
  0,1,...,\textbf{\{8,...,15\}}               & {} & {} & {} \\
  \textbf{\{8,9,...,15\}},16,...,23           & 0.3172 & 0.4142 & 0.6418 \\
  \hline     \hline                          


  0,1,...,\textbf{\{8,...,11,\}}...,14,15     & {} & {} & {} \\
  \textbf{\{8,...,11\}},12,...,22,23          & 0.3172 & 0.4142& 0.6381 \\
  \hline
  0,1,...,\textbf{\{10,...,13,\}}14,15,       & {} & {} & {} \\
  8,9,\textbf{\{10,...,13\}},14,...,22,23     & 0.3209 &0.4179   & 0.6418 \\
  \hline
  0,1,...,\textbf{\{12,...,15,\}}             & {} & {} & {} \\
  8,9,...,\textbf{\{12,...,15\}},16,...,23,   & 0.3284 & 0.4216 & 0.6381 \\
  \hline     \hline                          

  0,1,...,\textbf{\{8,...,11,\}},...,14,15,   & {} & {} & {} \\
  \textbf{\{8,9,...,11,\}}12,...,22,23        & 0.3172   & 0.4142& 0.6381 \\
  \hline
  0,1,...,\textbf{\{10,...,13,\}}14,15,       & {} & {} & {} \\
  \textbf{\{10,...,13\}},14,...,22,23         & 0.3209 &0.4179   & 0.6418 \\
  \hline
  0,1,...,\textbf{\{12,...,15\}}              & {} & {} & {} \\
  8,9,...,\textbf{\{12,...,15\}},16,...       & 0.3284 & 0.4216 & 0.6381 \\
  \hline \hline
  
  0,1,...,\textbf{\{8,9,\}},10,...,14,15,     & {} & {} & {} \\
  \textbf{\{8,9,\}}10,11,...,22,23            & 0.3134   & 0.4104 & 0.6381 \\
  \hline
  0,1,...,\textbf{\{10,11,\}},12,...,14,15,   & {} & {} & {} \\
  8,9,\textbf{\{10,11,\}}12,...,22,23         & 0.3209   & 0.4216 & 0.6418 \\
  \hline
  0,1,...,\textbf{\{12,13,\}},14,15,          & {} & {} & {} \\
  8,9,...,\textbf{\{12,13,\}}14,...,22,23     & 0.3246   & 0.4179 & 0.6418 \\
  \hline
  0,1,...,13,\textbf{\{14,15,\}}              & {} & {} & {} \\
  8,9,...,\textbf{\{14,15,\}}16,...,22,23     & 0.3321   & 0.4216 & 0.6306 \\
  \hline \hline
  \caption{Recall results for step = 8}
  \label{table:step8_16}
\end{longtable} 
\end{center}

As we can from the above table, generally we get very bad performance and we got the best performance when we calculate the overlap between only 2 frames (either \textit{14,15} or \textit{12,13}).
So, we thought that we should increase the video step because, probably, the connection algorithm is too strict into big movement variations during  the video. As a result, we set video step = 12 which
means that we have only 4 frames overlap. In this case,  for \#4 frames, we only have the combination \textit{(12...15)}, for \#2 frames we have \textit{(12,13), (13,14) and (14,15)} as shown in
Table \ref{table:step12_16}.

\begin{center}
\begin{longtable}{||c||c c c||}

  \hline
  \multirow{2}{5em}{combination} & {} &overlap thresh & {} \\
                                    &  0.5  &  0.4 &  0.3 \\         
  \hline  \hline
  0,1,...,11,\textbf{\{12,...,15\}}           & {} & {} & {} \\
  \textbf{\{12,13,...,15\}},16,...,26,27         & 0.3769 & 0.4627 & 0.6828 \\
  \hline     \hline                          

  0,1,...,\textbf{\{12,13,\}},14,15,          & {} & {} & {} \\
  \textbf{\{12,13,\}}14,15,...,26,27          & 0.3694   & 0.4627 & 0.6903 \\
  \hline                          
  0,1,...,12\textbf{\{13,14,\}},15,           & {} & {} & {} \\
  12,\textbf{\{13,14,\}}15,...,26,27          & 0.3843   & 0.4627 & 0.6828 \\
  \hline                          
  0,1,...,12,13\textbf{\{14,15,\}}            & {} & {} & {} \\
  12,13,\textbf{\{14,15,\}}16,...,26,27       & 0.3694   & 0.459 & 0.6828 \\
  \hline     \hline                          

  \caption{Recall results for step = 12}
  \label{table:step12_16}
\end{longtable} 
\end{center}

As we can see, recall performance is increased so that means that our assumption was correct. So again, we increase video step into 14, 15 and 16 frames
and recall score is shown in Table \ref{table:step14_16}
\begin{center}
\begin{longtable}{||c||c c c||}

  \hline
  \multirow{2}{5em}{combination} & {} &overlap thresh & {} \\
                                    &  0.5  &  0.4 &  0.3 \\         
  \hline  \hline
  0,1,...,13\textbf{\{14,15\}}                & {} & {} & {} \\
  \textbf{\{14,15\}},16,...,28,29                & 0.3731 & 0.5336 & 0.6493 \\
  \hline     \hline                          

  0,1,...,13,\textbf{\{14,\}}15,              & {} & {} & {} \\
  \textbf{\{14,\}}15,...,28,29                & 0.3694   & 0.5299 & 0.6455 \\
  \hline                          
  0,1,...,14,\textbf{\{15\}}                  & {} & {} & {} \\
  14,\textbf{\{15,\}}16,...,28,29             & 0.3731   & 0.5187 & 0.6381 \\
  \hline  \hline

  0,1,...,14,\textbf{\{15\}}                & {} & {} & {} \\
  \textbf{\{15\}},16,...,30                 & 0.3918 & 0.5187 & 0.6381 \\
  \hline     \hline                          
  0,1,...,14,\textbf{\{15\}}                & {} & {} & {} \\
  \textbf{\{16\}},17,...,31                 & 0.4067 & 0.7313 & 0.8731 \\
  \hline                          
  \caption{Recall results for steps = 14, 15 and 16}
  \label{table:step14_16}
\end{longtable} 
\end{center}

The results show that we get the best recall performance when we have no overlapping steps and video step = 16 = sample duration. We try to improve
more our results, using smaller duration because, as we saw from TPN recall performance, we get better results when we have sample duration = 8 or 4.

\paragraph{Sample duration equal to  8}
We wanted to confirm that we get the best results, when we have no overlapping frames and step = sample duration. So Table \ref{table:step4_8}
shows recall performance for sample duration = 8 and video step = 4 and Table \ref{table:step8_678 } for video steps = 6, 7 and 8.


\begin{center}
\begin{longtable}{||c||c c c||}

  \hline
  \multirow{2}{5em}{combination} & {} &overlap thresh & {} \\
                                    &  0.5  &  0.4 &  0.3 \\         
  \hline  \hline
  0,1,2,3,13\textbf{\{4,5,6,7\}}                & {} & {} & {} \\
  \textbf{\{4,5,6,7\}},8,9,10,11                & 0.2015   & 0.3582 & 0.5858 \\
  \hline     \hline                          

  0,1,2,3,\textbf{\{4,5,\}}6,7                  & {} & {} & {} \\
  \textbf{\{4,5,\}}6,7,8,9,10,11                & 0.1978   & 0.3582 & 0.5933 \\
  \hline                          
  0,1,2,3,4\textbf{\{5,6,\}}7                   & {} & {} & {} \\
  4,\textbf{\{5,6,\}}7,8,9,10,11                & 0.1978   & 0.3507 & 0.5821 \\
  \hline                          
  0,1,2,3,4,5\textbf{\{6,7\}}                   & {} & {} & {} \\
  4,5,\textbf{\{6,7,\}}8,9,10,11                & 0.194   & 0.3433 & 0.585 \\
  \hline                           
  \caption{Recall results for step = 4}
  \label{table:step4_8}
\end{longtable} 
\end{center}

\begin{center}
\begin{longtable}{||c||c c c||}

  \hline
  \multirow{2}{5em}{combination} & {} &overlap thresh & {} \\
                                    &   0.5  &  0.4 &  0.3 \\         
  \hline  \hline
  0,1,2,3,4,5\textbf{\{6,7\}}           & {} & {} & {} \\
  \textbf{\{6,7\}},8,9,10,11,12,13      & 0.3134  & 0.7015 & 0.8619 \\
  \hline     \hline                          

  0,1,2,3,4,5,\textbf{\{6,\}}7          & {} & {} & {} \\
  \textbf{\{6,\}}7,8,9,10,11,12,13      & 0.3209  & 0.6679 & 0.847 \\
  \hline                          
  0,1,2,3,4,5,6,\textbf{\{7\}}          & {} & {} & {} \\
  6,\textbf{\{7\}}8,9,10,11,12,13       & 0.3172  & 0.6567 & 0.8507 \\
  \hline                          
  0,1,2,3,4,5,6\textbf{\{7\}}           & {} & {} & {} \\
  \textbf{\{7,\}}8,9,10,11,12,13,14     & 0.5597  & 0.7687 & 0.903 \\
  \hline                           
  0,1,2,3,4,5,6\textbf{\{7\}}           & {} & {} & {} \\
  \textbf{\{8\}}9,10,11,12,13,14,15     & 0.653	  & 0.8396 &0.9179  \\
  \hline                           
  \caption{Recall results for steps = 6, 7 and 8}
  \label{table:step8_678 }
\end{longtable} 
\end{center}

According to Tables \ref{table:step4_8} and \ref{table:step8_678 }, it is clearly shown that, we achieve the  best results, for $step = sample duration$ and overlapping scores is calculated between the last box of the current tubes and the first box of next tubes.

\subsection{UCF-101 dataset}
In previous steps, we tried to find the best overlap policy for our algorithm in JHMDB dataset. After that, it's time to apply our algorithm in UCF-101 dataset using the best scoring
overlap policy. We did some modifications in the code, in order to use less memory and we moved most parts of the code to the GPU. This happened by using tensors instead of lists for scores while
most operations are, from now on, matrix operations. On top that, the last step of the algorithm, which is the modification from indices to actual action tubes is written, now, in CUDA code so
it takes place in the GPU, too. So, we are now able to increase the number of ToIs returned by TPN, the max number of active tubes before updating threshold and the max number of final
tubes. \par
The first experiments we performed were related to the number of the final tubes, our network proposes alongside with TPN's proposed
tubes' number. We experiment for cases, in which TPN proposes 30, 100 and 150 ToIs, our final network proposes 500, 2000 and 4000
candidate action tubes for sample durations equal to 8 and 16 frames.
For the sample duration equal to 8 we return 100 ToIs because, when we tried to return 150 proposed ToIs, we got OutOfMemory error.
Table \ref{table:ucf_recall} show the spatiotemporal recall and MABO performance of those approaches. Furthermore, Table \ref{table:ucf_temp_recall } show their temporal recall and MABO performance. We are interested in temporal performance, because UCF-101 consists of
untrimmed videos, unlike JHMDB which has only trimmed videos. So, we want to know how well our network is able to propose action tubes that
overlap temporally with the groundtruth action tubes over a ``big'' threshold. For temporal localization, we don't use 0.5, 0.4 and 0.3
overlapping threshold, but instead, we use 0.9, 0.8 and 0.7, because it is very important our network to be able to propose tubes that
 contain an action, at least from the temporal perspective. In order to calculate the temporal overlap, we use IoU for 1 dimension as described before.

\begin{center}
\begin{longtable}{||c | c | c ||c c c | c|}

  \hline
  \multirow{2}{*}{combination} & \multirow{2}{2.5em}{TPN tubes} & \multirow{2}{2.5em}{Final tubes} &  {} & overlap thresh & {} & \multirow{2}{*}{MABO} \\
  {} & {} & {} &  0.5 &  0.4 & 0.3 & {}\\         
  \hline
  
  \multirow{6}{7em}{0,1,...,6,\textbf{\{7,\}}
  \textbf{\{8,\}}9,...,14,15 }  & \multirow{3}{*}{30} & 500   & 0.2829  & 0.4395 & 0.5817  & 0.3501 \\
  \cline{3-7}
  {} &  {}   & 2000   & 0.3567  & 0.4996 & 0.6289 & 0.3815\\
  \cline{3-7}
  {} &  {}   & 4000   & 0.3749  & 0.5316 & 0.6487 & 0.3934 \\
  \cline{2-7}
  {} &  \multirow{3}{*}{100}   & 500   & 0.2966  & 0.451 & 0.5947 & 0.356 \\
  \cline{3-7}
  {} &  {}   & 2000   & 0.3757  & 0.5163 & 0.6471 & 0.3902 \\
  \cline{3-7}
  {} &  {}   & 4000  & 0.3977  & 0.5506 & 0.6624 & 0.4029 \\
  \hline                                    
  \multirow{6}{7em}{0,1,...,14,\textbf{\{15,\}}
  \textbf{\{16,\}}17,18,...,23 }  & \multirow{3}{*}{30} & 500   & 0.362  & 0.5042 & 0.6243 & 0.3866 \\
  \cline{3-7}
  {} &  {}   & 2000   & 0.416  & 0.5468 & 0.6631 & 0.4108  \\
  \cline{3-7}
  {} &  {}   & 4000   & 0.4281  & 0.5589  & 0.6779 & 0.4182 \\
  \cline{2-7}
  {} &  \multirow{3}{*}{150}   & 500 & 0.3589  & 0.4981 & 0.6198 & 0.3845 \\
  \cline{3-7}
  {} &  {}   & 2000   & 0.4129  & 0.5392  & 0.6563 & 0.4085 \\
  \cline{3-7}
  {} &  {}   & 4000   & 0.4266  & 0.5521 & 0.6722 & 0.4162\\
  \hline                                    

  \caption{Recall results for UCF-101 dataset}
  \label{table:ucf_recall}
\end{longtable} 
\end{center}

\begin{center}
\begin{longtable}{||c | c | c ||c c c| c|}

% \begin{table}
%   \centering
%   \setlength{\tabcolsep}{4pt}
%   \begin{tabular}{||c | c | c ||c c c| c|}

  \hline
  \multirow{2}{*}{combination} & \multirow{2}{2.5em}{TPN tubes} & \multirow{2}{2.5em}{Final tubes} &  {} &overlap thresh & {} & \multirow{2}{*}{MABO} \\
  {} & {} & {} &  0.9 &  0.8 & 0.7 & {}\\         
  \hline
  
  
  \multirow{6}{7em}{0,1,...,6,\textbf{\{7,\}}
  \textbf{\{8,\}}9,...,15 }  & \multirow{3}{*}{30} & 500   & 0.4464  & 0.581 & 0.6844  & 0.7787 \\
  \cline{3-7}
  {} &  {}   & 2000   & 0.635  & 0.7665 & 0.8403 & 0.8693 \\
  \cline{3-7}
  {} &  {}   & 4000   & 0.7034  & 0.8228 & 0.8875 & 0.8973 \\
  \cline{2-7}
  {} &  \multirow{3}{*}{100}   & 500   & 0.454 & 0.5924 & 0.692 & 0.783 \\
  \cline{3-7}
  {} &  {}   & 2000   & 0.651 & 0.7696 & 0.8441 &0.8734 \\
  \cline{3-7}
  {} &  {}   & 4000   & 0.7209 &0.8312 & 0.8913 & 0.9026 \\

  \hline                                    
  \multirow{6}{7em}{0,1,...,14,\textbf{\{15,\}}
  \textbf{\{16,\}}17,18,...,23 }  & \multirow{3}{*}{30} & 500   & 0.6844 &0.8327 & 0.9027 & 0.8992 \\
  \cline{3-7}
                                    {} &  {}   & 2000   & 0.7475 & 0.8684 & 0.9217 & 0.9175 \\
  \cline{3-7}
                                    {} &  {}   & 4000   & 0.7567  & 0.8745  & 0.9255 & 0.9211 \\
  \cline{2-7}
                                    {} &  \multirow{3}{*}{150}   & 500   & 0.7498 &0.8707 &0.9171 & 0.9125 \\
  \cline{3-7}
                                    {} &  {}   & 2000   & 0.8243 & 0.911 & 0.9392 & 0.9342\\
  \cline{3-7}
                                    {} &  {}   & 4000   &  0.8403  & 0.9179 & 0.9437 & 0.9389\\
  \hline                                    
  % \end{tabular}
  \caption{Temporal Recall results for UCF-101 dataset}
  \label{table:ucf_temp_recall }
% \end{table}
\end{longtable} 
\end{center}

According to Table \ref{table:ucf_recall}, we achieve better recall and MABO performance when we set sample duration equal to 16.
In all cases,  recall performance of simulations with sample duration equal to  16 outweight the corresponding with 8, with the difference
varying from 2\% to 8\%. In addition, we get best recall and MABO performance when our system proposes 4000 tubes. As we can see,
the ratio of good proposals increases about 5\%-7\% when we change number of proposed tubes from 500 to 2000. This ratio increases more
when we double returned action tubes, from 2000 to 4000. However, this increase is only about 1\%-2\%, which make us rethink if this increase
is worth to be performed. That's because, this modification increases memory usage, because of 4000 proposed action tubes, instead of
2000. Finally, Table \ref{table:ucf_recall} shows that, for sample duration = 8, changing the number of ToIs produced by TPN, slightly
helps our network to achieve better results. This contribution is measured about 1\%-2\%.
On the contrary, when we set sample duration equal to 16, it slightly reduces network's performance. Taking all the aforementioned
results into account, we think that the most suitable choices for connection approaches are, for the sample duration equal to 8, the one in
which TPN returns 100 ToIs and our network proposes 4000 action tubes, and for the sample duration equal to 16, the one in which,
TPN returns 30 ToIs and the network 4000 action tubes. \par
Additionally, Table \ref{table:ucf_temp_recall } shows some interesting facts, too. At first, it confirms that increasing the number of proposed
action tubes, from 500 to 4000, increases recall and MABO performance. Also, we get better results when network has 16 frames as sample
duration, too. However, unlike Table \ref{table:ucf_recall}, Table \ref{table:ucf_temp_recall } shows that when we increase TPN's number
of proposed ToIs, it increases performances for both sample durations. For sample duration equal to 8, this increase results in improving
recall performances by 2\% and MABO performance by 1\% like spatiotemporal recall and MABO. For the sample duration equal to 16, recall
performance is increasing by about 8\% and MABO by 1\%-2\%.  \par
Taking both tables into consideration, we think that the best approach is TPN returning 30 proposed ToIs, network returning 4000 proposed
action tubes and sample duration equals with 16. We didn't choose TPN returning 150 proposed ToIs because, based on MABO performances,
they different only by 1\%, difference which is insignificant.


\subsubsection{Adding NMS algorithm}

Previous section describes the performances of network's proposals for variations in the number of  TPN's  returned ToIs, number of returned
proposed action tubes and sample duration. For each situation, we choose the k-best scoring action tubes, without taking into account
any relation between these action tubes, like their spatiotemporal overlap. So, like TPN's approach, we thought that we should apply
nms algorithm before choosing k-best scoring tubes, in order to further improve  spatiotemporal and temporal, recall and MABO  performance.
We experiment using again two sample durations, 16 and 8 frames per video segment, number or TPN's returning tubes equal to 30 and the
number of final picked action tubes equal to 4000. NMS algorithm uses a threshold in order to choose if 2 action tubes overlap enough. We
experiment setting this threshold equal to 0.7 and 0.8 and  the results are shown in Table \ref{table:ucf_nms_recall} for Spatiotemporal
performance and at Table \ref{table:ucf_nms_temp_recall} for temporal performance.

\begin{center}
  \setlength{\tabcolsep}{2pt}
\begin{longtable}{||c | c | c | c c c| c|}

  \hline
  \multirow{2}{*}{combination} & \multirow{2}{2.5em}{NMS thresh} & \multirow{2}{3.5em}{PreNMS tubes} &  {} &overlap thresh & {} & \multirow{2}{*}{MABO} \\
  {} & {} & {} &  0.5 &  0.4 & 0.3 & {}\\         
  \hline
  \multirow{3}{7em}{0,1,...,6,\textbf{\{7,\}}
  \textbf{\{8,\}}9,...,15 }  & 0.7 &\multirow{3}{*}{20000}  & 0.346 & 0.5202 & 0.657 & 0.3824685269 \\
  \cline{2-2} \cline{4-7} 
  {} &  0.8   & {}   & 0.3643 & 0.5392 & 0.6578 & 0.3904727407 \\
  \cline{2-2} \cline{4-7} 
  {} &  0.9   & {}   & 0.397  & 0.5574 & 0.6677 & 0.4031543642 \\
  \hline                                    
  \multirow{3}{7em}{0,1,...,14,\textbf{\{15,\}}
  \textbf{\{16,\}}17,...,23 }  & 0.7 & \multirow{3}{*}{20000}   & 0.3939 & 0.5559  & 0.6882 & 0.404689056 \\
  \cline{2-2} \cline{4-7} 
                                    {} &  0.8   & {}   & 0.4259 & 0.5764 & 0.6981 & 0.419487652 \\
  \cline{2-2} \cline{4-7} 
                                    {} &  0.9   & {}   & 0.4494 & 0.5856 & 0.7019 & 0.4302611039 \\

  \hline                                    

  \caption{Spatiotemporal Recall results for UCF-101 dataset}
  \label{table:ucf_nms_recall}
\end{longtable} 
\end{center}

\newpage
\begin{center}
  \setlength{\tabcolsep}{2.2pt}
\begin{longtable}{||c | c | c | c c c| c|}

  \hline
  \multirow{2}{*}{combination} & \multirow{2}{2.5em}{NMS thresh} & \multirow{2}{3.5em}{PreNMS tubes} &  {} &overlap thresh & {} & \multirow{2}{*}{MABO} \\
  {} & {} & {} &  0.9 &  0.8 & 0.7 & {}\\         
  \hline
  \multirow{3}{7em}{0,1,...,6,\textbf{\{7,\}}
  \textbf{\{8,\}}9,...,15 }  & 0.7 &\multirow{3}{*}{20000}  & 0.6281 & 0.8251 & 0.9027 & 0.8885141223  \\
  \cline{2-2} \cline{4-7} 
  {} &  0.8   & {}   & 0.7369 & 0.8616 & 0.9148 & 0.9106069806 \\
  \cline{2-2} \cline{4-7} 
  {} &  0.9   & {}   &  0.7787 & 0.8753 & 0.9209 & 0.9212593589 \\
  \hline                                    
  \multirow{3}{7em}{0,1,...,14,\textbf{\{15,\}}
  \textbf{\{16,\}}17,...,23 }  & 0.7 & \multirow{3}{*}{20000}   & 0.7452 & 0.8920 & 0.9361 & 0.920331595 \\
  \cline{2-2} \cline{4-7} 
  {} &  0.8   & {}   & 0.8160 & 0.9278 & 0.9506 & 0.93612757 \\
  \cline{2-2} \cline{4-7} 
  {} &  0.9   & {}   & 0.854 & 0.9346 & 0.9529 & 0.9434986107 \\
  \hline                                    

  \caption{Temporal Recall results for UCF-101 dataset}
  \label{table:ucf_nms_temp_recall}
\end{longtable} 
\end{center}

Comparing Table \ref{table:ucf_nms_recall} with Table \ref{table:ucf_recall},  we notice that NMS algorithm improves recall and MABO
performance when NMS threshold is equal to 0.9. When we set it equal to 0.7 or 0.8, we get worse results. This happens, probably, because
nms algorithm removes some good proposals. Comparing these results with results presented at Tables \ref{table:ucf_recall} and \ref{table:ucf_temp_recall } it becomes clear that using NMS algorithm is very useful. That's because, even though we get the same number of proposed action tubes,
these tubes are not very close spatiotemporally, so this makes proposed action tubes more likely to contain actual foreground action tubes.

\subsubsection{Stop updating threshold}

In previous approaches, scoring threshold was updated each time our algorithm gathered a significant number of ``active'' tubes in order not
to add action tubes with score below this score. However, after serious consideration, we came to the conclusion that sometimes, the updated
threshold leads to not detecting action tubes that start after some frames. That's because, until then, linking threshold may be too big that
won't let new action tubes to be created. So, we came with the modification of not updating linking threshold, but just filtering proposed
tubes, by keeping k-best scoring each time their number is bigger that the a specific number. The rest algorithm remains the same. Tables
\ref{table:ucf_nms_noup_recall} and \ref{table:ucf_nms_noup_temp_recall} show spatiotemporal and temporal recall and MABO performance respectively.
We experiment for cases in which either we don't use the  NMS algorithm at all, either we set overlap threshold equal to 0.7 and 0.9 as shown
below.

\begin{center}
  \setlength{\tabcolsep}{2pt}
\begin{longtable}{||c | c | c | c c c| c|}

  \hline
  \multirow{2}{*}{combination} & \multirow{2}{2.5em}{NMS thresh} & \multirow{2}{3.5em}{PreNMS tubes} &  {} &overlap thresh & {} & \multirow{2}{*}{MABO} \\
  {} & {} & {} &  0.5 &  0.4 & 0.3 & {}\\         
  \hline
  \multirow{3}{7em}{0,1,...,6,\textbf{\{7,\}}
  \textbf{\{8,\}}9,...,15 }   &   \multicolumn{2}{|c|}{-}     &  0.3779 & 0.5316 & 0.6471 & 0.393082961 \\
  \cline{2-7}
  {} & 0.7 &\multirow{2}{*}{20000}  & 0.3483  & 0.5194 & 0.6471 & 0.3783524086 \\
  \cline{2-2} \cline{4-7} 
  {} &  0.9   & {}   & 0.416 & 0.5605 & 0.6722 & 0.4074053106 \\
  \hline                                    
  \multirow{3}{7em}{0,1,...,14,\textbf{\{15,\}}
  \textbf{\{16,\}}17,...,23 }  &   \multicolumn{2}{|c|}{-} & 0.438 & 0.5635 & 0.6829 & 0.4231788 \\
  \cline{2-7}
  {} & 0.7 & \multirow{2}{*}{20000}   & 0.4525 & 0.5848 & 0.7034 & 0.429747438 \\
  \cline{2-2} \cline{4-7} 
  {} &  0.9   & {}   & 0.3802 & 0.5133 & 0.6068 & 0.3862278851848662 \\

  \hline                                    

  \caption{Spatiotemporal Recall results for UCF-101 dataset}
  \label{table:ucf_nms_noup_recall}
\end{longtable} 
\end{center}

\begin{center}
  \setlength{\tabcolsep}{2.2pt}
\begin{longtable}{||c | c | c | c c c| c|}

  \hline
  \multirow{2}{*}{combination} & \multirow{2}{2.5em}{NMS thresh} & \multirow{2}{3.5em}{PreNMS tubes} &  {} &overlap thresh & {} & \multirow{2}{*}{MABO} \\
  {} & {} & {} &  0.9 &  0.8 & 0.7 & {}\\         
  \hline

  \multirow{3}{7em}{0,1,...,6,\textbf{\{7,\}}
    \textbf{\{8,\}}9,...,15 }  &   -   & -    & 0.7087 & 0.8281 & 0.8913 & 0.899210587 \\
  \cline{2-7} 
  {} & 0.7 &\multirow{2}{*}{20000}  & 0.6586 & 0.854 & 0.9278 & 0.903373468 \\
  \cline{2-2} \cline{4-7} 
  {} &  0.9   & {}   &  0.8137 & 0.8973 & 0.9361 & 0.9333068498 \\
  \hline                                    
  \multirow{3}{7em}{0,1,...,14,\textbf{\{15,\}}
  \textbf{\{16,\}}17,...,23 }  &   \multicolumn{2}{|c|}{-} & 0.8327 & 0.9156 &0.9399 & 0.940143272 \\
  \cline{2-7}
  {} & 0.7 & \multirow{2}{*}{20000}& 0.8646 & 0.9369 & 0.9567 & 0.946701832 \\
  \cline{2-2} \cline{4-7} 
  {} &  0.9   & {}   & 0.6183 & 0.7696 & 0.8388 & 0.8628507037919737 \\
  \hline                                    

  \caption{Temporal Recall results for UCF-101 dataset}
  \label{table:ucf_nms_noup_temp_recall}
\end{longtable} 
\end{center}

Comparing recall and MABO performances shown at Table \ref{table:ucf_nms_noup_recall} with those included in Tables \ref{table:ucf_nms_recall}
and \ref{table:ucf_recall}, we deduce that for the sample duration equal to 8, stopping updating linking threshold results in worse performance
when we set nms threshold equal to 0.7, but it achieves the best performances  when setting NMS threshold equal to 0.9 . Furthermore, for sample duration
equal to 16, we get, now,  best performance when we set nms threshold equal to 0.7 and worse performance for nms threshold equal to 0.9 .

\subsubsection{Soft-nms instead of nms}

After widely experiment using NMS algorithm, we thought that we should try to use Soft-NMS algorithm, introduced by \cite{DBLP:journals/corr/BodlaSCD17} and described in chapter 2. We implement our own soft-nms algorithm modifying it in order to calculate spatiotemporal overlapping
scores, and not just spatial, like the one implemented by \cite{DBLP:journals/corr/BodlaSCD17}. As mentioned before, instead of removing action tubes, Soft-NMS algorithm just reduces their score for those which overlap over a predefined threshol. We experiment for the sample duration
equal to 8 and thresholds equal to 0.7 and 0.9, because, our implementation ran out of memory for sample duration equal to 16.
Recall and MABO performance are presented in Tables \ref{table:ucf_softnms_recall} and \ref{table:ucf_softnms_temp_recall}

\begin{center}
  \setlength{\tabcolsep}{2pt}
\begin{longtable}{||c | c | c | c c c| c|}

  \hline
  \multirow{2}{*}{combination} & \multirow{2}{2.5em}{NMS thresh} & \multirow{2}{3.5em}{PreNMS tubes} &  {} &overlap thresh & {} & \multirow{2}{*}{MABO} \\
  {} & {} & {} &  0.9 &  0.8 & 0.7 & {}\\         
  \hline
  \multirow{2}{7em}{0,1,...,6,\textbf{\{7,\}}
  \textbf{\{8,\}}9,...,15 }  & 0.7 &\multirow{2}{*}{20000}  & 0.3916 & 0.5384 & 0.6464 & 0.3964639 \\
  \cline{2-2} \cline{4-7} 
  {} &  0.9   & {}   & 0.4023 & 0.5430 & 0.6502 & 0.398845313 \\
  \hline                                    

  \caption{Spatiotemporal Recall results for UCF-101 dataset using Soft-NMS}
  \label{table:ucf_softnms_recall}
\end{longtable} 
\end{center}

\begin{center}
  \setlength{\tabcolsep}{2.2pt}
\begin{longtable}{||c | c | c | c c c| c|}

  \hline
  \multirow{2}{*}{combination} & \multirow{2}{2.5em}{NMS thresh} & \multirow{2}{3.5em}{PreNMS tubes} &  {} &overlap thresh & {} & \multirow{2}{*}{MABO} \\
  {} & {} & {} &  0.9 &  0.8 & 0.7 & {}\\         
  \hline

  \multirow{2}{7em}{0,1,...,6,\textbf{\{7,\}}
    \textbf{\{8,\}}9,...,15 }  & 0.7 &\multirow{2}{*}{20000}  & 0.7521 & 0.8586 & 0.9110 & 0.915746097  \\
  \cline{2-2} \cline{4-7} 
  {} &  0.9   & {}   & 0.7741 & 0.8768 & 0.9255 & 0.922677864 \\
  \hline                                    

  \caption{Temporal Recall results for UCF-101 dataset using SoftNMS}
  \label{table:ucf_softnms_temp_recall}
\end{longtable} 
\end{center}

Taking results at Tables \ref{table:ucf_softnms_recall} and \ref{table:ucf_softnms_temp_recall} into consideration, alongside with those
at Tables \ref{table:ucf_nms_noup_recall} and \ref{table:ucf_nms_noup_temp_recall} for sample duration equal to 8, we notice that
using softNMS results in slightly better results. This happens when we set overlapping threshold equal to 0.9, otherwise, for
overlapping threshold 0.7, we get worst performance. Despite the fact that softnms results in better recall and MABO performance,
our implementation is very slow, which means that for a 201-frame video, softNMS part lasts about 32 seconds on the contrary with
standard NMS algorithm without updating linking threshold, in which this part last only 2 seconds. So, we choose to use the standard NMS
algorithm without updating liking threshold as an algorithm for  removing overlapping action tubes.

\section{Second approach: use progression and progress rate}
As we saw before, our first connecting algorithm doesn't have very good recall results. So, we created another algorithm which is based on
the one proposed by \cite{DBLP:journals/corr/abs-1903-00304}. This
algorithm introduces two new metrics according to \cite{DBLP:journals/corr/abs-1903-00304}:

% TODO add more description
\begin{description}
\item[ Progression,  ] which describes the probability of a specific action being performed in the ToI. 
  We add this factor because we have noticed that actioness is tolerant to false positives. Progression is
  mainly a rescoring mechanism for each class (as mentioned in \cite{DBLP:journals/corr/abs-1903-00304})

\item [ Progress rate, ] which is defined as the progress proportion that each action class has been performed.
  
\end{description}

So, each action tube is described as a set of ToIs
\[  T = {\{ {\bf t}_i^{(k)} | {\bf t}_i^{(k)} = ( t_i^{(k)}, s_i^{(k)}, r_i^{(k)} ) \}}_{i=1:n^{(k)},k=1:K} \]
where $ t_i^{(k)} $ contains ToI's spatiotemporal information, $ s_i^{(k)} $ its confidence score and $ r_i^{(k)} $ its progress rate.

In this approach, each class is handled separately, so for the rest section, we discuss action tube generation for one class only. In order to link 2 ToIs, for
a video with N video segments, the following steps are applied:
\begin{enumerate}
\item For the first video segment (k = 1), initialize an array with the M best scoring ToIs, which will be considered as active action tubes ( AT ).
  Correspondingly, initialize an array with M progress rates  and M confidence scores.
\item For k = 2:N, execute (a) to (c) steps:
  \begin{enumerate}
  \item Calculate overlaps between $ AT^{(k)} $ and $ ToIs^{(k)}. $
  \item Connect all tubes which satisfy the following criteria:
    \begin{enumerate}
    \item $ overlap score(at_i^{(k)},t_j^{(k)})   > \theta, 
      at  \in AT^{(k)}, t \in ToIs^{(k)}  $
    \item $r(at_i^{(k)}) < r(t_j^{(k)}) $ or 
      $r(t_i^{(k)}) - r(at_i{(k)}) < \lambda $
    \end{enumerate}
    
  \item For all new tubes update confidence score and progress rate as follows:
    \begin{description}
    \item The new confidence score is the average score of all connected ToIs:
      \[  s_z^{(k+1)} = \frac {1} {n} \sum_{n=0}^{k} s_i^{(n)}\]
    \item New progress rate is the highest progress rate:
      \[r(at_z^{(k+1)} = max(r(at_i^{(k)}), r(t_j^{(k)})) \]
    \end{description}
    % \item If $ progressrate(at_i^{(k)}) < progressrate(t_i^{(k)}) $ then $ progressrate(at_i^{(k+1)}) =
  \item Keep M best scoring action tubes as active tubes and keep K best scoring action tubes for classification.
  \end{enumerate}
  
\end{enumerate}

This approach has the advantage that we don't need to perform classification again because we already know the class of
each final tube. In order to validate our results, now, we calculate the recall only from the tubes which have the same
class as the groundtruth tube. Again, we considered a groundtruth  tube as positive if there is at least one proposed  tube
that overlaps with it over the predefined threshold

\begin{center}
\begin{longtable}{||c c||c c c||}
  \hline
  \multicolumn{2}{||c||}{\textbf{combination}} &\multicolumn{3}{|c||}{\textbf{overlap thresh}}\\

  \hline
  sample dur & step &   0.5  &  0.4 &  0.3 \\
  \hline   \hline
  8 & 6 & 0.3284 & 0.5 & 0.6082  \\
  \hline
  8 & 7 & 0.209	& 0.459 & 0.6119 \\
  \hline
  8 & 8 & 0.3060 & 0.5672 & 0.6866 \\
  \hline
  16 & 8  & 0.194 & 0.4366 & 0.7164 \\
  \hline
  16 & 12 & 0.3358 & 0.5336 & 0.7537 \\
  \hline
  16 & 16 & 0.2649 & 0.4664 & 0.709 \\
  
  \hline 

  \caption{Recall results for second approach with step = 8, 16 and their corresponding steps }
  \label{table:conn_app2}
\end{longtable} 
\end{center}

According to  Table \ref{table:conn_app2}, we get the best performance when we set sample duration equal to  16 and overlap step equal to 12.
Comparing this performance with the first approach, for both sample durations equal to 8 and 16, we notice that second approach falls short
comparing to the first one. 

\section{Third approach : use naive algorithm - only for JHMDB}

As mention in the first approach, \cite{DBLP:journals/corr/HouCS17} calculate all possible sequences of ToIs in order to the find the best
candidates. We rethought about this approach and we concluded that it could be implemented for JHMDB dataset if we reduce the number of proposed
ToIs, produced by TPN,  to 30 for each video clip. We exploited the fact that JHMDB dataset's videos are trimmed, so we do not need to look
for action tubes starting in the second video clip which saves us a lot of memory. On top of that, we modified our code
in order to be more memory efficient  writing some parts using CUDA programming language, saving a lot of processing power, too. \par
So, after computing all possible combinations starting of the first video clip and ending in the last video clip, we keep only the
\textbf{k-best scoring tubes (k = 500) }. We run experiments which have sample duration equal to 8 and 16 frames and we modify the video step each time.
For sample duration = 8, we return only 15 ToIs and for sample duration = 16, we return 30 because, if we return more, we get ``out of memory error''.
In the following table, we can see the recall results. \par  

\begin{center}
\begin{longtable}{||c c||c c c||}

  \hline
  \multicolumn{2}{||c||}{\textbf{combination}} &\multicolumn{3}{|c||}{\textbf{overlap thresh}}\\
  \hline
  sample dur & step &  0.5  &  0.4 &  0.3 \\
  \hline   \hline

  8 & 6 & 0.7873 & 0.8657 & 0.9366  \\
  \hline
  8 & 7 & 0.7836 & 0.8731 & 0.9366  \\
  \hline
  8 &  8 & 0.7910 & 0.8806 & 0.9515 \\
  \hline 

  16 & 8  & 0.7873 & 0.8843 & 0.9291 \\
  \hline
  16 & 12 & 0.7948 & 0.8881 & 0.9403 \\
  \hline
  16 & 16 & 0.7985 & 0.8918 & 0.9515 \\
  \hline 
  \caption{Recall results for third approach with  step = 8, 16 and their
corresponding steps}
  \label{table:conn_app3}
\end{longtable} 
\end{center}

From the above table, firstly, we confirm that when video step is equal to the sample duration gives us the best recall results.
Also, we notice that when sample duration is equal to 16  frames recall gets slightly better that when sample duration is
equal to 8. However, using  16 frames per video segment sample increases the memory usage even though it reduces the number of video segments, because of the
need to process bigger videos, bigger feature maps etc. So for the classification stage we will experiment using mostly sample duration equal
with 8 frames.

\section {General comments}

\begin{figure}[h]
  % \includegraphics[scale=0.7]{convolutional_neural_network_structure} \]
  \centering
  % \includegraphics[scale=0.2]{tube_ex2_half}
  \includegraphics[width= 0.8\textwidth, height=0.45\textheight]{tube_ex2_half}
  \caption{Example of connected tubes}
  \label{fig:tube_ex}
\end{figure}

Figure \ref{fig:tube_ex} shows the example presented in chapter 3, after linking first video segment's first ToI with a ToI proposed for the second video segment. For this case, we used the third proposed method, including calculating all possible combinations. As shown in the Figure,
our algorithm manages to track the actor performing the action efficiently enough. This means that even though the actor moves during the
video, our approach manages not to loose contact with him. On top of that, it clear that the silhouette of the actor changes, and so does the area of proposed action tubes. The only problem which appears is that proposed action tubes sometimes exceeds the area of the actual video.
In order to deal with this problem, we set bounding boxes not to exceed these areas by keeping the limits of the original picture. So,
from now on, no bounding box will overlap with padding area.

% \end{document}

\documentclass{report}

\usepackage{subcaption} % package for subfigures
\usepackage{hyperref}  % package for linking figures etc
\usepackage{enumitem}  % package for description with bullets
\usepackage{graphicx}  % package for importing images
\usepackage{mathtools} % package for math equation
\usepackage{mathrsfs}  % package for math font
\usepackage{indentfirst} % package for getting ident after section or paragraph
\usepackage[export]{adjustbox}
\usepackage{longtable} % package for multi pages tables
\usepackage{multirow}  % package for tables, multirow
% \usepackage{amsmath}

\setlength{\parindent}{2em} % how much indent to use when we start a paragraph

\graphicspath{ {./theory/figures/} }       % path for images

\begin{document}

\chapter{Classification stage}
\section{Description}
After getting all proposed tubes, it's time to do classification. As classifiers we use several approaches includingn
a Recursive Neural Network (RNN) Classifier, a Support Vector Machine (SVM) Classifier and a Multilayer perceptron (MLP).

\begin{figure}[h]
  % \includegraphics[scale=0.7]{convolutional_neural_network_structure} \]
  \centering
  \includegraphics[scale=0.42]{model_prenms}
  \caption{Structure of the whole network}
  \label{fig:whole_network}
\end{figure}

The whole procedure of classification is consisted from the following steps:
\begin{enumerate}
\item Seperate video into small video clips. Feed TPN network those video clips and get as output
  k-proposed ToIs and their corresponding features for each video clip.
\item Connect the proposed ToIs in order to get video tubes which may contain an action.
\item For each candidate video tube, which is a sequence of ToIs, feed it into the classifier
  for verification.
\end{enumerate}

The general structure of the whole network is depicted in figure \ref{fig:whole_network}, in which we can see the previous steps if we
follow the arrows.  \par
In first steps of classification stage we refer only to JHMDB dataset because it has smaller number of video than UCF dataset which
helped us save a lot of time and resources. That's because  we performed most experiments only JHMDB and after we found the optimal
situation, we implemented to UCF-dataset, too. 

\section{Preparing data for classification -  RNN Classifier}

% \textbf{(Pending... Introduction about Linear and RNN classifiers)}
\textbf{(Pending.. also an image of RNN classifier)}

In order to train our classifier, we have to execute the previous steps for each video. However, each video
has different number of frames and reserves too much memory in the GPU. In order to deal with this situation,
we give as input one video per GPU. So we can handle 4 videos simultaneously. This means that a regular
training session takes too much time for just 1 epoch. \par
The solution we came with, is to precompute the features for both positive video tubes and negative video tubes.
Then we feed those features to our classifier and we train it in order to disciminate their classes.
At first, we extract only groundtruth video tubes' features and the double number of background video tubes. We chose this
ratio between positive and negative tubes inspired by \cite{jjfaster2rcnn}, in which it has 0.25 ratio between foreground
and background rois and chooses 128 roi in total. Respectively, we chose a little bigger ratio because we have only 1 groundtruth
video tube in each video. So, for each video we got 3 video tubes in total, 1 for positive and 2 for background. We considered
background tubes those whose overlap scores with groundtruth tubes are $ \ge 0.1 $ and $ \le 0.3 $. \par

Then, after extracting those features, we trained both linear and RNN classifiers. The Linear classifier needs a fixed input size,
so we used a pooling function in the dimension of the videos. So, at first we had a feature map of 3,512,16 dimensions and then we
get as output a feature maps of 512,16 dimensions. We used both max and mean pooling as show in the results below. For the RNN
classifier, we do not use any pooling function before feeding it. For both classifiers, at first, we didn't considered a fixed
threshold for confidence score.

\textbf{(Pending results in Linear... Table)}

The results are disappointing.
As we can see in the table, RNN classifier cannot classify very well because, probably, the duration of the videos are so small
so we stopped using it in jHMDB dataset. In the Linear classifier, we noticed that every tube is considered as background tube.
That means that Linear classifier gets overfitted with trained data and cannot handle unknown data. So, we thought that we
need a classifier which can \"learn\" very easily, with little data. So we chose to try a support vector machine classifier.

\section{Support Vector Machine (SVM)}
\subsection{First steps}
SVMs are classifiers defined by a separating hyperplane between trained data in a N-dimensional space. The main advantage of using a SVM
is that can get very good classification results when we have few data available. 
\textbf{write more introduction and a pic, Pending...} \par
The use of SVM is inspired from \cite{Girshick:2015:FR:2919332.2920125} and it is trained using hard negative mining. 
This means that we have 1 classifier per class which has only 2 labels, positive and negative. We mark as positive the feature maps of the
groundtruth action, and as negative groundtruth actions from other classes, and feature maps from background classes.
As we know, SVM is driven by small number of examples near decision boundary. Our goal is to find a set of negatives that are the closest to
the seperating hyperplane. So in each iteration, we update this set of negatives adding those which our SVM didn't perform very well. Each
SVM is trained independently. \par
SVM code is take from Microsoft's Azure \href{https://github.com/Azure/ObjectDetectionUsingCntk} {github page} in which there is an implementation
of Fast RCNN using a SVM classifier. We didn't modify its parameters which means that it has a linear kernelr, uses  L2-norm as penalty and L1-norm
as loss during training. Also, we consider as hard-negatives the tubes that got score $ >  -1.0 $ during classification.\par
This whole process makes the choise of the negatives a crutial factor. In order to find the best policy,  we came with 5 different cases to consider
as negatives:
\begin{enumerate}
\item Negatives are other classes's positives and all the background tubes
\item Negatives are only all the background videos
\item Negatives are only other classes's positives
\item Negatives are other classes's positives and background tubes taken only from videos that contain a positive tube
\item Negatives are only background tubes taken from videos that contain a positive tube
\end{enumerate}

On top of that, we use 2 pooling functions in order to have a fixed input size. \par
In the next tables, we show our architecture's  mAP performance when we follow each one of the above policies. Also,
we experimented for 2 feature maps, \textit{(64,8,7,7)} and \textit{(256,8,7,7)} where 8 equals with the sample duration.
Both feature maps were extracted by using 3D RoiAlign procedure from feature maps with dimensions \textit{(64,8,28,28)} and
\textit{(256,8,7,7)} respectively (in the second case, we just add zeros in the feature map outsize from the bounding boxes for
each frame). Table \ref{table:svm_first_results} contains the first classification results. At first column we have the dimensions
of feature maps before pooling fuction, where k = 1,2,..5 . At second column we have feature maps' dimensions after pooling, and at
the next 2 column, the type of pooling function and the policy we followed. Finally in the last 3 collumns we have the mAP performance
when we have threshold equal with 0.3, 0.4 and 0.5 respectively. During validation, we keep only the best scoring tube because we know that
we have only 1 action per video.

\begin{center}
\begin{longtable}{||c | c | c| c||c c c||}

  \hline
  \multicolumn{2}{||c|}{\textbf{Dimensions}} & \multirow{2}{*}{ \textbf{Pooling}} &\multirow{2}{*}{\textbf{Type}} & \multicolumn{3}{|c||}{\textbf{mAP precision}}\\

   before & after &  {} & {} &  0.3 &  0.4 & 0.5 \\
 \hline   \hline
 \multirow{5}{*}{(k,64,8,7,7)} & \multirow{5}{*}{(1,64,8,7,7)} & \multirow{5}{*}{mean}  & 1 &  3.16 & 4.2 & 4.4    \\
  \cline{4-7}
  {} & {} & {} & 2 & 2.29 & 2.68 & 2.86    \\
    \cline{4-7}
  {} & {} & {} & 3 & 1.04 & 1.04 & 1.04    \\
    \cline{4-7}
  {} & {} & {} & 4 & TODO & TODO & TODO    \\
    \cline{4-7}
  {} & {} & {} & 5 & TODO & TODO & TODO    \\
  \hline
 \multirow{5}{*}{(k,64,8,7,7)} & \multirow{5}{*}{(1,64,8,7,7)} & \multirow{5}{*}{max}  & 1 & TODO & TODO & TODO \\
    \cline{4-7}
  {} & {} & {} & 2 & TODO & TODO & TODO    \\
    \cline{4-7}
  {} & {} & {} & 3 & TODO & TODO & TODO    \\
    \cline{4-7}
  {} & {} & {} & 4 & TODO & TODO & TODO    \\
    \cline{4-7}
  {} & {} & {} & 5 & TODO & TODO & TODO     \\

  \hline   \hline

 \multirow{5}{*}{(k,256,8,7,7)} & \multirow{5}{*}{(1,256,8,7,7)} & \multirow{5}{*}{mean}  & 1 &  11.41 & 11.73 & 11.73 \\

    \cline{4-7}
  {} & {} & {} & 2 & 10.35 & 10.92 &11.89 \\
    \cline{4-7}
  {} & {} & {} & 3 & 8.93 9.64 9.94 \\
    \cline{4-7}
  {} & {} & {} & 4 & TODO & TODO & TODO \\
    \cline{4-7}
  {} & {} & {} & 5 & 5.92 & 6.92 & 7.79 \\
    \hline
 \multirow{5}{*}{(k,256,8,7,7)} & \multirow{5}{*}{(1,256,8,7,7)} & \multirow{5}{*}{max}  & 1  & 22.07 & 24.4 & 25.77  \\
    \cline{4-7}
  {} & {} & {} & 2  & 14.07 & 16.56 & 17.74 \\
    \cline{4-7}
  {} & {} & {} & 3  & 14.22 & 18.94 &21.6 \\
    \cline{4-7}
  {} & {} & {} & 4  & 21.05 & 24.63 & 25.93 \\
    \cline{4-7}
  {} & {} & {} & 5  & 11.6 & 13.92 & 15.81 \\
  \hline   
  
  \caption{Our architecture's performance using 5 different policies and 2 different feature maps while pooling in
  tubes' dimension. With bold is the best scoring case}
  \label{table:svm_first_results}

\end{longtable} 
\end{center}

\subsection{Modifying 3D Roi Align}
As we mentioned before, we extract from each tube its activation maps using 3D Roi Align procedure and we set equal to zero the pixels outside
of bounding boxes for each frame. We came with the idea that the enviroment surrounding the actor sometimes help us determine the class
of the action which is performed. This is base in the idea that 3D Convolutional Networks use the whole scene in order to classify the action
that is performed. We thought to extend a little each bounding box both in width and height. So, during Roi Align procedure, after resizing
the bounding box into the desired spatial scale  ( in our case 1/16 because original sample size = 112 and resized sample size = 7 )
we increase by 1 both width and height. According to that if we have a resized bounding box $( x_1,y_1,x_2,y_2) $ our new bounding box becomes
$ (max(0,x_1-0.5),max(0,y_1-0.5),min(7,x_2+0.5),min(7,y_2+0.5)) $ ( we use \textit{ min} and \textit{max} functions in order to avoid exceeding feature maps' limits).
We just experiment in policies 1 and 4 for both (256,8,7,7) and (64,8,7,7) feature maps as show in  Table \ref{table:svm_mod_roialign}


\begin{center}
\begin{longtable}{||c | c | c| c||c c c||}

  \hline
  \multicolumn{2}{||c|}{\textbf{Dimensions}} & \multirow{2}{*}{ \textbf{Pooling}} &\multirow{2}{*}{\textbf{Type}} & \multicolumn{3}{|c||}{\textbf{mAP precision}}\\

   before & after &  {} & {} &  0.3 &  0.4 & 0.5 \\
 \hline   \hline
 \multirow{5}{*}{(k,64,8,7,7)} & \multirow{5}{*}{(1,64,8,7,7)} & \multirow{5}{*}{mean}  & 1 & TODO & TODO & TODO  \\
  \cline{4-7}
  {} & {} & {} & 2 & TODO & TODO & TODO \\
    \cline{4-7}
  {} & {} & {} & 3 & 0.96 & 2.11 & 2.9   \\
    \cline{4-7}
  {} & {} & {} & 4 &  2.4  & 4.53 & 5.16    \\
    \cline{4-7}
  {} & {} & {} & 5 &  0.6  & 0.76 & 0.93    \\
  \hline
 \multirow{5}{*}{(k,64,8,7,7)} & \multirow{5}{*}{(1,64,8,7,7)} & \multirow{5}{*}{max}  & 1 &  1.07 & 1.85 & 2.26    \\
    \cline{4-7}
  {} & {} & {} & 2 &  3.2 & 3.39 & 3.66    \\
    \cline{4-7}
  {} & {} & {} & 3 &  0.86 & 1.98 & 1.98    \\
    \cline{4-7}
  {} & {} & {} & 4 &  1.24 & 2.68 & 3.09    \\
    \cline{4-7}
  {} & {} & {} & 5 &  0.21 & 0.27 & 0.32    \\

  \hline   \hline

 \multirow{5}{*}{(k,256,8,7,7)} & \multirow{5}{*}{(1,256,8,7,7)} & \multirow{5}{*}{mean}  & 1 &  10.62 & 10.94 & 10.94    \\

    \cline{4-7}
  {} & {} & {} & 2 &  9.09  & 10.02 & 10.83   \\
    \cline{4-7}
  {} & {} & {} & 3 &  9.05  & 9.65  & 9.69    \\
    \cline{4-7}
  {} & {} & {} & 4 &  11.19 & 11.51 & 11.51   \\
    \cline{4-7}
  {} & {} & {} & 5 &  4.84  & 6.13  & 6.13   \\
    \hline
 \multirow{5}{*}{(k,256,8,7,7)} & \multirow{5}{*}{(1,256,8,7,7)} & \multirow{5}{*}{max}  & 1  & \bf 20.94 & \bf 24.96 & \bf 26.46   \\
    \cline{4-7}
  {} & {} & {} & 2  & 14.94 & 17.78 & 19.38   \\
    \cline{4-7}
  {} & {} & {} & 3  & 14.9 & 17.39 & 19.88   \\
    \cline{4-7}
  {} & {} & {} & 4  & 19.43 & 23.91 & 25.31   \\
    \cline{4-7}
  {} & {} & {} & 5  & 10.41 & 10.46 & 11.29   \\
  \hline   
  
  \caption{Our architecture's performance using 5 different policies and 2 different feature maps extracted using modified Roi Align.}
  \label{table:svm_mod_roialign}

\end{longtable} 
\end{center}

From the above results we notice that features map with dimension (256,8,7,7) outperform in all cases, both for mean and max pooling and
for all the policies. Also, we can see that max pooling outperforms mean pooling in all cases, too. Last but not least, we notice that policies
2, 3 and 5 give us the worst results which means that svm needs both data from other classes positives and from background tubes. 

\subsection{Temporal pooling}
After getting first results, we implement a temporal pooling function inspired from \cite{DBLP:journals/corr/HouCS17}. We need a
fixed input size for the SVM. However, our tubes' temporal stride varies from 2 to 5. So we use as fixed temporal pooling equal
with 2. As pooling function we use 3D max pooling, one for each filter of the feature map.  So for example, for an action tube
with 4 consecutive ToIs, we  have 4,256,8,7,7 as feature size. We seperate the feature map into 2 groups using \textit{linspace}
function and we reshape the feature map into 256,k,8,7,7 where k is the size of each group, After using 3D max pooling, we get
a feature map 256,8,7,7 so finally we concat them and get 2,256,8,7,7. In this case we didn't experiment with (64,8,7,7) feature
maps because it wouldn't performed better that (256,8,7,7) ferature maps as noticed from the previous section.

\begin{center}
\begin{longtable}{||c | c| c| c||c c c||}

  \hline
 \multicolumn{2}{||c|}{\textbf{Dimensions}} & \multirow{2}{*}{\textbf{Pooling}} &\multirow{2}{*}{ \textbf{Type}} &\multicolumn{3}{|c||}{\textbf{mAP precision}}\\

  before & after & {} & {} & 0.3 &  0.4 & 0.5 \\
  \hline   \hline

  \multirow{5}{*}{k,256,8,7,7} & \multirow{5}{*}{2,256,8,7,7} & \multirow{5}{*}{mean}  & 1 & TODO & TODO & TODO \\
  \cline{4-7}

  {} & {} & {} & 2 & TODO & TODO & TODO \\
  \cline{4-7}
  {} & {} & {} & 3 & TODO & TODO & TODO \\
  \cline{4-7}
  {} & {} & {} & 4 & TODO & TODO & TODO \\
  \cline{4-7}
  {} & {} & {} & 5 & TODO & TODO & TODO \\
  \hline

  \multirow{5}{*}{k,256,8,7,7} & \multirow{5}{*}{2,256,8,7,7} & \multirow{5}{*}{max}  & 1 & 25.07 & 26.91 & 29.11 \\
  \cline{4-7}

  {} & {} & {} & 2 & TODO & TODO & TODO \\
  \cline{4-7}
  {} & {} & {} & 3 & TODO & TODO & TODO \\
  \cline{4-7}
  {} & {} & {} & 4 & TODO & TODO & TODO \\
  \cline{4-7}
  {} & {} & {} & 5 & TODO & TODO & TODO \\
  \hline

  \caption{mAP results for second approach using temporal pooling }
  \label{table:svm_temp_pooling}
\end{longtable} 
\end{center}

\subsection{Increasing sample duration to 16 frames}

\begin{center}
\begin{longtable}{||c | c| c| c||c c c||}

  \hline
 \multicolumn{2}{||c|}{\textbf{Dimensions}} & \multirow{2}{4.5em}{\textbf{Temporal Pooling}} &\multirow{2}{*}{ \textbf{Type}} &\multicolumn{3}{|c||}{\textbf{mAP precision}}\\

  before & after & {} & {} & 0.3 &  0.4 & 0.5 \\
  \hline   \hline

  \multirow{5}{*}{k,256,16,7,7} & \multirow{5}{*}{1,256,16,7,7} & \multirow{5}{*}{No}  & 1 & 23.4 & 27.57 &28.65  \\
  \cline{4-7}

  {} & {} & {} & 2 & TODO & TODO & TODO \\
  \cline{4-7}
  {} & {} & {} & 3 & TODO & TODO & TODO \\
  \cline{4-7}
  {} & {} & {} & 4 & 22.7 & 26.95 & 28.05 \\
  \cline{4-7}
  {} & {} & {} & 5 & TODO & TODO & TODO \\
  \hline

  \multirow{5}{*}{k,256,16,7,7} & \multirow{5}{*}{2,256,16,7,7} & \multirow{5}{*}{Yes}  & 1 & 21.12 & 24.07 & 24.36  \\
  \cline{4-7}

  {} & {} & {} & 2 & TODO & TODO & TODO \\
  \cline{4-7}
  {} & {} & {} & 3 & TODO & TODO & TODO \\
  \cline{4-7}
  {} & {} & {} & 4 & 18.36 & 23.09 & 23.75 \\
  \cline{4-7}
  
  {} & {} & {} & 5 & TODO & TODO & TODO \\
  \hline
  \caption{mAP results for   }
  \label{table:svm_temp_pooling_16}
\end{longtable} 
\end{center}

\subsection{Adding more groundtruth tubes}
From above results, we notice that SVM improve a lot the perfomance of our model. In order to futher improve our results, we will
add more groundtruth action tubes. We consider as groundtruth action tubes all the tubes whose overlap score  with a groundtruth tube is
greater that 0.7 . Also, we increase the total number of tube to 8. Table \ref{table:svm_increased}

\begin{center}
\begin{longtable}{||c | c| c| c||c c c||}

  \hline
 \multicolumn{2}{||c|}{\textbf{Dimensions}} & \multirow{2}{*}{\textbf{Pooling}} &\multirow{2}{*}{ \textbf{Type}} &\multicolumn{3}{|c||}{\textbf{mAP precision}}\\

  before & after & {} & {} & 0.3 &  0.4 & 0.5 \\
  \hline   \hline

  \multirow{2}{*}{k,64,8,7,7} & \multirow{2}{*}{2,64,7,7} & \multirow{2}{*}{max}  & 1 & TODO & TODO & TODO \\
  \cline{4-7}
  {} & {} & {} & 4 & TODO & TODO & TODO   \\
  \hline   
  \multirow{2}{*}{k,256,8,7,7} & \multirow{2}{*}{2,256,7,7} & \multirow{2}{*}{max}  & 1 & TODO & TODO & TODO \\
  \cline{4-7}
  {} & {} & {} & 4 & TODO & TODO & TODO   \\
  \hline   
  \caption{Results after increasing the number of total tubes }
  \label{table:svm_increased}

\end{longtable} 
\end{center}

\section{MultiLayer Perceptron (MLP)}
Last but not least approach is a Multilayer perceptorn (MLP). More specifically, we extract the 3 last residulal layers of 3D ResNet34
and we add a classification layer.  

\subsection{Extract features}
We

\subsection{Regular training}

\section{Classifying dataset UCF}
As mentioned before, all the above results are from JHMDB dataset. 
\section{Final Improvements}
After classification, we relize that a lot of classified tubes overlap and represent the same action. So, we use again NMS algorithm in order
to remove unnecessary tubes. The new model can be seen in figure \ref{fig:network_nms}.

\begin{figure}[h]
  % \includegraphics[scale=0.7]{convolutional_neural_network_structure} \]
  \centering
  \includegraphics[scale=0.7]{model_nms}
  \caption{Structure of the network with NMS}
  \label{fig:network_nms}
\end{figure}

\end{document}

% \documentclass{report}

% \usepackage{subcaption} % package for subfigures
% \usepackage{hyperref}  % package for linking figures etc
% \usepackage{enumitem}  % package for description with bullets
% \usepackage{graphicx}  % package for importing images
% \usepackage{mathtools} % package for math equation
% \usepackage{mathrsfs}  % package for math font
% \usepackage{indentfirst} % package for getting ident after section or paragraph
% \usepackage[export]{adjustbox}
% \usepackage{multirow}  % package for tables, multir
% \usepackage{amssymb}
% % \usepackage{tabu}   % for tables 
% \setlength{\parindent}{2em} % how much indent to use when we start a paragraph

% \graphicspath{ {./theory/figures/} }       % path for images

% \begin{document}

\chapter{Conclusion - Future work}

\section{Conclusion}
In this thesis, we explored the problem of action recognition and localization. We design a network based on the approach proposed by \cite{DBLP:journals/corr/HouCS17}
combined with some elements presented by \cite{DBLP:journals/corr/abs-1712-09184}, \cite{Ren:2015:FRT:2969239.2969250}, \cite{Girshick:2015:FR:2919332.2920125},
\cite{DBLP:journals/corr/abs-1903-00304} and \cite{hara3dcnns}. \par

We wrote a pytorch implementation, expanding code only from \cite{jjfaster2rcnn}. Furthermore, we wrote our own code using some CUDA functions designed by us (like
calculating connection scores, modifying tubes etc). \par

We tried to  a design the  Tube Proposal Network for proposing ToIs in given video segments, inspired by Faster R-CNN's RPN.
We designed it using general anchors and not dataset specific anchors. This means that we try to generalize our approach for several datasets, on the contrary with
the approach proposed by \cite{DBLP:journals/corr/abs-1712-09184}, in which it uses the most frequently appearing anchors for each dataset.

On top of that, we designed a naive connection algorithm for connecting  proposed ToIs based on the algorithm proposed by \cite{DBLP:journals/corr/abs-1712-09184}.
In our approach, we use the same scoring policy, which is a combination between actioness and overlapping scores. The main difference is that we avoid to calculate
all the possible combinations using an updatable threshold. We, also, tried another connection algorithm inspired by \cite{DBLP:journals/corr/abs-1903-00304}. However,
our implementation wasn't very good so, we didn't explore all of its potentials. \par

Finally, we explored several classifiers for the classification stage of our network, which are a RNN, an SVM and an MLP.  We used an implementation taken from Fast RCNN
for the SVM classifier, which included hard negatives mining training procedure. We explored some training techniques for best classification performance and
2 training approaches for MLP classifier, the classic one and using pre-extracted features. 

\section{Future work}
There is a lot of room for improvement for our network, in order to achieve state-of-the-art results. The most important are described in the next paragraphs.

\paragraph{Improving TPN proposals} We implemented 2 networks for proposing action tubes in a video segment. We managed to achieve about 63\% recall score for
sample duration = 16 and about 80\% recall for sample duration = 8. Theses scores show that there is plenty room for improvement, especially for sample duration = 16.
Even though a lot of networks' architectures have been explored for regression, a good idea would be to try other networks, not necessarily inspired by object detection
networks like we did. On top of that, adding a $\lambda$ factor in training loss would be a good idea and exploring which is the best approach.
So training loss could be defined as:
\begin{equation} 
\begin{split}
 L  =  \sum_iL_{cls}(p_i, p_i^*) + \lambda_1 \sum_ip_i^*L_{reg}(t_i,t_i^*) + \lambda_2  \sum_iq_i^*L_{reg}(c_{i}, c_{i}^*) \\
\end{split}
\end{equation}

Furthermore, it would be a good idea to use SSD's (\cite{DBLP:journals/corr/LiuAESR15}) proposal network instead of RPN, in order to compare result. Finally,
we could experiment using Feature Pyramid Networks(\cite{8099589}), which could be exteded in 3 dimensions as another feature extractor or some other type of 3D ResNet.

\paragraph{Changing Connection algorithm}
In this thesis, another challenge we came was connecting proposed ToIs for proposing action tubes. We implemented a very naive algorithm, which wasn't
able to give us very good proposals despite the changes we tried to do. We implemented another connection algorithm which was based on estimationing temporal
progress of an action tube  and its overlap with others. Although it also didn't give us very good proposals, we believe that we should explore this algorithm's potential. That's
because it takes advantage of the progress of the action, which the previous algorithm didn't.

\paragraph{Explore other  classification techniques}
For classification stage, we experimented mainly on an SVM classifier for JHMDB dataset and we didn't get involved with UCF dataset. Our first
goal is to be able to get good classification results for UCF dataset, too.  We think that we should explore UCF's feature maps and techniques applied at feature maps before  classification. In addition, we could try other classification techniques like random forest or experiment more with RNN classifier for the UCF dataset.
Finally, another classification procedure would be a good idea, like extracting first all the possible action tubes and then using other network's features for classification
stage.

% \end{document}




\en
\addcontentsline{toc}{chapter}{\bibname}
\printbibliography

%%%  Bibliography

\end{document}
